\documentclass{article}
\usepackage[utf8]{inputenc}
\usepackage{enumerate}
\usepackage{ragged2e}
\usepackage{etoc}
\usepackage{amsmath}
\usepackage{graphicx}
\graphicspath{ {./images/} }

\renewcommand{\thesection}{}
\newcommand{\sent}[1]{$^{#1}$}
\counterwithout{subsection}{section}
\renewcommand{\thesubsection}{§\arabic{subsection}}

\title{Kaiserpakt}
\author{Kaiser Friedrich IV. von Preußen}
\date{27. Februar 1920}

\begin{document}
\maketitle
\begin{center}
    \includegraphics[scale=.15]{dr_wappen}
\end{center}
\newpage
\topskip0pt
\vspace*{\fill}
\begin{Center}
\textbf{1. Fassung}
\vspace*{\fill}
\end{Center}
\newpage
\tableofcontents
\newpage
\section{Vertragliche Definitionen}

\subsection{Vertragliche Gültigkeit}
\begin{enumerate}[(1)]
    \item Die Parteien dieses Abkommens sind die unterzeichnenden Staaten.
    \item Der nachfolgende Vertrag ist gültig, bis von allen Vertragsparteien ein Abkommen zur Aufhebung des Abkommens von Tokiun aufgesetzt und unterschrieben wird.
    \item Das Madagassische Verfassungsgericht ist in der Lage, die Vertragsgültigkeit im Zweifelsfalle für alle Mitglieder ausnahmslos und zeitweise auszusetzen.
    \item Die Aussetzung darf höchstens zwei Monate dauern.
    \item Entscheidungen im Zuge dieses Abkommens müssen von der Mehrheit der Vertragsmitglieder bewilligt werden.
    \item Die vertragliche Anerkennung durch autonome Staaten erfolgt nur durch  Unterschrift durch die ihnen übergeordnete souveräne Vertragsnation, sofern sie nicht explizit im Vertrag mitsamt ihrer Ansprüche erwähnt werden.
    \item Die Bezeichnung der Staaten entspricht deren Namen zum Zeitpunkt der erstmaligen Unterzeichnung.
    \item Nur rechtmäßige Nachfolger der Staaten haben das Recht, die Mitgliedschaft ihres Vorgängers im Vertrag fortzuführen, ohne zu unterzeichnen.
    \item Dies bedeutet jedoch auch die damit einhergehende vollständige Anerkennung des gesamten Inhalts.
    \item Interventionen bezüglich Anspruchsstellungen können nur mit nachvollziehbarem Grund anerkannt werden.    
\end{enumerate}

\subsection{Freiwilligkeit der Unterzeichnung weiterer Abkommen}
Unterzeichnende Staaten dürfen nicht auf Grundlage dieses Abkommens zur Unterzeichnung anderer Abkommen gezwungen werden.

\subsection{Anerkennung weiterer Ansprüche}
\begin{enumerate}[(1)]
    \item Es werden nur Gebietsansprüche anerkannt, die vom Abkommen von Tokiun vorgesehen sind.
    \item Es ist den unterzeichnenden Mitgliedern verboten, Ansprüche anzuerkennen, die im Konflikt mit Absatz 1 stehen.
    \item Alle Vertragsparteien dürfen nur Gebiete auf der Erde beanspruchen.
    \item Die Außenwelt gehört nicht zur Erde.
    \item Alle Vertragsparteien verpflichten sich auch, die Ansprüche anderer Staaten auf Gebiete außerhalb der Erde ebenfalls abzuerkennen.
    \item Absatz 3 ff. gilt auch für die Nichtbeanspruchung der Antarktis.
    \item Hoheitsgewässer sind gültige Ansprüche gelten gemäß angehängter Karte.
    \item Die Hoheitsgewässer entsprechen den eingekreisten Bereichen.
    \item Für Landflächen, die nicht eingekreist sind, gilt dass alles in einem Radius von fünfzig Blöcken Hoheitsgewässer sind.
    \item Besteht ein Konflikt zwischen den Radien zweier verschiedener Länder, so wird in der Mitte der Schnittfläche die Abgrenzung verlaufen.
    \item Wird ein gesamter Ozean oder ein Teil des Ozeans als Gesamtanspruch formuliert, so ist dieser das Hoheitsgewässer des Staats.
    \item Ausnahme zu Absatz 11 sind die Gebiete in einem Radius von fünfzig Blöcken um fremde Küsten herum.
    \item Ein Bereich gilt auch als Hoheitsgewässer, wenn er zwischen zwei oder mehr Gebieten desselben Staats liegt und nicht in Konflikt mit einem fremden Hoheitsgewässer kommt. 
\end{enumerate}

\subsection{Anspruchsverletzungen durch Vertragsparteien}
\begin{enumerate}[(1)]
    \item Jegliche Verletzungen von Ansprüchen gemäß Abkommen von Tokiun, die von Vertragsparteien begangen werden, werden notfalls durch den Ausschluss des Aggressors aus dem Vertrag geahndet.
    \item Durchgesetzte Ansprüche müssen von allen Parteien militärisch verteidigt werden.
\end{enumerate}

\subsection{Anspruchsverletzungen durch andere Staaten}
Wer Ansprüche gemäß Abkommen von Tokiun verletzt und keine Vertragspartei ist, muss militärisch bekämpft werden und notfalls vollkommen erobert werden.

\subsection{Verteidigungspflicht}
\begin{enumerate}[(1)]
    \item Jeder Staat ist gemäß §§4, 5 zur Verteidigung der Abkommensansprüche verpflichtet.
    \item Man kann von der Verteidigungspflicht mit einem offiziellen Schreiben zurücktreten.
    \item Macht man von Absatz 2 Gebrauch, so sind die anderen Abkommensstaaten nicht mehr zur Verteidigung der eigenen Ansprüche verpflichtet.
\end{enumerate}

\subsection{Anspruchsdefinitionen}
\begin{enumerate}[(1)]
    \item Ein durchgesetzter Anspruch ist ein Anspruch, der mittels gemäß Serverkriegsrecht legaler Methoden erworben wurde.
    \item Ein vorgesehener Anspruch ist ein Anspruch, den der Staat in nicht näher zu definierender Zukunft gemäß Abkommen von Tokiun erwerben darf. 
    \item Die Ansprüche werden über am 17. Juli 2023 völkerrechtlich anerkannte Staaten und sonstige geographischen Gegebenheiten der realen Welt definiert.
\end{enumerate}

\section{Gestelle Ansprüche}
\subsection{Keosu Teikoku}
Die Ansprüche von Keosu Teikoku lauten wie folgt:
\begin{enumerate}
    \item Grönland
    \item Svalbard und Jan Mayen
    \item Die Nordhälfte von Nowaja Semlja
    \item Wisconsin gemäß Grenzen zu seiner größten Ausdehnung
\end{enumerate}

\subsection{China}
Die Ansprüche Chinas lauten wie folgt:
\begin{enumerate}
    \item China mitsamt Hongkong und der Insel Taiwan
\end{enumerate}

\subsection{Deutsches Kaiserreich}\label{german}
\begin{enumerate}[(1)]
    \item Die Ansprüche des Deutschen Reichs lauten wie folgt:
    \begin{enumerate}[1.]
        \item Das geographische Europa einschließlich aller Außengebiete und Überseeterritorien, die zu den dazugehörigen Ländern gehören, mit Ausnahme der Inseln Großbritanniens und der Insel Irland samt der dazugehörigen Außengebiete und Überseeterritorien, sowie Island, Svalbard und Jan Mayen
        \item Russland bis zur, in Absatz 2 definierten Grenze
        \item Die südliche Hälfte von Nowaja Semlja
        \item Die Türkei
        \item Syrien
        \item Der Libanon
        \item Israel
        \item Ägypten
        \item Libyen
        \item Algerien
        \item Tunesien
        \item Marokko
        \item Indonesien
        \item Australien
        \item Eine Fläche mit einem 100-Blöcke-Radius um den Mount Everest
        \item Eine Fläche mit einem 100-Blöcke-Radius um Machu Picchu
        \item Eine Fläche mit einem 100-Blöcke-Radius um den Mount St. Elias
        \item Japan mitsamt aller dazugehörigen Inseln
        \item Nordkorea
        \item Südkorea
    \end{enumerate}
    \item Die Grenze zwischen Russland und dem Deutschen Kaiserreich verläuft genau mittig zwischen den Küstenlinien des Obbusens bis zur Ob. Dieser folgt die Grenze exakt mittig zwischen den Ufern bis zur Stadt Labytnangi. Von hier aus verläuft die Grenze parallel zum Äquator bis genau vor das Ural-Gebirge. Diesem folgt die Grenze in gleichem Abstand bis zur untersten Spitze. Von hier aus verläuft die Grenze parallel zum Meridian bis zur Grenze von Kasachstan.    
\end{enumerate}

\subsection{Kasachstan}
Kasachstans Ansprüche lauten wie folgt:
\begin{enumerate}
    \item Kasachstan
\end{enumerate}

\subsection{Russland}
Russlands Ansprüche lauten wie folgt:
\begin{enumerate}
    \item Gesamt Russland östlich der, in \ref{german} Abs. 2 beschriebenen Grenze
\end{enumerate}

\subsection{Orden}
Die Ordensansprüche lauten wie folgt:
\begin{enumerate}
    \item Nepal
    \item Bhutan
    \item Tibet bis unterhalb der chinesischen Autobahn S301
\end{enumerate}

\end{document}