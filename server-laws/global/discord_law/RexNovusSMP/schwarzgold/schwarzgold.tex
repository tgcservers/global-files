\documentclass{article}
\usepackage[utf8]{inputenc}
\usepackage{enumerate}
\usepackage{ragged2e}
\usepackage{etoc}
\usepackage{amsmath}
\usepackage{graphicx}
\graphicspath{ {./images/} }

\renewcommand{\thesection}{}
\newcommand{\sent}[1]{$^{#1}$}
\counterwithout{subsection}{section}
\renewcommand{\thesubsection}{§ \arabic{subsection}}

\title{Schwarzgold-Konkordat}
\author{Kaiser Friedrich IV. von Preußen}
\date{29. Juni 1920}

\begin{document}
\maketitle
\begin{center}
    \includegraphics[scale=.22]{coa_blackgold_concordat.png}
\end{center}
\begin{center}
    \textbf{Zwischen den Parteien\\}\textbf{\\}
    Das \textbf{Deutsche Kaiserreich\\}
    Das Keosunische Reich, im Folgenden \textbf{``Keosu Teikoku''}\\
    \textbf{\\}

    Gemeinsam im Folgenden \textbf{``Die Parteien''}, \textbf{``Die Staaten''}, \textbf{``Die Vertragsparteien''}, \textbf{``Die Abkommensstaaten''}
\end{center}
\newpage
\topskip0pt
\vspace*{\fill}
\begin{Center}
\textbf{Vertragssignatur:\\}
\texttt{DE-988F383C89CF28BFA04065864D5AD122}
\vspace*{\fill}
\end{Center}
\newpage
\tableofcontents
\newpage
\section{Vertragliche Definitionen}

\subsection{Vertragliche Gültigkeit}
\begin{enumerate}[(1)]
    \item Die Parteien dieses Abkommens sind die unterzeichnenden Staaten.
    \item Der nachfolgende Vertrag ist gültig, bis von allen Vertragsparteien ein Abkommen zur Aufhebung des Abkommens aufgesetzt und unterschrieben wird.
    \item ${^1}$Entscheidungen im Zuge dieses Abkommens müssen von der Mehrheit der Vertragsmitglieder bewilligt werden.
    \item Die vertragliche Anerkennung durch autonome Staaten erfolgt nur durch Unterschrift durch die, ihnen übergeordnete souveräne Vertragsnation.
    \item ${^1}$Nur rechtmäßige Nachfolger der Staaten haben das Recht, die Mitgliedschaft ihres Vorgängers im Vertrag fortzuführen, ohne zu unterzeichnen. ${^2}$Als rechtmäßiger Nachfolger gilt, wer von den Mitgliedern des Kaiserpakts den Bestimmungen des Kaiserpakts entsprechend als solcher anerkannt wurde.
    \item Dies bedeutet jedoch auch die damit einhergehende vollständige Anerkennung des gesamten Inhalts.
    \item Dieser Vertrag fordert die vollständige Anerkennung der internationalen Gerichtsbarkeit gemäß Abkommen von Berlin und kann somit Gegenstand internationaler Rechtsprechung werden.
\end{enumerate}

\subsection{Anforderungen an die Vertragsparteien}
Die Bestimmungen dieses Abkommens müssen sinngemäß und vollständig in die Verfassung der unterzeichnenden Staaten oder einen anderweitigen, vollständig für diesen Staat gültigen Gesetzestext übernommen werden.

\section{Vertragsbestimmungen}
\begin{enumerate}[(1)]
    \item Beide Vertragsparteien verpflichten sich, sich wenn möglich gegenseitig militärische Unterstützung zu bieten.
    \item Absatz 1 ist in Hinsicht auf Verteidigungskriege auf eigener Seite verpflichtend und muss auf Anfrage hin geleistet werden.
    \item In Angriffskriegen durch eigene Seite wird es den Parteien lediglich angeraten, Unterstützung auf Anfrage hin zu leisten.
    \item Es dürfen keine Angriffskriege durch die Parteien geführt werden, die nicht einstimmig durch beide Vertragsparteien befürwortet worden sind.
\end{enumerate}

\subsection{Militärische Außenposten}
\begin{enumerate}[(1)]
    \item ${^1}$Beide Vertragsparteien sichern dem jeweils anderen das Recht zu, höchstens zehn Militärbasen auf dem Territorium des jeweils anderen einzurichten. ${^2}$Die Anzahl der Militärbasen kann durch Zusatzbeschlüsse jederzeit angepasst werden, sofern von beiden Parteien unterzeichnet.
    \item Außerdem muss ein zusätzlicher militärischer Außenposten von der jeweils anderen Partei bei sich eingerichtet werden.
    \item Die Standorte und Eigentümer dieser Außenposten unterliegen strengster Geheimhaltung und dürfen nur unter Zustimmung beider Parteien weitergegeben werden.
    \item Sämtlicher Luftraum über und Untergrund unter dem Grundstück sind Eigentums des Lands, dessen militärische Einrichtung dies ist und sind damit Gegenstand von dessen Rechtsprechung.
\end{enumerate}

\subsection{Spionage und Aufklärung}
\begin{enumerate}[(1)]
    \item Spionage und sonstige Arten der Aufklärung durch die Vertragsparteien sind in den Staaten der Vertragsparteien in jedweder Hinsicht ohne Einwilligung der Vertragspartei, auf dessen Territorium dies stattfinden soll, verboten.
    \item ${^1}$Die Parteien im Abkommen verpflichten sich, jegliche Resultate eigener Spionage und Aufklärung den anderen Vertragsparteien ausnahmslos zur Verfügung zu stellen. ${^2}$Dies betrifft auch Erkenntnisse, die vor Unterzeichnung des Abkommens gesammelt wurden.
    \item Insbesondere muss man sämtliche, vertragsfremde Spionageaktivitäten, die sich gegen die Vertragsparteien ohne dessen Genehmigung richten, melden sofern man von diesen erfährt.
\end{enumerate}

\subsection{Militärische Versorgung}
\begin{enumerate}[(1)]
    \item Es soll ständiger technologischer und sonstiger wissenschaftlicher Austausch zwischen den Vertragsparteien bestehen.
    \item Hierzu gehört auch die Lieferung militärischer Güter, die nicht dem Waffenkontrollgesetz unterliegen.
    \item Die Parteien verpflichten sich, von jeglichen Handelssanktionen jeglichen Wirtschaftssektors, die sich gegen eine Vertragspartei richten, abzusehen und den Handel weiter fortzuführen.
\end{enumerate}

\subsection{Territoriale Vereinbarungen}
\begin{enumerate}[(1)]
    \item Das Deutsche Reich verpflichtet sich in diesem Abkommen, Keosu Teikoku die Nordhälfte von Nowaja Semlja, sowie Jan Mayen zu überstellen.
    \item Im Gegenzug erhält es von Keosu Teikoku die Osthälfte von Island, sowie sämtliches Gebiet innerhalb eines hundert Blöcke Radius um den Mount Saint Elias in Alaska.
    \item Die hier genannten Gebiete müssen ausnahmslos entmilitarisiert werden und dessen Durchquerung, sowie die Durchquerung jeglicher Gebiete der Vertragsparteien oder Gewässer, die in mindestens hundert Blöcken Entfernung von diesen liegen, erfordert eine Genehmigung beider Vertragsparteien.
    \item Eine Ausnahme hierzu bilden die Einrichtung gemeinsamer Militärbasen, die Bewegungen von Truppen zu diesen und die Durchquerung besagter Gebiete durch zeremonielle Truppen der Abkommensstaaten.
\end{enumerate}

\end{document}