\documentclass{article}
\usepackage[utf8]{inputenc}
\usepackage{enumerate}
\usepackage{ragged2e}
\usepackage{etoc}
\usepackage{amsmath}
\usepackage{graphicx}
\graphicspath{ {./images} }

\renewcommand{\thesection}{}
\newcommand{\sent}[1]{$^{#1}$}
\counterwithout{subsection}{section}
\renewcommand{\thesubsection}{§ \arabic{subsection}}

\title{Kaiserpakt}
\author{Kaiser Friedrich IV. von Preußen}
\date{27. Februar 1920}

\begin{document}
\maketitle
\begin{center}
    \includegraphics[scale=.15]{dr_wappen}
\end{center}
\begin{center}
    \textbf{Zwischen den Parteien\\}\textbf{\\}
    Das Chinesische Reich, im Folgenden bezeichnet als \textbf{``China''\\}
    Das \textbf{Deutsche Kaiserreich\\} auch im Folgenden bezeichnet als \textbf{``Deutsches Reich''\\}
    \textbf{Kasachstan\\}
    Das \textbf{Keosunische Reich\\}
    Die Deutsch-Keosunische Sonderverwaltungszone der Mönchsrepublik Ratosurya, im Folgenden bezeichnet als \textbf{``Mönchsrepublik Ratosurya''}, \textbf{``Der Orden''\\}
    Das Autonome Hochkönigreich Russland unter Deutscher Krone, im Folgenden bezeichnet als \textbf{``Autonomes Hochkönigreich Russland''}, \textbf{``Russland''\\}
    \textbf{\\}

    Gemeinsam im Folgenden \textbf{``Die Parteien''}, \textbf{``Die Staaten''}, \textbf{``Die Vertragsparteien''}, \textbf{``Die Abkommensstaaten''}
\end{center}
\newpage
\topskip0pt
\vspace*{\fill}
\begin{Center}
\textbf{1. Fassung}
\vspace*{\fill}
\end{Center}
\newpage
\tableofcontents
\newpage
\section{Vertragliche Definitionen}

\subsection{Vertragliche Gültigkeit}
\begin{enumerate}[(1)]
    \item Die Parteien dieses Abkommens sind die unterzeichnenden Staaten.
    \item Der nachfolgende Vertrag ist gültig, bis von allen Vertragsparteien ein Abkommen zur Aufhebung des Kaiserpakts aufgesetzt und unterschrieben wird.
    \item Das Internationale Verfassungsgericht ist in der Lage, die Vertragsgültigkeit im Zweifelsfalle für alle Mitglieder ausnahmslos und zeitweise auszusetzen.
    \item Die Aussetzung darf höchstens zwei Monate andauern.
    \item ${^1}$Entscheidungen im Zuge dieses Abkommens müssen von der Mehrheit der Vertragsmitglieder bewilligt werden. ${^2}$Dies gilt nicht für Entscheidungen der zugrundeliegenden Gerichte.
    \item Die vertragliche Anerkennung durch autonome Staaten erfolgt nur durch Unterschrift durch die, ihnen übergeordnete souveräne Vertragsnation, womit diese auch die Ansprüche des autonomen Gebiets bewilligen.
    \item Die Bezeichnung der Staaten entspricht deren Namen zum Zeitpunkt der erstmaligen Unterzeichnung.
    \item Nur rechtmäßige Nachfolger der Staaten haben das Recht, die Mitgliedschaft ihres Vorgängers im Vertrag fortzuführen, ohne zu unterzeichnen.
    \item Dies bedeutet jedoch auch die damit einhergehende vollständige Anerkennung des gesamten Inhalts.
    \item Interventionen bezüglich Anspruchsstellungen können nur mit nachvollziehbarem Grund anerkannt werden.
    \item Es darf nur den Kaiserpakt unterzeichnen, wer das Abkommen von Berlin unterzeichnet hat.
    \item Als Anspruchsverletzung gilt ebenfalls das Bewegen bewaffneter Einheiten durch fremdes Territorium einer Vertragspartei ohne dessen Erlaubnis.
\end{enumerate}

\subsection{Rechtmäßiger Nachfolger}
\begin{enumerate}[(1)]
    \item Als rechtmäßiger Nachfolger wird derjenige definiert, der die Traditionen und den Namen des Staates weiterführt und durch die Abkommensmitglieder mehrheitlich als Nachfolger legitimiert wurde.
    \item Im Falle einer Sezession und Inkorporation gilt es keinen rechtmäßigen Nachfolger zu bestimmen, zumal dieser weiterhin besteht.
    \item Kommt es zu einer Dismembration oder Fusion, so obliegt es den Vertragsmitgliedern, über die Nachfolge zu entscheiden.
    \item Es ist nicht zwingend notwendig, einen Nachfolgestaat zu bestimmen.
\end{enumerate}

\subsection{Anerkennung weiterer Ansprüche}
\begin{enumerate}[(1)]
    \item Es werden nur Gebietsansprüche anerkannt, die vom Kaiserpakt vorgesehen sind.
    \item Es ist den unterzeichnenden Mitgliedern verboten, Ansprüche anzuerkennen, die im Konflikt mit Absatz 1 stehen.
    \item Alle Vertragsparteien dürfen nur Gebiete auf der Erde beanspruchen.
    \item Die Außenwelt gehört nicht zur Erde.
    \item Alle Vertragsparteien verpflichten sich auch, die Ansprüche anderer Staaten auf Gebiete außerhalb der Erde ebenfalls abzuerkennen.
    \item Absatz 3 ff. gilt auch für die Nichtbeanspruchung der Antarktis.
    \item Hoheitsgewässer sind gültige Ansprüche und gelten gemäß angehängter Karte.
    \item Die Hoheitsgewässer entsprechen den eingekreisten Bereichen.
    \item ${^1}$Für Landflächen, die nicht eingekreist sind, gilt dass alles in einem Radius von fünfzig Blöcken Hoheitsgewässer sind. ${^2}$Im Falle einer Insel wird von der Mitte der Insel als Ursprung des Radius und dem Radius entsprechend der größten Ausdehnung von der Mitte aus bemessen und zusätzlichen fünfzig Blöcken ausgegangen.
    \item Besteht ein Konflikt zwischen den Radien zweier verschiedener Länder, so wird in der Mitte der Schnittfläche die Abgrenzung verlaufen.
    \item Ein Bereich gilt auch als Hoheitsgewässer, wenn er zwischen zwei oder mehr Gebieten desselben Staats liegt und nicht in Konflikt mit einem fremden Hoheitsgewässer kommt, sowie zwischen den, von der Mitte aus gesehen, größten Ausdehnungen keinen Durchmesser von über dreihundert Blöcken hat. 
    \item Ansprüche, die serverrechtlich durchgesetzt sind und nicht in Konflikt mit den Abkommensbestimmungen stehen, bilden eine Ausnahme zu Absatz 1.
\end{enumerate}

\subsection{Anspruchsverletzungen durch Vertragsparteien}
\begin{enumerate}[(1)]
    \item Jegliche Verletzungen von Ansprüchen gemäß Kaiserpakt, die von Vertragsparteien begangen werden, werden notfalls durch den Ausschluss des Aggressors aus dem Vertrag geahndet.
    \item Durchgesetzte Ansprüche müssen von allen Parteien militärisch verteidigt werden.
\end{enumerate}

\subsection{Anspruchsverletzungen durch andere Staaten}
Wer Ansprüche gemäß Kaiserpakt verletzt und keine Vertragspartei ist, muss militärisch bekämpft werden und notfalls vollkommen erobert werden.

\subsection{Verteidigungspflicht}
\begin{enumerate}[(1)]
    \item Jeder Staat ist gemäß §§4, 5 zur Verteidigung der Abkommensansprüche verpflichtet.
    \item Man kann von der Verteidigungspflicht mit einem offiziellen Schreiben zurücktreten.
    \item Macht man von Absatz 2 Gebrauch, so sind die anderen Abkommensstaaten nicht mehr zur Verteidigung der eigenen Ansprüche verpflichtet.
\end{enumerate}

\subsection{Anspruchsdefinitionen}
\begin{enumerate}[(1)]
    \item Ein durchgesetzter Anspruch ist ein Anspruch, der mittels gemäß Serverkriegsrecht legaler Methoden erworben wurde.
    \item Ein vorgesehener Anspruch ist ein Anspruch, den der Staat in nicht näher zu definierender Zukunft gemäß Kaiserpakt erwerben darf. 
    \item Die Ansprüche werden über am 17. Juli 2023 völkerrechtlich anerkannte Staaten und sonstige geographischen Gegebenheiten der realen Welt definiert.
\end{enumerate}

\section{Gestelle Ansprüche}

\subsection{China}
Die Ansprüche Chinas lauten wie folgt:
\begin{enumerate}
    \item China mitsamt Hongkong und mit Ausnahme der Gebiete, wie in \ref{order} Abs. 2 definiert.
    \item Die Insel Taiwan
\end{enumerate}

\subsection{Deutsches Kaiserreich}\label{german}
\begin{enumerate}[(1)]
    \item Die Ansprüche des Deutschen Reichs lauten wie folgt:
    \begin{enumerate}[1.]
        \item Das geographische Europa einschließlich aller Außengebiete und Überseeterritorien, die zu den dazugehörigen Ländern gehören, mit Ausnahme von Svalbard und Jan Mayen
        \item Russland bis zur, in Absatz 2 definierten Grenze
        \item Die südliche Hälfte von Nowaja Semlja
        \item Die Türkei
        \item Syrien
        \item Der Libanon
        \item Israel
        \item Ägypten
        \item Libyen
        \item Algerien
        \item Tunesien
        \item Marokko
        \item Indonesien
        \item Die Salomonen
        \item Vanuatu
        \item Die Region Mikronesien mit Ausnahme von Guam und den Nördlichen Marianen
        \item Australien
        \item Eine Fläche mit einem 100-Blöcke-Radius um den Mount Everest
        \item Eine Fläche mit einem 100-Blöcke-Radius um Machu Picchu
        \item Eine Fläche mit einem 100-Blöcke-Radius um den Mount St. Elias
        \item Japan
        \item Nordkorea
        \item Südkorea
        \item Trinidad und Tobago
        \item Suriname
        \item Guyana
        \item Venezuela mit Ausnahme aller Gebiete westlich des sechsundsechzigsten westlichen Längengrades
    \end{enumerate}
    \item Die Grenze zwischen Russland und dem Deutschen Kaiserreich verläuft genau mittig zwischen den Küstenlinien des Obbusens bis zur Ob. Dieser folgt die Grenze exakt mittig zwischen den Ufern bis zur Stadt Labytnangi. Von hier aus verläuft die Grenze parallel zum Äquator bis genau vor das Ural-Gebirge. Diesem folgt die Grenze in gleichem Abstand bis zur untersten Spitze. Von hier aus verläuft die Grenze parallel zum Meridian bis zur Grenze von Kasachstan.    
\end{enumerate}

\subsection{Kasachstan}
Kasachstans Ansprüche lauten wie folgt:
\begin{enumerate}
    \item Kasachstan
\end{enumerate}

\subsection{Keosunisches Reich}
Die Ansprüche des Keosunischen Reichs lauten wie folgt:
\begin{enumerate}
    \item Grönland
    \item Svalbard und Jan Mayen
    \item Die Nordhälfte von Nowaja Semlja
    \item Kanada, ausschließlich der, Wisconsin zugesicherten Gebiete
    \item Alaska
\end{enumerate}

\subsection{Mönchsrepublik Ratosurya}\label{order}
\begin{enumerate}[(1)]
    \item Die Ordensansprüche lauten wie folgt:
    \begin{enumerate}[1.]
        \item Nepal
        \item Bhutan
        \item Tibet bis unterhalb der, in Absatz 2 definierten Grenze
        \item Die indischen Bundesstaaten Bihar, Assam, Maharashtra, Manipur, Meghalaya, Mizoram, Sikkim, Westbengalen, Jharkhand und Tripura
        \item Bangladesch
    \end{enumerate}
    \item Die Grenze verläuft anfangend an der chinesischen Westgrenze in einer geraden Linie entlang des achtzigsten südlichen Breitengrades
    bis zum Beginn der Straße S301. Dieser folgt sie bis zum Punkt, an dem sie auf die S203 trifft. Von hier aus verläuft sie entlang der Straße,
    bis sie in die G562 übergeht und an dieser weiter entlang. Vom neunundachtzigsten östlichen Längengrad abwärts mündet die Grenze in die S204, der sie weiterhin Richtung Osten folgt.
    Am Ende verläuft sie in einer geraden Linie den neunundachtzigsten östlichen Längengrad entlang nach Süden. 
    \item Die Mönchsrepublik ist unter Deutsch-Keosunischer Administration und darf nicht Gegenstand eines Krieges sein.
    \item Beide Staaten sind zudem verpflichtet, die Mönchsrepublik Ratosurya auch in Kriegszeiten weiterhin kooperativ und friedlich gemeinsam zu führen.
\end{enumerate}

\subsection{Autonomes Hochkönigreich Russland}
\begin{enumerate}[(1)]
    \item Russlands Ansprüche lauten wie folgt:
    \begin{enumerate}[1.]
        \item Gesamt Russland östlich der, in \ref{german} Abs. 2 beschriebenen Grenze
        \item Georgien
        \item Armenien
        \item Aserbaidschan
    \end{enumerate}
    \item Russland ist ein autonomes Gebiet innerhalb des Deutschen Kaiserreichs.
\end{enumerate}

\subsection{Ausnahmefall Island}
Island gilt aufgrund der ausbleibenden Übereinkunft zwischen dem Deutschen Kaiserreich und Keosu Teikoku unbesetzt. Allerdings dürfen nur die beiden
Staaten Anspruch auf dieses Land erheben.

\end{document}