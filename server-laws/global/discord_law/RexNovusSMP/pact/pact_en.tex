\documentclass{article}
\usepackage[utf8]{inputenc}
\usepackage{enumerate}
\usepackage{ragged2e}
\usepackage{etoc}
\usepackage{amsmath}
\usepackage{graphicx}
\graphicspath{ {./images} }

\renewcommand{\thesection}{}
\newcommand{\sent}[1]{$^{#1}$}
\counterwithout{subsection}{section}
\renewcommand{\thesubsection}{§\arabic{subsection}}

\title{Imperial Pact}
\author{Emperor Friedrich IV of Prussia}
\date{February 27, 1920}

\begin{document}
\maketitle
\begin{center}
    \includegraphics[scale=.15]{dr_wappen}
\end{center}
\begin{center}
    \textbf{Between the Parties\\}\textbf{\\}
    The Chinese Empire, hereinafter referred to as \textbf{``China''\\}
    The \textbf{German Empire\\} also hereinafter referred to as \textbf{``German Empire''\\}
    \textbf{Kazakhstan\\}
    The \textbf{Empire of Keosu\\}
    The German-Keosunian Special Administrative Zone of the Monastic Republic of Ratosurya, hereinafter referred to as \textbf{``Monastic Republic of Ratosurya''}, \textbf{``The Order''\\}
    The Autonomous High Kingdom of Russia under German Crown, hereinafter referred to as \textbf{``Autonomous High Kingdom of Russia''}, \textbf{``Russia''\\}
    \textbf{Wisconsin\\}
    \textbf{\\}

    Collectively hereinafter referred to as \textbf{``The Parties''}, \textbf{``The States''}, \textbf{``The Contracting Parties''}, \textbf{``The Treaty States''}
\end{center}
\newpage
\topskip0pt
\vspace*{\fill}
\begin{Center}
\textbf{1st Version}
\vspace*{\fill}
\end{Center}
\newpage
\tableofcontents
\newpage
\section{Contractual Definitions}

\subsection{Contractual Validity}
\begin{enumerate}[(1)]
    \item The parties to this treaty are the signing states.
    \item The following treaty is valid until an agreement to terminate the Imperial Pact is drafted and signed by all contracting parties.
    \item The International Constitutional Court is able to suspend the treaty validity temporarily for all members without exception in case of doubt.
    \item The suspension may last no longer than two months.
    \item ${^1}$Decisions under this treaty must be approved by a majority of the contracting members. ${^2}$This does not apply to decisions of the underlying courts.
    \item Contractual recognition by autonomous states is only effective upon signature by their superior sovereign contracting nation, thereby also approving the claims of the autonomous territory.
    \item The designation of the states corresponds to their names at the time of initial signing.
    \item Only lawful successors of the states have the right to continue the membership of their predecessor in the treaty without signing.
    \item This also implies full recognition of the entire content.
    \item Interventions regarding claims can only be recognized with a justifiable reason.
    \item Only those who have signed the Treaty of Berlin may sign the Imperial Pact.
    \item Moving armed units through foreign territory of a contracting party without permission is also considered a violation of claims.
\end{enumerate}

\subsection{Lawful Successor}
\begin{enumerate}[(1)]
    \item A lawful successor is defined as one who continues the traditions and name of the state and has been legitimately recognized as a successor by a majority of the treaty members.
    \item In the event of a secession and incorporation, it is not necessary to determine a lawful successor, as it still exists.
    \item In case of dismemberment or fusion, it is up to the treaty members to decide on the succession.
    \item It is not mandatory to determine a successor state.
\end{enumerate}

\subsection{Recognition of Further Claims}
\begin{enumerate}[(1)]
    \item Only territorial claims provided for by the Imperial Pact are recognized.
    \item Signing members are prohibited from recognizing claims that conflict with paragraph 1.
    \item All contracting parties may only claim territories on Earth.
    \item Outer space does not belong to Earth.
    \item All contracting parties also commit to denying the claims of other states to territories outside Earth.
    \item Paragraph 3 ff. also applies to the non-claiming of Antarctica.
    \item Territorial waters are valid claims and apply according to the attached map.
    \item The territorial waters correspond to the circled areas.
    \item ${^1}$For land areas not circled, everything within a radius of fifty blocks is considered territorial waters. ${^2}$In the case of an island, the radius is measured from the center of the island, corresponding to its greatest extent from the center plus fifty blocks.
    \item If there is a conflict between the radii of two different countries, the boundary will run in the middle of the overlap.
    \item An area is also considered territorial waters if it lies between two or more areas of the same state and does not conflict with foreign territorial waters, and the diameter between the greatest extents from the center does not exceed three hundred blocks.
    \item Claims that are enforced under server law and do not conflict with the provisions of the agreement are an exception to paragraph 1.
\end{enumerate}

\subsection{Violations of Claims by Contracting Parties}
\begin{enumerate}[(1)]
    \item Any violations of claims under the Imperial Pact committed by contracting parties will, if necessary, be sanctioned by the expulsion of the aggressor from the treaty.
    \item Enforced claims must be defended militarily by all parties.
\end{enumerate}

\subsection{Violations of Claims by Other States}
Any state that violates claims under the Imperial Pact and is not a contracting party must be fought militarily and, if necessary, completely conquered.

\subsection{Duty of Defense}
\begin{enumerate}[(1)]
    \item Each state is obliged to defend the treaty claims in accordance with §§4, 5.
    \item One may withdraw from the duty of defense with an official letter.
    \item If one makes use of paragraph 2, the other treaty states are no longer obliged to defend one's own claims.
\end{enumerate}

\subsection{Claim Definitions}
\begin{enumerate}[(1)]
    \item An enforced claim is a claim acquired by legally permissible methods according to server war law.
    \item A prospective claim is a claim that the state may acquire in the indefinite future according to the Imperial Pact.
    \item The claims are defined over internationally recognized states and other geographical conditions of the real world as of July 17, 2023.
\end{enumerate}

\section{Stated Claims}

\subsection{China}
China's claims are as follows:
\begin{enumerate}
    \item China, including Hong Kong and excluding the areas defined in \ref{order} paragraph 2.
    \item The island of Taiwan.
\end{enumerate}

\subsection{German Empire}\label{german}
\begin{enumerate}[(1)]
    \item The claims of the German Empire are as follows:
    \begin{enumerate}[1.]
        \item Geographical Europe, including all outer and overseas territories belonging to the associated countries, except for the islands of Great Britain and Ireland, along with their outer and overseas territories, as well as Iceland, Svalbard, and Jan Mayen.
        \item Russia up to the border defined in paragraph 2.
        \item The southern half of Novaya Zemlya.
        \item Turkey.
        \item Syria.
        \item Lebanon.
        \item Israel.
        \item Egypt.
        \item Libya.
        \item Algeria.
        \item Tunisia.
        \item Morocco.
        \item Indonesia.
        \item The Solomon Islands.
        \item Vanuatu.
        \item The Micronesia region except for Guam and the Northern Mariana Islands.
        \item Australia.
        \item An area with a 100-block radius around Mount Everest.
        \item An area with a 100-block radius around Machu Picchu.
        \item An area with a 100-block radius around Mount St. Elias.
        \item Japan.
        \item North Korea.
        \item South Korea.
        \item Trinidad and Tobago.
        \item Suriname.
        \item Guyana.
        \item Venezuela except for all areas west of the 66th western meridian.
    \end{enumerate}
    \item The border between Russia and the German Empire runs exactly midway between the coastlines of the Ob Bay up to the Ob River. From there, the border follows the midpoint between the banks to the city of Labytnangi. From here, the border runs parallel to the equator until just before the Ural Mountains. The border then follows the same distance to the lowest tip. From here, the border runs parallel to the meridian to the border of Kazakhstan.
\end{enumerate}

\subsection{Kazakhstan}
Kazakhstan's claims are as follows:
\begin{enumerate}
    \item Kazakhstan.
\end{enumerate}

\subsection{Empire of Keosu}
The claims of the Empire of Keosu are as follows:
\begin{enumerate}
    \item Greenland.
    \item Svalbard and Jan Mayen.
    \item The northern half of Novaya Zemlya.
    \item Canada, except for the areas allocated to Wisconsin.
    \item Alaska.
\end{enumerate}

\subsection{Monastic Republic of Ratosurya}\label{order}
\begin{enumerate}[(1)]
    \item The claims of the Order are as follows:
    \begin{enumerate}[1.]
        \item Nepal.
        \item Bhutan.
        \item Tibet up to the border defined in paragraph 2.
        \item The Indian states of Bihar, Assam, Maharashtra, Manipur, Meghalaya, Mizoram, Sikkim, West Bengal, Jharkhand, and Tripura.
        \item Bhutan.
        \item Bangladesh.
    \end{enumerate}
    \item The border runs from the western border of China in a straight line along the 80th southern parallel to the beginning of road S301. From here, it follows road S301 until it meets S203. From there, it follows the road until it merges into G562 and continues along it. From the 89th eastern meridian downwards, the border merges into road S204, which it continues to follow eastward. Finally, it runs in a straight line along the 89th eastern meridian southwards.
    \item The Monastic Republic is under German-Keosunian administration and may not be subject to war.
    \item Both states are also obliged to continue to manage the Monastic Republic of Ratosurya cooperatively and peacefully together, even in times of war.
\end{enumerate}

\subsection{Autonomous High Kingdom of Russia}
\begin{enumerate}[(1)]
    \item Russia's claims are as follows:
    \begin{enumerate}[1.]
        \item All of Russia east of the border described in \ref{german} paragraph 2.
        \item Georgia.
        \item Armenia.
        \item Azerbaijan.
    \end{enumerate}
    \item Russia is an autonomous territory within the German Empire.
\end{enumerate}

\subsection{Valkyrie Trading Corporation}
The claims of the Valkyrie Trading Corporation are as follows:
\begin{enumerate}
    \item Argentina.
    \item Chile.
    \item Paraguay.
    \item Uruguay.
\end{enumerate}

\subsection{Exception Case Iceland}
Due to the lack of agreement between the German Empire and the Keosu Empire, Iceland is considered unoccupied. However, only the two states may claim this land.

\end{document}
