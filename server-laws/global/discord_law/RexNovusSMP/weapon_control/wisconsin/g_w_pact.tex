\documentclass{article}
\usepackage[utf8]{inputenc}
\usepackage{enumerate}
\usepackage{ragged2e}
\usepackage{etoc}
\usepackage{amsmath}
\usepackage{graphicx}
\graphicspath{ {./images} }

\renewcommand{\thesection}{}
\newcommand{\sent}[1]{$^{#1}$}
\counterwithout{subsection}{section}
\renewcommand{\thesubsection}{§\arabic{subsection}}

\title{Deutschland-Wisconsin-Pakt}
\author{Kaiser Friedrich IV. von Preußen}
\date{19. Juli 1920}

\begin{document}
\maketitle
\begin{center}
    \begin{minipage}{0.45\textwidth}
        \centering
        \includegraphics[scale=.15]{dr_wappen}
    \end{minipage}
    \hfill
    \begin{minipage}{0.45\textwidth}
        \centering
        \includegraphics[scale=.10]{wi_wappen.png}
    \end{minipage}
\end{center}


\begin{center}
    \textbf{Zwischen den Parteien\\}\textbf{\\}
    Das \textbf{Deutsche Kaiserreich} auch im Folgenden bezeichnet als \textbf{``Deutsches Reich''\\}
    Das Königreich Wisconsin, im Folgenden \textbf{``Wisconsin''\\}
    \textbf{\\}
    Gemeinsam im nachfolgenden Vertrag bezeichnet als \textbf{``Die Staaten''}, \textbf{``Die Parteien''}

\end{center}
\newpage
\topskip0pt
\vspace*{\fill}
\begin{Center}
\textbf{Vertragssignatur:\\}
\texttt{DE-D1B6AA043F8E2B72BFE19B67F2370C7A}
\vspace*{\fill}
\end{Center}
\newpage
\section{Allgemeines}
\subsection{Gültigkeit}

\subsection{Vertragliche Gültigkeit}
\begin{enumerate}[(1)]
    \item Der nachfolgende Vertrag ist gültig, bis von allen Vertragsparteien ein Abkommen zur Aufhebung des Abkommens aufgesetzt und unterschrieben wird.
    \item Dieser Vertrag ordnet sich den nationalen Verfassungen und Rechtsprechungen unter.
    \item Entscheidungen im Zuge dieses Abkommens müssen von der Mehrheit der Vertragsmitglieder bewilligt werden.
    \item Die Bezeichnung der Staaten entspricht deren Namen zum Zeitpunkt der erstmaligen Unterzeichnung.
    \item Nur rechtmäßige Nachfolger der Staaten haben das Recht, die Mitgliedschaft ihres Vorgängers im Vertrag fortzuführen, ohne zu unterzeichnen.
    \item Dies bedeutet jedoch auch die damit einhergehende vollständige Anerkennung des gesamten Inhalts.
    \item Jeder Staat muss souverän sein, um zu unterzeichnen.
\end{enumerate}

\subsection{Rechtmäßiger Nachfolger}
Rechtmäßiger Nachfolger ist, wer durch die Mitglieder des Kaiserpakts als dieser anerkannt wird.

\section{Militärische Vereinbarungen}
\subsection{Nichtangriffspakt}
\begin{enumerate}[(1)]
    \item Beide Parteien sichern sich zu, gegen den anderen keine militärischen Handlungen zu unternehmen, sofern diese nicht vom Kaiserpakt vorgeschrieben werden.
    \item Zudem sichern sich die Vertragsparteien zu, keine militärischen Handlungen gegen den anderen zu unterstützen.
    \item Das Finanzieren und anderweitige Voranbringen derartiger Operationen fällt ebenfalls unter diese Regelung.
\end{enumerate}

\subsection{Defensivabkommen}
\begin{enumerate}[(1)]
    \item Beide Parteien sichern sich zu, den anderen im Falle nicht eindeutig und beinahe ausnahmslos selbstverschuldeter Verteidigungsmaßnahmen aktiv zu unterstützen.
    \item Dies beinhält die Erhebung umfangreicher Sanktionen gegen die angreifende Nation.
\end{enumerate}

\subsection{Weitere militärische Verpflichtungen}
Beide Parteien sind nicht gezwungen, den anderen in anderweitigen militärischen Situationen zu unterstützen.

\section{Vereinbarungen bezüglich weiterer Abkommen}
\subsection{Waffenkontrollgesetz}
\begin{enumerate}[(1)]
    \item Das Deutsche Reich hebt die Gültigkeit des Waffenkontrollgesetzes für Wisconsin ausnahmslos auf und genehmigt ihnen, die Garantie dahingehend rückgängig zu machen.
    \item Im Kriegsfall, ungeachtet des Grundes und der Eigenbeteiligung an der Kriegserklärung, garantiert das Deutsche Reich Wisconsin mit kostenlosen Waffen und kostenloser Munition einmalig zu versorgen.
\end{enumerate}

\subsection{Karibischer Pakt}
Das Deutsche Reich willigt ein, die Insel Bermuda zu entmilitarisieren und unter gemeinsame Verwaltung zu stellen.

\subsection{Handel magischer Waffen}
Das Deutsche Reich willigt ein, den Handel mit magischen Waffen und anderen magischen Gütern für Wisconsin zu arrangieren, aufgrund des Waffenkontrollgesetzes mit Keosu Teikoku jedoch nicht selbst zu vollziehen.

\end{document}