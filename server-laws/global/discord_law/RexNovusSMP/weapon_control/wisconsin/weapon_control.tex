\documentclass{article}
\usepackage[utf8]{inputenc}
\usepackage{enumerate}
\usepackage{ragged2e}
\usepackage{etoc}
\usepackage{amsmath}
\usepackage{graphicx}
\graphicspath{ {./images} }

\renewcommand{\thesection}{}
\newcommand{\sent}[1]{$^{#1}$}
\counterwithout{subsection}{section}
\renewcommand{\thesubsection}{§\arabic{subsection}}

\title{Abkommen zur Einführung eines Waffenkontrollgesetzes}
\author{Kaiser Friedrich IV. von Preußen}
\date{08. Juli 1920}

\begin{document}
\maketitle
\begin{center}
    \includegraphics[scale=.15]{dr_wappen}
\end{center}
\begin{center}
    \textbf{Zwischen den Parteien\\}\textbf{\\}
    Das \textbf{Deutsche Kaiserreich} auch im Folgenden bezeichnet als \textbf{``Deutsches Reich''\\}
    \textbf{Wisconsin\\}\textbf{\\}

    Gemeinsam im Folgenden \textbf{``Die Parteien''}, \textbf{``Die Vertragsparteien''}
\end{center}
\newpage
\topskip0pt
\vspace*{\fill}
\begin{Center}
\textbf{Vertragssignatur:\\}
\texttt{DE-938D59F9992094CA0E1D3B15F625D22C}
\vspace*{\fill}
\end{Center}
\newpage
\tableofcontents
\newpage
\section{Vertragliche Definitionen}

\subsection{Vertragliche Gültigkeit}
\begin{enumerate}[(1)]
    \item Dieses Abkommen setzt die Bestimmungen des Abkommens zur Einführung eines Deutsch-Keosunischen Waffenkontrollgesetzes in Bezug auf Wisconsin aus.
    \item Der nachfolgende Vertrag ist gültig, bis von allen Vertragsparteien ein Abkommen zur Aufhebung des Abkommens aufgesetzt und unterschrieben wird.
    \item Dieser Vertrag ordnet sich den nationalen Verfassungen und Rechtsprechungen unter.
    \item Entscheidungen im Zuge dieses Abkommens müssen von der Mehrheit der Vertragsmitglieder bewilligt werden.
    \item Die Bezeichnung der Staaten entspricht deren Namen zum Zeitpunkt der erstmaligen Unterzeichnung.
    \item Nur rechtmäßige Nachfolger der Staaten haben das Recht, die Mitgliedschaft ihres Vorgängers im Vertrag fortzuführen, ohne zu unterzeichnen.
    \item Dies bedeutet jedoch auch die damit einhergehende vollständige Anerkennung des gesamten Inhalts.
\end{enumerate}

\subsection{Rechtmäßiger Nachfolger}
Rechtmäßiger Nachfolger ist, wer durch die Mitglieder des Kaiserpakts als dieser anerkannt wird.

\subsection{Vertragliche Verpflichtungen der Parteien}
Die Parteien verpflichten sich dazu, die Vertragsbestimmungen in ihre Verfassungen oder anderweitig vollständig gültige Gesetzestexte ihrer Nation sinngemäß zu übernehmen.

\section{Waffenkontrollgesetze}
\subsection{Monopole}
\begin{enumerate}[(1)]
    \item Die Vertragsparteien sichern sich zu, dass
    \begin{enumerate}[1.]
        \item jegliche Schusswaffen aus dem Deutschen Reich importiert werden müssen
        \item jegliche magische Waffen aus Keosu Teikoku importiert werden müssen
    \end{enumerate}
    \item Es ist den Staatsbürgern der Parteien nicht erlaubt, über andere Mittel an diese zu gelangen.
    \item Anderweitig erlangte Waffen, sowie der Handel mit derartigen Waffen erfordern die Genehmigung der Partei, die für diese zuständig ist.
    \item Die Waffenkontrollgesetze schließen jegliches weiteres Zubehör und Material ein, die lediglich in Verbindung mit den jeweiligen Waffen verwendet werden.
\end{enumerate}

\subsection{Verkaufszwang}
\begin{enumerate}[(1)]
    \item Die Parteien müssen auf Nachfrage hin Waffen an die andere Partei ausliefern.
    \item ${^1}$Dies verfällt, sofern die fragliche Partei unter Aufführung nachvollziehbarer und verständlicher Gründe die Auslieferung verweigert. ${^2}$Dies rechtfertigt eine Rückerstattung des gezahlten Geldes.
\end{enumerate}

\subsection{Durchsuchungsbeschluss}
\begin{enumerate}[(1)]
    \item Untersuchungen bezüglich der Verletzung des Waffenkontrollgesetzes auf fremdem Territorium erfordern einen Durchsuchungsbeschluss.
    \item Dieser Beschluss muss durch den Internationalen Strafgerichtshof in einer geheimen Anhörung unter sinnhafter und verständlicher Darlegung eines dringenden Verdachts oder der Notwendigkeit einer Präventivmaßnahme erwirkt werden.
    \item ${^1}$Der Durchsuchungsbefehl muss der Person, dessen Territorium durchsucht wird, bei Betreten dessen vorgelegt werden. ${^2}$Die Einsatzkräfte müssen die Kenntnisnahme des Dokuments durch die fragliche Person nicht abwarten, um mit der Untersuchung zu beginnen.
\end{enumerate}

\subsection{Untersuchungen}
\begin{enumerate}[(1)]
    \item Besteht der Verdacht auf den illegalen Besitz eines Gutes, das unter die Bestimmungen dieses Abkommens fällt, durch einen der Staatsbürger der Parteien, oder die zukünftige illegale Anschaffung dieser Güter, so rechtfertigt dies einen zeitnahen Eingriff durch die, für diese Waffen zuständige Partei.
    \item Bei Einsätzen muss eine neutrale Partei teilnehmen, die weder zu den betroffenen Parteien, noch zum herrschenden Haus von Kasachstan gehört und die Einhaltung des Abkommens überwacht.
    \item Ist zum Zeitpunkt des Einsatzes keine neutrale Partei anwesend, so muss ohne diese verfahren werden.
    \item Zu Beginn der Kooperation müssen die Einsatzkräfte betroffene Personen in einen sicheren Bereich eskortieren, in dem sie durch keine feindlichen Entitäten angegriffen werden.
    \item Die Einsatzkräfte müssen Personen mit Respekt behandeln, sofern sie eindeutig Kooperationswillen zeigen.
    \item Während dieses Einsatzes dürfen die Einsatzkräfte Personen, die den Eingriff aktiv behindern, erschießen.
    \item Weiterhin darf Schießbefehl durch die Einsatzkräfte erteilt werden, sofern eine, nicht zu diesen gehörige Person, eine Waffe ausrüstet oder, sofern sie diese bereits ausgerüstet hat, trotz eindeutiger Warnung nicht aushändigt.
    \item Die Einsatzkräfte dürfen jegliche, notwendigen Maßnahmen ergreifen, um die illegalen Waffen sicherzustellen.
    \item Wurde eine Präventivmaßnahme angemeldet, so dürfen ebenfalls jegliche Gegenstände konfisziert werden, die ausschließlich zur Herstellung von Waffen dienen.
    \item Für jegliche Ausschreitungen und aufkommende Schäden während des Einsatzes können die Einsatzkräfte nur belangt werden, falls diese unter Außerachtlassung nötiger Vorsicht entstanden oder eindeutig vermeidbar waren und die Einsatzkräfte sich der Vermeidbarkeit nachweislich bewusst waren.
    \item Einsätze müssen mit Bild und Ton aufgezeichnet werden.
    \item Die Vernichtung der Einsatzaufnahmen ohne Genehmigung der anderen Partei, oder dessen Fehlen stellt einen Vertragsbruch dar.
\end{enumerate}

\subsection{Illegaler Waffenbesitz}
\begin{enumerate}[(1)]
    \item Des illegalen Waffenbesitzes macht sich schuldig, wer ohne Zustimmung, der für diese Waffen, Material oder dessen Zubehör zuständigen Partei, Waffen, Material oder Zubehör beschafft, verkauft oder anderweitig vermittelt.
    \item Dies wird mit einer Freiheitsstrafe von nicht unter einer Stunde und einem Bußgeld geahndet.
    \item Wer gewerbsmäßig illegal Waffen, Material oder Zubehör verkauft oder anderweitig vermittelt muss ein Bußgeld zahlen und erhält eine Freiheitsstrafe von nicht unter vier Stunden.
\end{enumerate}

\end{document}