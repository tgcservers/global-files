\documentclass{article}
\usepackage[utf8]{inputenc}
\usepackage{enumerate}
\usepackage{ragged2e}
\usepackage{etoc}
\usepackage{amsmath}
\usepackage{graphicx}
\graphicspath{ {./images} }

\renewcommand{\thesection}{}
\newcommand{\sent}[1]{$^{#1}$}
\counterwithout{subsection}{section}
\renewcommand{\thesubsection}{§\arabic{subsection}}

\title{Garantie zur zwischenzeitlichen Aufrechterhaltung der Achtung der deutschen Waffenindustrie}
\author{König Drachenherz von Wisconsin}
\date{15. Juli 1920}

\begin{document}
\maketitle
\begin{center}
    \begin{minipage}{0.45\textwidth}
        \centering
        \includegraphics[scale=.15]{dr_wappen}
    \end{minipage}
    \hfill
    \begin{minipage}{0.45\textwidth}
        \centering
        \includegraphics[scale=.10]{wi_wappen.png}
    \end{minipage}
\end{center}


\begin{center}
    \textbf{Zwischen den Parteien\\}\textbf{\\}
    \textbf{Wisconsin} auch im Folgenden bezeichnet als \textbf{``Wir''}, \textbf{``Unser Staat''\\}
    Das \textbf{Deutsche Kaiserreich} auch im Folgenden bezeichnet als \textbf{``Deutsches Reich''\\}
    \textbf{\\}

\end{center}
\newpage
\topskip0pt
\vspace*{\fill}
\begin{Center}
\textbf{Vertragssignatur:\\}
\texttt{WI-F42E464B8A485445D36D2EA27304D547}
\vspace*{\fill}
\end{Center}
\newpage
\vspace*{\fill}
\paragraph{Ich, König Drachenherz von Wisconsin garantiere hiermit dem Deutschen Reich,\\\\}
dass wir bis zum Inkrafttreten des neuen Waffenkontrollgesetzes zwischen uns und dem Deutschen Reich,
dessen Richtlinien bezüglich des Verbots des, durch das Deutsche Reich ungenehmigten Waffenbesitzes und Besitzes
des jeweiligen Zubehörs, sowie Materials, weiter bedingungslos einhalten werden.\\
Zudem erklären wir hiermit offiziell, dass wir uns nicht im Besitz von, durch das Deutsche Reich ungenehmigte,
Waffen und Zubehör befinden und nicht die Absicht verfolgen, dies in Zukunft zu tun.\\
Diese Erklärung erfolgt nur unter Garantie durch das Deutsche Reich, dass sie bis zur Einführung des neuen Waffenkontrollgesetzes
zwischen uns und dem Deutschen Reich, keine unbefugten Einsätze mehr auf unserem Territorium vornehmen, sofern diese nicht eindeutig durch uns provoziert wurden. Die Eindeutigkeit ist hierbei von einer neutralen Partei zu entscheiden, die keinem unserer Staaten angehört.\\
Wir, wie auch das Deutsche Reich garantieren zudem, das Waffenkontrollgesetz vom 15. Juli 1920 an innerhalb der nächsten zwei Wochen das erneuerte Abkommen zur Einführung eines Waffenkontrollgesetzes zu unterzeichnen.
\vspace*{\fill}
\end{document}