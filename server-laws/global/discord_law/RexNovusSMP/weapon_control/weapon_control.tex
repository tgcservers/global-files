\documentclass{article}
\usepackage[utf8]{inputenc}
\usepackage{enumerate}
\usepackage{ragged2e}
\usepackage{etoc}
\usepackage{amsmath}
\usepackage{graphicx}
\graphicspath{ {./images} }

\renewcommand{\thesection}{}
\newcommand{\sent}[1]{$^{#1}$}
\counterwithout{subsection}{section}
\renewcommand{\thesubsection}{§\arabic{subsection}}

\title{Abkommen zur Einführung eines Deutsch-Keosunischen Waffenkontrollgesetzes}
\author{Kaiser Friedrich IV. von Preußen}
\date{08. Juli 1920}

\begin{document}
\maketitle
\begin{center}
    \includegraphics[scale=.15]{dr_wappen}
\end{center}
\begin{center}
    \textbf{Zwischen den Parteien\\}\textbf{\\}
    Das \textbf{Deutsche Kaiserreich\\} auch im Folgenden bezeichnet als \textbf{``Deutsches Reich''\\}
    \textbf{Keosu Teikoku\\}\textbf{\\}

    Gemeinsam im Folgenden \textbf{``Die Parteien''}, \textbf{``Die Vertragsparteien''}
\end{center}
\newpage
\topskip0pt
\vspace*{\fill}
\begin{Center}
\textbf{1. Fassung}
\vspace*{\fill}
\end{Center}
\newpage
\tableofcontents
\newpage
\section{Vertragliche Definitionen}

\subsection{Vertragliche Gültigkeit}
\begin{enumerate}[(1)]
    \item Der nachfolgende Vertrag ist gültig, bis von allen Vertragsparteien ein Abkommen zur Aufhebung des Kaiserpakts aufgesetzt und unterschrieben wird.
    \item Dieser Vertrag ordnet sich den nationalen Verfassungen und Rechtsprechungen unter.
    \item Entscheidungen im Zuge dieses Abkommens müssen von der Mehrheit der Vertragsmitglieder bewilligt werden.
    \item Die vertragliche Anerkennung durch autonome Staaten erfolgt nur durch Unterschrift durch die, ihnen übergeordnete souveräne Vertragsnation, womit diese auch die Ansprüche des autonomen Gebiets bewilligen.
    \item Die Bezeichnung der Staaten entspricht deren Namen zum Zeitpunkt der erstmaligen Unterzeichnung.
    \item Nur rechtmäßige Nachfolger der Staaten haben das Recht, die Mitgliedschaft ihres Vorgängers im Vertrag fortzuführen, ohne zu unterzeichnen.
    \item Dies bedeutet jedoch auch die damit einhergehende vollständige Anerkennung des gesamten Inhalts.
\end{enumerate}

\subsection{Rechtmäßiger Nachfolger}
Rechtmäßiger Nachfolger ist, wer durch die Mitglieder des Kaiserpakts als dieser anerkannt wird.

\subsection{Vertragliche Verpflichtungen der Parteien}
Die Parteien verpflichten sich dazu, die Vertragsbestimmungen in ihre Verfassungen oder anderweitig vollständig gültige Gesetzestexte ihrer Nation sinngemäß zu übernehmen.

\section{Waffenkontrollgesetze}
\subsection{Monopole}
\begin{enumerate}[(1)]
    \item Die Vertragsparteien sichern sich zu, dass
    \begin{enumerate}[1.]
        \item jegliche Schusswaffen aus dem Deutschen Reich importiert werden müssen
        \item jegliche magische Waffen aus Keosu Teikoku importiert werden müssen
    \end{enumerate}
    \item Es ist den Staatsbürgern der Parteien nicht erlaubt, über andere Mittel an diese zu gelangen.
    \item Anderweitig erlangte Waffen erfordern die Genehmigung der Partei, die für diese zuständig ist.
\end{enumerate}

\subsection{Verkaufszwang}
\begin{enumerate}[(1)]
    \item Die Parteien müssen auf Nachfrage hin Waffen an die andere Partei ausliefern.
    \item ${^1}$Dies verfällt, sofern die fragliche Partei unter Aufführung nachvollziehbarer und verständlicher Gründe die Auslieferung verweigert. ${^2}$Dies rechtfertigt eine Rückerstattung des gezahlten Geldes.
\end{enumerate}

\subsection{Illegaler Waffenbesitz}
\begin{enumerate}[(1)]
    \item Besteht der Verdacht auf illegalen Waffenbesitz durch einen der Staatsbürger der Parteien, so rechtfertigt dies einen sofortigen Eingriff durch die, für diese Waffen zuständige Partei.
    \item Dieser Eingriff muss unter Inkenntnissetzung, nicht jedoch Zustimmung der anderen Partei geschehen.
    \item Während dieses Einsatzes dürfen die Einsatzkräfte Personen, die den Eingriff behindern, erschießen.
    \item Weiterhin darf der Schießbefehl bei Ausrüstung einer Waffe durch eine Person erteilt werden.
    \item Die Einsatzkräfte dürfen jegliche, notwendigen Maßnahmen ergreifen, um die illegalen Waffen sicherzustellen.
    \item Personen, die sich des illegalen Waffenbesitzes schuldig gemacht haben, erhalten die Todesstrafe.
    \item Für jegliche Ausschreitungen während des Einsatzes können die Einsatzkräfte nur belangt werden, falls diese eindeutig vermeidbar und sich der Vermeidbarkeit nachweislich bewusst waren.
    \item Einsätze müssen mit Bild und Ton aufgezeichnet werden.
    \item Die Vernichtung der Einsatzaufnahmen ohne Genehmigung der anderen Partei, oder dessen Fehlen stellt einen Vertragsbruch dar.
\end{enumerate}

\end{document}