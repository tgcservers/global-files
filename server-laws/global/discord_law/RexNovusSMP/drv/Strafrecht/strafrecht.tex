\documentclass{article}
\usepackage[utf8]{inputenc}
\usepackage{enumerate}
\usepackage{ragged2e}
\usepackage{etoc}

\renewcommand{\thesection}{}
\renewcommand{\thesubsection}{§\arabic{subsection}}

\title{Strafrecht}
\author{Kaiser Friedrich IV.}
\date{07. Juli 1920}

\begin{document}
\maketitle
\vspace*{\fill}
\paragraph{Strafrecht des Deutschen Kaiserreichs gemäß Reichsstrafrechtsbeschluss vom 07. Juli 1920}

\newpage
\topskip0pt
\vspace*{\fill}
\begin{Center}
\textbf{1. Fassung}
\vspace*{\fill}
\end{Center}
\newpage
\tableofcontents
\newpage

\section{Strafgesetzbuch (StGB)}
\localtableofcontents
\subsection{Diebstahl}
\begin{enumerate}[(1)]
    \item Stiehlt man vom Territorium des Kaiserreichs, so muss man die Ware mitsamt ihres doppelten Warenwerts, sofern vorhanden, zurückerstatten.
    \item Dies gilt für alle Gegenstände, die dem Staatsgebiet entstammen oder einer Person auf dem Staatsgebiet gehören und widerrechtlich entwendet wurden.
    \item Auch gilt dies für Gegenstände, die gelöscht oder anderweitig vernichtet wurden.
\end{enumerate}

\subsection{Mord}
\begin{enumerate}[(1)]
    \item Tötet man eine Person aus niederen Beweggründen, so muss man dessen Erbe mit 10 000 Kaisermark entschädigen und wird hingerichtet.
\end{enumerate}

\subsection{Körperverletzung}
Wer eine Person auf dem Gebiet des Kaiserreichs physisch verletzt, muss mit einer Strafe von 15 HTK rechnen.

\subsection{Schwere Körperverletzung}
Verletzt man eine Person vorsätzlich so schwer, dass sie mindestens die Hälfte ihrer Leben verloren hat, so muss man 50 HTK zahlen.

\subsection{Verunglimpfung fraktioneller Insignien und Symbole}
\begin{enumerate}[(1)]
    \item Wer fraktionelle Symbole von Hamavar, dessen Vasallen oder Verbündeten verunglimpft oder absichtlich entfernt, muss 50 HTK zahlen.
    \item Hierzu zählt ebenfalls das unerlaubte Tragen von Orden und Uniformen, beziehungsweise das Tragen von Orden zu einer inoffiziellen Uniform.    
\end{enumerate}

\subsection{Effekte und Fähigkeiten}
Man darf keine Effekte ohne Genehmigung haben. Verstöße werden mit 30 HTK Bußgeld vergolten.

\subsection{Verbotene Gegenstände}
Man darf keine verbotenen Gegenstände mit sich führen, ansonsten droht eine Hinrichtung.

\subsection{Pferde}
\begin{enumerate}[(1)]
    \item Pferde sind innerhalb der Stadt nicht als Fortbewegungsmittel gestattet. Jeglicher Verstoß wird mit einer Bußgeldstrafe von 2 HTK geahndet.
    \item Reitet man mit einem Pferd in den Palasthof des Weißen Palasts, so muss man 6 HTK zahlen.    
\end{enumerate}

\subsection{Betrug}
Wer sich oder einen Dritten durch Vorspiegelung falscher Tatsachen bereichert oder einen Vorteil verschafft, muss Bußgeld zahlen. Der Betrag wird an die Schwere der Straftat angepasst.

\subsection{Sklaverei}
Sklaverei und Menschenhandel werden mit 500 HTK Bußgeld und der Todesstrafe bestraft.

\subsection{Menschenexperimente}
\begin{enumerate}[(1)]
    \item Menschenexperimente sind nur unter staatlicher Aufsicht erlaubt.
    \item Dies erfordert kein Einverständnis der Testperson.
    \item Der Staat kann Einspruch gegen die Wahl der Testperson erheben.
\end{enumerate}

\subsection{Geldwäsche}
Wer sich ohne Genehmigung der zuständigen Bank eine, durch das Kaiserreich Hamavar genehmigte Währung prägt, muss eine Haftstrafe absitzen. Weiterhin wird das Konto der Person geleert und ihr temporär alle Geldzufuhren abgestellt. Die Person verliert somit ihre Kreditfähigkeit und all ihre Immobilien.

\subsection{Siegelfälschung}
Wer ein Schwarzsiegel, staatliches Zertifikat oder einen historischen Gegenstand ungenehmigt dupliziert, muss 1000 HTK Strafe zahlen. Zudem muss der Gewinn, der dadurch erwirtschaftet wurde, zurückgezahlt werden.

\subsection{Hehlerei}
\begin{enumerate}[(1)]
    \item Wer illegale Waren verkauft, muss 50 HTK Strafe zahlen.
    \item Gewerbsmäßige Hehlerei wird zusätzlich mit dem Tode vergolten.
\end{enumerate}

\subsection{Beleidigung}
Wer eine Person beleidigt und dadurch die Würde des Betroffenen verletzt, muss ein Bußgeld in Höhe von 50 HTK entrichten.

\subsection{Verleumdung}
Es verleumdet, wer wider besseren Wissens Unwahrheiten über eine Person verbreitet und somit dessen Ruf nachhaltig schädigt.

\subsection{Schulden}
\begin{enumerate}[(1)]
    \item Jegliche Schulden, die man beim Kaiserreich, dem Adel oder den Bürgern des Kaiserreichs hat, müssen innerhalb von 10 Tagen zurückgezahlt werden, wenn keine anderweitige Frist angegeben wurde.
    \item Tut man dies nicht, verliert man bis zur Rückzahlung zusammen mit zusätzlichen 80 HTK oder Gegenständen mit äquivalentem Wert die Kreditfähigkeit im Kaiserreich.
    \item Die Strafe nach dreifachem Aufschub liegt im Ermessen des zuständigen Gerichts.    
\end{enumerate}

\subsection{Veruntreuung}
\begin{enumerate}[(1)]
    \item Es veruntreut, wer die Treuhandspflicht über Vermögen oder Wertanlagen verletzt und diese ohne Genehmigung des Eigentümers an andere überschreibt oder ausgibt, ohne die Möglichkeit, diese gleichwertig zu entschädigen.
    \item Dies wird nach wirtschaftsrechtlichem Bußgeldsatz und einer zusätzlichen Bußgeldstrafe von nicht unter 2000 HTK geahndet.
\end{enumerate}

\subsection{Sachbeschädigung}
Wer fremdes Eigentum auf dem Gebiet von Hamavar beschädigt, muss für die Schäden vollständig aufkommen und zusätzlich 100 HTK zahlen.

\newpage
\section{Zivilgesetzbuch (ZivilGB)}
\localtableofcontents

\subsection{Steuerhinterziehung}\label{hinterz}
Wer Steuern nicht vorschriftsgemäß bezahlt, muss die Steuern in Form einer Bußgeldstrafe mit zusätzlichen 100 HTK entrichten.

\subsection{Rechte des Eigentümers}
\begin{enumerate}[(1)]
    \item Wer auf hamavarischem Grund rechtmäßig Eigentum erworben hat, darf dieses nutzen und verändern, wie er möchte, solange diese Handlungen ausschließlich gesetzeskonform sind.
    \item Erwirbt man ein Haus, so gehört einem nur das Innere des Hauses und nicht die Fassade, weshalb diese nicht verändert werden darf.
    \item Für Territorien gilt, dass man sie erst mit Genehmigung des Lehnsherrn bebauen darf.
\end{enumerate}

\subsection{Hausfriedensbruch}\label{hausfr}
\begin{enumerate}[(1)]
    \item Es begeht Hausfriedensbruch, wer sich ohne die Genehmigung des Besitzers oder einer, durch diesen bevollmächtigten Person, Zutritt zu dessen Immobilie verschafft.
    \item Der Besitzer einer Immobilie muss die Bußgeldstrafe, die Personen bei Hausfriedensbruch von da an zahlen sollen, einmalig in die Staatskasse einzahlen.
    \item Absatz 2 stellt eine Ausnahme zu §20 GerO dar.
    \item Erfolgt keine Einzahlung, wird der Hausfriedensbruch durch eine Freiheitsstrafe geahndet.
\end{enumerate}

\subsection{Landfriedensbruch}
\begin{enumerate}[(1)]
    \item Wer sich auf öffentlichen Boden unter gewalttätiger Absicht oder gewalttätiger Ausschreitungen zusammenrottet oder einzeln befindet, begeht Landfriedensbruch.
    \item Der Täter wird mit einer Freiheitsstrafe oder Bußgeldstrafe von nicht unter 1000 HTK bestraft. In schweren Fällen gilt es die Bußgeldstrafe durch Todesstrafe zu ersetzen.
\end{enumerate}

\newpage
\section{Staatsdeliktsgesetzbuch (SDelGB)}
\localtableofcontents

\subsection{Betreten des Staatsgebietes}
Das Betreten des Staatsgebietes darf nur mit einer ausdrücklichen Genehmigung erfolgen. Betritt man das Staatsgebiet ohne diese Aufenthaltsgenehmigung, so muss man 50 HTK Strafe zahlen.

\subsection{Spionage}
\begin{enumerate}[(1)]
    \item Strategische Aufklärung und Spionage auf dem Staatsgebiet sind nicht erlaubt und daher strafbar. Aufgrund der besonderen Schwere wird dies mit einer Hinrichtung und 1000 HTK Strafe vergolten.
    \item Dies gilt nicht für Operationen, die durch den Staat ausdrücklich genehmigt wurden. 
    \item Man darf ebenso wenig ohne Genehmigung das hamavarische Territorium im Zuschauermodus durchqueren, denn gilt dies ebenfalls als Spionage.    
\end{enumerate}

\subsection{Strafen in den Reichsstädten}
In den Reichsstädten gelten die fünffachen Bußgeldsätze.

\subsection{Finanzeller Status des Reichs}
Der Kaiser kann auf nationaler Ebene nicht verschuldet sein.

\subsection{Hochverrat}
\begin{enumerate}[(1)]
    \item Als Hochverräter gilt, wer
    \begin{enumerate}[1.]
        \item Staatsgeheimnisse ohne Genehmigung verbreitet oder versucht auf diese unerlaubt zuzugreifen.
        \item Eine absichtliche Schwächung des Staates herbeiführt
        \item Die Befehle des Kaisers verweigert
    \end{enumerate}
    \item Der Strafsatz gleicht dem Strafsatz des Mordes an dem Kaiser.
\end{enumerate}

\subsection{Religiöse Gegenstände}
\begin{enumerate}[(1)]
    \item Wer Gegenstände religiöser Natur beschädigt oder zerstört oder auf religiösem Boden Verbrechen begeht, muss eine Bußgeldstrafe in Höhe von 200 HTK zahlen und wird hingerichtet.
    \item Entweiht man religiöse Gebäude kommt dies dem dreifachen Strafsatz gleich.
\end{enumerate}

\subsection{Majestätsbeleidigung}
Beleidigt man den Kaiser, den Staat oder übergeordnete Staatsvertreter, so muss man 1000 HTK Strafe zahlen und wird hingerichtet.

\subsection{Regizid}
Tötet man den Kaiser, so muss die Familie des Täters, wie auch dieser selbst, ihren gesamten Besitz an das Kaiserreich übergeben und wird hingerichtet.

\end{document}