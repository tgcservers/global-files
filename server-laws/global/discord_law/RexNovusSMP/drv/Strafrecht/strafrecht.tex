\documentclass{article}
\usepackage[utf8]{inputenc}
\usepackage{enumerate}
\usepackage{ragged2e}
\usepackage{etoc}

\renewcommand{\thesection}{}
\renewcommand{\thesubsection}{§\arabic{subsection}}

\title{Strafrecht}
\author{Kaiser Friedrich IV.}
\date{07. Juli 1920}

\begin{document}
\maketitle
\vspace*{\fill}
\paragraph{Strafrecht des Deutschen Kaiserreichs gemäß Reichsstrafrechtsbeschluss vom 07. Juli 1920}

\newpage
\topskip0pt
\vspace*{\fill}
\begin{Center}
\textbf{1. Fassung}
\vspace*{\fill}
\end{Center}
\newpage
\tableofcontents
\newpage

\section{Strafgesetzbuch (StGB)}
\localtableofcontents
\subsection{Diebstahl}
\begin{enumerate}[(1)]
    \item Stiehlt man vom Territorium des Kaiserreichs, so muss man die Ware mitsamt ihres doppelten Warenwerts, sofern vorhanden, zurückerstatten.
    \item Dies gilt für alle Gegenstände, die dem Staatsgebiet entstammen oder einer Person auf dem Staatsgebiet gehören und widerrechtlich entwendet wurden.
    \item Auch gilt dies für Gegenstände, die gelöscht oder anderweitig vernichtet wurden.
\end{enumerate}

\subsection{Mord}
\begin{enumerate}[(1)]
    \item Tötet man eine Person aus niederen Beweggründen, so muss man dessen Erbe mit 300 Kaisermark entschädigen und wird hingerichtet.
\end{enumerate}

\subsection{Körperverletzung}
Wer eine Person auf dem Gebiet des Kaiserreichs physisch verletzt, muss mit einer Strafe von 5 Kaisermark rechnen.

\subsection{Schwere Körperverletzung}
Verletzt man eine Person vorsätzlich so schwer, dass sie mindestens die Hälfte ihrer Leben verloren hat, so muss man 10 Kaisermark zahlen.

\subsection{Verunglimpfung fraktioneller Insignien und Symbole}
\begin{enumerate}[(1)]
    \item Wer fraktionelle Symbole von Hamavar, dessen Vasallen oder Verbündeten verunglimpft oder absichtlich entfernt, muss 20 Kaisermark zahlen.
    \item Hierzu zählt ebenfalls das unerlaubte Tragen von Orden und Uniformen, beziehungsweise das Tragen von Orden zu einer inoffiziellen Uniform.    
\end{enumerate}

\subsection{Effekte und Fähigkeiten}
Man darf keine Effekte ohne Genehmigung haben. Verstöße werden mit 5 Kaisermark Bußgeld vergolten.

\subsection{Verbotene Gegenstände}
Man darf keine verbotenen Gegenstände mit sich führen, ansonsten droht eine Hinrichtung.

\subsection{Pferde}
\begin{enumerate}[(1)]
    \item Pferde sind innerhalb der Stadt nicht als Fortbewegungsmittel gestattet. Jeglicher Verstoß wird mit einer Bußgeldstrafe von 2 Kaisermark geahndet. 
\end{enumerate}

\subsection{Betrug}
Wer sich oder einen Dritten durch Vorspiegelung falscher Tatsachen bereichert oder einen Vorteil verschafft, muss Bußgeld zahlen. Der Betrag wird an die Schwere der Straftat angepasst.

\subsection{Sklaverei}
Sklaverei und Menschenhandel werden mit 100 Kaisermark Bußgeld und der Todesstrafe bestraft.

\subsection{Menschenexperimente}
\begin{enumerate}[(1)]
    \item Menschenexperimente sind nur unter staatlicher Aufsicht erlaubt.
    \item Dies erfordert kein Einverständnis der Testperson.
    \item Der Staat kann Einspruch gegen die Wahl der Testperson erheben.
\end{enumerate}

\subsection{Geldwäsche}
\begin{enumerate}[(1)]
    \item Geldwäsche betreibt, wer
    \begin{enumerate}[1.]
        \item Wer sich ohne Genehmigung der zuständigen Bank eine, durch das Deutsche Reich genehmigte Währung prägt.
        \item Wer zur eigenen Bereicherung oder zur Erzielung von Vergünstigungen den Wert eines Objekts absichtlich in die Höhe treibt oder den Preis drückt. 
    \end{enumerate}
    \item Dies hat eine Haftstrafe zur Folge. Weiterhin wird das Konto der Person geleert und ihr temporär alle Geldzufuhren abgestellt. Die Person verliert somit ihre Kreditfähigkeit und all ihre Immobilien.
\end{enumerate}

\subsection{Siegelfälschung}
Wer ein Schwarzsiegel, staatliches Zertifikat oder einen historischen Gegenstand ungenehmigt dupliziert oder anderweitig fälscht, muss 300 Kaisermark Strafe zahlen.

\subsection{Hehlerei}
\begin{enumerate}[(1)]
    \item Wer rechtswidrig erwirtschaftete Waren verkauft, muss 10 Kaisermark Strafe zahlen.
    \item Gewerbsmäßige Hehlerei wird zusätzlich mit dem Tode vergolten.
\end{enumerate}

\subsection{Beleidigung}
Wer eine Person beleidigt und dadurch die Würde des Betroffenen verletzt, muss je nach Schwere ein Bußgeld in Höhe von nicht unter 5 Kaisermark entrichten.

\subsection{Verleumdung}
Es verleumdet, wer wider besseren Wissens Unwahrheiten über eine Person verbreitet und somit dessen Ruf nachhaltig schädigt. Dies hat eine Bußgeldstrafe von nicht unter 5 Kaisermark zur Folge.

\subsection{Sachbeschädigung}
Wer fremdes Eigentum auf dem Gebiet des Deutschen Reichs beschädigt, muss für die Schäden vollständig aufkommen und zusätzlich 10 Kaisermark zahlen, sowie die Versicherungsgebühren entrichten.\footnote{Reichsstrafversicherungsbeschluss vom 05. August 1920}

\newpage
\section{Zivilgesetzbuch (ZivilGB)}
\localtableofcontents

\subsection{Steuerhinterziehung}\label{hinterz}
Wer Steuern nicht vorschriftsgemäß bezahlt, muss die Steuern in Form einer Bußgeldstrafe mit zusätzlichen 50 Kaisermark entrichten.

\subsection{Rechte des Eigentümers}
\begin{enumerate}[(1)]
    \item Wer auf deutschem Grund rechtmäßig Eigentum erworben hat, darf dieses nutzen und verändern, wie er möchte, solange diese Handlungen ausschließlich gesetzeskonform sind.
    \item Erwirbt man ein Haus oder eine Wohnung, so gehört einem nur das Innere des Hauses und nicht die Fassade, weshalb diese nicht verändert werden darf.
    \item Für Territorien gilt, dass man sie erst mit Genehmigung des Lehnsherrn bebauen darf.
\end{enumerate}

\subsection{Hausfriedensbruch}\label{hausfr}
\begin{enumerate}[(1)]
    \item Es begeht Hausfriedensbruch, wer sich ohne die Genehmigung des Besitzers oder einer, durch diesen bevollmächtigten Person, Zutritt zu dessen Immobilie verschafft.
    \item Besteht keine Versicherung, wird der Hausfriedensbruch durch eine Freiheitsstrafe geahndet. Andernfalls muss die Person zusätzlich die Versicherungsgebühren entrichten.\footnote{Reichsstrafversicherungsbeschluss vom 05. August 1920}
\end{enumerate}

\subsection{Landfriedensbruch}
\begin{enumerate}[(1)]
    \item Wer sich auf öffentlichen Boden unter gewalttätiger Absicht oder gewalttätiger Ausschreitungen zusammenrottet oder einzeln befindet, begeht Landfriedensbruch.
    \item Der Täter wird mit einer Freiheitsstrafe oder Bußgeldstrafe von nicht unter 200 Kaisermark bestraft. In schweren Fällen gilt es die Bußgeldstrafe durch Todesstrafe zu ersetzen.
\end{enumerate}

\subsection{Schulden}
\begin{enumerate}[(1)]
    \item Jegliche Schulden, die man beim Kaiserreich, dem Adel, den Bürgern oder jeglicher anderen Entität auf dem Gebiet des Kaiserreichs hat, müssen innerhalb von 10 Tagen zurückgezahlt werden, wenn keine anderweitige Frist angegeben wurde.
    \item Tut man dies nicht, verliert man bis zur Rückzahlung zusammen mit zusätzlichen 10 Kaisermark oder Gegenständen mit äquivalentem Wert die Kreditfähigkeit im Kaiserreich.
    \item Die Strafe nach dreifachem Aufschub liegt im Ermessen des zuständigen Gerichts.
    \item Auf dieses Gesetz kann eine Strafversicherung angewendet werden.\footnote{Reichsstrafversicherungsbeschluss vom 05. August 1920}    
\end{enumerate}

\subsection{Veruntreuung}
\begin{enumerate}[(1)]
    \item Es veruntreut, wer die Treuhandspflicht über Vermögen oder Wertanlagen verletzt und diese ohne Genehmigung des Eigentümers an andere überschreibt oder ausgibt, ohne die Möglichkeit, diese gleichwertig zu entschädigen.
    \item Dies wird nach wirtschaftsrechtlichem Bußgeldsatz und einer zusätzlichen Bußgeldstrafe von nicht unter 200 Kaisermark geahndet.
    \item Auf dieses Gesetz kann eine Strafversicherung angewendet werden.\footnote{Reichsstrafversicherungsbeschluss vom 05. August 1920}
\end{enumerate}

\newpage
\section{Staatsdeliktsgesetzbuch (SDelGB)}
\localtableofcontents

\subsection{Betreten des Staatsgebietes}
Das Betreten des Staatsgebietes darf nur mit einer ausdrücklichen Genehmigung erfolgen. Betritt man das Staatsgebiet ohne diese Aufenthaltsgenehmigung, so muss man 50 Kaisermark Strafe zahlen.

\subsection{Spionage}
\begin{enumerate}[(1)]
    \item Strategische Aufklärung und Spionage auf dem Staatsgebiet sind nicht erlaubt und daher strafbar. Aufgrund der besonderen Schwere wird dies mit einer Hinrichtung und 100 Kaisermark Strafe vergolten.
    \item Dies gilt nicht für Operationen, die durch den Staat ausdrücklich genehmigt wurden. 
    % \item Man darf ebenso wenig ohne Genehmigung das deutsche Territorium im Zuschauermodus durchqueren, denn gilt dies ebenfalls als Spionage.    
\end{enumerate}

\subsection{Strafen in den Reichsstädten}
In den Reichsstädten gelten die fünffachen Bußgeldsätze.

\subsection{Vergehen am Kaiser}
Wer sich am Kaiser vergeht, muss den zehnfachen Bußgeldsatz entrichten, sofern das Vergehen ein Staatsdelikt ist.

% \subsection{Finanzeller Status des Reichs}
% Der Kaiser kann auf nationaler Ebene nicht verschuldet sein.

\subsection{Hochverrat}
\begin{enumerate}[(1)]
    \item Als Hochverräter gilt, wer
    \begin{enumerate}[1.]
        \item Staats- oder regierungsfeindliche Inhalte verbreitet oder auf diese unerlaubt zugreift.
        \item Staatsgeheimnisse ohne Genehmigung verbreitet oder auf diese unerlaubt zuzugreift.
        \item Eine absichtliche Schwächung des Staates herbeiführt
        \item Die Befehle des Kaisers verweigert
    \end{enumerate}
    \item Der Strafsatz gleicht dem Strafsatz des Mordes an dem Kaiser.
\end{enumerate}

\subsection{Religiöse Gegenstände}
\begin{enumerate}[(1)]
    \item Wer Gegenstände religiöser Natur beschädigt oder zerstört oder auf religiösem Boden Verbrechen begeht, muss eine Bußgeldstrafe in Höhe von 200 Kaisermark zahlen und wird hingerichtet.
    \item Entweiht man religiöse Gebäude kommt dies dem dreifachen Strafsatz gleich.
\end{enumerate}

\subsection{Majestätsbeleidigung}
Beleidigt man den Kaiser, den Staat oder übergeordnete Staatsvertreter, so muss man 100 Kaisermark Strafe zahlen und wird hingerichtet.

\subsection{Regizid}
Tötet man den Kaiser, so muss die Familie des Täters, wie auch dieser selbst, ihren gesamten Besitz an das Kaiserreich übergeben und wird hingerichtet.

\section{Strafversicherungsgesetz (StVersG)}\footnote{Reichsstrafversicherungsbeschluss vom 05. August 1920}
\localtableofcontents

\subsection{Anwendung}
\begin{enumerate}[(1)]
    \item Die Strafversicherung ist eine Versicherungsmaßnahme zur Entschädigung von Vermögens- und Eigentumsverletzungen.
    \item Man kann bei dem Reichsschatzamt eine Strafversicherung beantragen, indem man die betreffende Kontonummer angibt, sowie gegen welche Art des Verstoßes, mit wie viel Geld und für welches Eigentum oder Vermögen man sich versichern möchte.
    \item Es kann die Strafversicherung nur auf ausgewiesene Gesetze angewandt werden.
    \item Für jeden Strafbestand muss zur Wirksamkeit auf diesen bei einem bestimmten Vermögen oder Eigentum, für das jeweilige Vermögen oder Eigentum und den Strafbestand ein eigener Antrag gestellt werden.
\end{enumerate}

\subsection{Inkrafttreten}
\begin{enumerate}[(1)]
    \item Die Versicherung tritt in Kraft, sobald er durch das Reichsschatzamt bewilligt, im Strafversicherungsregister eingetragen und der angegebene Betrag vom angegebenen Konto einmalig an das Reichsschatzamt überwiesen wurde.
    \item Sie kann nur durch den Eigentümer des geschützten Vermögens oder Eigentums oder einen, durch diesen durch ein Rechtsgeschäft gemäß § 1 HGB bevollmächtigte Person oder Entität angemeldet und abgemeldet werden.
    \item Im Falle einer Abmeldung hat man keinen Anspruch auf die Rückerstattung der Einzahlung, sofern man die Rückerstattung nicht innerhalb von zwei Wochen nach Antragstellung beantragt und die Versicherung in dieser Zeit nicht zur Anwendung gekommen ist.
\end{enumerate}

\subsection{Auszahlung}
Jede Person, die gegen strafversichertes Eigentum oder Vermögen eine Straftat verübt, auf die die Strafversicherung angemeldet wurde, muss für die Einzahlung vollständig erstatten.

\subsection{Missbrauch}
${^1}$Wer die Strafversicherung missbraucht, um sich oder einen Dritten zu bereichern, zahlt 100 Kaisermark und bekommt eine Freiheitsstrafe. ${^2}$Zusätzlich wird der Versicherungssatz ausgetragen.

\subsection{Strafversicherungsregister}
\begin{enumerate}[(1)]
    \item Im Strafversicherungsregister wird jede gültige Strafversicherungsgebühr mitsamt der Referenz auf das versichterte Eigentum oder Vermögen und der verwaltenden Kontonummer vermerkt.
    \item Eine Austragung aus dem Strafversicherungsregister kann gerichtlich verordnet werden.
\end{enumerate}

\end{document}