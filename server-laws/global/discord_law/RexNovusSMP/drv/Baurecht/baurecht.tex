\documentclass{article}
\usepackage[utf8]{inputenc}
\usepackage{enumerate}
\usepackage{ragged2e}
\usepackage{etoc}
\usepackage{amsmath}
\usepackage{graphicx}

\graphicspath{ {F:/global-files/server-laws/global/discord_law/RexNovusSMP/images} }

\renewcommand{\thesection}{}
\renewcommand{\thesubsection}{§\arabic{subsection}}

\title{Baurecht}
\author{Kaiser Friedrich IV.}
\date{05. August 1920}

\begin{document}
\maketitle
\begin{center}
    \includegraphics[scale=.15]{dr_wappen}
\end{center}
\vspace*{\fill}
\paragraph{Baurecht des Deutschen Kaiserreichs gemäß Reichsbaurechtsbeschluss vom 05. August 1920.}

\newpage
\topskip0pt
\vspace*{\fill}
\begin{Center}
\textbf{1. Fassung}
\vspace*{\fill}
\end{Center}
\newpage
\tableofcontents
\newpage
\section{Baugesetzbuch (BauGB)}
\localtableofcontents

\subsection{Antragstellung}
\begin{enumerate}[(1)]
    \item Um einen Bau vollziehen zu dürfen, muss man einen Bauantrag stellen, der gemäß \ref{baugen} genehmigt werden muss.
    \item Er muss nachfolgende Informationen beinhalten:
    \begin{enumerate}[1.]
        \item Den Auftraggeber
        \item Den Grundstücksbesitzer
        \item Das Grundstück
        \item Kurzgefasste Beschreibung des Gebäudes, die unter anderem den Verwendungszweck angibt
        \item Das Bauunternehmen gemäß \ref{bauunt}
    \end{enumerate}
    \item Für mehrere Projekte kann man auch einen zusammengefassten Gesamtantrag stellen, der allerdings alle Details für jedes Gebäude beinhält.
\end{enumerate}

\subsection{Besitzungshoheit}
\begin{enumerate}[(1)]
    \item Man darf nur Grundstücke erwerben, die im Deutschen Reichsgrundbuch eingetragen wurden.
    \item Ist man nicht Besitzer des Grundstücks, so benötigt man eine Vollmacht vom Grundstücksbesitzer, die den Regulatorien von § 1 HGB entspricht.
\end{enumerate}

\subsection{Baugenehmigung} \label{baugen}
\begin{enumerate}[(1)]
    \item Eine Baugenehmigung kann nur durch ein staatliches Rechtsgeschäft gemäß § 1 HGB erteilt werden.
    \item Dieses Rechtsgeschäft muss mit Genehmigung durch den Lehnsherrn geschehen.
    \item Ihm direkt übergeordnete Lehnsherrn dürfen die Genehmigung für ungültig erklären.
    \item Dies ist kein absoluter Akt und kann daher vor dem für diese Region und Ebene zuständigen Gericht angefochten werden.
\end{enumerate}

\subsection{Eingetragenes Bauunternehmen} \label{bauunt}
\begin{enumerate}[(1)]
    \item Grundstücksbebauungen können nur unter Beaufsichtigung durch ein staatlich anerkanntes Bauunternehmen vorgenommen werden.
    \item Unternehmen können sich durch die hamavarische Regierung als Bauunternehmen eintragen lassen.
\end{enumerate}

\subsection{Baumaterial}
\begin{enumerate}[(1)]
    \item Das Baumaterial muss den geltenden Gesetzen entsprechen und muss auf legalem Wege erworben und zur Baustelle transportiert worden sein.
    \item Es ist nicht genehmigt, gefährliches oder aus sonstigem Grund verbotenes Material im Haus zu verbauen.
\end{enumerate}
\newpage
\section{Grundbesitzordnung (GrBO)}
\localtableofcontents
\subsection{Gültigkeit}
\begin{enumerate}[(1)]
    \item Gültig ist ein Grundbesitz, solange er durch eine explizit für diese Region zuständige oder einer solchen übergeordneten, von dem
    Deutcshen Reich als solche ernannten Behörde gemäß \ref{rgbuch} in das Reichsgrundbuch eingetragen wurde.
    \item Mit ausreichender und nachvollziehbarer Begründung können die in Absatz 1 genannten Behörden diesen Eintrag aus dem Reichsgrundbuch austragen.
    \item Eine Eintragung im Reichsgrundbuch gewährt vollständige Eigentumsrechte gemäß GrBO.
    \item Die Eigentumsrechte schließen in Städten nur Innenräume, nicht jedoch Fassaden oder Dächer ein.
\end{enumerate}
\subsection{Reichsgrundbuch}\label{rgbuch}
\begin{enumerate}[(1)]
    \item Jegliche Eintragung in das Reichsgrundbuch, die durch die zuständige Behörde vorgenommen wurde, gilt im Deutschen Kaiserreich als rechtskräftiges Eigentum.
    \item Ein Eintrag muss folgende Informationen beinhalten:
    \begin{enumerate}[1.]
        \item Eintragsnummer
        \item Adresse
        \item Eigentümer
        \item Aktuelle Bevollmächtigte
        \item Angabe, ob es sich um ein Gewerbe handelt
    \end{enumerate}
\end{enumerate}
\subsection{Grundsteuern}
\begin{enumerate}[(1)]
    \item Die Eigentümer eines Grundstücks beziehungsweise Gebäudes müssen Grundsteuern entrichten.
    \item Die Grundsteuern richten sich nach den Bodenrichtwerten, die vom Reichsschatzamt beschlossen werden.
    \item Jegliche Errechnung und Rechnungsstellung zu den Grundsteuern geht von einer, durch das Deutsche Reich ernannten Behörde aus.
\end{enumerate}
\subsection{Gewerbebesitz}
\begin{enumerate}[(1)]
    \item Als Gewerbebesitz gilt jegliches Grundstück, beziehungsweise Gebäude, welches sich im Eigentum eines Gewerbes befindet.
    \item Gewerbe sind im Reichshandelsregister eingetragene Gesellschaften.
    \item Es kann nur als Gewerbebesitz eingetragen werden, was sich nachweislich in aktiver Nutzung des Unternehmens befindet.
    \item Jegliche Eintragung von Gewerbebesitz muss explizit angefordert werden und wird nicht selbstständig vom Staat übernommen.
    \item Gewerbebesitz ist von den Grundsteuern ausgenommen.
\end{enumerate}
\subsection{Hauptwohnsitz}
Als Hauptwohnsitz dient ein Wohnsitz gemäß Art. 24 Abs. 4 DRV, der mehr als die Hälfte des Jahres durch den Eigentümer bezogen wird.

\end{document}