\documentclass{article}
\usepackage[utf8]{inputenc}
\usepackage{enumerate}
\usepackage{ragged2e}
\usepackage{tocloft}
\usepackage{etoc}
\usepackage{amsmath}
\usepackage{chngcntr}

\renewcommand{\thesection}{}
\newcommand{\sent}[1]{$^{#1}$}
\newcommand{\ph}{\textbf{PLACEHOLDER}\ }
\counterwithout{subsection}{section}
\renewcommand{\thesubsection}{§\ \arabic{subsection}}
\cftsetindents{subsection}{3em}{4em}
\newenvironment*{pg}{\begin{enumerate}[(1)]}{\end{enumerate}}

\title{Gerichtsordnung (GerO)}
\author{Kaiser Friedrich IV.}
\date{23. Mai 1920}

\begin{document}
\maketitle
\vspace*{\fill}
\paragraph{Gerichtsordnung des Deutschen Kaiserreichs gmeäß Reichsjudikativordnung vom 23. Mai 1920}

\newpage
\topskip0pt
\vspace*{\fill}
\begin{Center}
\textbf{1. Fassung}
\vspace*{\fill}
\end{Center}
\newpage
\tableofcontents
\newpage
\section{Beweisführung}\label{zeugen}
\begin{enumerate}[(1)]
    \item Man darf Personen in den Zeugenstand rufen.
    \item Diese darf man unter den gegebenen Regeln befragen.
    \item Diese Regeln lauten:
        \begin{enumerate}[1.]
            \item Die Zeugen stehen unter Eid, sobald sie ihr erstes Wort im Zeugenstand erheben.
            \item Die Zeugen müssen daher alles wahrheitsgemäß beantworten.
            \item Jegliche ungenauen Aussagen der Zeugen werden nicht ins Protokoll aufgenommen (siehe hierzu \ref{eordnung} Absatz 5).
            \item Zeugen dürfen dem Prozess nur während ihrer Aussage beiwohnen. Davor und danach dürfen sie diesem erst zur Urteilsverkündung wieder beiwohnen.
        \end{enumerate}
    \item Zeugen, wie auch Beweise müssen vor dem Prozess angemeldet werden.
    \item Wenn die Beweise bei Prozessbeginn noch nicht vorliegen, muss deren Beschaffung angemeldet werden. Der Richter muss dann einen weiteren Prozess anberaumen, in dem diese Beweismittel auch geklärt werden können.
    \item Beweise und Zeugen dürfen nicht manipuliert werden.
\end{enumerate}

\subsection{Anwälte}
Man darf einen Anwalt einstellen. Hierbei muss jedoch beachtet werden, dass kein Anrecht auf einen Pflichtverteidiger besteht.

\subsection{Anwaltszulassung}
\begin{enumerate}[(1)]
    \item Einem jeden deutschen Bürger steht auch ohne Studium eine Anwaltszulassung zu.
    \item Bei Missbrauch oder mangelhafter, beziehungsweise fehlerhafter Wahrnehmung der Pflichten des Anwalts, kann einem die Zulassung entzogen werden.
    \item Die Entziehung kann durch ein bestandenes Staatsexamen widerrufen werden. Im Falle, dass man es bereits bestanden hat, muss man es wiederholen und erneut bestehen.
\end{enumerate}

\subsection{Einspruchsordnung}\label{eordnung}
\begin{enumerate}[(1)]
    \item Einsprüche sind erlaubt und bilden eine Ausnahme zu \ref{gordnung} Absatz 1.
    \item Sie können durch die Richterschaft abgewiesen werden.
    \item Bei einmaliger Ablehnung eines Einspruchs darf dieser nicht auf dieselbe Aussage erneut angewandt werden.
    \item Auf die Ankündigung eines Einspruchs muss stets die Ankündigung des Grundes folgen.
    \item Rechtlich zulässige Gründe sind:
        \begin{enumerate}[1.]
            \item Nicht aussagekräftig/unverständlich/mehrdeutig: Die Aussage oder Frage ist aufgrund seiner nicht aussagekräftigen Natur unzulässig.
            \item Bereits beantwortet: Die gleiche Frage wurde mehrfach gestellt, obwohl sie bereits beantwortet wurde.
            \item Unbewiesene Vermutung: Der Anwalt behauptet etwas, ohne sich auf vorliegende Beweise zu stützen.
            \item Fordert Spekulationen: Der Anwalt fordert den Zeugen auf, zu spekulieren.
            \item Zu viele Fragen: Der Anwalt fragt mehr als eine Frage gleichzeitig.
            \item Mangelnde Kenntnisse: Die Kenntnisse des Zeugens über das gefragte Thema sind unzureichend nachgewiesen.
            \item Ohne Priorität: Die Frage ist dem Prozess beziehungsweise der Befragung nicht dienlich.
            \item Gerücht: Die Antwort der Partei baut auf außergerichtlichen Aussagen auf.
            \item Hinterfragt die Staatsautorität: Eine Partei fechtet, hinterfragt oder beleidigt die Staatsautorität beziehungsweise die Autorität des Kaisers. Wird dieser Einspruch bewilligt, wird derjenige, der die Aussage gebracht hat, hinterher wegen Verstoßes gegen §5 SDelGB vor Gericht gestellt.
            \item Anfechtung: Der getätigte Einspruch ist nicht rechtmäßig, da dessen Gegenstand in diesem spezifischen Kontext keinen Verstoß gegen die Befragungsordnung darstellt.
        \end{enumerate}
    \item Wird ein Einspruch stattgegeben, so muss der Befragende bei der Befragung mit der nächsten Frage fortfahren. Der Zeuge darf die vorherige Frage nicht beantworten oder seine Aussage wird im Fall, dass er sie bereits getätigt hat oder dennoch antwortet, gestrichen. Erhebt ein Richter diesen Einspruch, so ist dem sofort stattgegeben, sofern der Gerichtsvorsitzende dem nicht widerspricht. 
    \item Der Richterschaftsvorsitz darf den Einspruch nur ablehnen, wenn der Gegenstand des Einspruchs in keinem vermutbaren Kontext einen Verstoß gegen die Befragungsordnung darstellt.   
\end{enumerate}

\subsection{Plädoyer}\label{pleed}
\begin{enumerate}[(1)]
    \item Jede, zur Aussage verpflichteten Partei, hat die Pflicht, ihre Position durch ein Schlussplädoyer zu verteidigen.
    \item Dieses Plädoyer muss den geltenden Konventionen entsprechen.
    \item Die Schlussfolgerung des Plädoyers ist die Einlassung des Mandanten, sofern es sich um das Schlussplädoyer der Beklagten handelt, und die Strafempfehlung.
    \item Eine Strafempfehlung darf nur durch einen Anwalt geäußert werden.
\end{enumerate}

\subsection{Prozessverlauf}\label{verlauf}
Das deutsche Recht sieht den nachfolgenden Verlauf für Gerichtsverfahren vor:
\begin{enumerate}[1.]
    \item Alle Parteien mit Ausnahme der Richterschaft betreten den Raum.
    \item Die Richterschaft versammelt sich. Währenddessen muss jeder Anwesende stehen.
    \item Der Gerichtsvorsitzende eröffnet den Prozess und die weiteren Richter setzen sich.
    \item Die Beklagte verliest die Anklageschrift.
    \item Der Kläger muss den Strafbestand aus seiner Sicht darlegen.
    \item Der Beklagte hat das Wort und darf seine Darstellung des Sachverhalts darlegen.
    \item Von nun an entscheidet der Gerichtsvorsitzende, wer das Wort erhält.
    \item Sobald alle Beweise und Aussagen der beiden Parteien dargelegt wurden, müssen die Schlussplädoyers gemäß \ref{pleed} gehalten werden.
    \item Die Richterschaft tritt zurück und berät sich in einem separaten Gespräch. Hierbei wird über die Strafe beratschlagt und anschließend entschieden. Bei Stimmgleichheit verfügt der Gerichtsvorsitzende eine zweite Stimme.
    \item Die Richterschaft betritt den Saal, wobei erneut jeder stehen muss, und verkündet im Anschluss die Strafe. Daraufhin fragt der Gerichtsvorsitzende, ob eine Partei in Berufung gehen möchte, sofern denn eine höhere Instanz besteht. Andernfalls ist die Strafe final.
    \item Bis der letzte Richter den Saal verlassen hat müssen alle Teilnehmer stehen und dürfen den Saal nicht verlassen.    
\end{enumerate}

\subsection{Ordnungspflicht}\label{gordnung}
\begin{enumerate}[(1)]
    \item Man darf nicht unaufgefordert sprechen.
    \item Man muss sich für den Prozess angemessen kleiden. Dementsprechend dürfen die Anwesenden keine Kopfbedeckungen mit sich führen und müssen einen Anzug in einer angemessenen Farbe tragen.
    \item Richter müssen schwarze Anzüge tragen.
    \item Im Falle des Reichsverfassungsgerichts müssen die Richter rote Anzüge tragen.
    \item Verstöße gegen die Gerichtsordnung unter Inbezugnahme von \ref{zeugen} und \ref{verlauf} werden, sofern bereits eine Verwarnung erteilt wurde mit 10 Kaisermark Bußgeld geahndet. Liegen nach Ermessen der Richterschaft zu viele Verstöße vor, können sie die schuldige Partei ungeachtet ihrer Relevanz für diesen Prozess aus dem Saal verweisen und das Verfahren anschließend in dessen Abwesenheit fortfahren.
    \item Von Absatz 5 ist lediglich der Kaiser ausgenommen.
\end{enumerate}

\subsection{Gerichtliche Vorladung}
\begin{enumerate}[(1)]
    \item Sofern ein Verfahren bestätigt wurde kann unter Vereinbarung mit beiden Parteien ein Gerichtstermin festgelegt werden. Dies wird als außerordentliche Vorladung angesehen.
    \item Legt das Gericht einen Termin fest, so muss dieses beide Parteien in einem Schreiben deutlich über das Verhandlungsdatum informieren. Hierbei handelt es sich um eine ordentliche Vorladung
    \item Der Termin und Ort einer Verhandlung muss spätestens zwölf Stunden vor Prozessbeginn bekanntgegeben werden.
    \item Ein Antrag auf Aufschub kann bis zu zwei Stunden vor Prozessbeginn eingereicht werden.
    \item Wird diesem Antrag durch den Gerichtsvorsitzenden des Verfahrens stattgegeben, so wird das Verfahren vertagt.
    \item Andernfalls, oder wenn kein Antrag besteht, müssen die Parteien erscheinen, ansonsten wird in ihrer Abwesenheit verhandelt.
    \item Erscheint keine Partei, so wird der Termin ebenfalls vertagt.
    \item Jeder gemäß Absatz 6 abwesenden Partei droht eine Bußgeldstrafe in Höhe von 20 Kaisermark.
    \item Der Verhandlungsort wird gemäß Wohnsitz der beklagten Partei entschieden.
    \item Verfügt die beklagte Partei über keinen Wohnsitz im deutschen Kaiserreich, so wird gemäß Wohnsitz der klagenden Partei entschieden.
    \item Können weder Absatz 10 noch Absatz 11 erfüllt werden, so entscheidet der Staat über den Verhandlungsort.    
\end{enumerate}

\subsection{Anrede des Richters}
Steht man vor Gericht, so hat man den Richter als `Euer Ehren' anzureden. Tut man dies nicht, muss man fünf Kaisermark zahlen.

\subsection{Verbannung}
\begin{enumerate}[(1)]
    \item Verbannung dient im Falle von Zahlungsunfähigkeit als Ersatz für hohe Bußgeldstrafen. Die verzehnfachte Bußgeldstrafe entspricht der Anzahl der Tage einer Verbannung.
    \item Als Verbannter darf man das Gebiet des Kaiserreichs nicht betreten.
\end{enumerate}

\subsection{Entschädigungssteuern}
\begin{enumerate}[(1)]
    \item Auf Entschädigungen werden zusätzlich zu den, im Recht verankerten Bußgeldsätzen, eine Steuer erhoben.
    \item Der Steuersatz wird alle 30 Tage von dem Reichsschatzmeister festgelegt.
    \item Die Steuern umfassen einen Mindestbetrag von 1 Kaisermark und werden stets aufgerundet.
    \item Nur Bürger und Personen, bei denen durch den entstandenen Schaden Kosten aufkommen, haben Anspruch auf Entschädigung.
    \item Alle, von Absatz 4 ausgenommenen Bußgelder gehen an den Staat.
\end{enumerate}

\subsection{Freiheitsstrafe}
\begin{enumerate}[(1)]
    \item Eine Haftstrafe kann bei Beschluss des Gerichts entweder als Strafersatz oder Strafzusatz angewendet werden.
    \item Bei Ausbruchsversuchen und Ausbrüchen werden stets zehn Minuten zusätzliche Haft angeordnet.
    \item Beihilfe zu Ausbrüchen werden mit dem Verordnen der gleichen Haftstrafe für die helfende Partei bestraft.
    \item Abgesessen hat man die Strafe, sobald man die jeweilige Zeit nachweislich anwesend war.
    \item Der Staat haftet für keine Gegenstände, die während der Haftstrafe verlorengehen, sofern für den Häftling oder den zu Inhaftierenden genügend Zeit bestand, die Gegenstände anderweitig zu lagern.
    \item Der Strafsatz bemisst sich in 5-Minuten-Sätzen
\end{enumerate}

\subsection{Hinrichtung}
\begin{enumerate}[(1)]
    \item Hinrichtungen sind als Strafmaßnahme für Kapitalverbrechen vorgesehen, können jedoch auch als Strafmaßnahme für andere Verbrechen verhängt werden.
    \item Sie sind erst dann erlaubt, wenn das zuständige Gericht eindeutig die Todesstrafe verhängt hat.
    \item Nur Gerichte ab der Landesebene können Todesstrafen verhängen.
    \item Kapitalverbrechen müssen in erster Instanz vor dem zuständigen Landesgericht verhandelt werden.
\end{enumerate}

\subsection{Verbindlichkeit von Strafsätzen}
\begin{enumerate}[(1)]
    \item Strafsätze sollten an der Schwere und Häufigkeit des Verbrechens des Einzelnen bemessen werden.
    \item Die aufgeführten Strafsätze dienen lediglich der Orientierung und sind daher nicht verbindlich.
    \item Dies gilt nicht für Hinrichtungen.
    \item Bei Wiederholungstaten liegt es je nach Häufigkeit und Schwere der Tat im Ermessen des zuständigen Gerichts, ob weiterhin derselbe oder ein verhärteter Strafsatz geltend gemacht werden sollte.
    \item Bei äußerster Häufigkeit oder relativer Häufigkeit von Taten besonderer Schwere, haben Wiederholungstaten die Todesstrafe zur Folge.
    \item Sofern die Verhängung einer Strafmaßnahme außerhalb der Vollmachten des zuständigen Gerichts steht, ist das Gericht verpflichtet, ein Berufungsverfahren zu erzwingen und den Fall an die nächste
    Instanz weiterzugeben, die zur Verhängung der Strafmaßnahme fähig ist. Dieses Gericht darf den Prozess nicht abweisen und muss den Fall verhandeln, die Strafmaßnahme allerdings nicht verhängen.
\end{enumerate}

\subsection{Untersuchungshaft}
Besteht die Gefahr, dass ein Tatverdächtiger bis zu seinem Prozess flieht oder befragt werden muss, muss eine Unterbringung in der Untersuchungshaft angeordnet werden.

\subsection{Unterbringung in einer Hochsicherheitseinrichtung}
\begin{enumerate}[(1)]
    \item Freiheitsstrafen in Höhe von mehr als dreißig Minuten müssen in Hochsicherheitseinrichtungen abgesessen werden.
    \item Besteht eine akute Fluchtgefahr, so kann dies auch bei kürzerer Haft angeordnet werden.
\end{enumerate}

\subsection{Sonderverwahrung}
Personen, die sich eines Kapitalverbrechens schuldig gemacht haben und daher hingerichtet werden sollen, müssen sofern zusätzlich eine Freiheitsstrafe angeordnet wurde, in einer Todeszelle untergebracht werden. Mit Ende ihrer Haftstrafe werden sie hingerichtet.

\subsection{Entzug von Titeln}
\begin{enumerate}[(1)]
    \item Es ausschließlich ist dem Reichsgerichtshof gestattet, bestimmten Personen den Titel zu entziehen, sofern sie dessen Macht missbrauchen oder mit ihr anderweitig nicht umgehen können.
    \item Dies bezieht den Fall ein, dass ein Amtsträger sein Gebiet über lange Zeit hinweg nicht bebaut oder kein Interesse an dem Titel selbst, sondern lediglich an dessen Vollmachten aufzeigt.
    \item Anklagen wegen Verstoßes dieses Gesetzes gehen vom Stand der Anklage aus, nicht vom Stand des darauffolgenden Gerichtsprozesses.
\end{enumerate}

\subsection{Präzedenzfälle}
\begin{enumerate}[(1)]
    \item Sofern ein rechtlicher Ausnahmefall vorliegt, ist der Fall unter sofortiger Wirkung dem Reichsgerichtshof zu übertragen.
    \item Entscheidet dieser, dass es sich bei dem vorliegenden Fall um eine Straftat handelt, so muss dies umgehend in die Gesetze aufgenommen werden und
    sofern nach Ermessen des Reichsgerichtshof ein Bewusstsein des Verstoßes gegen geltende moralische Normen durch die Beklagte vorliegen sollte, der Strafe entsprechend
    geurteilt werden.
\end{enumerate}

\subsection{Generationenrecht}
\begin{enumerate}[(1)]
    \item Verstirbt ein Kläger oder Opfer eines Verbrechens, so darf das Haus des Geschädigten Anklage erheben oder die Geschädigte vor Gericht vertreten.
    \item Verstirbt ein Täter, so muss sich das Haus des Täters für dessen Straftaten verantworten.
    \item Das Haus wird stets durch dessen Oberhaupt gemäß Erbrecht vertreten. Besteht keins, so wird dieses vom zuständigen Gericht gewählt.
    \item Gemäß Absatz 2 können demnach auch die nachfolgenden Oberhäupter zur Rechenschaft gezogen werden.    
\end{enumerate}

\end{document}