\documentclass{article}
\usepackage[utf8]{inputenc}
\usepackage{enumerate}
\usepackage{amsmath}
\usepackage{ragged2e}
\usepackage{etoc}

\renewcommand{\thesection}{}
\renewcommand{\thesubsection}{\Roman{subsection}}
\counterwithin{subsubsection}{section}
\counterwithout{subsubsection}{subsection}
\renewcommand{\thesubsubsection}{§\arabic{subsubsection}}

\title{Wirtschaftsrecht}
\author{Kaiser Friedrich IV.}
\date{05. August 1920}

\begin{document}
\maketitle
\vspace*{\fill}
\paragraph{Wirtschaftsrecht des Deutschen Kaiserreichs gemäß Reichswirtschaftsrechtsbeschluss vom 05. August 1920.}
\newpage
\newpage
\topskip0pt
\vspace*{\fill}
\begin{Center}
\textbf{1. Fassung}
\vspace*{\fill}
\end{Center}
\newpage
\tableofcontents
\newpage
\section{Handelsgesetzbuch (HGB)}
\localtableofcontents
\subsection{Grundregelungen}
\subsubsection{Rechtsgeschäft}\label{rechtsg}
\begin{enumerate}[(1)]
    \item Als gültiges und damit auch verbindliches Rechtsgeschäft gilt jeder der nachfolgenden Rechtsakte, sofern dieser gänzlich gesetzeskonform ist:
    \begin{enumerate}[1.]
        \item Testamentarische Verfügung
        \item Vertragsgeschäfte
    \end{enumerate}
    \item Ein Rechtsgeschäft verliert im Kontext von § 18 GerO seine Gültigkeit auch dann, wenn nach gerichtlichem Urteil kein Bewusstsein der Schuld vorliegt.
    \item Es bedarf einer Beglaubigung durch einen Notar, der vom Deutschen Reich ernannt wurde, um vor Gericht gültig zu sein.
    \item Finanztransaktionen müssen immer mit dem Konto des Betroffenen ausgeführt werden.
    \item Ein Rechtsgeschäft ist null und nichtig, sobald dieser die Würde oder das Recht auf Unversehrtheit einer unterzeichnenden oder betroffenen Partei angreift oder einschränkt oder einen Dritten betrifft, der dem Rechtsgeschäft nicht nachweislich zugestimmt hat.
\end{enumerate}

\subsubsection{Allgemeine Regelung bezüglich wirtschaftsrechtlicher Bußgeldsätze und sonstiger Strafsätze}
\begin{enumerate}[(1)]
    \item Für jegliche Straftat, die im Kontext des Wirtschaftsrechts begangen wird, gilt dass die Bußgeldstrafen durch die Gesellschaft und bei dessen Zahlungsunfähigkeit durch dessen Gesellschafter ungeachtet der Gesellschaftsform verrichtet werden müssen.
    \item ${^1}$Strafen, die durch eine Gesellschaft wegen Zahlungsunfähigkeit oder aus anderen Gründen unmöglich verbüßt werden können, müssen von den Vertretungsberechtigten anteilsgemäß verbüßt werden. ${^2}$Es müssen nur Anteilseigner haften, die über 1\% Anteile besitzen. Die restlichen Anteile müssen durch diese ebenfalls gemäß Unternehmensanteilen verrichtet werden. ${^3}$Ein Unternehmen muss über mindestens einen Anteilseigner mit über 1\% Anteilen verfügen, um geschäftsfähig zu sein.
    \item Wird eine Freiheitsstrafe verordnet, so müssen nur diejenigen diese absitzen, die die Tat begangen haben.
    \item Im Falle eines Bankrotts können haftungsbeschränkte Gesellschaften nicht wegen Verstoßes gegen § 5 ZivilGB angeklagt werden.
    \item Wirtschaftsrechtliche Bußgeldsätze sind anhand des Umsatzes des Gesellschafters oder der Gesellschaft zu errechnen.
    \item Gerichte können die sofortige Austragung aus dem Handelsregister veranlassen.
\end{enumerate}

\subsubsection{Rechtswidrige Bereicherung}
Wer sich rechtswidrig bereichert muss zusätzlich zu den geltenden Strafsätzen die dadurch erwirtschafteten Güter an die Betroffenen ausnahmslos zurückerstatten.

\subsubsection{Gesellschaftsführung}\label{firm}
\begin{enumerate}[(1)]
    \item Wer eine Gesellschaft gründet, muss diese in das Deutsche Reichshandelsregister (\ref{register}) eintragen lassen.
    \item Diese Gesellschaften müssen präzise Buch führen (\ref{buch}).
    \item Eine Person kann erst dann rechtskräftig zum Eigentümer einer bereits gegründeten Gesellschaft ernannt werden, sofern ein Rechtsgeschäft (\ref{rechtsg}) vorliegt, in welchem der vorherige Eigentümer die Gesellschaft dem neuen Eigentümer nachweislich überträgt und der neue Eigentümer in das Reichshandelsregister (\ref{register}) eingetragen wurde.
    \item Sobald ein Eigentümer beabsichtigt zurückzutreten und kein neuer Eigentümer gemäß Absatz 3 nachfolgt, ist der Eigentümer für die offenen Geschäfte des Unternehmens verantwortlich. Laufen diese aus, so darf dieser zurücktreten.
    \item Solange kein Eigentümer gemäß Absatz 3 nachfolgt, darf die Gesellschaft keine neuen Geschäfte aufnehmen.
    \item Absatz 3ff. gelten nur dann, wenn es sich um kein Familienunternehmen (\ref{familien}) handelt.
    \item Anteilsunbeschränkte Gesellschaften müssen von einem Vorstand, bestehend aus allen Anteilhabern geführt werden, der alle zwei Monate einen Vorstandsvorsitz wählt.

\end{enumerate}


\subsubsection{Handelsregister}\label{register}
\begin{enumerate}[(1)]
    \item Einträge im Handelsregister werden von dem Reichsschatzamt vorgenommen.
    \item Jegliche Gesellschaft muss mit folgenden Informationen eingetragen werden:
    \begin{enumerate}[1.]
        \item Name der Gesellschaft
        \item Vertretungsberechtigte, falls es sich um eine anteilsbeschränkte Gesellschaft handelt
        \item Adresse der Hauptzweigstelle
        \item Adressen weiterer Zweigstellen
        \item Handelsregistereintragsnummer
        \item Bestätigendes Gericht
        \item Rechtsform und Rechtsformzusätze
        \item Stammkapital
        \item Nummern der Unternehmenskonten
        \item Datum der Gründung
        \item Datum der Eintragung, falls abweichend vom Datum der Gründung
        \item Unternehmensgegenstand
        \item Sonstige Rechtsverhältnisse
    \end{enumerate}
    \item Hoflieferanten können zusätzlich die, ihnen zustehenden Garantiesätze eintragen lassen.
    \item Fehlerhafte Angaben durch denjenigen, der die Eintragung veranlasst, sind strafbar und können zur sofortigen Austragung des Unternehmens führen.
    \item Umwidmungen und Eintragungen werden nur unter Zahlung einer Gebühr gemäß Reichsgebührenkatalog veranlasst.
    \item Das Deutsche Reichssonderhandelsregister ist für Gesellschaften vorbehalten, die insbesondere im geheimen Staatsdienst tätig sind. Jegliche Informationen aus diesem dürfen nur auf Genehmigung des Kaisers hin bereitgestellt werden.
    \item Prozesse bezüglich derartiger Gesellschaften erfordern die Schließung des Prozesses für die Öffentlichkeit.
\end{enumerate}


\subsubsection{Auslandshandelsgebühren}
\begin{enumerate}[(1)]
    \item Gesellschaften, die im Ausland Tochterunternehmen eröffnen, müssen zusätzliche Gebühren an das Deutsche Reich zahlen.
    \item Diese Gebühren müssen der Eintragung im Reichsgebührenkatalog entsprechen. 
    \item Sind die Gebühren zur Eröffnung der Zweigstelle im Ausland billiger, so muss die Gesellschaft die Kostendifferenz zum Gebührensatz (Absatz 2) an das Kaiserreich zahlen.
    \item Andernfalls muss es die Zusatzgebühren der Empfehlung entsprechend erstatten.
    \item Von dieser Regelung ausgenommen sind leistungsbeschränkte Gesellschaften. Im Falle von Absatz 3 müssen sie keine Zusatzgebühren bezahlen und im Falle von Absatz 4 übernimmt das Kaiserreich Hamavar die Kostendifferenz zur Empfehlung.
    \item Letzteres verliert seine Wirksamkeit, sofern das Reichsschatzamt die Unterstützungen innerhalb des fraglichen Staates generell verwehrt und der Antrag nach der öffentlichen Bekanntgabe dieser Verwehrung gestellt wurde.
    \item Eröffnet eine ausländische Gesellschaft im Deutschen Reich eine Zweigstelle, so muss sie die Gebühren vollständig erstatten.
    \item Jegliche Erstattungen gemäß Abs. 4f. erfordern eine Antragsstellung beim zuständigen Amt.
\end{enumerate}


\subsubsection{Familienunternehmen}\label{familien}
\begin{enumerate}[(1)]
    \item Familienunternehmen dürfen nur von deutschen Bürgern gegründet werden, die seit mehr als zehn Jahren in Deutschland ansässig sind und dort auch ihren Hauptwohnsitz haben.
    \item Sie dürfen lediglich von Familienmitgliedern geführt werden
    \item Die Gesellschaft kann nur dann an andere Familien ausgehändigt werden, sofern der Eigentümer verfügt, dass die Gesellschaft zu einer nicht-familiären Gesellschaft umgewidmet und anschließend an den außerfamiliären Eigentümer übertragen wird.
    \item Im Falle von Absatz 3 kann das Unternehmen nicht zu Lebzeiten des neuen Eigentümers zu einem Familienunternehmen umgewidmet werden.
    \item Verstirbt das letzte Mitglied der Familie, so verliert das Familienunternehmen seine Geschäftsfähigkeit gemäß \ref{firm} Abs. 5, sofern keine rechtsgültige Nachfolge bewirkt wurde.
    \item Gesellschaften mit erweiterten Anteilsrechten können nicht als Familienunternehmen eingetragen werden.
\end{enumerate}

\subsubsection{Buchführung} \label{buch}
\begin{enumerate}[(1)]
    \item Es besteht allgemeine Buchhaltungspflicht.
    \item Zu jedem Geschäft muss in diesem Falle folgendes vermerkt werden:
    \begin{enumerate}[1.]
        \item Verkäufer (sofern er von der Hauptgesellschaft abweicht)
        \item Kunde (sofern er von der Hauptgesellschaft abweicht)
        \item Gesamtpreis (dies schließt auch Tauschwaren ein)
        \item Gehandelte Gegenstände, beziehungsweise Kommentar zu gegenstandslosen Transaktionen
        \item Anmerkung, ob es sich um einen Realgewinn oder einen Imaginärgewinn handelt.
    \end{enumerate}
    \item Liegt eine gegenstandlose Transaktion, beispielsweise Schenkung oder Spenden in finanzieller oder gegenständlicher Form  vor, so müssen die Kommentare sinnig sein und für die bearbeitende Behörde ersichtlich sein.
    \item Die Gesellschaft muss monatlich der zuständigen Behörde die Buchhaltung zukommen lassen.
    \item Verstöße gegen die Buchhaltungbestimmungen haben ein Strafmaß gemäß \ref{apored} Abs. 3 zur Folge.
    \item In den ersten drei Monaten nach der Gründung muss insgesamt ein eindeutiger Gewinn von fremder Seite vorliegen.
    \item Danach muss ein insgesamt bestehender Gewinn im Abstand von einem Jahr regelmäßig nachgewiesen werden.
    \item Andernfalls muss die Gesellschaft Strafgebühren zahlen, die von dem Reichsschatzamt beschlossen werden.
    \item Im Falle, dass nur Verluste registriert werden, muss die Gesellschaft geschlossen und ausgetragen werden.
    \item Mit Ausnahme von Absatz 9 werden jegliche Verluste, die Gesellschaften registrieren, bei fehlerfreier Buchhaltung vom Staat zu dem aktuellen Kostenerstattungssatz erstattet.
    \item Bei Verdacht, dass eine Gesellschaft keine Gewinne macht, kann der Staat Buchführung verordnen.
    \item Im Kontext dieses Gesetzes gelten nur Realgewinne als Gewinne. Dennoch müssen die Imaginärgewinne ebenfalls registriert werden.
\end{enumerate}

\subsubsection{Realgewinne}\label{real}
\begin{enumerate}[(1)]
    \item Als Realgewinne werden Gewinne bezeichnet, die nicht als gesammelte Auszahlung eingestuft werden können.
    \item Im Gegensatz zu diesen steht der Imaginärgewinn.
\end{enumerate}

\subsubsection{Gesammelte Auszahlungen}
Eine gesammelte Auszahlung ist eine Auszahlung, dessen Ursprung nicht auf individuelle Konten zurückführbar ist.

\subsubsection{Verkettungsverbot}
\begin{enumerate}[(1)]
    \item Es herrscht das Prinzip, dass ein Gewinn nur dann gemäß \ref{real} Abs. 1 ein Realgewinn ist, wenn dieser Gewinn nicht direkt oder durch Gesellschaftsbesitzverkettungen auf einen gleichen Transaktionsempfänger und Transaktionssender zurückführbar ist.
    \item Absatz 1 gilt auch dann, wenn festgestellt wird, dass zwei parallele Gesellschaftsbesitzverkettungen zu einem Gesamtgesellschaftsbesitzkreislauf zusammengeschlossen wurden und so durch Inbezugnahme mindestens eines Zweiten die obere Regel umgangen werden soll.
\end{enumerate}

\subsubsection{Handelswert}
\begin{enumerate}[(1)]
    \item Als Handelswert wird das monatliche Handelsvolumen einer Gesellcshaft bezeichnet.
    \item Die Einstufung des Handelswerts wird von einer unabhängigen, vom Staat ernannten Gesellschaft vorgenommen.
    \item Der Handelswert ist eine gesammelte Auszahlung.
\end{enumerate}

\subsubsection{Wirtschaftsbetrug}
\begin{enumerate}[(1)]
    \item Es begeht Wirtschaftsbetrug, wer:
    \begin{enumerate}[1.]
        \item Sich oder einen Dritten durch Beisteuerung internen Wissens bereichert.
        \item Mittels durch einen Dritten beigesteuerten internen Wissens sich selbst bereichert hat.
        \item Unternehmen mit dem Ziel führt, Grundsteuern zu umgehen.
        \item Die Begünstigungen einer nicht gewinnorientierten Organisation gezielt ausnutzt und sich daran bereichert.
        \item Dem Verkettungsverbot entgegen zur eigenen Bereicherung, Bereicherung eines dritten oder einer eigenen Gesellschaft oder der eines dritten Überweisungen tätigt, die keinem legitimen oder legitimierbaren Zweck dienen.
        \item Sich oder einen Dritten durch Kauf geringstpreisigster Wertpapiere trotz hoher Vermögensklasse bereichert.
        \item Wer Aktienwerte gezielt und zur Bereicherung von sich oder einem Dritten manipuliert.
    \end{enumerate}
    \item Dies wird mit einer Bußgeldstrafe von nicht unter 200 Reichsmark zusätzlich zur Auszahlung sämtlichen, aus dem Wirtschaftsbetrug
    resultierenden, Gewinns und einer Freiheitsstrafe geahndet.
\end{enumerate}


\subsubsection{Insolvenz}\label{apored}
\begin{enumerate}[(1)]
    \item Verfügt eine Gesellschaft nur noch über die Hälfte des Stammkapitals, muss es Konkurs anmelden.
    \item Eine Gesellschaft, welche bankrott geht und zuvor nicht an einen neuen Eigentümer überschrieben wurde, verliert die Genehmigung, Geschäfte auszuüben und wird aus dem Handelsregister ausgetragen.
    \item Geht die Gesellschaft Absatz 1 oder 2 nicht nach, so muss der Eigentümer die Haftung ungeachtet der Rechtsform übernehmen und die Gesellschaft wird umgehend aus dem Handelsregister ausgetragen und ist somit nicht länger fähig, ihren Eigentümer zu wechseln. Zudem muss der Gesellschafter ein Bußgeld von nicht unter 100 Reichsmark zahlen.
    \item Die Insolvenzschuld wird unter anderem unter Betrachtung des Verkettungsprinzips beurteilt.
\end{enumerate}


\subsubsection{Internationaler Handel}
\begin{enumerate}[(1)]
    \item Um Transaktionen in das Ausland und Inland vorzunehmen, muss man eine Zweigstelle auf deutschem Territorium und im entsprechenden Empfängerland im Reichshandelsregister registriert haben.
    \item Zweigstellen im Deutschen Rechtsraum müssen über eine Deutsche Rechtsform verfügen.
    \item Es ist deutschen Staatsbürgern untersagt, die Hauptzweigstelle im Ausland zu gründen.
    \item Der Buchführungspflicht gemäß \ref{buch} unterliegen jegliche deutschen Hauptzweigstellen und deren Zweigstellen, sowie jegliche Hauptzweigstellen und deren Zweigstellen, sofern sie eine Zweigstelle im Deutschen Kaiserreich haben.
    \item Nur logistische Gesellschaftsformen dürfen Waren über Staatsgrenzen bewegen.
\end{enumerate}

\subsubsection{Haftung}
\begin{enumerate}[(1)]
    \item Haftung für die Waren übernimmt derjenige, der sie zurzeit besitzt.
    \item Dies gilt sowohl auf deutschem Grunde als auch für Gesellschaften mit Zweigstelle oder Hauptzweigstelle auf deutschem Grund.
    \item Die Haftung gegenüber dem Staat unterliegt stets dem Gesellschafter, was in diesem Kontext jegliche vertretungsberechtigte Person einbezieht.
    \item Im Falle, dass der Gesellschafter nicht die juristische Person der Gesellschaft ist, trifft Absatz 3 nur dann zu, wenn die juristische Person zahlungsunfähig ist.
\end{enumerate}

\subsubsection{Banken}
\begin{enumerate}[(1)]
    \item Deutsche Banken müssen die Kontoinformationen der Kunden bei Anfrage durch das Reichsschatzamt oder andere zuständige Behörden offenlegen.
    \item Ohne Beschluss des Reichsschatzamts dürfen Banken keine Kontensperrung vornehmen.
    \item Deutschen Gesellschaften und Privatkunden ist es untersagt, sich bei Banken zu registrieren, die nicht den notwendigen Grad an Kontoinformationseinsicht für das Deutsche Reich bereitstellen.
    \item Verstöße gegen Absatz 3 werden für Gesellschaften gemäß \ref{apored} Abs. 3 geahndet.
    \item Als Bank gilt jegliche Gesellschaft, die Konten verwaltet oder Währungen herausgibt oder diese Dienstleistungen anbietet.
    \item Banken können nicht die Rechtsform einer nicht gewinnorientierten Gesellschaft haben.
    \item Im Falle einer unverschuldeten Insolvenz müssen Banken das Geld der Kunden nicht auszahlen.
    \item Eine Bank verfügt über allgemeine Treuhandspflicht für die Konten, die bei ihr angemeldet sind.
\end{enumerate}

\subsubsection{Währungsschutz}
\begin{enumerate}[(1)]
    \item Es ist untersagt, ohne staatliche Genehmigung eine Währung im deutschen Handelsraum zu führen.
    \item Ebenso ist es untersagt, mit einer derartigen Währung dort zu handeln.
    \item Als Währung gilt, was im Tausch den Rang einer Währung besitzt.
    \item Regelmäßiger Tauschhandel mit dieser Ware bei Registrierung einer Wertänderung entspricht dem Rang einer Währung.
    \item Ausgeschlossen ist, was ausschließlich im Tausch gegen eine genehmigte Währung oder vertraglich und durch staatliche Genehmigung gehandelt wird.
    \item Sowohl der Handel als auch die Führung einer ungenehmigten Währung gilt als Geldwäsche besonderer Schwere und wird gemäß § 12 StGB geahndet.
\end{enumerate}

\subsubsection{Gesellschaftskapital}
\begin{enumerate}[(1)]
    \item Jede haftungsbeschränkte Gesellschaft muss über ein Stammkapital verfügen, welches die gründende Partei zusammen mit den Gründungsgebühren entrichtet.
    \item Im Gegensatz zu den Gründungsgebühren wird dieses Stammkapital als Gesellschaftskapital verwendet.
    \item Gesellschaften, die in den Rahmen von Absatz 1 fallen, sind verpflichtet, separate Bankkonten für die Gesellschaft zu eröffnen.
    \item Die Gründungsgebühren werden im Reichsgebührenkatalog geregelt.
\end{enumerate}

\subsubsection{Umsatzsteuern}
\begin{enumerate}[(1)]
    \item Jeder Umsatz eines gewinnorientierten Unternehmens muss gemäß aktuellem Steuersatz vom Deutschen Reich besteuert werden.
    \item Fehlerhafte Besteuerung ist strafbar und wird gemäß \ref{apored} Abs. 3 geahndet, sofern die Schuld bei der Gesellschaft liegt.
\end{enumerate}

\subsection{Rechtsformen}
\subsubsection{Rechtsform}\label{gesellform}
\begin{enumerate}[(1)]
    \item Die Rechtsform einer Gesellschaft bestimmt die Haftungs-, sowie Handelsbedingungen.
    \item Im Deutschen Reich anerkannte Rechtsformen sind:
        \begin{enumerate}[1.]
            \item Transportgesellschaft (TrG)
            \item Handelsgesellschaft deutschen Rechts (HGdR)
            \item Reichsgesellschaft (RG)
            \item Privatgesellschaft (PG)
            \item Aktiengesellschaft (AG)
            \item Transportgesellschaft auf Aktien (TrG a.A.)
            \item Reichsgesellschaft gesonderten Besitzes (RGgB)
            \item Gemeinnützige Organisation (Org)
        \end{enumerate}
\item Gesellschaften haften mit dem Kapital der juristischen Person.
\item Ebenso gehört der Umsatz der Gesellschaft der juristischen Person.
\item Die juristische Person von haftungsbeschränkten Gesellschaften ist die Gesellschaft selbst.
\item Gesellschaften haftungsbeschränkter Rechtsformen müssen bei ihrer Gründung über ein Mindeststammkapital verfügen, welches von dem Reichsschatzamt beschlossen wird.
\item Nur haftungsbeschränkte Gesellschaften dürfen über ein eigenes Konto verfügen.
\item Transportunternehmen unterliegen den staatlichen Warenhandelsbestimmungen.
\end{enumerate}

\subsubsection{Privatgesellschaft}
\begin{enumerate}[(1)]
    \item Die Privatgesellschaft ist eine haftungsunbeschränkte Rechtsform.
    \item Gesellschaften dieser Rechtsform verfügen über kein separates Stammkapital.
    \item Die juristische Person dieser Gesellschaft ist der Gesellschafter selbst.
    \item Gesellschaften dieser Rechtsform können keine Gesellschaftsanteile verkaufen.
\end{enumerate}

\subsubsection{Handelsgesellschaft deutschen Rechts}\label{llc}
\begin{enumerate}[(1)]
    \item Die Handelsgesellschaft deutschen Rechts ist eine haftungsbeschränkte Rechtsform.
    \item Die juristische Person dieser Gesellschaft ist die Gesellschaft selbst.
    \item Es bestehen keine Leistungseinschränkungen für diese Rechtsform.
    \item Gesellschaften dieser Rechtsform können nicht weniger als Zehntelanteile verkaufen.
\end{enumerate}

\subsubsection{Transportgesellschaft}
\begin{enumerate}[(1)]
    \item Das Gründungsrecht und die juristische Person einer Transportgesellschaft entspricht der Handelsgesellschaft deutschen Rechts.
    \item Gesellschaften mit Transportrecht unterliegen nur teilweise den Auslandshandelsgebühren.
    \item Sie dürfen nur mit Dienstleistungen handeln, die mit dem Transport von Waren und Personen zusammenhängen. Sachleistungen dürfen sie nicht erbringen. 
\end{enumerate}

\subsubsection{Reichsgesellschaft}
\begin{enumerate}[(1)]
    \item Reichsgesellschaften sind Gesellschaften im staatlichen Besitz.
    \item Sie unterliegen keinen Leistungseinschränkungen.
    \item Jegliche Einnahmen gehören dem Kaiserreich.
    \item Gesellschaften dieser Rechtsform dürfen keine Anteile verkaufen.
\end{enumerate}

\subsubsection{Reichsgesellschaften gesonderten Besitzes}
\begin{enumerate}[(1)]
    \item Reichsgesellschaften gesonderten Besitzes sind Unternehmen gesonderten staatlichen Besitzes.
    \item Der Staat ist hier lediglich Hauptanteilhaber, jedoch können weitere Gesellschaftsanteile auch weiterverkauft werden.
    \item Reichsgesellschaften gesonderten Besitzes unterliegen keinen Leistungs- oder Anteilsbeschränkungen.
    \item Es können keine anteilsunbeschränkten Gesellschaften als Reichsgesellschaft gesonderten Besitzes registriert werden.
\end{enumerate}

\subsubsection{Aktiengesellschaft}\label{ext}
\begin{enumerate}[(1)]
    \item Eine Aktiengesellschaft verfügt über das Recht, eigene Unternehmensaktien auszustellen.
    \item Dies bedingt, dass das Unternehmen beliebig große Anteile an ihrem Unternehmen verkaufen dürfen.
    \item Diese Rechtsform ist nicht leistungsbeschränkt.
    \item Gesellschaften dieser Rechtsform sind haftungsbeschränkte Gesellschaften gemäß \ref{llc}, verfügen allerdings über erweiterte Anteilsrechte, wie in Absatz 1 beschrieben.
    \item Aktiengesellschaften sind nicht verpflichtet, einen, den Anteilen entsprechenden Gewinnanteil auszuzahlen, sondern können sich diesem mit einer regelmäßigen Auszahlung pro Aktie annähern.
    \item Diese Gesellschaften müssen durch einen Vorstand vertreten werden, in denen gewählte Vertreter der Unternehmensbereiche, sowie jeder Aktionär mit über 25.1\% über eine Stimme verfügt.
    \item Dieser Vorstand muss alle zwei Monate einen Vorstandsvorsitz wählen.
    \item Die Kosten der Unternehmensanteile werden durch die deutschen Börsen nur angenähert, nicht jedoch vorgeschrieben.
\end{enumerate}

\subsubsection{Transportgesellschaft auf Aktien}
Diese Rechtsform entspricht einer Transportgesellschaft mit erweiterten Anteilsrechten gemäß \ref{ext} Abs. 1f., 5ff.

\subsubsection{Organisation}
\begin{enumerate}[(1)]
    \item Eine Organisation ist eine nicht gewinnorientierte Organisation, die keine Privateinnahmen generieren darf.
    \item Sie dürfen unversteuerte Spenden entgegennehmen.
    \item Sie dürfen nur gemeinschaftsorientierte Dienstleistungen erbringen.
    \item Jegliche Einnahmen müssen zum Ende des Jahres entweder ausgegeben worden sein oder eindeutig als Rücklage für zukünftige Investitionen angegeben werden.
    \item Für Organisationen besteht Buchführungspflicht. 
\end{enumerate}

\subsubsection{Hoflieferanten}
\begin{enumerate}[(1)]
    \item Jegliches Haus und jeglicher Titel, der vom Kaiser das Recht ausgeschrieben bekommen hat, Hoflieferanten auszuwählen, darf nur bei diesen einkaufen, die sie ausgewählt haben.
    \item Hoflieferanten müssen mit einem offiziellen Schreiben vom Ausstellenden oder einem, von ihm ausgewählten Vertreter, ernannt werden.
    \item Diese dürfen das Wappen des Ernennenden in Verbindung mit ihrer Marke tragen, dürfen dies allerdings weder als eigene Marke, noch anderweitig als alleinstehendes Symbol ohne Garantiesatz genutzt werden.
    \item Der Garantiesatz verweist auf den Dienst des Unternehmens als Hoflieferant. Er beinhält sowohl den Ernennenden als auch die Kategorie des Unternehmens und den ausgeschriebenen Markennamen.
    \item Der Garantiesatz kann beliebig mit oder ohne Wappen vom Unternehmen auf Produkten, Produktbeschreibungen, Dokumenten und im Handelsregister vermerkt werden.
    \item Lieferanten des Kaiserhauses tragen den Rechtsformzusatz ``B.I.A.`` (``By imperial appointment``)
\end{enumerate}

\subsubsection{Handelsmarke}
\begin{enumerate}[(1)]
    \item Als Handelsmarke wird eine registrierte Handelsmarke bezeichnet.
    \item Diese Handelsmarke muss im Handelsregister eindeutig einer eingetragenen Gesellschaft zugeordnet werden.
    \item Diese Gesellschaft darf die Handelsmarke separat als Tochtergesellschaft aufführen, doch kann diese nicht ohne Verbindung zur Hauptgesellschaft als vollwertige Marke genannt werden.
    \item Der Handelsmarke kommen keine gewerblichen Fähigkeiten oder sonstige Kompetenzen einer Rechtsform zu.
    \item Sie müssen im Handelsregister anhand ihrer Eintragsnummer eindeutig als Eigentum der Hauptgesellschaft erkennbar sein.
\end{enumerate}

\newpage

\section{Börsengesetz (BörsenG)}
\localtableofcontents
\subsubsection{Börse}
\begin{enumerate}[(1)]
    \item Als Börse wird eine Einrichtung bezeichnet, die als Handelsplatz für standardisierte Handelsobjekte dient.
    \item Börsen müssen über ein Gebäude verfügen, das als obligatorischer Handelsplatz dient.
    \item Es darf kein Handel außerhalb dieses Gebäudes vorgenommen werden.
    \item Die Börse muss sich im vollständigen Eigentum einer Gesellschaft befinden und darf lediglich unter Genehmigung durch den deutschen Staat gegründet und geführt werden.
    \item Der Handelsplatz darf sich nicht durch Erwerb von Handelsobjekten am eigenen Handelsplatz bereichern.
\end{enumerate}

\subsubsection{Börsenaufsichtsbehörde}
\begin{enumerate}[(1)]
    \item Jegliche Transaktionen an der Börse unterliegen der Betreuung der Börsenaufsichtsbehörde.
    \item Sind diese Transaktionen rechtswidrig oder verstoßen gegen die geltenden Auflagen, so ist diese befugt, diese Transaktionen rückgängig zu machen und müssen die Staatsanwaltschaft informieren.
\end{enumerate}

\subsubsection{Handelslizenz}
\begin{enumerate}[(1)]
    \item Es darf nur an einem Handelsplatz handeln, wer für diesen eine Lizenz beantragt und durch die, für diesen zuständige Behörde, bewilligt bekommen hat.
    \item Es steht den Handelsplätzen frei, eine Gebühr für die Ausstellung oder Führung zu verlangen.
    \item ${^1}$Besteht ein grober und umfangreicher Verstoß gegen das Wirtschaftsrecht durch den Inhaber einer derartigen Lizenz, können ihm alle Handelslizenz per gerichtlichem Beschluss entzogen werden. ${^2}$Dies geht mit der Sperrung einer erneuten Erwerbung einher.
    \item Die erneute Erwerbung einer Handelslizenz auf deutschem Grund erfordert der Aufhebung der Sperrung durch den Reichsschatzmeister.
    \item Die ungenehmigte Ausstellung einer Handelslizenz ist strafbar und zieht eine Bußgeldstrafe von nicht unter 200 Kaisermark nach sich.
    \item Es steht einem Handelsplatz frei, unter Auszahlung der besessenen Handelsobjekte und Erstattung der letzten Lizenzgebühr, sofern eine regelmäßig erhoben wird, die Handelslizenz jederzeit zu entziehen.
    \item Wird eine einmalige Lizenzgebühr erhoben, so ist dies nicht möglich.
\end{enumerate}

\subsubsection{Handelsgebühren}
\begin{enumerate}[(1)]
    \item Es steht den zuständigen Börsen zu, eine Gebühr auf Transaktionen zu erheben.
    \item Diese Gebühren oder die Referenz zu den Gebühren müssen in der Handelslizenzvereinbarung spezifiziert werden.
\end{enumerate}

\subsubsection{Registrierung von Handelsobjekten}
\begin{enumerate}[(1)]
    \item Die Registrierung von Handelsobjekten darf nur durch anteilsunbeschränkte Gesellschaften mit Zweigstelle im Deutschen Reich und im Falle von Wertpapieren nur für die eigenen Wertpapiere veranlasst werden.
    \item Gesellschaften mit Hauptzweigstelle im Deutschen Reich dürfen ihre Wertpapiere nur an deutschen Börsen registrieren.
    \item Handelsobjekte müssen beim Reichsschatzamt angemeldet werden.
    \item Für die Anmeldung von Handelsobjekten, die nicht Unternehmensanteilen entsprechen, erfordert es einer zusätzlichen Zulassung durch das Reichsschatzamt.
\end{enumerate}

\subsubsection{Handel von Handelsobjekten}
\begin{enumerate}[(1)]
    \item Handelsobjekte dürfen nur an ausgewiesenen Handelsplätzen gehandelt werden.
    \item Lizenzierte Händler dürfen auch stellvertretend für Dritte handeln, unterliegen in diesem Falle jedoch der Treuhandspflicht.
    \item Der Handel darf nur zu den einheitlichen Handelszeiten geschehen, die durch das Reichsschatzamt beschlossen werden.
\end{enumerate}
\end{document}