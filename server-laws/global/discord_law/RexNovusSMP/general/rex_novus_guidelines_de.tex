\documentclass{article}
\usepackage[utf8]{inputenc}
\usepackage{enumerate}
\usepackage{ragged2e}
\usepackage{etoc}
\usepackage{chngcntr}
\usepackage{amsmath}

\renewcommand{\thesection}{}
\newcommand{\sent}[1]{$^{#1}$}
\counterwithout{subsection}{section}
\renewcommand{\thesubsection}{§\arabic{subsection}}


\title{Richtlinien des Rex Novus SMP}
\author{the.god.emperor}
\date{26. Februar 2024}

\begin{document}
\maketitle

\newpage
\topskip0pt
\vspace*{\fill}
\begin{Center}
\textbf{1. Fassung}
\vspace*{\fill}
\end{Center}
\newpage
Der Server und jegliche andere Dienstleistungen, die mit dem Rex Novus SMP Discord-Server verbunden sind,
laufen unter dem Namen `The God Complex Servers'. Dieser Name ist weder eine eingetragene Handelsmarke, noch
anderweitig registriert.\\
Das ungenehmigte Bereitstellen von Inhalten von The God Complex Servers, die explizit und einzigartig mit dem Server und
den Dienstleistungen in Verbindung stehen, ist eine Urheberrechtsverletzung und somit eine Straftat.
\\\\
\copyright\ The God Complex Servers 2024\\
\copyright\ God Emperor Networks 2024
\\\\
\textbf{Urheberrechtsinhaber: } 
the.god.emperor, $\text{ryurai\_jokhin\_pc}$
\\\\
\textbf{Projektregistrierung: }
DE000-MC3
\newpage
\tableofcontents
\newpage
\section{Discord-Server}
\subsection{Gültigkeit}
\begin{enumerate}[(1)]
	\item Tritt man diesem Server bei, akzeptiert man die hier festgesetzten Bestimmungen.
	\item Die Server-Administration behält sich das Recht vor, diese Regeln jederzeit zu ändern.
	\item \textit{weggefallen}
	\item Die Regelungen treten erst in Kraft, sobald sie in dem Textkanal für Regeln veröffentlicht werden. Dementsprechend gilt keine Regelung rückwirkend.
	\item Mangelnde oder fehlerhafte Kenntnisse der Serverbestimmungen gewähren keine rechtliche Immunität, da das Informieren über die aktuelle Gesetzeslage des Servers Pflicht ist.
	\item Ebenfalls muss man sich bei Unklarheiten an den Zuständigen wenden (\ref{members}).
	\item Verstößt eine der Bestimmungen gegen die Verfassung des Landes einer betroffenen Person, so wird für diese lediglich die rechtswidrige Passage aufgehoben\footnote{Salvatorische Klausel}.
	\item $^{1}$Die Urheberrechte werden wie in der Präambel beschrieben verwaltet. $^{2}$Urheberrechtsverletzungen werden bei VIRTSTAX gemeldet und führen zu einem permanenten Ausschluss aus allen, von
	VIRTSTAX verwalteten Domänen und Inhaberschaften.
\end{enumerate}

\subsection{Chatverhalten}\label{verhalten}
\begin{enumerate}[(1)]
	\item $^{1}$Spam, Beleidigungen, Drohungen und Provokationen gegen andere Spieler sind verboten und werden zu Sanktionen führen. $^{2}$Als Spammen wird das Verschicken von mehreren Nachrichten in einem geringen Zeitintervall bezeichnet. Ab fünf Nachrichten in kürzester Zeit kann es Konsequenzen nach sich ziehen. $^{3}$Das unerlaubte Nutzen von Pings ist aufgrund seiner provokanten Natur ebenfalls untersagt.
	\item $^{1}$Rassistische, politische und ethisch inakzeptable Inhalte (Äußerungen, Bilder, etc.) sind verboten und führen zu einem permanenten Ausschluss auf dem gesamten Discord-Server. $^{2}$Dies gilt auch für pornografische Inhalte. $^{3}$Auch gilt dies für die absichtliche Anordnung von Reaktionen der Kategorie "Regional Indicator" zu derartigen Äußerungen.\footnote{Message Reaction Act vom 19. Juli 2023} $^{4}$Das Senden von GIFs ist verboten.\footnote{Channel Usage Act vom 12. Februar 2023} $^{5}$ Symbole und Verharmlosungen \textbf{jeglicher} politisch extremistischer Taten oder Gruppierungen sind untersagt.\footnote{Dies gilt auch für linksextreme Gruppierungen; Anti-Extremism Act vom 20. Februar 2024}
	\item Für pornografische Zwecke explizit angelegte Textkanäle sind von Absatz 2 Satz 2 ausgeschlossen.
	\item Für Memes vorgesehene Textkanäle sind von Absatz 2 Satz 4 ausgenommen.
	\item $^{1}$ Weiterhin dürfen Kanäle nur für den Zweck verwendet werden, für den sie vorgesehen sind. Bestehen Unklarheiten über den Verwendungszweck, so muss man sich vor dem Verfassen einer Nachricht an den Support (\ref{support}) wenden. 
		  $^{2}$ Ebenfalls muss man sich an die Beschreibung der Kanäle und die korrekte Verwendung derer Funktionen halten.\footnote{Channel Usage Act Amendment vom 15. März 2024}
\end{enumerate}

\subsection{Teammitglieder}\label{members}
\begin{enumerate}[(1)]
	\item Anweisungen von befehlsbefugten Teammitgliedern sind verbindlich und stets zu befolgen.
	\item Teammitglieder werden durch eine Rangbezeichnung, beziehungsweise Rolle gekennzeichnet.
	\item Zu den befehlsbefugten Teammitgliedern gehören:
	\begin{enumerate}
		\item Der CEO
		\item Der CLO
		\item Der CSO\footnote{Domain Security Act vom 29. März 2024}
		\item Der CTO
		\item Die Investoren
		\item Der COO
		\item Der CAO
		\item Die Techniker
		\item Die DC-Moderatoren
		\item Die MC-Moderatoren
	\end{enumerate}
	\item Die befehlsbefugten Mitglieder dürfen nur Befehle in ihrem Zuständigkeitsbereich erteilen.
\end{enumerate}

\subsection{Verhalten im Sprachchat}
\begin{enumerate}[(1)]
	\item Abgesehen von den Regelungen aus \ref{verhalten} gelten für Sprachkanäle zusätzlich folgende Bestimmungen.
	\item Die Nutzung von Stimmenverzerrern und Soundboards ist erlaubt, sofern die Teammitglieder keine Einwände erheben.
	\item Es ist nicht gestattet, Personen ohne deren Einverständnis aufzuzeichnen.
	\item Fehlverhalten in Sprachkanälen rechtfertigt eine temporäre serverweite Stummschaltung.
\end{enumerate}

\subsection{Zweitaccounts}
Man muss Zweitaccounts als solche markieren. Hierfür muss man sein Serverprofil derartig bearbeiten, dass es jedem möglich ist, anhand dieses Profils nachvollziehen zu können, um wessen Zweitaccount es sich handelt.

\subsection{Strafmaß}
\begin{enumerate}[(1)]
	\item Es wird im Allgemeinen zwischen drei Strafen differenziert:
	\begin{enumerate}[1.]
		\item Eine Verwarnung ist eine Vorstufe zu tatsächlichen Strafmaßnahmen. Jedes Mitglied bekommt für minder schwere Verstöße eine Verwarnung.
		\item Ein Timeout bezeichnet einen temporären Ausschluss vom Server.
		\item Ein permanenter Bann ist ein unwiderruflicher, zeitlich unbegrenzter Ausschluss vom Server.
	\end{enumerate}
	\item Das Strafmaß wird selten nach der Schwere des Verstoßes, sondern zumeist nach folgender Vorgabe bemessen:
	\begin{enumerate}[1.]
		\item Erste Verwarnung
		\item Zweite Verwarnung
		\item 24-Stunden-Timeout
		\item Einwöchiges Timeout
		\item Ein-Monat-Timeout
		\item 1-Jahr-Timeout
		\item Permanenter Bann
	\end{enumerate}
	\item Jede Strafe muss ausnahmslos widerrufen werden, sofern die bestrafte Person die Unrechtmäßigkeit der Strafe nachweisen kann.
	\item Unrechtmäßigkeit liegt vor, sofern es sich bei der fraglichen Tat um keinen Verstoß seitens des Bestraften handelt, bei der Bestrafung gegen Absatz 1 und 2 verstoßen wurde oder die Tat fälschlicherweise als Straftat besonderer Schwere eingestuft wurde.
	\item Handelt es sich bei der Tat um einen schweren Verstoß, so kann je nach Schwere des Verstoßes ein sofortiges Timeout bishin zu einem sofortigen permanenten Bann erfolgen. Die Einschätzung der Schwere unterliegt dem Zuständigen, muss jedoch nachvollziehbar sein. 
	\item Sofern Zweifel bestehen, kann das Urteil von dem CLO oder durch eine qualifizierte Mehrheit durch den Vorstand aufgehoben und rückgängig gemacht oder in eine andere Strafe umgewandelt werden.
	\item Jegliche rechtswidrigen Nachrichten müssen in Form eines Screenshots dauerhaft samt der Beschreibung des Vorfalls im Strafregister zwischengespeichert werden\footnote{Dauerhaftigkeit und Eintragungen im Strafregister seit dem Criminal Records Act vom 15. März 2024}, damit im Zweifelsfall die Rechtswidrigkeit angefochten werden kann\footnote{Dies begründet sich in vergangenen Schwierigkeiten, die Rechtswidrigkeit von Aussagen im Nachhinein zu bewerten.}, danach kann man bei dem Vorstand eine Löschung beantragen, die jedoch mit einer qualifizierten Mehrheit bestätigt werden muss.
	\item Nach jedem Timeout steigt die Schwere der Straftat so, dass der erste Verstoß nach einem Timeout gemäß Strafhierarchie aus Absatz 2 Nummer 1 - 7 aufgrund seiner Schwere im Verhältnis zur vorherigen Grundstrafe\footnote{Die erste Strafe nach einem Timeout, beziehungsweise die insgesamt erste Strafe.} erhöht wird.
	\item Von Absatz 8 sind permanente Banns teils ausgeschlossen. Diese sollen lediglich im Falle äußerster Schwere oder beim Zutreffen von Absatz 8 verhängt werden, sofern keine akute Verhaltensbesserung vorliegt und der Vorstand in einem einfachen Mehrheitsbeschluss dafür stimmt.
	\item Es ist nicht gestattet, entgegen der ausdrücklichen Erlaubnis des CLO, Personen auf dem Server zu entbannen. Dies gilt als strafbar und wird ungeachtet der Position in der Strafhierarchie nach Interpretation gemäß Absatz 8 mit einem Timeout bestraft. Die widerrechtlich entbannte Person ist zudem umgehend gebannt zu werden.
	\item $^{1}$Verstöße besonderer Schwere durch Mitglieder eines Unternehmensbereichs resultieren in einem permantenten Ausschluss aus diesem und weiteren Unternehmensbereichen. $^{2}$Diese Regelung betrifft weder the.god.emperor noch ryurai\_jokhin\_pc oder deren Zweitaccounts.
	\item Mittäterschaft, Beihilfe, Anstiftung und Versuch werden äquivalent betraft.
\end{enumerate}

\subsection{Inhaber}
\begin{enumerate}[(1)]
	\item Der Begriff des Inhabers entspricht dem Begriff des Teilhabers gemäß §3 Abs. 1 TeilhB.
	\item Jegliche Beschlüsse der Inhaberschaft erfordern eine qualifizierte Mehrheit, es sei denn, die Teilhaberschaftsbestimmungen widersprechen dem.
	\item Auf Anordnung der Inhaber hin kann ein Adminkonzil einberufen werden, bei welchem die Admins und Moderatoren, die keine Teilhaber sind, je eine Stimme bekommen und die Entscheidung für die Teilhaberversammlung übernehmen.
\end{enumerate}

\subsection{Vorstandsvorsitz}
\begin{enumerate}[(1)]
	\item Den Vorstandsvorsitz gemäß §8 Abs. 2 TeilhB nimmt der Chief executive officer ein. Diese Rangbezeichnung wird mit `CEO' abgekürzt.
	\item Er wird ständig durch die beiden Gründer des Servers belegt.
	\item Die Gründer sind the.god.emperor und ryurai\_jokhin\_pc (705020938810294392, 744269509770346626).
	\item Sollten beide Gründer aus dem Vorstand austreten, so muss durch den Vorstand mittels einer einfachen Minderheit ein neuer Vorstandsvorsitz gewählt werden.
\end{enumerate}

\subsection{Unternehmensbereiche}
\begin{enumerate}[(1)]
	\item Die nachfolgenden Unternehmensbereiche gemäß §9 TeilhB bestehen auf dem Server:
		\begin{enumerate}[1.]
			\item Buchhaltung (Accounting)
			\item Rechtsabteilung (Legal Department)
			\item Moderation
			\item Technische Abteilung (Technical Department)
			\item Sicherheitsabteilung (Security Department)\footnote{Domain Security Act vom 29. März 2024}
		\end{enumerate}
	\item An den Vorsitz dieser Abteilungen wird die dementsprechende Rolle vergeben.
	\item In den Unternehmensbereichen dürfen nur Personen angestellt werden, die auf dem jeweiligen Gebiet über ausreichende Kenntnisse und Fähigkeiten verfügen.
\end{enumerate}

\subsection{Buchhaltung}
\begin{enumerate}[(1)]
	\item Die Buchhaltung ist dafür verantwortlich, die Serverfinanzen im Blick zu behalten.
	\item Dies bezieht auch die Verwaltung der Unternehmensanteile ein.
	\item Der Chief accounting officer (CAO) ist der Transaktionsberechtigte gemäß §7 TeilhB und der Abteilungsleiter.
\end{enumerate}

\subsection{Rechtsabteilung}
\begin{enumerate}[(1)]
	\item In den Aufgabenbereich der Rechtsabteilung fallen:
	\begin{enumerate}[1.]
		\item Rechtliche Fragen zur Serververfassung
		\item Anfragen rechtlichen Beistands
		\item Anfechtungen servergerichtlicher und sonstiger Urteile
		\item Verfassungsbeschwerden
		\item Gesetzesvorschläge
		\item Behandlung von Verstößen gegen Serverrichtlinien
		\item Behandlung von Verstößen gegen die Teilhaberschaftsbestimmungen
		\item Prüfung der Urteile der Moderation
	\end{enumerate}
	\item Gesetzesvorschläge, die von der Änderung oder Abschaffung bereits bestehender Gesetzer sprechen, gelten als Verfassungsbeschwerden.
	\item Sowohl Gesetzesverstöße und Berufung, als auch Verfassungsbeschwerden gelten als ausreichende Begründung für einen vollwertigen Prozess.
	\item Der Vorsitzende der Abteilung ist der Chief legal officer (CLO).
	\item In seiner Funktion als richtendes Organ wird die Rechtsabteilung als `Supreme Court' (Oberster Gerichtshof) bezeichnet.
	\item Mitglieder des Obersten Servergerichts tragen die Amtsbezeichnung `Supreme Judge' (Oberster Richter). Außerhalb ihrer richtenden Tätigkeit werden sie als `Server lawyer' (Serveranwalt) bezeichnet.
	\item Dem Hohen Gericht steht der `President of the Supreme Court' (Präsident am Obersten Gerichtshof) vor.
	\item Der Präsident am Obersten Gerichtshof ist der CLO.\@
	\item Sofern keine neuen, ausreichenden Beweise vorliegen, trifft Absatz 5 nicht zu.
	\item Urteile durch die Moderation und der Oberste Gerichtshof müssen von dem Präsidenten am Obersten Gerichtshof bestätigt werden und können daher abgewiesen werden.
	\item Die Abweisung von Urteilen muss gerechtfertigt sein und begründet werden.
	\item Die Rechtsabteilung ist in der Lage, mittels Urteilen die Interpretation einzelner Gesetze zu bestimmen, sofern sich dies in einem nachvollziehbaren Rahmen bewegt.\footnote{Prejudice Act vom 17. März 2024}
\end{enumerate}

\subsection{Moderation}\label{support}
\begin{enumerate}[(1)]
	\item Jegliche Fragen bezüglich des Discord- und Minecraft-Servers, die nicht in den rechtlichen Bereich fallen, fallen in den Aufgabenbereich der Moderation. Bestehen Unklarheiten bezüglich des Zuständigkeitsbereichs, sollte man sich ebenfalls an die Moderation wenden.
	\item Die Moderationsabteilung dient zur Kontrolle der Einhaltung der Serverrichtlinien.
	\item Dies bedingt, dass sie in der Lage sind, ohne eine Genehmigung Strafen zu vollziehen, die allerdings an die Rechtsabteilung mitsamt des Kontexts weitergeleitet werden und endgültig bestätigt werden müssen.
	\item Den Vorstand der Moderation hat der Chief operating officer (COO), welcher auch administrative Aufgaben erfüllt.
	\item Von Absatz 3 ausgeschlossen sind jegliche Verstöße gegen die Teilhaberschaftsbestimmungen, da diese von der Rechtsabteilung und VIRTSTAX behandelt werden.
	\item Mitglieder der Moderation werden je nach Zuständigkeitsbereich als `MC Moderator' (Minecraft-Moderator) oder als `DC Moderator' (Discord-Moderator) aufgeführt.
	\item Moderatoren dürfen Urteile gemäß Absatz 3 nur innerhalb ihres Zuständigkeitsbereichs vollziehen.
	\item Minecraft-Moderatoren haben im Gegensatz zu Discord-Moderatoren zwar keine Discord-Administratorberechtigungen, allerdings Operator-Berechtigungen auf dem Minecraft-Server.
\end{enumerate}

\subsection{Technische Abteilung}
\begin{enumerate}[(1)]
	\item Die technische Abteilung dient der Wartung aller Server und Dienstleistungen, die The God Complex Servers unterstehen.
	\item Dies bezieht auch die Realisierung neuer Funktionalitäten auf diesen Servern ein.
	\item Den Vorsitz hat der Chief technology officer (CTO).
	\item Mitglieder dieser Abteilung werden als `Technicians' (Techniker) bezeichnet.
\end{enumerate}

\subsection{Sicherheitsabteilung}\footnote{Domain Security Act vom 29. März 2024}
\begin{enumerate}[(1)]
	\item Die Sicherheitsabteilung dient der Sicherung der Domäneneigentümer und der Überwachung der Nutzeraktivitäten.
	\item Jegliche Mittel zur Realisierung ihrer Aufgaben müssen nur von den amtierenden CEOs bewilligt werden.
	\item Den Vorsitz hat der Chief security officer (CSO).
	\item Die Vorgänge innerhalb dieser Abteilung unterstehen strengster Geheimhaltung und dürfen nur unter Erlaubnis der CEOs veröffentlicht werden.
\end{enumerate}

\section{Minecraft-Server}
\subsection{Grundsätzliche Regeln}
\begin{enumerate}[(1)]
	\item Das Minecraft-Serverrecht untersteht dem Discordserverrecht.
	\item Es ist verboten, auf Methoden zurückzugreifen, die gegenüber anderen Spielern, ungeachtet dessen, ob sie die Methode einsetzen oder nicht, einen Vorteil verschaffen, die allgemein nicht als gerecht anerkannt werden.
	\item Jegliches, von derartigen Methoden nicht betroffenes Verhalten, ist nicht strafbar.
	\item Von Absatz 3 ist der Bau von Konstruktionen, die dem Zwecke dienen, die Serverleistung zu verringern, ausgeschlossen.
	\item Auf dem Minecraft-Server muss man sich den jeweiligen Regeln des Discord-Servers entsprechend verhalten.
	\item Das generelle Serverrecht unterscheidet nicht zwischen Fraktionen, weshalb diese lediglich eine interne Organisation darstellt, die keine Deckung durch jegliche serverweite Gesetze erfährt und somit Verbrechen gegen diese im Einzelnen kein Gegenstand serverweiter Urteile sein können.
	\item Absatz 3 tritt nur ein, wenn sich die Verstöße nicht gegen die Regeln des Discordservers oder Abs. 1f. richten.
\end{enumerate}

\subsection{Bürgschaft}\footnote{Guarantee Act vom 30. Juni 2023}
\begin{enumerate}[(1)]
	\item Es dürfen nur Personen auf die Whitelist gesetzt werden, für die jemand nachweislich bürgt.
	\item Begeht eine Person einen Verstoß besonderer Schwere, so werden die bestehenden Bürgschaftsverhältnisse überprüft und können bei Bedarf aufgelöst werden\footnote{Abs. 2f.: Guarantee Act Amendment vom 20. Februar 2024}.
	\item Wird ein Bürgender vom Server gebannt, so werden alle Bürgschaftsverhältnisse mit diesem terminiert.
	\item Bürgschaften kann man nicht nachträglich zurückziehen.
	\item Die Administration ist von Abs. 2f. ausgenommen.
	\item $^{1}$ Man muss einen Antrag zur Aufnahme auf die Whitelist stellen.$^{2}$ Dies muss über den Support-Kanal geschehen.$^{3}$ Anderweitig gestellte Anträge sind ausnahmslos ungültig.\footnote{Guarantee Act Amendment II vom 06. Mai 2024}
\end{enumerate}

\subsection{Rechtliche Separation}
\begin{enumerate}[(1)]
	\item Das Serverrecht ist eindeutig von der internen Rechtssituation auf den Ablegern des Rex Novus SMP zu unterscheiden.
	\item Als internes Recht werden nicht von der Inhaberschaft in ihrer Funktion als Teilhaberversammlung anerkannte Verfassungen und Regeln, wie beispielsweise fraktionseigene Gesetzestexte bezeichnet.
	\item Die Einsicht und Nutzung von, internen Regelungen übergeordneten, Serverdaten und sonstigen, nur für die Administration zugänglichen Informationen, wie Spielerdaten oder Logs, darf nicht zur Beweisführung für Prozesse und ähnliches dienen, die nicht von dem Obersten Servergericht in dessen Funktion vollzogen werden\footnote{So dürfen beispielsweise In-Game-Morde nicht über Logs nachgewiesen werden}.
	\item Dies gilt nicht in dem Bereich um den Welteinstiegspunkt herum, einschließlich dessen selbst. Dieser ist domänenrechtlich verwaltetes Gebiet, weshalb dementsprechend die Domänenrechtsprechung gilt.\footnote{Spawn Protection Act vom 16. Mai 2024}
\end{enumerate}

\subsection{Serverweite Haftbefehle und Haftzwang}\footnote{Law Enforcement Act vom 27. Mai 2024}
\begin{enumerate}[(1)]
	\item Hat eine Person Verbrechen gegen die geschriebene Verfassung einer Nation oder eines Staatenbundes begangen, so kann diese bei der Rechtsabteilung des Servers einen Antrag
	auf einen Haftbefehl stellen.
	\item Die Rechtsabteilung muss daraufhin für die Person eine Anordnung zum Erscheinen vor dem zuständigen Gericht ausstellen, welcher die Person nachkommen muss.
	\item Bei dem Prozess muss ein, nicht im Prozess beteiligtes Mitglied der Rechtsabteilung als Zuschauer teilnehmen.
	\item Sieht der Serveranwalt einen Haftbefehl auf Grundlage des Prozesses als zu genüge gerechtfertigt, so muss überprüft werden, wo sich die fragliche Person befand, als der Haftbefehl beantragt wurde.
	\item $^{1}$Befand sich die Person zu dieser Zeit im Inland oder der Außenwelt, so kann umstandslos weiterverfahren werden. $^{2}$Unter dem Umstand, dass die Person sich im, von einem Servermitglied besetzten,
	Ausland aufhielt, muss der Staat, in dem sich die Person aufhielt, der Auslieferung der Person zustimmen. $^{3}$Befand sich die Person auf anderweitigem Territorium, so ist per Zufallsverfahren zu entscheiden, ob die Nation der Auslieferung zustimmt.
	\item Befand sich die Person im Inland, der Außenwelt oder wurde der Auslieferung zugestimmt, so wird per Zufallsverfahren entschieden, ob die Verhaftung erfolgreich verlief.
	\item Wurde der Auslieferung nicht zugestimmt, so steht der Haftbefehl aus, bis der Auslieferung zugestimmt wurde. Dies hat nur Wirkung, sofern sich die fragliche Person im jeweiligen Land aufhält. Andernfalls muss die Genehmigung von dem neuen Aufenthaltsland eingeholt werden. Verlief die Verhaftung nicht erfolgreich, so verliert der Haftbefehl seine Wirkung.
	\item Sofern die Haft umstandslos erfolgte, wird ein Haftzwang erwirkt. Hierbei handelt es sich um die serverrechtliche Aufforderung, sich der Strafe, zu stellen.
	\item Geht eine Person der Aufforderung nicht nach, hat dies serverrechtliche Konsequenzen und die Verhaftung wird mittels Operatorbefehlen erzwungen.
\end{enumerate}

\subsection{Ausbruch aus serverrechtlich verhängtem Haftzwang}\footnote{Law Enforcement Act vom 27. Mai 2024}
\begin{enumerate}[(1)]
	\item Ist man der Haftaufforderung nachgekommen, so darf man, nachdem man mindestens eine halbe Stunde und höchstens die Hälfte der Strafe die Haft abgesessen hat,
	bei der Rechtsabteilung des Servers einen Antrag auf Ausbruch stellen.
	\item Die Rechtsabteilung ermittelt anschließend nach Zufallsverfahren, ob der Ausbruch erfolgen darf.
	\item Wenn er erfolgen darf, muss die Person von der Serveradministration befreit werden.
\end{enumerate}

\subsection{Ausnahmeregelung zu antragsbasierten Eingriffen}\footnote{Law Enforcement Act vom 27. Mai 2024}
Die Serveradministration darf keine antragsbasierten Eingriffe in die projektinterne Umgebung unternehmen, sofern der Antrag oder der Bearbeitungs- oder Bewilligungsprozess
fehlerhaft oder regelwidrig erfolgte.

\section{Fraktionsrecht}
\subsection{Fraktionen}
\begin{enumerate}[(1)]
	\item Als Fraktion gilt jegliche Gruppierung mit mindestens einem Spieler.
	\item Der Begriff des Spielers ist nicht mit dem Begriff des Minecraft-Kontos synonym und rechtfertigt daher keine weiteren Gebietsansprüche, wenn es sich bei dem anderen Konto um ein Zweitkonto der Person handelt.
	\item Fraktionen haben das Anrecht auf eine eigene Kategorie, in der sie jegliche Kanäle auf Anfrage hin einrichten können.
	\item Die Moderation kann derartige Anliegen ablehnen, sofern diese keinen gerechtfertigten Grund für eine Einrichtung feststellen können.
	\item Aufgrund der Tätigkeit und Aufgaben der Moderation ist diese jederzeit berechtigt, in die Kanäle einzusehen, um Verstöße gegen das geltende Recht erkennen zu können.
	\item Jeder Fraktion wird eine eigene Rolle zugesichert, die zum Zweck hat, dass diese Kanäle nicht durch Mitglieder anderer Fraktionen eingesehen werden können.
    \item Die Teilnahme am Server erfordert die Gründung oder den Beitritt zu einer, von der Administration genehmigten Fraktion\footnote{Abs. 7-10, Abs. 15-20: Faction Establishment Act vom 20. Februar 2024}.
    \item Um eine anerkannte Fraktion zu gründen, muss man der Administration folgende Informationen mitteilen:
    \begin{enumerate}[1.]
        \item Das Fraktionsgebiet
        \item Den Namen des Staats
        \item Die Insignien des Staats
        \item Alle Staatsmitglieder und ihre rollenspieltechnischen Namen
        \item Der erste Anführer des Staats
    \end{enumerate}
	\item Verliert ein bereits anerkannter Staat eines der genannten Kriterien, so hat dieser zwei Monate Zeit, diese wiederzuerlangen.
    \item Schafft der Staat es nicht, rechtzeitig das Kriterium zurückzuerlangen, so wird er offiziell aufgelöst.
	\item Das Serverrecht legt keine Nachfolgeregelungen fest und bewertet nicht die Legitimität eines Nachfolgeanspruchs.\footnote{Abs. 11-14: Faction Establishment Act Amendment vom 26. Februar 2024}
	\item Es ist möglich, die Regierungsnachfolge durch interne Ereignisse ohne Einwilligung des derzeitigen Staatsoberhaupts zu ändern.
	\item Revolutionskriege unterliegen dem Kriegsrecht.
	\item Rebellierende und revoltierende Fraktionen dürfen jegliche Territorien als anfängliches Fraktionsgebiet wählen, die ihnen vom vorigen Staatsoberhaupt zuerkannt wurden. In diesem Falle ist dieses Gebiet rechtskräftig.
    \item ${^1}$Das Gebiet und die Staatsinsignien, sowie jegliche anderen Eigentümer und Besitztümer des Staats werden nicht von den serverweiten Richtlinien beschützt und können daher von anderen Staaten beansprucht werden. ${^2}$ Dies gilt nicht für obligatorische Insignien.\footnote{Faction Establishment Act Amendment III vom 27. Juni 2024}
    \item Die Staaten dürfen sich an jeglichen reellen Staaten orientieren, müssen dies allerdings nicht.
    \item Es ist nicht erlaubt, Insignien oder Namen zu wählen, die eindeutig einem historischen oder aktuellen Staat zuordbar sind, ohne diesen zu repräsentieren.	
    \item Außerhalb eines serverrechtlich anerkannten Kriegszustands und eines aktiv von einer Schlacht betroffenen Gebiets, darf man keinen fremden Besitz ungenehmigt zerstören oder bebauen.\footnote{War Regulatory Act Amendment III vom 17. Juni 2024} \footnote{Property Protection Act vom 08. Juli 2024}  
    \item Jede staatliche Verfassung gilt als anerkannt, sobald sie öffentlich zugänglich ist und der Zugang auf dem Discord-Server im jeweiligen Textkanal freigegeben wurde.
    \item Es ist nicht möglich, serverrechtlich gegen staatliche Verfassungen und Abkommen vorzugehen, sofern diese dem Serverrecht nicht widersprechen.
    \item Man kann fraktionsinterne Urteile serverrechtlich anfechten, sofern diese keine rechtfertigbare schriftliche Grundlage haben.\footnote{Faction Establishment Act Amendment II vom 24. März 2024}
	\item ${^1}$Ein Staat darf bis exakt zwei Monate nach Gründung von anderen Staaten nicht angegriffen werden. ${^2}$Sobald der Staat einen anderen Staat angreift verfällt diese Immunität. ${^3}$Unter diese Regelung fallen nur neu gegründete Staaten, die aus keinem anderen Staat hervortraten.\footnote{Faction Establishment Act Amendment IV of 08 July 2024}
\end{enumerate}

\subsection{Fraktionsinsignien}\footnote{Faction Establishment Act vom 20. Februar 2024}
\begin{enumerate}[(1)]
    \item Jegliche Fraktionsinsignien müssen einzigartig und deutlich von den Fraktionsinsignien anderer Fraktionen unterscheidbar sein.
    \item Die Fraktionsinsignien dürfen gegen keine Verhaltensregeln verstoßen und müssen klar erkenntlich dem Zwecke der ernsthaften Repräsentation der Fraktion dienen.
    \item Flaggen und Insignien reeller Staaten bilden keine Ausnahme zu Abs. 2.
    \item Zu den obligatorischen Fraktionsinsignien gehören:
    \begin{enumerate}[1.]
        \item Die Flagge der Fraktion
        \item Das Wappen der Fraktion
        \item Die Uniformen der Fraktion
    \end{enumerate}
\end{enumerate}

\subsection{Territoriale Expansion und Kriegsrecht}
\begin{enumerate}[(1)]
    \item Territoriale Expansion muss von der Administration anerkannt werden und erfordert entweder Abkommen zwischen oder Zusammenschlüsse von Staaten oder eine Eroberung des Staats.
    \item Die Eroberung eines Staats erfordert Kriegsführung, die dem geltenden serverweiten Kriegsrecht entspricht.
    \item Die Serveradministration kann bei Einstimmigkeit Ausnahmen zum Kriegsrecht beschließen. Es obliegt zudem der Serveradministration, zu beurteilen, ob gegen die Regeln, die ein subjektives Urteil erfordern, verstoßen wurde. % Admins oder Rechtsabteilung?
    \item Man muss im Krieg die Uniformen der eigenen Nation tragen.\footnote{War Regulatory Act Amendment IV vom 27. Juni 2024}
\end{enumerate}

\subsection{Kriegsbeginn}\footnote{War Regulatory Act vom 02. Juni 2023}
\begin{enumerate}[(1)]
	\item \sent{1}Eine Nation kann monatlich nur einen Krieg gegen eine andere Nation führen. \sent{2}Dies gilt nur bei Angriffs- und Konterkriegen.
	\item Unbesetzte Staaten lassen sich durch Nutzung der Kriegspunkte besetzen.
	\item Man muss dem Staat, den man angreifen möchte, öffentlich den Krieg erklären. % Frist bis tatsächlichen Krieg???
	\item Bei Kriegserklärung müssen neutrale Parteien ermittelt werden, die den Krieg beobachten und auf seine Regelkonformität prüfen.
	\item Können keine angemessenen neutralen Beobachter ernannt werden, so müssen indirekt verwickelte Parteien\footnote{Beispielsweise Bündnisparteien} auf der Seite des Angreifers und auf der Seite des Verteidigers aus dem Krieg austreten und als Beobachter fungieren.
	\item Gegen Beobachterentscheidungen kann man vor der Rechtsabteilung des Servers Einspruch einlegen.\footnote{Abs. 6-9: War Regulatory Act Amendment vom 21. Februar 2024}
	\item Befangene und manipulierte Beobachterentscheidungen sind regelwidrig.
	\item Hat ein Beobachter einmal nachweislich nicht neutral geurteilt, kann er nicht mehr zum Beobachter ernannt werden.
	\item Man kann Einspruch gegen die Wahl der Beobachter einlegen.
	\item Zu Beginn müssen beide Parteien Bedingungen aushandeln, die zum Ende des Kriegs eintreten wenn der Verteidiger gewinnen sollte. % Abschaffen??
	\item Können die Parteien sich innerhalb einer Woche nach Kriegserklärung nicht einigen, darf der Verteidiger die Bedingungen diktieren.
	\item Die Bedingungen müssen immer verhältnismäßig sein.
	\item Hat der Verteidiger unentschuldigt bei allen Verhandlungen innerhalb dieser Woche gefehlt, beginnt der Krieg ohne Verhandlungen. In diesem Falle kann der Verteidiger keine Bedingungen einklagen.
	\item Man darf keine unbesetzten Länder während eines laufenden Kriegs einnehmen.
	\item Man darf zu jeder Zeit unbesetzte Gebiete serverrechtlich ohne militärische Aktion besetzen, sofern man einen rechtmäßigen Anspruch nachweisen kann, da das Gebiet dem reellen Staat gehört.
	\item Es gilt, dass bei der Betrachtung von Regionen, Städten und ähnlichem in Hinsicht auf das Kriegsrecht vom Stand des Kriegsbeginns ausgegangen wird.\footnote{Ab hier: War Regulatory Act Amendment II vom 28. Februar 2024}
	\item Es gelten nur Grenzen, die öffentlich bekanntgemacht wurden.
\end{enumerate}

\subsection{Kriegspunkte}\footnote{War Regulatory Act vom 02. Juni 2023}
\begin{enumerate}[(1)]
	\item Kriegspunkte sind eine Einheit, um zu ermitteln, wie viele Kriege von einem Staat pro Monat gegen unbesetzte Länder geführt werden können.\footnote{Sie sind eine Methode, um zu verhindern, dass man gegen riesige Länder genauso oft kämpfen kann, wie gegen winzige Länder.}
	\item \sent{1}Jeden Monat erhält eine Fraktion drei Kriegspunkte. \sent{2}Am Ende des Monats wird die Zahl der Kriegspunkte nicht zurückgesetzt. % Anpassung der Zahl der Kriegspunkte??
	\item Es ist möglich, nur bestimme Regionen eines unbesetzten Gebiets einzunehmen. In diesem Falle muss man nur dem Anteil an der Gesamtfläche entsprechende Kriegspunkte ausgeben.\footnote{War Regulatory Act Amendment II vom 27. Februar 2024}
	\item Besteht der Verdacht, dass mehrere souveräne Staaten hauptsächlich deshalb souverän sind, um mehr Kriegspunkte zu erhalten, muss die Serveradministration diese souveränen Staaten im Sinne der Kriegspunkte als eine einzige Fraktion behandeln und demnach nur die einfache Kriegspunktezahl vergeben. % Frage der Rechtsabteilung?
\end{enumerate}

\subsection{Kriegsführung}\footnote{War Regulatory Act vom 02. Juni 2023}
\begin{enumerate}[(1)]
	\item Man kann nur Länder angreifen, die direkt an eine Durchgangspassage angrenzen
	\item Eine Durchgangspassage bezeichnet einen Weg, den die eigenen Truppen durch serverrechtlich besetzte Gebiete gerechtfertigt nutzen können.
	\item Als gerechtfertigt nutzbar gilt eine Durchgangspassage erst dann, wenn sie durchgängig entweder zum eigenen Gebiet oder zum Gebiet eines Staates gehört, welcher nachweislich die Erlaubnis zur Durchquerung gibt.
	\item \sent{1}Man darf ebenfalls das Gebiet eines Staats gerechtfertigt betreten, dem man den Krieg erklärt hat. \sent{2}Dies gilt nur für besetzte Gebiete.
\end{enumerate}

\subsection{Besatzung}\footnote{War Regulatory Act vom 02. Juni 2023}
\begin{enumerate}[(1)]
	\item Als besetzt gilt jedes Gebiet, das man mit Truppen erfolgreich einnehmen konnte.
	\item Wurde dieses nicht verteidigt, gilt es ebenfalls als eingenommen.
	\item Ein besetztes Gebiet gilt erst dann als serverrechtlich besetzt, wenn es von der Administration als rechtmäßig erlangtes Gebiet anerkannt und dem Besatzer zugesprochen wurde.
\end{enumerate}

\subsection{Sieg}\footnote{War Regulatory Act vom 02. Juni 2023}
\begin{enumerate}[(1)]
	\item Gewonnen hat man eine Region, sofern man mindestens die Hälfte ihrer Städte einschließlich ihrer Hauptstadt eingenommen hat.
	\item Ist das Ziel eines Kriegs nicht die Eroberung eines ganzen Staats, so reicht es, die geforderten Regionen einzunehmen.
	\item Einen Staat hat man eingenommen, sofern man mindestens zwei Drittel seiner Regionen und die Hauptstadt eingenommen hat. Die Hauptstadt gilt je nach nationalrechtlichem Status als eigene Region und muss dementsprechend in den zwei Dritteln einbezogen werden oder nicht.
	\item Gewonnen hat man einen Verteidigungskrieg, sofern der Gegner nicht in der Lage war, die besetzten Gebiete zu halten oder einzunehmen.
	\item Eine Besatzungszone gilt als verloren, sofern man sie nach den erforderlichen und drei zusätzlicher Schlachten nicht einnehmen oder zurückerobern konnte.
	\item Als erforderliche Schlacht gilt jede Schlacht, die dazu dient, die eigenen Bedingungen direkt zu erfüllen.\footnote{Beispielsweise der Angriff einer bestimmten Stadt innerhalb einer Region, damit man diese einnimmt}
	\item Der sofortige Sieg erfolgt auch, sofern ein Staat dessen Kapitulation öffentlich bekanntgibt.
	\item Der Krieg kann nicht in Abwesenheit der Soldaten eines Staats geführt werden.
	\item Es müssen mehr als die Hälfte aller Soldatren anwesend sein.
	\item Erfolgte innerhalb von zwei aufeinanderfolgenden Wochen aus unentschuldigten Gründen keine nennenswerte Kampfhandlung von einem Staat oder sind dessen Truppen in dieser Zeit abwesend gewesen, so hat dieser verloren, sofern es der Verteidiger ist.
	\item Hat man einen Krieg rechtmäßig gewonnen, so wird einem das Gebiet, das man erobern wollte, serverrechtlich zugesprochen.
\end{enumerate}

\subsection{Schlacht}\footnote{War Regulatory Act vom 02. Juni 2023}
\begin{enumerate}[(1)]
	\item Eine Schlacht gilt als gewonnen, wenn alle Gegner tot sind.
	\item Als tot gilt, wer außerhalb des Schlachtfelds respawnt.\footnote{Ab hier: War Regulatory Act Amendment vom 21. Februar 2024}
	\item Vor Beginn einer Schlacht wird ein festes Schlachtfeld bestimmt, welches höchstens einhundert mal einhundert Blöcke groß sein darf.
	\item Das Übertreten der Schlachtfeldgrenzen resultiert im sofortigen Ausscheiden aus der Schlacht.
	\item Jede Schlacht muss von einem Beobachter aufgenommen und überwacht werden.
\end{enumerate}

\subsection{Städte}\footnote{War Regulatory Act vom 02. Juni 2023}
\begin{enumerate}[(1)]
	\item Als Stadt gilt jede, öffentlich als solche gekennzeichnete Stadt, die den Richtlinien entspricht.
	\item Als öffentlich gekennzeichnet gilt sie, wenn man ihre Position im jeweiligen Kanal geteilt hat oder nachweislich der anderen Kriegspartei diese hat zukommen lassen.
	\item Außenposten müssen nicht öffentlich gekennzeichnet werden.
	\item Der Hauptaußenposten einer Region gilt als Stadt.
	\item Als Stadt oder Außenposten gelten nur gebaute Städte und Außenposten, die über eine angemessene Größe verfügen.
	\item Wurde eine Stadt innerhalb der zwei Monate nach Bekanntgabe ihrer Existenz nicht errichtet, fällt ihr Status als Stadt weg.
	\item Eine reine Bekanntgabe der Position eines Außenpostens reicht nicht aus, um diesen gültig zu erklären.
	\item Der Unterschied zwischen einem Außenposten und einer Stadt ist dieser, dass ein Außenposten lediglich diplomatischer oder militärischer Zwecke dient.
	\item Die Karte einer Stadt muss ebenfalls öffentlich zugänglich sein.
	\item Es gelten nur die Teile einer Stadt als Stadt, die von einem Staat als solche anerkannt werden.
	\item Die Hauptstadt darf nur einmal im Jahr umverlegt werden.
	\item Eine gebaute Stadt und ein gebauter Außenposten ist erst dann gültig, wenn sie alle Anforderungen dieses Abschnitts erfüllt.
	\item Dies gilt auch für geplante Städte.
	\item Gründungen können nicht während eines Kriegs erfolgen.
	\item Es dürfen nicht zu viele Städte in einem zu kurzen Zeitraum gegründet werden.
\end{enumerate}

\section{Plenarordnung}\footnote{Plenary Act vom 23. Mai 2024}
\subsection{Nutzung der Plenarräumlichkeiten}
\begin{enumerate}[(1)]
	\item Die Plenarräumlichkeiten, die am Welteinstiegspunkt zur Verfügung gestellt werden, sind für alle Servermitglieder öffentlich nutzbar.
	\item Um diese nutzen zu dürfen, muss man dessen Nutzung bei der Administration anmelden.
	\item Nutzt man diese Räumlichkeiten, so verpflichtet man sich, die Plenarordnung einzuhalten.
\end{enumerate}

\subsection{Versammlungsvorsitz}
\begin{enumerate}[(1)]
	\item Bei Anmeldung der Raumnutzung muss man ebenfalls bekanntgeben, wer den Versammlungsvorsitz einnehmen wird.
	\item Der Versammlungsvorsitz erteilt das Wort und darf Versammlungsmitglieder zur Ordnung rufen, sowie deren Redezeit beenden.
	\item Ist der Versammlungsvorsitz Redner, so muss dieser zunächst einen stellvertretenden Vorsitz wählen, der in der Zwischenzeit den Platz einnimmt.
	\item Der Versammlungsvorsitz darf auch in seiner Funktion als Versammlungsvorsitzender Zwischenfragen stellen.\footnote{Herr Abgeordneter/Frau Abgeordnete \textit{Name}, gestatten Sie mir eine Zwischenfrage?}
\end{enumerate}

\subsection{Antragstellungen}
\begin{enumerate}[(1)]
	\item Bis zu zwei Tage vor der Versammlung müssen bei dem Versammlungsvorsitz Anträge auf Themen eingereicht werden.
	\item Der Versammlungsvorsitz darf diese nur ablehnen, sofern sie offensichtlich gegen die Richtlinien des Rex Novus SMP verstoßen.
	\item Eilanträge können nach Ermessen des Versammlungsvorsitzes abgelehnt werden.
	\item Die Anträge werden zunächst nach Zeitpunkt der Einreichung und anschließend nach Thematik sortiert.
	\item Zwei Tage vor Versammlungsbeginn muss der Versammlungsvorsitz die Liste aller Themen an die Versammlungsmitglieder weiterreichen.
	\item Wurde einem Eilantrag stattgegeben, so muss der Versammlungsvorsitz die Mitglieder über diesen informieren.
	\item Jede Rede, die zu einem Antrag gehalten werden soll, muss bis zum Versammlungsbeginn dem Versammlungsvorsitz angekündigt werden.
\end{enumerate}

\subsection{Reden}
\begin{enumerate}[(1)]
	\item Jedes Versammlungsmitglied hat eine Redezeit von höchstens exakt fünfzehn Minuten.
	\item Wird die Redezeit überschritten, muss der Versammlungsvorsitz darauf hinweisen.
	\item Geht das Versammlungsmitglied der Aufforderung, die Rede zu beenden in den nächsten zwei Minuten nicht nach, muss der Versammlungsvorsitz die Rede unterbrechen und die Person muss sich vom Pult entfernen.
	\item Jede Rede muss mit der Addressierung des Versammlungsvorsitzes und der weiteren Versammlungsmitglieder beginnen.\footnote{Sehr geehrter Herr Präsident (o. Sehr geehrte Frau Präsidentin), sehr geehrte Abgeordnete}
	\item Zwischenrufe werden während Reden nicht geduldet.
\end{enumerate}

\subsection{Zwischenfragen und Zwischenrufe}
\begin{enumerate}[(1)]
	\item Während einer Rede dürfen sich Versammlungsmitglieder melden und somit signalisieren, dass sie Zwischenfragen stellen wollen.
	\item Der Versammlungsvorsitz muss in diesem Falle den Redner fragen, ob die Person die Frage stellen darf.\footnote{Herr Abgeordneter/Frau Abgeordnete \textit{Name}, lassen Sie eine Zwischenfrage des Kollegen Herrn Abgeordneten/der Kollegin Frau Abgeordnete \textit{Name} zu?}
	\item Es steht dem Redner frei, zu entscheiden, ob er die Frage annimmt. Er ist nicht verpflichtet, seine Entscheidung zu begründen.
	\item Zwischenrufe werden nicht geduldet.
\end{enumerate}

\subsection{Ordnungsrufe}
\begin{enumerate}[(1)]
	\item Verletzt ein Versammlungsmitglied die Plenarordnung oder die allgemeingültigen Verhaltensregeln, so ist ein Ordnungsruf zu erteilen.
	\item Bei schweren Verletzungen der zugrundeliegenden Ordnungen oder drei Ordnungsrufen, muss das jeweilige Versammlungsmitglied den Saal verlassen.
\end{enumerate}

\end{document}