\documentclass{article}
\usepackage[utf8]{inputenc}
\usepackage{enumerate}
\usepackage{ragged2e}
\usepackage{etoc}
\usepackage{chngcntr}
\usepackage{amsmath}

\renewcommand{\thesection}{}
\newcommand{\sent}[1]{$^{#1}$}
\counterwithout{subsection}{section}
\renewcommand{\thesubsection}{§\arabic{subsection}}


\title{Rex Novus SMP Guidelines}
\author{the.god.emperor}
\date{February 26\textsuperscript{th}, 2024}

\begin{document}
\maketitle

\newpage
\topskip0pt
\vspace*{\fill}
\begin{Center}
\textbf{Translated: May 20\textsuperscript{th}, 2024}
\vspace*{\fill}
\end{Center}
\newpage
The server and any other services associated with the Rex Novus SMP Discord server,
operate under the name `The God Complex Servers'. This name is not a registered trade mark, nor is it
otherwise registered.\\
The unauthorised provision of content from The God Complex Servers that is explicitly and uniquely associated with the server and services is a
services is a copyright infringement and therefore a criminal offence.
\\\\
\copyright\ The God Complex Servers 2024\\
\copyright\ God Emperor Networks 2024
\\\\
\textbf{Copyright holders: } 
the.god.emperor, $\text{ryurai\_jokhin\_pc}$
\\\\
\textbf{Project registration: }
DE000-MC3
\newpage
\tableofcontents
\newpage
\section{Discord server}
\subsection{Validity}
\begin{enumerate}[(1)]
	\item By joining this server, you accept the terms and conditions set out here.
	\item The server administration reserves the right to change these rules at any time.
	\item \textit{omitted}
	\item The regulations only come into force as soon as they are published in the text channel for rules. Accordingly, no regulation applies retroactively.
	\item Lack of or incorrect knowledge of the server regulations does not grant legal immunity, as it is mandatory to be informed about the current legal situation of the server.
	\item You must also contact the person responsible (\ref{members}) if anything is unclear.
	\item If one of the provisions violates the constitution of the country of a person concerned, only the unlawful passage is cancelled for that person.
	\item $^{1}$Copyrights are managed as described in the preamble. $^{2}$Copyright infringements will be reported to VIRTSTAX and will result in permanent exclusion from all domains and ownerships 
	managed by VIRTSTAX.
\end{enumerate}

\subsection{Chat behaviour}\label{verhalten}
\begin{enumerate}[(1)]
	\item $^{1}$Spam, insults, threats and provocations against other players are prohibited and will lead to sanctions. $^{2}$Sending several messages in a short time interval is referred to as spamming. Sending five or more messages in a short period of time can have consequences. $^{3}$The unauthorised use of pings is also prohibited due to its provocative nature.
	\item $^{1}$Racist, political and ethically unacceptable content (statements, images, etc.) is prohibited and will result in permanent exclusion from the entire Discord server. $^{2}$This also applies to pornographic content. $^{3}$This also applies to the deliberate arrangement of reactions of the category `Regional Indicator' to such statements. \footnote{Message Reaction Act of 19 July 2023} $^{4}$The sending of GIFs is prohibited.\footnote{Channel Usage Act of 12 February 2023} $^{5}$ Symbols and trivialisation of \textbf{any} politically extremist acts or groups are prohibited.\footnote{This also applies to left-wing extremist groups; Anti-Extremism Act of 20 February 2024}
	\item Text channels explicitly created for pornographic purposes are excluded from paragraph 2 sentence 2.
	\item Text channels intended for memes are excluded from paragraph 2 sentence 4.
	\item $^{1}$ Furthermore, channels may only be used for the purpose for which they are intended. If there are any uncertainties about the intended use, you must contact support (\ref{support}) before writing a message. 
		  $^{2}$ You must also adhere to the description of the channels and the correct use of their functions.\footnote{Channel Usage Act Amendment of 15 March 2024}
\end{enumerate}

\subsection{Team members}\label{members}
\begin{enumerate}[(1)]
	\item Instructions from authorised team members are binding and must always be followed.
	\item Team members are identified by a rank or role.
	\item The authorised team members include:
	\begin{enumerate}
		\item The CEO
		\item The CLO
		\item The CSO\footnote{Domain Security Act of 29 March 2024}
		\item The CTO
		\item The investors
		\item The COO
		\item The CAO
		\item The Technicians
		\item The DC moderators
		\item The MC moderators
	\end{enumerate}
	\item The authorised members may only issue orders within their area of responsibility.
\end{enumerate}

\subsection{Behaviour in voice channels}
\begin{enumerate}[(1)]
	\item Apart from the regulations from \ref{verhalten}, the following additional provisions apply to voice channels.
	\item The use of voice distorters and soundboards is permitted, provided the team members do not object.
	\item It is not permitted to record persons without their consent.
	\item Misbehaviour in voice channels justifies a temporary server-wide mute.
\end{enumerate}

\subsection{Secondary accounts}
Secondary accounts must be marked as such. To do this, you must edit your server profile in such a way that it is possible for anyone to see whose secondary account it is based on this profile.

\subsection{Penalty}
\begin{enumerate}[(1)]
	\item A distinction is generally made between three penalties:
	\begin{enumerate}[1.]
		\item A warning is a preliminary stage to actual punitive measures. Every member receives a warning for less serious offences.
		\item A timeout refers to a temporary exclusion from the server.
		\item A permanent ban is an irrevocable, indefinite exclusion from the server.
	\end{enumerate}
	\item The penalty is rarely assessed according to the severity of the offence, but usually according to the following guideline:
	\begin{enumerate}[1.]
		\item First warning
		\item Second warning
		\item 24-hour timeout
		\item One-week timeout
		\item One-month timeout
		\item One-year timeout
		\item Permanent ban
	\end{enumerate}
	\item Every sentence must be revoked without exception if the person punished can prove that the sentence was unlawful.
	\item Unlawfulness exists if the offence in question is not a violation on the part of the person punished, if the punishment violates paragraphs 1 and 2 or if the offence was wrongly classified as an offence of particular gravity.
	\item If the offence is a serious one, an immediate timeout or even an immediate permanent ban may be imposed, depending on the severity of the offence. The assessment of the severity is subject to the person responsible, but must be comprehensible. 
	\item In case of doubt, the judgement may be overturned and reversed or converted into another penalty by the CLO or by a qualified majority of the Board of Directors.
	\item Any unlawful messages must be permanently stored in the criminal record in the form of a screenshot together with the description of the incident\footnote{Permanence and entries in the criminal record since the Criminal Records Act of 15 March 2024}, so that in case of doubt the unlawfulness can be challenged\footnote{This is due to past difficulties in assessing the unlawfulness of statements in retrospect}, after which an application for deletion can be made to the Board, which must be confirmed by a qualified majority.
	\item After each timeout, the severity of the offence increases in such a way that the first offence after a timeout is increased in accordance with the penalty hierarchy from paragraph 2 numbers 1 - 7 due to its severity in relation to the previous basic penalty\footnote{The first penalty after a timeout, or the first penalty in total}.
	\item Permanent bans are partially excluded from paragraph 8. These should only be imposed in cases of extreme severity or if paragraph 8 applies, provided there is no acute improvement in behaviour and the Executive Board votes in favour in a simple majority resolution.
	\item It is not permitted to unban people from the server without the express permission of the CLO. This is considered a criminal offence and will be punished with a timeout regardless of the position in the penalty hierarchy according to the interpretation in paragraph 8. The illegally unbanned person must also be banned immediately.
	\item $^{1}$Violations of particular severity by members of a company division will result in permanent exclusion from this and other company divisions. $^{2}$This regulation does not apply to the.god.emperor or ryurai\_jokhin\_pc or their secondary accounts.
	\item Complicity, aiding and abetting, incitement and attempt are penalised equivalently.
\end{enumerate}

\subsection{Owner}
\begin{enumerate}[(1)]
	\item The term `owner' corresponds to the term `shareholder' in accordance with §3 para. 1 OwnP.
	\item Any resolutions of the owners require a qualified majority, unless the provisions of the ownership provisions contradict this.
	\item An admin council can be convened by order of the owners, at which the admins and moderators who are not shareholders each receive one vote and take the decision for the shareholders' meeting.
\end{enumerate}

\subsection{Chairmanship}
\begin{enumerate}[(1)]
	\item The Chief Executive Officer is the Chairman of the Executive Board in accordance with §8 para. 2 OwnP. This rank is abbreviated to `CEO'.
	\item It is constantly occupied by the two founders of the server.
	\item The founders are the.god.emperor and ryurai\_jokhin\_pc (705020938810294392, 744269509770346626).
	\item Should both founders resign from the Executive Board, a new Chairman of the Executive Board must be elected by the Executive Board by a simple minority vote.
\end{enumerate}

\subsection{Corporate divisions}
\begin{enumerate}[(1)]
	\item The following divisions according to §9 OwnP exist on the server:
		\begin{enumerate}[1.]
			\item Accounting (Accounting)
			\item Legal Department
			\item Moderation
			\item Technical Department
			\item Security Department\footnote{Domain Security Act of 29 March 2024}
		\end{enumerate}
	\item The chair of these departments is assigned the corresponding role.
	\item Only persons with sufficient knowledge and skills in the respective field may be employed in the divisions.
\end{enumerate}

\subsection{Accounting}
\begin{enumerate}[(1)]
	\item Accounting is responsible for keeping an eye on the server finances.
	\item This also includes the management of company shares.
	\item The chief accounting officer (CAO) is the authorised transaction officer in accordance with §7 OwnP and the head of department.
\end{enumerate}

\subsection{Legal department}
\begin{enumerate}[(1)]
	\item The responsibilities of the legal department include
	\begin{enumerate}[1.]
		\item Legal questions about the server constitution
		\item Requests for legal assistance
		\item Challenges to server court and other judgements
		\item Constitutional complaints
		\item Proposals for legislation
		\item Handling of violations of server policies
		\item Handling of violations of the partnership provisions
		\item Examination of the judgements of the moderation
	\end{enumerate}
	\item Legislative proposals that seek to amend or abolish existing laws are considered constitutional complaints.
	\item Both statutory offences and appeals, as well as constitutional complaints, are considered sufficient grounds for a full trial.
	\item The chairman of the department is the Chief Legal Officer (CLO).
	\item In its function as a judicial body, the Legal Division is referred to as the `Supreme Court'.
	\item Members of the Supreme Court have the official title `Supreme Judge'. Outside of their judicial activities, they are referred to as `server lawyer'.
	\item The High Court is presided over by the President of the Supreme Court.
	\item The President of the Supreme Court shall be the CLO.\@
	\item Unless there is new and sufficient evidence, paragraph 5 does not apply.
	\item Judgements by the Moderation and the Supreme Court must be confirmed by the President of the Supreme Court and may therefore be dismissed.
	\item The dismissal of judgements must be justified and reasoned.
	\item The Legal Division is able to determine the interpretation of individual laws by means of judgements, provided that this is within a comprehensible framework.\footnote{Prejudice Act of 17 March 2024}
\end{enumerate}

\subsection{Moderation}\label{support}
\begin{enumerate}[(1)]
	\item Any questions regarding the Discord and Minecraft server that do not fall within the legal area are the responsibility of the moderation team. If there are any uncertainties regarding the area of responsibility, you should also contact the moderation department.
	\item The moderation department is responsible for monitoring compliance with the server guidelines.
	\item This means that they are able to enforce penalties without authorisation, but these must be forwarded to the legal department together with the context and finally confirmed.
	\item The Chief Operating Officer (COO), who also fulfils administrative tasks, is the head of moderation.
	\item Excluded from paragraph 3 are any violations of the partnership regulations, as these are dealt with by the legal department and VIRTSTAX.
	\item Members of the moderation are listed as `MC Moderator' (Minecraft Moderator) or `DC Moderator' (Discord Moderator) depending on their area of responsibility.
	\item Moderators may only execute judgements in accordance with paragraph 3 within their area of responsibility.
	\item Minecraft Moderators, unlike Discord Moderators, do not have Discord Administrator permissions, but do have Operator permissions on the Minecraft server.
\end{enumerate}

\subsection{Technical department}
\begin{enumerate}[(1)]
	\item The technical department is responsible for the maintenance of all servers and services under The God Complex Servers.
	\item This also includes the realisation of new functionalities on these servers.
	\item The Chief technology officer (CTO) chairs the department.
	\item Members of this department are referred to as `technicians'.
\end{enumerate}

\subsection{Security department}\footnote{Domain Security Act vom 29. März 2024}
\begin{enumerate}[(1)]
	\item The security department is responsible for securing the domain owners and monitoring user activities.
	\item Any means for the realisation of their tasks must only be approved by the acting CEOs.
	\item It is chaired by the Chief Security Officer (CSO).
	\item The processes within this department are subject to the strictest secrecy and may only be published with the permission of the CEOs.
\end{enumerate}

\section{Minecraft server}
\subsection{Basic rules}
\begin{enumerate}[(1)]
	\item Minecraft server law is subject to Discord server law.
	\item It is forbidden to use methods that give an advantage to other players, regardless of whether they use the method or not, that are not generally recognised as fair.
	\item Any behaviour not affected by such methods is not punishable.
	\item Paragraph 3 excludes the construction of structures intended to reduce server performance.
	\item On the Minecraft server, you must behave in accordance with the respective rules of the Discord server.
	\item The general server law does not differentiate between factions, which is why they are merely an internal organisation that is not covered by any server-wide laws and therefore crimes against them cannot be the subject of server-wide judgements.
	\item Paragraph 3 only applies if the offences are not directed against the rules of the Discord server or Paragraph 1f.
\end{enumerate}

\subsection{Guarantee}\footnote{Guarantee Act of 30 June 2023}
\begin{enumerate}[(1)]
	\item Only persons for whom someone can be proven to be a guarantor may be placed on the whitelist.
	\item If a person commits an offence of particular gravity, the existing guarantee relationships will be reviewed and may be terminated if necessary\footnote{para. 2 - 3: Guarantee Act Amendment of 20 February 2024}.
	\item If a guarantor is banned from the server, all guarantee relationships with this guarantor are terminated.
	\item Guarantees cannot be withdrawn retrospectively.
	\item The administration is exempt from para. 2 - 4.
	\item $^{1}$ You must submit an application for inclusion on the whitelist.$^{2}$ This must be done via the support channel.$^{3}$ Applications submitted otherwise are invalid without exception.\footnote{Guarantee Act Amendment II of 06 May 2024}
\end{enumerate}

\subsection{Legal separation}
\begin{enumerate}[(1)]
	\item The server right must be clearly distinguished from the internal legal situation on the branches of the Rex Novus SMP.
	\item Internal law refers to constitutions and rules that are not recognised by the owners in their function as a shareholders' meeting, such as the group's own legal texts.
	\item The inspection and use of server data and other information that is only accessible to the administration, such as player data or logs, may not be used as evidence for trials and the like that are not carried out by the Supreme Server Court in its function\footnote{for example, in-game murders may not be proven via logs}.
	\item This does not apply in the area around the world entry point, including the world entry point itself. This is domain-governed territory, so domain jurisdiction applies accordingly.\footnote{Spawn Protection Act of 16 May 2024}
\end{enumerate}

\subsection{Server-wide Arrest Warrants and Detention Orders}\footnote{Law Enforcement Act dated May 27, 2024}
\begin{enumerate}[(1)]
    \item If a person has committed crimes against the written constitution of a nation or a union of states, they may apply for an arrest warrant with the server's legal department.
    \item The legal department must then issue an order for the person to appear before the competent court, which the person must comply with.
    \item During the trial, a member of the legal department not involved in the trial must attend as an observer.
    \item If the server attorney deems an arrest warrant based on the trial to be sufficiently justified, it must be verified where the person was located at the time the arrest warrant was requested.
	\item $^{1}$If the person was in domestic territory or the external world at that time, the process can proceed without hindrance. $^{2}$If the person was in a foreign country occupied by a server member, the state where the person was located must consent to the extradition. $^{3}$If the person was on other territory, it is to be decided by random procedure whether the nation consents to the extradition.
	\item If the person was in domestic territory, the outside world or if extradition was authorised, a random decision is made as to whether the arrest was successful.
	\item If extradition was not authorised, the arrest warrant is pending until extradition is authorised. This only has effect if the person in question is in the respective country. Otherwise, authorisation must be obtained from the new country of residence. If the arrest was unsuccessful, the arrest warrant loses its effect.
	\item If the arrest was unsuccessful, a detention order is issued. This is a request under server law to submit to punishment.
	\item If a person does not comply with the request, this has consequences under server law and the arrest is enforced by means of operator orders.
\end{enumerate}

\subsection{Escape from Server-Legally Imposed Detention Order}\footnote{Law Enforcement Act dated May 27, 2024}
\begin{enumerate}[(1)]
	\item If a person has complied with the detention directive, after having served at least half an hour and at most half of the sentence, they may apply for an escape to the legal department of the server.
	\item The legal department will then determine by random procedure whether the escape is allowed.
	\item If the escape is allowed, the person must be released by the server administration.
\end{enumerate}

\subsection{Exception to Application-Based Interventions}\footnote{Law Enforcement Act dated May 27, 2024}
The server administration may not undertake application-based interventions in the project’s internal environment if the application or the processing or approval process was faulty or irregular.

\section{Faction Rules}
\subsection{Factions}\label{factions}
\begin{enumerate}[(1)]
	\item Any group consisting of at least one player shall be considered a faction.
	\item The term "player" is not synonymous with the term "Minecraft account" and therefore does not justify additional territorial claims if the other account is a secondary account of the same person.
	\item Factions are entitled to their own category where they can set up any channels upon request.
	\item Moderation may reject such requests if they find no justified reason for establishment.
	\item Due to the duties and responsibilities of the moderation team, they are entitled at any time to inspect the channels to identify violations of the applicable guidelines.
	\item Each faction is guaranteed its own role, which serves the purpose of preventing channels from being viewed by members of other factions.
    \item Participation on the server requires the formation or joining of a faction approved by the administration\footnote{Paragraphs 7, 8: Faction Establishment Act of February 20, 2024}.
    \item To establish a recognized faction, the following information must be provided to the administration:
    \begin{enumerate}[1.]
        \item The faction's territory
        \item The name of the state
        \item The insignia of the state
        \item All state members and their role-playing names
        \item The first leader of the state
    \end{enumerate}
	\item The act of establishing a faction and the right to its continuation shall be referred to as a faction declaration.
\end{enumerate}

\subsection{Nullity of the Faction Declaration}
A faction declaration pursuant to \ref{factions} paragraph 9 is null and void if the faction
\begin{enumerate}
	\item Loses one of the criteria from \ref{factions} paragraph 8 and is unable to regain all the criteria within two months.\footnote{Faction Establishment Act of February 20, 2024}
	\item Has not been significantly active for at least one month without providing an understandable and comprehensible reason.\footnote{Paragraphs 2, 3: Faction Establishment Act Amendment VI of July 17, 2024}
	\item Does not comply with the server rules or defies direct orders from the team.
\end{enumerate}

\subsection{Faction Insignia}\footnote{Faction Establishment Act of February 20, 2024}
\begin{enumerate}[(1)]
    \item Any faction insignia must be unique and clearly distinguishable from the insignia of other factions.
    \item The faction insignia must not violate any behavioral rules and must clearly serve the purpose of serious representation of the faction.
    \item Flags and insignia of real states are not exempt from paragraph 2.
    \item The mandatory faction insignia include:
    \begin{enumerate}[1.]
        \item The flag of the faction
        \item The coat of arms of the faction
        \item The uniforms of the faction
    \end{enumerate}
\end{enumerate}

\subsection{Succession Claims and Revolutions}\footnote{Faction Establishment Act Amendment of February 26, 2024}
\begin{enumerate}[(1)]
	\item The server law does not establish succession rules and does not evaluate the legitimacy of a succession claim.
	\item It is possible to change the government succession through internal events without the consent of the current head of state.
	\item Revolutionary wars are subject to the laws of war.
	\item Rebellious and revolting factions may choose any territories as their initial faction territory that were granted to them by the previous head of state. In this case, this territory is legally binding.
\end{enumerate}

\subsection{Intervention of Server Law Regarding Faction Rules and Constitutions}
\begin{enumerate}[(1)]
	\item Any state constitution is considered recognized once it is publicly accessible and access is granted on the Discord server in the respective text channel.
    \item It is not possible to legally challenge state constitutions and agreements as long as they do not contradict server law.
    \item Internal faction judgments can be legally contested if they do not have a justifiable written basis.\footnote{Faction Establishment Act Amendment II of March 24, 2024}
\end{enumerate}

\subsection{Protection of Faction Integrity}
\begin{enumerate}[(1)]
	\item ${^1}$The territory and state insignia, as well as any other property and possessions of the state, are not protected by the server-wide guidelines and can therefore be claimed by other states. ${^2}$ This does not apply to mandatory insignia.\footnote{Faction Establishment Act Amendment III of June 27, 2024}
    \item States may model themselves after any real states but are not required to do so.
    \item It is not permitted to choose insignia or names that are clearly attributable to a historical or current state without representing that state.
    \item Outside of a server-legally recognized state of war and an actively affected battle zone, no foreign property may be destroyed or built upon without permission.\footnote{War Regulatory Act Amendment III of June 17, 2024} \footnote{Property Protection Act of July 8, 2024}
	\item ${^1}$A state may not be attacked by other states for exactly two months after its founding. ${^2}$Once the state attacks another state, this immunity lapses. ${^3}$This rule applies only to newly founded states that did not emerge from another state.\footnote{Faction Establishment Act Amendment IV of July 8, 2024}
\end{enumerate}

\subsection{Territorial expansion and martial law}
\begin{enumerate}[(1)]
    \item Territorial expansion must be recognised by the administration and requires either agreements between or mergers of states or a conquest of the state.
    \item The conquest of a state requires warfare that complies with the applicable server-wide laws of war.
    \item The server administration can decide on exceptions to martial law by unanimous vote. It is also up to the server administration to judge whether the rules requiring a subjective judgement have been violated. % Admins or legal department?
    \item One must wear the uniforms of their own nation during war.\footnote{War Regulatory Act Amendment IV of 27 June 2024}
	\item One shall not utilize invisibility potions, effects, or similar means during times of war.\footnote{War Regulatory Act Amendment V of 10 July 2024}
\end{enumerate}

\subsection{Begin of war}\footnote{War Regulatory Act vom 02. Juni 2023}
\begin{enumerate}[(1)]
	\item \sent{1}A nation can only wage one war against another nation per month. \sent{2}This only applies to offensive and counterattack wars.
	\item Unoccupied states can be occupied by using the war points.
	\item You must publicly declare war on the state you wish to attack. % Deadline until actual war?
	\item When war is declared, neutral parties must be identified to observe the war and check that it conforms to the rules.
	\item If no appropriate neutral observers can be appointed, indirectly involved parties\footnote{for example, alliance parties} on the side of the attacker and on the side of the defender must withdraw from the war and act as observers.
	\item Observer decisions can be appealed to the server's legal department.\footnote{Sections 6-9: War Regulatory Act Amendment of 21 February 2024}
	\item Biased and manipulated observer decisions are against the rules.
	\item Once an observer has demonstrably not made a neutral judgement, he can no longer be appointed as an observer.
	\item An objection can be lodged against the election of observers.
	\item At the beginning, both parties must negotiate conditions that will apply at the end of the war if the defender wins. % Abolish?
	\item If the parties cannot reach an agreement within one week of the declaration of war, the defender may dictate the terms.
	\item The conditions must always be proportionate.
	\item If the defender has been absent without excuse from all negotiations within this week, the war begins without negotiations. In this case, the defender cannot claim any conditions.
	\item You may not take unoccupied lands during an ongoing war.
	\item You may occupy unoccupied territories at any time under server law without military action, provided you can prove a legitimate claim, as the territory belongs to the real state.
	\item When considering regions, cities and the like with regard to the law of war, the status of the beginning of the war is assumed.\footnote{From here on: War Regulatory Act Amendment II of 28 February 2024}
	\item Only boundaries that have been publicly announced apply.
\end{enumerate}

\subsection{War Points}\footnote{War Regulatory Act of 02 June 2023}
\begin{enumerate}[(1)]
	\item War points are a unit to determine how many wars can be fought by a state per month against unoccupied countries.\footnote{They are a method to prevent fighting huge countries as often as tiny countries.}
	\item \sent{1}Each month, a faction receives three war points. \sent{2}At the end of the month, the number of war points is not reset. % Adjustment of the number of war points?
	\item It is possible to capture only certain regions of an unoccupied territory. In this case, you only have to spend war points corresponding to the proportion of the total area.\footnote{War Regulatory Act Amendment II of 27 February 2024}
	\item If it is suspected that several sovereign states are sovereign mainly in order to receive more war points, the server administration must treat these sovereign states as a single faction for the purposes of war points and therefore only award the single war point number. % Question from the legal department?
	\item You have to wear the uniforms of your own nation in war.\footnote{War Regulatory Act Amendment IV of 27 June 2024}
\end{enumerate}

\subsection{Warfare}\footnote{War Regulatory Act of 02 June 2023}
\begin{enumerate}[(1)]
	\item You can only attack countries that are directly adjacent to a passageway
	\item A passageway is a path that your troops can justifiably use through territories occupied by servers.
	\item A passageway is only considered to be justifiably usable if it is either part of your own territory or the territory of a state that demonstrably gives permission to cross it.
	\item \sent{1}You may also justifiably enter the territory of a state that you have declared war on. \sent{2}This only applies to occupied territories.
\end{enumerate}

\subsection{Occupation}\footnote{War Regulatory Act of 02 June 2023}
\begin{enumerate}[(1)]
	\item Any territory that could be successfully occupied with troops is considered occupied.
	\item If this territory was not defended, it is also considered occupied.
	\item An occupied territory is only considered occupied under server law if it has been recognised by the administration as legally acquired territory and awarded to the occupier.
\end{enumerate}

\subsection{Victory}\footnote{War Regulatory Act of 02 June 2023}
\begin{enumerate}[(1)]
	\item You have won a region if you have conquered at least half of its cities, including its capital.
	\item If the aim of a war is not the conquest of an entire state, it is sufficient to capture the required regions.
	\item A state is conquered when at least two thirds of its regions and the capital have been captured. Depending on your national status, the capital is considered a separate region and must be included in the two thirds or not.
	\item A defensive war is won if the opponent was unable to hold or capture the occupied territories.
	\item An occupied zone is considered lost if it could not be captured or recaptured after the required and three additional battles.
	\item A required battle is any battle that directly fulfils your conditions.\footnote{For example, attacking a specific city within a region to capture it}
	\item Immediate victory also occurs if a state publicly announces its surrender.
	\item War cannot be waged in the absence of the soldiers of a state.
	\item More than half of all soldiers must be present.
	\item If no significant combat action has been taken by a state within two consecutive weeks for unexcused reasons or if its troops have been absent during this time, the state has lost if it is the defender.
	\item If you have legally won a war, the territory you wanted to conquer is awarded to you under server law.
\end{enumerate}

\subsection{Battle}\footnote{War Regulatory Act vom 02 June 2023}
\begin{enumerate}[(1)]
	\item A battle is considered won when all opponents are dead.
	\item Anyone who respawns outside the battlefield is considered dead.\footnote{From here on: War Regulatory Act Amendment of 21 February 2024}
	\item Before the start of a battle, a fixed battlefield is determined, which may not be larger than one hundred by one hundred blocks.
	\item Exceeding the battlefield limits results in immediate elimination from the battle.
	\item Each battle must be recorded and monitored by an observer.
\end{enumerate}

\subsection{Städte}\footnote{War Regulatory Act of 02 June 2023}
\begin{enumerate}[(1)]
	\item A city is any publicly labelled city that complies with the guidelines.
	\item A city is considered to be publicly labelled if its position has been shared in the respective channel or if it can be proven that the other warring party has received it.
	\item Outposts do not need to be publicly labelled.
	\item The main outpost of a region is considered a city.
	\item Only built cities and outposts of an appropriate size are considered cities or outposts.
	\item If a city has not been built within two months of its existence being announced, its status as a city is cancelled.
	\item Simply announcing the position of an outpost is not sufficient to declare it valid.
	\item The difference between an outpost and a city is that an outpost only serves diplomatic or military purposes.
	\item The map of a city must also be publicly accessible.
	\item Only those parts of a city that are recognised as such by a state count as a city.
	\item The capital may only be moved once a year.
	\item A built city and outpost is only valid if it fulfils all the requirements of this section.
	\item This also applies to planned cities.
	\item Foundations cannot be built during a war.
	\item Too many cities may not be founded in too short a period of time.
\end{enumerate}

\section{Plenary Rules}\footnote{Plenary Act of 23 May 2024}
\subsection{Use of Plenary Premises}
\begin{enumerate}[(1)]
	\item The plenary premises made available at the worldspawn are publicly accessible to all server members.
	\item To use these premises, one must register their use with the administration.
	\item By using these premises, one commits to complying with the plenary rules.
\end{enumerate}

\subsection{Chairing of Assemblies}
\begin{enumerate}[(1)]
	\item Upon registering the use of the premises, one must also declare who will chair the assembly.
	\item The chair of the assembly grants speaking rights and may call assembly members to order and terminate their speaking time.
	\item If the chair of the assembly is a speaker, they must first appoint a deputy chair to take over in the interim.
	\item The chair of the assembly may also pose interjections in their capacity as chair.\footnote{Mr. Deputy/Madam Deputy \textit{Name}, may I interject with a question?}
\end{enumerate}

\subsection{Submission of Proposals}
\begin{enumerate}[(1)]
	\item Proposals on topics must be submitted to the chair of the assembly up to two days before the assembly.
	\item The chair of the assembly may only reject proposals if they clearly violate the guidelines of Rex Novus SMP.
	\item Urgent proposals may be rejected at the discretion of the chair of the assembly.
	\item Proposals are initially sorted by submission time and subsequently by topic.
	\item Two days before the start of the assembly, the chair must distribute the list of all topics to the assembly members.
	\item If an urgent proposal is accepted, the chair must inform the members about it.
	\item Every speech related to a proposal must be announced to the chair of the assembly before the start of the assembly.
\end{enumerate}

\subsection{Speeches}
\begin{enumerate}[(1)]
	\item Each assembly member has a maximum speaking time of exactly fifteen minutes.
	\item If the speaking time is exceeded, the chair of the assembly must indicate this.
	\item If the assembly member does not comply with the request to end the speech within the next two minutes, the chair must interrupt the speech, and the person must leave the podium.
	\item Every speech must begin with addressing the chair and the other assembly members.\footnote{Esteemed Mr. President (or Esteemed Madam President), esteemed deputies}
	\item Interruptions are not tolerated during speeches.
\end{enumerate}

\subsection{Interjections and Interruptions}
\begin{enumerate}[(1)]
	\item During a speech, assembly members may signal their desire to ask questions by raising their hand.
	\item In such cases, the chair must ask the speaker if they allow the question.\footnote{Mr. Deputy/Madam Deputy \textit{Name}, do you allow a question from the colleague Mr. Deputy/Madam Deputy \textit{Name}?}
	\item The speaker is free to decide whether to accept the question and is not obliged to justify their decision.
	\item Interjections are not tolerated.
\end{enumerate}

\subsection{Calls to Order}
\begin{enumerate}[(1)]
	\item If an assembly member violates the plenary rules or the generally applicable conduct rules, a call to order must be issued.
	\item In the case of severe violations of the underlying rules or three calls to order, the respective assembly member must leave the hall.
\end{enumerate}

\end{document}