\documentclass{article}
\usepackage[utf8]{inputenc}
\usepackage{enumerate}
\usepackage{ragged2e}
\usepackage{tocloft}
\usepackage{etoc}
\usepackage{amsmath}
\usepackage{chngcntr}

\renewcommand{\thesection}{}
\newcommand{\sent}[1]{$^{#1}$}
\newcommand{\ph}{\textbf{PLACEHOLDER}\ }
\counterwithout{subsection}{section}
\renewcommand{\thesubsection}{Art.\ \arabic{subsection}}
\cftsetindents{subsection}{3em}{4em}
\newenvironment*{pg}{\begin{enumerate}[(1)]}{\end{enumerate}}

\title{Verfassung der Mönchsrepublik Ratosurya}
\author{Sangabat Jyoti}
\date{20. Februar 1920}

\begin{document}
\maketitle
\vspace*{\fill}
\paragraph{Verfassung des Deutschen Kaiserreichs}

\newpage
\topskip0pt
\vspace*{\fill}
\begin{Center}
\textbf{1. Fassung}
\vspace*{\fill}
\end{Center}
\newpage
\topskip0pt
\vspace*{\fill}
\paragraph*{Wir, Friedrich IV., von Gottes Gnaden Deutscher Kaiser, König von Preußen, etc., verordnen
hiermit im Namen des Deutschen Reichs, nach erfolgter Zustimmung durch den Reichsrath und den Reichstag,
was folgt:}
\vspace*{\fill}
\newpage
\tableofcontents
\newpage
\section{Hoheitsordnung}
\subsection{Unveränderlichkeit der Gesetze}\footnote{Reichshoheitsordnungsbeschluss vom 10. März 1920}
Jegliche Gesetze der Hoheits- und Grundordnung können durch andere Gesetze oder Beschlüsse nicht in ihrer Gültigkeit beeinträchtigt oder erweitert werden.

\subsection{Das Deutsche Reich}
\begin{pg}
    \item Das Deutsche Reich ist eine freie und unabhängige Nation unter der Hand und Schirmherrschaft Seiner Majestät des Kaisers und somit der Deutschen Kaiserkrone zur Treue verpflichtet.\footnote{Reichskaisertumsbeschluss vom 23. Juni 1920}
    \item Das Reichsgebiet umfasst das Gebiet Deutschland, Frankreich, Spanien, die Ukraine und Japan.
    \item Das Deutsche Reich erhebt unanfechtbaren Anspruch auf die Gebiete, die ihnen im Kaiserpakt zugesprochen werden.
    \item Alle Gebiete, die dem Deutschen Reich angehören, können dieses nicht verlassen.\footnote{Reichshoheitszusatzbeschluss vom 10. März 1920}
\end{pg}

\subsection{Der Kaiser}
\begin{pg}
    \item Der Deutsche Kaiser ist das ständige Staatsoberhaupt des Deutschen Reichs.
    \item Der Nachfolger wird erbrechtlich geregelt.
    \item Der Kaiser ernennt die Reichsleitung und kann diese unbegründet entlassen.
    \item Er ist die höchste Reichsgewalt und verfügt über absolute und uneingeschränkte Entscheidungsvollmachten.\footnote{Reichskaisertumsbeschluss vom 23. Juni 1920}
    \item Der Deutsche Kaiser ist Oberbefehlshaber der Reichsstreitkräfte.
    \item Der Deutsche Kaiser kann Einspruch gegen Entscheidungen des Reichsraths und des Reichstags einlegen.
    \item Gesetze und Beschlüsse können erst dann inkrafttreten, wenn der Deutsche Kaiser sie unterzeichnet.
    \item Der Begriff des Kaisers ist synonym mit dem Begriff des Deutschen Kaiserreichs, des Staates und aller weiteren Synonyme jener Begriffe.\footnote{Ab hier: Reichskaisertumsbeschluss vom 23. Juni 1920}
    \item Allein der Kaiser verfügt über das Recht, Personen zu begnadigen.
\end{pg}

\subsection{Die Kaiserin}\footnote{Reichskaisertumsbeschluss vom 23. Juni 1920}
\begin{enumerate}[(1)]
    \item Es kann nur die, unter amtlicher Zeugschaft getraute Ehefrau des Kaisers den Titel der Kaiserin halten.
    \item Die Kaiserin unter dem Kaiser die zweithöchste Gewalt des Reichs.
    \item In den Ansprüchen und Reichsgebieten Japan und Korea und in Fragen, die ausschließlich diese Gebiete betreffen ist sie dem Kaiser übergeordnet.
\end{enumerate}

\section{Reichsgrundordnung}\footnote{Reichsgrundordnungsbeschluss vom 10. März 1920}
\subsection{Stimmrechte eines Versammlungsvorsitzes}
\begin{enumerate}[(1)]
    \item Bei Stimmgleichheit steht einem Versammlungsvorsitz eine zweite Stimme zu.
    \item Es ist einem Versammlungsvorsitz möglich, bei Einstimmung durch den Kaiser, die Anzahl der benötigten Stimmen auf die Hälfte der Versammlung einzugrenzen.
\end{enumerate}

\subsection{Keine Strafe ohne Gesetz}
\begin{enumerate}[(1)]
    \item Es ist nicht zulässig, eine Person auf Grundlage eines Gesetzes zu verurteilen, das zur Tatzeit noch nicht bestand.
    \item Einem Gericht ist es möglich, dies dennoch zu tun, sofern ein begründeter und nachvollziehbarer Verdacht besteht, dass der Täter die Tat im Gewissen verübte, gegen die geltende moralische Norm des Kaiserreichs zu verstoßen.
    \item Absatz 2 ist nur dann gültig, sofern es sich bei der Straftat um eine Straftat handelt, die das Recht auf körperliche Unversehrtheit verletzt oder die Wirtschaft nachhaltig beeinträchtigt.
\end{enumerate}

\subsection{Strafbarkeit des Versuchs und der Täterschaft}
\begin{enumerate}[(1)]
    \item Jeglicher Versuch, eine Straftat zu begehen, ist mit der Straftat selbst äquivalent zu behandeln.
    \item Ebenfalls gleicht vor dem Deutschen Gesetz die Mitwisserschaft, die Beihilfe und die Mittäterschaft der Täterschaft und ist daher mit gleichen Maßnahmen aufzuwiegen.
\end{enumerate}

\subsection{Rechte des Beschuldigten}
\begin{enumerate}[(1)]
    \item Mangelnde oder fehlerhafte Kenntnisse der Rechtslage gewähren keine rechtliche Immunität.
    \item Bei eigenverschuldeter und unentschuldbarer Abwesenheit vor Gericht, dürfen Prozesse in Abwesenheit der fehlenden Partei abgehalten werden.
    \item Einem Beschuldigten steht es zu, die Beschuldigung vor Gericht anzufechten.
    \item Es gibt kein rechtskräftiges Urteil ohne Gerichtsprozess.
    \item Advokaten müssen von der Beklagten oder dem Klagenden selbst gestellt werden.
    \item Es besteht kein Grundrecht auf einen Advokaten.
    \item Jedem, der sich durch ein Urteil nachvollziehbar geschädigt sieht, steht es zu, dieses anzufechten. Tut man dies, so wird die Rechtschaffenheit des Urteils von der nächsthöheren Instanz überprüft.
    \item Befindet man sich bereits in der höchsten Instanz, so ist das Urteil rechtskräftig und final.
    \item Diese Rechte dürfen keinem verwehrt werden.
\end{enumerate}

\section{Exekutive}
\subsection{Amtsbesetzung}\footnote{Reichsamtsbesetzungsbeschluss vom 10. März 1920}
\begin{enumerate}[(1)]
    \item Jedes Amt muss alle zwei Monate neu gewählt werden.
    \item Kein Amt ist erblich.
    \item Es wird demjenigen die Amtszulassung entzogen, wer sich nicht der Amtsmoral beugt.
    \item Ämter können nur durch Staatsbürger Deutscher Geburt besetzt werden.
\end{enumerate}

\subsection{Die Reichsleitung}
\begin{enumerate}[(1)]
    \item Die Reichsleitung ist das regierende Organ des Kaiserreichs.
    \item Sie besteht aus dem Reichskanzler und den Staatssekretären.
    \item HGegenüber dem Kaiser agiert die Reichsleitung als beratendes Organ.\footnote{Reichsexekutivaufgabenbeschluss vom 23. Juni 1920}
\end{enumerate}

\subsection{Der Reichskanzler}
\begin{enumerate}[(1)]
    \item Der Reichskanzler ist der Regierungschef und Stellvertreter des Deutschen Kaisers, sowie Oberhaupt der kaiserlichen Verwaltung.\footnote{Reichsexekutivaufgabenbeschluss vom 23. Juni 1920}
    \item Er wird vom Deutschen Kaiser auf fünf Jahre ernannt.
    \item Er kann jederzeit unbegründet durch den Deutschen Kaiser entlassen werden.
    \item Der Reichskanzler ist die höchste Reichsgewalt unter dem Gesetz.\footnote{Reichsexekutivaufgabenbeschluss vom 23. Juni 1920}
\end{enumerate}

\subsection{Die Staatssekretäre und Reichsämter}
\begin{enumerate}[(1)]
    \item Die Staatssekretäre sind die Vorsitzenden, der ihnen untergeordneten Reichsämter.
    \item Sie sind die obersten Berater in ihrem Ressort und können ein Veto gegen Entscheidungen des Reichsraths und des Reichstags einlegen, die ihr Ressort betreffen.
    \item Die Reichsämter verwalten das, ihnen zugewiesene Resort und können ohne Genehmigung des Reichsraths und des Reichstags in diesem auch vollwertige Beschlüsse dem Kaiser vorlegen.
    \item Die derzeitigen Reichsämter lauten wie folgt:
    \begin{enumerate}[1.]
        \item Reichskanzleramt
        \item Reichsamt des Innern
        \item Reichsamt für das Auswärtige
        \item Reichsjustizamt
        \item Reichsamt des Krieges
        \item Reichsheroldsamt
        \item Reichsschatzamt\footnote{Reichsamtsbeschluss vom 10. März 1920}
    \end{enumerate}
\end{enumerate}

\subsection{Reichskanzleramt}
\begin{enumerate}[(1)]
    \item Die Aufgaben des Reichskanzleramts umfassen:
    \begin{enumerate}[1.]
        \item Die Koordination der Regierungsgeschäfte
        \item Die Unterstützung des Reichskanzlers bei der Ausführung seiner Aufgaben
        \item Die Vorbereitung von Gesetzesvorlagen und Regierungsverordnungen
    \end{enumerate}
    \item Der Reichskanzler ist der Leiter des Reichskanzleramts.
\end{enumerate}

\subsection{Reichsamt des Innern}
\begin{enumerate}[(1)]
    \item Die Aufgaben des Reichsamt des Innern umfassen:
    \begin{enumerate}[1.]
        \item Die Verwaltung der inneren Reichsangelegenheiten
        \item Die Kommunalverwaltung
        \item Die Überwachung der Polizei und sonstiger Sicherheitskräfte
        \item Die Verwaltung jeglicher Angelegenheiten bezüglich Staatsbürgerschaften und Einreisegenehmigungen
        \item Die Organisation und Überwachung von Wahlen und politischen Prozessen
    \end{enumerate}
    \item Der Vorsitzende des Reichsamt des Innern ist der Staatssekretär des Innern.
\end{enumerate}

\subsection{Reichsamt für das Auswärtige}
\begin{enumerate}[(1)]
    \item Die Aufgaben des Reichsamts für das Auswärtige umfassen:
    \begin{enumerate}[1.]
        \item Die Verwaltung der Reichsaußenpolitik
        \item Die Verwaltung der diplomatischen Prozesse des Reichs
        \item Die Vertretung deutscher Interessen im Ausland
        \item Die Verhandlung und der Abschluss internationaler Verträge und Abkommen
        \item Die Organisation und Gewährleistung des Schutzes deutscher Bürger im Ausland
    \end{enumerate}
    \item Der Leiter dieses Reichsamts ist der Staatssekretär für das Auswärtige.
\end{enumerate}

\subsection{Reichsjustizamt}
\begin{enumerate}[(1)]
    \item Die Aufgaben des Reichsjustizamts umfassen:
    \begin{enumerate}[1.]
        \item Die Verwaltung der Justiz im Reich, einschließlich der Überwachung der Gerichte und der Rechtspflege
        \item Die Entwicklung von Gesetzesvorlagen und Rechtsvorschriften
        \item Die Gewährleistung der Rechtssicherheit und Durchsetzung der Gesetze
        \item Die Verwaltung von Haftanstalten und Justizvollzug
    \end{enumerate}
    \item Der Leiter des Reichsjustizamts ist der Staatssekretär der Justiz.
\end{enumerate}

\subsection{Reichsamt des Krieges}
\begin{enumerate}[(1)]
    \item Die Aufgaben des Reichsamts des Krieges umfassen:
    \begin{enumerate}[1.]
        \item Die Organisation und Verwaltung der Streitkräfte des Reiches
        \item Die Verwaltung der diplomatischen Prozesse des Reichs
        \item Die Vertretung deutscher Interessen im Ausland
        \item Die Verhandlung und der Abschluss internationaler Verträge und Abkommen
        \item Die Organisation und Gewährleistung des Schutzes deutscher Bürger im Ausland
    \end{enumerate}
    \item Der Leiter des Reichsamts des Krieges ist der Generalstabschef des Deutschen Reichsheers.
\end{enumerate}

\subsection{Reichsheroldamt}
\begin{enumerate}[(1)]
    \item Die Aufgaben des Reichsheroldsamts lauten:
    \begin{enumerate}[1.]
        \item Die Verwaltung und Verleihung von Adelstiteln
        \item Die Überwachung und Organisation von Zeremonien
        \item Die Verwaltung der Reichsinsignien
    \end{enumerate}
    \item Der Reichsamtsleiter ist der Reichsherold.
\end{enumerate}

\subsection{Reichsschatzamt}\footnote{Reichsamtsbeschluss vom 10. März 1920}
\begin{enumerate}[(1)]
    \item Die Aufgauben des Reichsschatzamt lauten wie folgt:
    \begin{enumerate}[1.]
        \item Überwachung und Verwaltung des Staatsschatzes
        \item Verwaltung der Geldinstitute im Reich
        \item Erlass von Staatsfonds und Investitionsgeldbeschlüssen
    \end{enumerate}
    \item Der reichsweite Leiter des Reichsschatzamts ist der Reichsschatzmeister.
\end{enumerate}

\section{Standesordnung}\footnote{Reichsständebeschluss vom 10. März 1920}
\subsection{Die Könige}
\begin{enumerate}[(1)]
    \item Könige verfügen auf ihrem Gebiet über absolute Entscheidungsvollmachten, die nur von der Reichsleitung in einem Mehrheitsvotum widerrufen werden können.
    \item Sie verfügen über das Anrecht, Gesetze innerhalb ihres Reichslehen zu erlassen, sofern diese nicht gegen die geltenden Gesetze des Deutschen Reichs sprechen.
    \item Könige können Reichslehen bis zur Stufe des Herzogtums verleihen und genehmigen.
\end{enumerate}

\subsection{Die Herzoge}
\begin{enumerate}[(1)]
    \item Herzoge verfügen auf ihrem Gebiet nur über eine Teilvollmacht, welche durch ihren Lehnsherren angefochten werden kann.
    \item Sie dürfen Reichslehen bis zur Stufe des Fürstentums verleihen.
    \item Nur Mitglieder des Hauses des Lehnsherrn dürfen den Titel des Großherzogs tragen.
\end{enumerate}

\subsection{Die Fürsten}
\begin{enumerate}[(1)]
    \item Fürsten verfügen auf ihrem Gebiet über geringe Vollmachten und müssen bei signifikanten Entscheidungen den Lehnsherrn um Erlaubnis bitten.
    \item Die Reichsfürsten dürfen Reichslehen bis hin zur Stufe der Grafschaft verleihen.
\end{enumerate}

\subsection{Die Staatsbürgerschaft}\footnote{Reichsständezusatzbeschluss vom 23. Mai 1920}
\begin{enumerate}[(1)]
    \item Die Staatsbürgerschaft erhält nur, wer im Deutschen Reich geboren ist, in diesem seit mehr als zehn Jahren seinen Hauptwohnsitz hat oder wem die Staatsbürgerschaft durch den Kaiser gewährt wurde.
    \item Ein, nicht behördlich als Hauptwohnsitz anerkannter Wohnsitz, oder Verwandtschaft oder Vorfahren mit deutscher Staatsbürgerschaft gewähren einem kein Anrecht auf eine Staatsbürgerschaft.
    \item Eine Staatsbürgerschaft erfordert einen Wohnsitz auf deutschem Boden.
    \item Als Wohnsitz gilt jegliche, sich im Eigentum oder Besitz der fraglichen Person befindliche oder durch diese bewohnte Residenz, die man weder zur Vermietung ausstellt, noch derzeit vermietet oder anderweitig anderen, nicht der Familie angehörigen 
    Personen zeitlich unbefristet oder über einen geregelten Zeitraum hinweg allein diesen zur Verfügung stellt, ohne diese Residenz nach Sinn und Verstand zu urteilen, selbst zu nutzen.
    \item Das Deutsche Reich erlaubt keine zusätzliche Staatsbürgerschaft.
    \item Wer die Staatsbürgerschaft ohne Erfüllung der Anforderungen trägt, dem ist die Staatsbürgerschaft ausnahmslos zu entziehen.
\end{enumerate}

\subsection{Erblicher Adel}\footnote{Zweiter Reichsständezusatzbeschluss vom 23. Mai 1920}
Jene Adelstitel gehören zum erblichen Adel, dessen Weitervergabe durch das reichsweite Erbrecht geregelt wird und somit keine Einschränkungen durch Legislatur erfahren.

\subsection{Amtlicher Adel}\footnote{Zweiter Reichsständezusatzbeschluss vom 23. Mai 1920}
Dem amtlichen Adel gehören jene Titel an, die an ein Amt gebunden sind und somit stets an den derzeitigen Amtsträger verliehen werden.

\subsection{Titularadel}\footnote{Zweiter Reichsständezusatzbeschluss vom 23. Mai 1920}
Titel gehören dem Titularadel an, sofern sie ohne jegliches Land oder sonstige Bevollmächtigungen verliehen werden.

\section{Judikative}\footnote{Reichsjudikativbeschluss vom 23. Mai 1920}
\subsection{Gerlichtliche Instanzen}
\begin{enumerate}[(1)]
    \item Die Instanzen im Deutschen Kaiserreich sind wie folgt geregelt:
    \begin{enumerate}[1.]
        \item Jegliche Fälle, die unter die Bezirksgerichtsbarkeit fallen, werden von einem Amtsgericht bearbeitet. In diesem sitzt ein, durch die Bezirksregierung gewählter Richter.
        \item Fälle der Landesgerichtsbarkeit werden von einem Landesgericht angehört. Dessen Richterschaft, bestehend aus zwei Richtern und einem Richterschaftsvorsitz, wird von dem amtierenden Herzog gewählt.
        \item Alles, das in den Bereich der Reichsgerichtsbarkeit fällt, unterliegt der Bearbeitung durch den Reichsgerichtshof. Dieser besteht aus vier Richtern einem Richterschaftsvorsitz. Die Richter werden durch den Kaiser ernannt.
    \end{enumerate}
    \item Das Reichsverfassungsgericht, bestehend aus dem Kaiser als Richterschaftsvorsitz, einem durch den Kaiser ernannten Richter und dem Staatssekretär der Justiz, gehört nicht zur Instanzenstruktur und
    bearbeitet lediglich Verfassungsbeschwerden und Urteile, in denen sich eine Partei seiner Grundrechte verletzt sieht.
    \item Damit es zu einem Gerichtsprozess kommt, muss das zuständige Gericht den Antrag der Klagenden prüfen und Anklage erheben.
    \item Man kann gegen die Erhebung und Nichterhebung der Anklage in Berufung gehen.
\end{enumerate}

\section{Repressivorgane}\footnote{Reichsrepressivbeschluss vom 23. Mai 1920}
\subsection{Reichspolizei}
\begin{enumerate}[(1)]
    \item Die Reichspolizei ist eine reichsweite Institution die der Bewahrung der öffentlichen Ordnung und Sicherheit dienlich ist.
    \item Der Reichspolizei sitzt der Polizeipräsident vor.
    \item Der Polizeipräsident wird durch den Staatssekretär des Innern eingesetzt und ist diesem direkt unterstellt.
\end{enumerate}

\subsection{Kaiserliche Streitkräfte}
\begin{enumerate}[(1)]
    \item Die Kaiserlichen Streitkräfte sind das höchste Repressivorgan des Deutschen Kaiserreichs.
    \item Sie unterstehen dem Generalstabschef.
    \item Es besteht allgemeine Wehrpflicht für alle männlichen, wehrfähigen Bürger ab einem Alter von achtzehn Jahren.
    \item Entzieht man sich der Wehrpflicht, muss man 100 Kaisermark zahlen.
    \item Im Krieg kann auf Beschluss des Kaisers hin die Wehrpflicht auf weitere Bevölkerungsgruppen ausgeweitet werden.
    \item Der Deutsche Kaiser ist Oberbefehlshaber der Streitkräfte.
    \item Es wird zwischen drei Teilstreitkräften unterschieden:
    \begin{enumerate}[1.]
        \item Das Kaiserliche Heer, welches der Reichsheeresleitung untersteht.
        \item Die Kaiserliche Marine, welche der Reichsadmiralität untersteht.
        \item Die Kaiserliche Luftwaffe, welche dem Reichsluftwaffenkommando untersteht.
    \end{enumerate}
    \item Die Dienstgradgruppen lauten mitsamt ihrer Befugnisse, sortiert nach aufsteigender Rangfolge, wie folgt:
    \begin{enumerate}[1.]
        \item Mannschaften, angehörig der Dienstgradgruppe VII, sind weder dazu befugt, allein aufgrund ihres Dienstgrades Befehle zu erteilen, noch sich
        selbst in Notstandssituationen zu Vorgesetzten erklären.
        \item Unteroffiziere ohne Portepee, angehörig der Dienstgradgruppe VI, können als Zug-, Kompanietrupp-, 
        Gruppen- oder Truppführer einberufen werden.
        \item Unteroffiziere mit Portepee, angehörig der Dienstgradgruppe V, haben gleiche Befugnisse wie die Unteroffiziere ohne Portepee.
        \item Leutnante, angehörig der Dienstgradgruppe IV, befinden sich in Ausbildung und werden daher nicht als Befehlshaber einer Teileinheit eingesetzt.
        \item Hauptleute, angehörig der Dienstgradgruppe III, werden entweder als Zugführer oder stellvertretender Kompaniechef oder im Stabsdienst
        als Unterstützung der Stabsoffiziere eingesetzt.
        \item Stabsoffiziere, angehörig der Dienstgradgruppe II, werden als Militärattachés, Verbindungsoffiziere oder Befehlshaber von Bataillonen, Regimenten oder Kompanien eingesetzt.
        \item Generale, angehörig der Dienstgradgruppe I, werden als Stabsleiter, Leiter der Teilstreitkräfte, Befehlshaber der Streitkräfte oder Kommandeure von Teileinheiten ab der Brigade eingesetzt.
    \end{enumerate}
    \item Die Dienstgrade der Mannschaften lauten wie folgt:
    \begin{enumerate}[1.]
        \item Soldat
        \item Gefreiter
        \item Obergefreiter
    \end{enumerate}
    \item Die Dienstgrade der Unteroffiziere ohne Portepee lauten:
    \begin{enumerate}[1.]
        \item Unteroffizier
    \end{enumerate}
    \item Die Dienstgrade der Unteroffiziere mit Portepee lauten:
    \begin{enumerate}[1.]
        \item Feldwebel
    \end{enumerate}
    \item Die Dienstgrade der Leutnante lauten wie folgt:
    \begin{enumerate}[1.]
        \item Leutnant
        \item Oberleutnant
    \end{enumerate}
    \item Die Dienstgrade der Hauptleute lauten:
    \begin{enumerate}[1.]
        \item Hauptmann
    \end{enumerate}
    \item Die Dienstgrade der Stabsoffiziere lauten wie folgt:
    \begin{enumerate}[1.]
        \item Major
        \item Oberst
    \end{enumerate}
    \item Die Dienstgrade der Generale lauten wie folgt:
    \begin{enumerate}[1.]
        \item Generalleutnant
        \item General
        \item Generalfeldmarschall
    \end{enumerate}
    \item Generale sind die Befehlshaber der, ihnen zugeteilten Teilstreitkraft.
    \item Der Generalstabschef muss Generalfeldmarschall sein.
\end{enumerate}

\subsection{Polizeiliche Kontrollen}
\begin{enumerate}[(1)]
    \item Polizeiliche Kontrollen werden durch diensthabende Beamte der Reichspolizei durchgeführt.
    \item Sie dürfen Personen, die illegale Gegenstände mit sich führen, töten sofern diese der dritten Aufforderung, sie abzugeben, nicht nachgehen.
    \item Auch dürfen sie Personen, die kein gültiges Visum mit sich tragen und dennoch sich nach der dritten Aufforderung noch auf dem Gebiet aufhalten, einsperren.
    \item Ausnahmen hierzu bilden eingeladene Personen.
    \item Widersetzt man sich der Polizeigewalt, so darf man durch einen diensthabenden Reichspolizisten getötet werden.
\end{enumerate}

\subsection{Strafverfolgung}
\begin{enumerate}[(1)]
    \item Entzieht man sich der Strafverfolgung des Reichs, wird man auf dem Gebiet für vogelfrei erklärt, es sei denn, man stellt sich freiwillig vor das zuständige Gericht.
    \item Man darf sich ebenfalls nicht der Strafverfolgung verbündeter Reiche auf dem Gebiet des Kaiserreichs entziehen.
    \item Absatz 1f. treten nur dann ein, wenn einer Person kein Asyl gewährt wurde.
    \item Einer Person darf Asyl gewährt werden, wenn sie in einem anderen Staat eine Straftat beging, die auf dem Gebiet des Kaiserreichs nicht als Verbrechen anerkannt wird.
    \item Das Recht auf Asyl darf einer Person jederzeit entzogen werden
    \item Behindert man die Justiz absichtlich, so muss man eine Bußgeldstrafe in Höhe von 100 Kaisermark zahlen. 
\end{enumerate}

\end{document}