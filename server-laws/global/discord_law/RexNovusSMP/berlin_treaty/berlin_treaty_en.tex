\documentclass{article}
\usepackage[utf8]{inputenc}
\usepackage{enumerate}
\usepackage{ragged2e}
\usepackage{etoc}
\usepackage{amsmath}
\usepackage{graphicx}
\graphicspath{ {./images/} }

\renewcommand{\thesection}{}
\newcommand{\sent}[1]{$^{#1}$}
\counterwithout{subsection}{section}
\renewcommand{\thesubsection}{§\arabic{subsection}}

\title{Treaty of Berlin for the Enforcement of International Jurisdiction}
\author{Emperor Friedrich IV of Prussia}
\date{June 29, 1920}

\begin{document}
\maketitle
\begin{center}
    \includegraphics[scale=.15]{dr_wappen}
\end{center}
\begin{center}
    \textbf{Between the Parties\\}\textbf{\\}
    The Chinese Empire, hereinafter referred to as \textbf{``China''\\}
    The \textbf{German Empire\\}
    \textbf{Kazakhstan\\}
    \textbf{Keosu Empire\\}
    The German-Keosunian Special Administrative Zone of the Monastic Republic of Ratosurya, hereinafter referred to as \textbf{``Monastic Republic of Ratosurya''\\}
    The Autonomous High Kingdom of Russia under German Crown, hereinafter referred to as \textbf{``Autonomous High Kingdom of Russia''\\}
    The \textbf{Valkyrie Trading Corporation\\}
    \textbf{Wisconsin\\}\textbf{\\}

    Collectively hereinafter referred to as \textbf{``The Parties''}, \textbf{``The States''}, \textbf{``The Contracting Parties''}, \textbf{``The Treaty States''}
\end{center}
\newpage
\topskip0pt
\vspace*{\fill}
\begin{Center}
\textbf{1st Version}
\vspace*{\fill}
\end{Center}
\newpage
\tableofcontents
\newpage
\section{Contractual Definitions}

\subsection{Contractual Validity}
\begin{enumerate}[(1)]
    \item The parties to this treaty are the signing states.
    \item The following treaty is valid until an agreement to terminate the treaty is drafted and signed by all contracting parties.
    \item ${^1}$Decisions under this treaty must be approved by a majority of the contracting members. ${^2}$This does not apply to decisions of the underlying courts.
    \item Contractual recognition by autonomous states is only effective upon signature by their superior sovereign contracting nation.
    % \item The designation of the states corresponds to their names at the time of initial signing.
    \item ${^1}$Only lawful successors of the states have the right to continue the membership of their predecessor in the treaty without signing. ${^2}$A lawful successor is one who has been recognized as such by the members of the Imperial Pact in accordance with the provisions of the Imperial Pact.
    \item This also implies full recognition of the entire content.
    \item The signing of the Treaty of Berlin and the validity of the signature come into effect only after the signing of the Imperial Pact.
\end{enumerate}

\subsection{Basic Obligations of the Signatory Parties}
\begin{enumerate}[(1)]
    \item The signatory parties are obliged to recognize international jurisdiction and to establish a constitution that explicitly acknowledges the contents of this treaty.
    \item If a state cannot meet its obligations within the first two months after signing, the validity of the Imperial Pact and the Treaty of Berlin temporarily lapses for that state until these obligations are fulfilled.
\end{enumerate}

\section{Berlin Laws}
\subsection{Nature of the Laws}
\begin{enumerate}[(1)]
    \item The following laws defined in this section are also referred to as ``Berlin Laws''.
    \item Only crimes that violate the Berlin Laws can be the subject of proceedings before the International Criminal Court.
    \item All contracting parties are obliged to incorporate the laws listed here into their constitution or other legal texts fully applicable within the state.
    \item Jurisdiction is carried out in accordance with the provisions of Section 3.
    \item Attempt, complicity, and knowledge are punishable in any case.
    \item Lack of or faulty knowledge of the applicable Berlin Laws neither justifies a reduction of guilt nor exoneration.
\end{enumerate}

\subsection{Murder}
\begin{enumerate}[(1)]
    \item Committed murder is when a person is killed using dangerous means and for base motives.
    \item Lawful executions are not recognized as murder.
\end{enumerate}

\subsection{Manslaughter}
\begin{enumerate}[(1)]
    \item Whoever kills a person without being a murderer commits manslaughter.
    \item Lawful executions are not recognized as manslaughter.
\end{enumerate}

\subsection{Theft}
If one steals, removes, or robs goods from or on the territory of a contracting party, one must return the goods along with their double value.

\subsection{Defamation of Factional Insignia and Symbols}
\begin{enumerate}[(1)]
    \item The intentional defamation or unauthorized removal of state symbols and insignia of a contracting party constitutes defamation of factional insignia and symbols.
    \item This also includes the unauthorized wearing of official decorations and uniforms of a contracting party.
\end{enumerate}

\subsection{Fraud}
Fraud is defined as enriching oneself or a third party by pretending false facts.

\subsection{Money Laundering}
The manufacture of coins or securities without the authorization of the state bank of the affected contracting party is punishable.

\subsection{Handling Stolen Goods}
\begin{enumerate}[(1)]
    \item The sale of stolen goods is punishable.
    \item Professional handling of stolen goods is a capital crime.
\end{enumerate}

\subsection{Debts}
Any debts owed to a contracting party or a citizen of the contracting party must be settled within a period of 10 days\footnote{Real time}.

\subsection{Entering the Territory}
\begin{enumerate}[(1)]
    \item Entry into the territory of a contracting party is only permitted with explicit authorization or proof of a valid entry permit.
    \item Violations legitimize a shoot-to-kill order without prior warning.
\end{enumerate}

\subsection{Espionage}
Tactical reconnaissance and espionage on the territory of a contracting party are strictly prohibited.

\subsection{High Treason}
\begin{enumerate}[(1)]
    \item A person is considered a traitor if they:
    \begin{enumerate}[1.]
        \item Spread state secrets without explicit authorization or attempt to access them unlawfully.
        \item Intentionally cause a weakening of the state.
        \item Disobey orders from the head of state or government.
    \end{enumerate}
    \item It is not considered high treason to stand against the state in an emergency situation caused by the state without justifiable reason.
\end{enumerate}

\subsection{Violations of the Treaties}
Anyone who violates the provisions of the Imperial Pact or the Treaty of Berlin is liable to prosecution.

\section{Jurisdiction}
\subsection{Terminological Definitions}
\begin{enumerate}[(1)]
    \item The party making the accusations is referred to as the plaintiff.
    \item The party against whom the accusations are made is referred to as the defendant.
    \item The party acting as judges to mediate and bring about a fair settlement is referred to as the judiciary.
    \item ${^1}$The presiding judge is the head of the judiciary, if it consists of more than one person. ${^2}$The presiding judge grants the floor and is responsible for conducting the proceedings. In the event of a tie, the presiding judge is authorized to cast a deciding vote.
\end{enumerate}

\subsection{Obligations of the Treaty States Regarding Their Constitution}
\begin{enumerate}[(1)]
    \item ${^1}$Each treaty state must provide the defendant with unrestricted opportunity to defend against judgments and accusations, unless these have been rendered by the highest instance. ${^2}$The highest instance for violations concerning the Berlin Laws is the International Criminal Court. ${^3}$The plaintiff must also be provided with unrestricted opportunity to bring accusations to court and to appeal judgments.
    \item The constitution must allow the defendant and the plaintiff to appeal against judgments that are not entirely or at all comprehensible, unless, as defined in paragraph 1, rendered by the highest instance.
    \item The constitution must also ensure that the mentioned states cannot exploit precedent rules to enforce a conviction.
    \item The use of legal loopholes must be legal in the states.
    \item Furthermore, the defendant and the plaintiff must not be subject to arbitrary justice.
\end{enumerate}

\subsection{Conviction}
\begin{enumerate}[(1)]
    \item To justify a conviction, the defendant must have committed the act intentionally or through gross or ordinary negligence.
    \item The sentence must be based on the severity of the act and the past criminality of the offender. The sentence must not exceed or fall short of the legal guidelines and must not be cumulated based on the number of violations.
    \item In the case of repeat offenses, it is at the discretion of the competent court, depending on the frequency and severity of the offense, whether the same or a harsher sentence should apply.
    \item The listed sentences serve only as a guideline and are therefore not mandatory.
\end{enumerate}

\subsection{International Criminal Court}
\begin{enumerate}[(1)]
    \item Jurisdiction regarding the Berlin Laws lies ultimately with the International Criminal Court.
    \item Each treaty state of the Treaty of Berlin appoints a judge.
    \item ${^1}$The president of the International Criminal Court presides over the judiciary. ${^2}$This position may only be held by individuals with comprehensive legal knowledge and knowledge of the Imperial Pact and the Treaty of Berlin.
    \item The International Criminal Court will be established in an area declared as neutral territory between all contracting parties.
\end{enumerate}

\subsection{International Constitutional Court}
\begin{enumerate}[(1)]
    \item The International Constitutional Court is responsible for receiving and handling complaints regarding the treaties and national constitutions of the contracting parties, as well as decisions that conflict with the rights granted to the parties under the Treaty of Berlin.
    \item The International Constitutional Court is also an instance superior to the constitutional courts of the contracting parties.
    \item ${^1}$The president of the International Constitutional Court presides over the judiciary. ${^2}$This position may only be held by individuals with comprehensive legal knowledge and knowledge of the Imperial Pact and the Treaty of Berlin.
    \item To avoid the enforcement of particular interests, two additional judges are elected monthly by the contracting states. An iteration ban applies.
\end{enumerate}

\subsection{Procedure in Case of Judicial Bias}
\begin{enumerate}[(1)]
    \item Only judges who are neither witnesses, nor defenders, defendants, or plaintiffs in the case may participate in the trial.
    \item If a contracting state is the plaintiff and no other contracting party is involved as the defendant, the judiciary nevertheless acts as the plaintiff.
    \item Otherwise, the involved parties must act as defendants and plaintiffs and may not claim their seats in the judiciary.
    \item In the case of the International Constitutional Court, unbiased judges must replace the biased judges for the duration of the bias.
    \item If the presiding judge is biased, or if another biased judge needs to be replaced, the contracting parties must elect a deputy to take on the role during this time.
\end{enumerate}

\subsection{Evidence Procedure}\label{zeugen}
\begin{enumerate}[(1)]
    \item Individuals may be called to the witness stand.
    \item These individuals may be questioned under the given rules.
    \item These rules are as follows:
        \begin{enumerate}[1.]
            \item Witnesses are under oath as soon as they speak their first word in the witness stand.
            \item Witnesses must therefore answer everything truthfully.
            \item Any inaccurate statements by witnesses will not be included in the record (see \ref{eordnung} paragraph 5).
            \item Witnesses may attend the trial only during their testimony. Before and after, they may only attend again at the pronouncement of the judgment.
        \end{enumerate}
    \item Witnesses, as well as evidence, must be registered before the trial.
    \item If evidence is not available at the start of the trial, its procurement must be registered. The judge must then schedule another trial to examine this evidence.
    \item Evidence and witnesses must not be tampered with.
\end{enumerate}

\subsection{Lawyers}
One may hire a lawyer. However, it should be noted that there is no right to a public defender.

\subsection{Lawyer Admission}
\begin{enumerate}[(1)]
    \item Any citizen of the contracting parties is entitled to a lawyer's admission even without a degree.
    \item In case of abuse or inadequate or faulty performance of the lawyer's duties, the admission can be revoked.
    \item The revocation can be overturned by passing a state exam recognized by the German Empire. If it has already been passed, it must be repeated and passed again.
\end{enumerate}
    
\subsection{Objection Rules}\label{eordnung}
\begin{enumerate}[(1)]
    \item Objections are allowed and form an exception to \ref{gordnung} paragraph 1.
    \item They can be overruled by the judiciary.
    \item Once an objection is overruled, it may not be applied again to the same statement.
    \item The announcement of an objection must always be followed by the announcement of the reason.
    \item Legally permissible reasons are:
        \begin{enumerate}[1.]
            \item Not significant/unclear/ambiguous: The statement or question is inadmissible due to its non-significant nature.
            \item Already answered: The same question has been asked multiple times despite being already answered.
            \item Unproven assumption: The lawyer claims something without relying on existing evidence.
            \item Calls for speculation: The lawyer asks the witness to speculate.
            \item Too many questions: The lawyer asks more than one question at the same time.
            \item Lack of knowledge: The witness's knowledge of the subject matter is inadequately demonstrated.
            \item Without priority: The question is not relevant to the trial or the interrogation.
            \item Rumor: The party's response is based on out-of-court statements.
            \item Questions state authority: A party disputes, questions, or insults state authority or the authority of the Emperor. If this objection is sustained, the person who made the statement will subsequently be tried for violating §5 SDelGB.
            \item Challenge: The raised objection is not lawful because its subject matter does not constitute a violation of the interrogation rules in this specific context.
        \end{enumerate}
    \item If an objection is sustained, the questioner must proceed with the next question. The witness may not answer the previous question or, if they have already answered or still answer, their statement will be struck. If a judge raises this objection, it is immediately sustained unless the presiding judge objects.
    \item The presiding judge may only overrule the objection if the subject of the objection does not constitute a violation of the interrogation rules in any conceivable context.
\end{enumerate}

\subsection{Plea}\label{pleed}
\begin{enumerate}[(1)]
    \item Each party obliged to testify must defend its position through a closing plea.
    \item This plea must comply with the applicable conventions.
    \item The conclusion of the plea is the client's statement if it is the defendant's closing plea, and the sentencing recommendation.
    \item A sentencing recommendation may only be made by a lawyer.
\end{enumerate}

\subsection{Trial Procedure}\label{verlauf}
The Treaty of Berlin prescribes the following procedure for court proceedings:
\begin{enumerate}[1.]
    \item All parties except the judiciary enter the room.
    \item The judiciary gathers. During this time, everyone present must stand.
    \item The presiding judge opens the proceedings, and the other judges take their seats.
    \item The defendant reads the indictment.
    \item The plaintiff must present the facts of the case from their perspective.
    \item The defendant has the floor and may present their account of the facts.
    \item From then on, the presiding judge decides who has the floor.
    \item Once all evidence and statements from both parties have been presented, the closing pleas must be held in accordance with \ref{pleed}.
    \item The judiciary retreats and deliberates in a separate discussion. The sentence is discussed and subsequently decided. In the event of a tie, the presiding judge casts a second vote.
    \item The judiciary re-enters the room, with everyone standing again, and then announces the sentence.
    \item All participants must remain standing and may not leave the room until the last judge has left.
\end{enumerate}

\subsection{Duty of Order}\label{gordnung}
\begin{enumerate}[(1)]
    \item One may not speak without being called upon.
    \item One must dress appropriately for the trial. Accordingly, attendees may not wear headgear and must wear a suit in an appropriate color.
    \item Judges must wear black judicial robes.
    \item In the case of the International Constitutional Court, judges must wear red suits.
    \item Violations of the court rules, considering \ref{zeugen} and \ref{verlauf}, will be fined if a warning has already been issued. If, at the discretion of the judiciary, there are too many violations, they may expel the guilty party from the courtroom regardless of their relevance to the trial and continue the proceedings in their absence.
\end{enumerate}

\subsection{Court Summons}
\begin{enumerate}[(1)]
    \item Once a proceeding has been confirmed, a court date can be set by agreement with both parties. This is considered an extraordinary summons.
    \item If the court president sets a date, both parties must be clearly informed of the hearing date in writing. This is an ordinary summons.
    \item The date and place of a hearing must be announced at least twelve hours before the start of the trial.
    \item A request for a postponement can be filed up to two hours before the start of the trial.
    \item If this request is granted by the presiding judge of the proceeding, the trial will be postponed.
    \item Otherwise, or if no request exists, the parties must appear, or the case will be heard in their absence.
    \item If neither party appears, the decision will be in favor of the defendant.
    \item Each party absent according to paragraph 6 faces a fine.
    \item Absent judges have no voting rights in the proceeding.
    \item Requests for postponement made by judges can be denied by the court president if the postponement exceeds three days or if the trial has already been postponed by more than two weeks.
    \item The judiciary can convene if at least three judges are present.
    \item Court summons must be confirmed by the court president. If this is not done within the next three days, it is still valid.
\end{enumerate}

\subsection{Procedural Errors}
In case of a procedural error, the court process must be repeated.

\subsection{Precedent Rules}\label{prec}
\begin{enumerate}[(1)]
    \item If the International Criminal Court decides that a legal exception case should be considered a crime, it must be immediately incorporated into the Berlin Laws, and if, at the discretion of the judiciary, there is an awareness of the violation of moral norms by the defendant and an act against the physical integrity of a person or the structural integrity of a contracting party, the punishment must be judged accordingly.
    \item For the enacted law to remain in effect, it requires a simple majority of votes cast within the next two weeks.
    \item If the law is rejected, convictions based on this draft law remain valid.
\end{enumerate}

\subsection{Jurisdiction of the States}
Cases where a state is the defendant are exclusively heard before the international courts of this treaty.

\subsection{Extrajudicial Legislative Decisions}
\begin{enumerate}[(1)]
    \item The enactment of Berlin Laws and other decrees outside the framework of \ref{prec} requires the majority approval of the contracting parties.
    \item Votes can only be taken in the meeting convened for the decision on the matter.
    \item The meeting and its agenda are announced by the court president at least two days in advance.
    \item If an urgent request is made, the court president may shorten the period mentioned in paragraph 2 to twelve hours.
    \item A request for postponement can be filed up to two hours before the start of the meeting.
    \item The meeting is quorate if at least one-third of the members are represented.
    \item In the case of non-judicial decision-making, the assembly members are the contracting members and the judiciary.
\end{enumerate}

\end{document}