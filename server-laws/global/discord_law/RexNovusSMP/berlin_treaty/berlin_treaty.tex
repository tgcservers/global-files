\documentclass{article}
\usepackage[utf8]{inputenc}
\usepackage{enumerate}
\usepackage{ragged2e}
\usepackage{etoc}
\usepackage{amsmath}
\usepackage{graphicx}
\graphicspath{ {./images/} }

\renewcommand{\thesection}{}
\newcommand{\sent}[1]{$^{#1}$}
\counterwithout{subsection}{section}
\renewcommand{\thesubsection}{§\arabic{subsection}}

\title{Abkommen von Berlin zur Durchsetzung einer internationalen Jurisdiktion}
\author{Kaiser Friedrich IV. von Preußen}
\date{29. Juni 1920}

\begin{document}
\maketitle
\begin{center}
    \includegraphics[scale=.15]{dr_wappen}
\end{center}
\begin{center}
    \textbf{Zwischen den Parteien\\}\textbf{\\}
    Das Chinesische Reich, im Folgenden bezeichnet als \textbf{``China''\\}
    Das \textbf{Deutsche Kaiserreich\\}
    \textbf{Kasachstan\\}
    Das \textbf{Keosunische Reich\\}
    Die Deutsch-Keosunische Sonderverwaltungszone der Mönchsrepublik Ratosurya, im Folgenden bezeichnet als \textbf{``Mönchsrepublik Ratosurya''\\}
    Das Autonome Hochkönigreich Russland unter Deutscher Krone, im Folgenden bezeichnet als \textbf{``Autonomes Hochkönigreich Russland''\\}
    Die \textbf{Valkyrie Trading Corporation\\}
    \textbf{Wisconsin\\}\textbf{\\}

    Gemeinsam im Folgenden \textbf{``Die Parteien''}, \textbf{``Die Staaten''}, \textbf{``Die Vertragsparteien''}, \textbf{``Die Abkommensstaaten''}
\end{center}
\newpage
\topskip0pt
\vspace*{\fill}
\begin{Center}
\textbf{1. Fassung}
\vspace*{\fill}
\end{Center}
\newpage
\tableofcontents
\newpage
\section{Vertragliche Definitionen}

\subsection{Vertragliche Gültigkeit}
\begin{enumerate}[(1)]
    \item Die Parteien dieses Abkommens sind die unterzeichnenden Staaten.
    \item Der nachfolgende Vertrag ist gültig, bis von allen Vertragsparteien ein Abkommen zur Aufhebung des Abkommens aufgesetzt und unterschrieben wird.
    \item ${^1}$Entscheidungen im Zuge dieses Abkommens müssen von der Mehrheit der Vertragsmitglieder bewilligt werden. ${^2}$Dies gilt nicht für Entscheidungen der zugrundeliegenden Gerichte.
    \item Die vertragliche Anerkennung durch autonome Staaten erfolgt nur durch Unterschrift durch die, ihnen übergeordnete souveräne Vertragsnation.
    % \item Die Bezeichnung der Staaten entspricht deren Namen zum Zeitpunkt der erstmaligen Unterzeichnung.
    \item ${^1}$Nur rechtmäßige Nachfolger der Staaten haben das Recht, die Mitgliedschaft ihres Vorgängers im Vertrag fortzuführen, ohne zu unterzeichnen. ${^2}$Als rechtmäßiger Nachfolger gilt, wer von den Mitgliedern des Kaiserpakts den Bestimmungen des Kaiserpakts entsprechend als solcher anerkannt wurde.
    \item Dies bedeutet jedoch auch die damit einhergehende vollständige Anerkennung des gesamten Inhalts.
    \item Die Unterzeichnung des Abkommens von Berlin und die Gültigkeit der Signatur tritt erst nach Unterzeichnung des Kaiserpakts in Kraft. 
\end{enumerate}

\subsection{Grundlegende Verpflichtungen der unterzeichnenden Parteien}
\begin{enumerate}[(1)]
    \item Die unterzeichnenden Parteien sind zur Anerkennung der internationalen Rechtsprechung und der einhergehenden Aufsetzung einer Verfassung verpflichtet, die die Inhalte dieses Abkommens ausdrücklich anerkennt.
    \item Kann ein Staat in den ersten zwei Monaten nach Unterzeichnung nicht den Verpflichtungen nachgehen, verfallen zeitweise für diesen Staat die Gültigkeit des Kaiserpakts und des Abkommens zu Berlin, bis diese Verpflichtungen erfüllt wurden.
\end{enumerate}

\section{Berliner Gesetze}
\subsection{Natur der Gesetze}
\begin{enumerate}[(1)]
    \item Die nachfolgenden Gesetze, die in diesem Abschnitt definiert werden, werden auch als ``Berliner Gesetze'' bezeichnet.
    \item Nur Straftaten, die gegen die Berliner Gesetze verstoßen, können Gegenstand von Prozessen vor dem Internationalen Strafgerichtshof werden.
    \item Alle Vertragsparteien sind dazu verpflichtet, die hier aufgeführten Gesetze sinngemäß in ihre Verfassung oder anderweitige, im Staat vollständig gültige Gesetzestexte aufzunehmen.
    \item Die Rechtsprechung erfolgt gemäß den Bestimmungen aus Abschnitt 3.
    \item Der Versuch, die Mittäterschaft und Mitwisserschaft sind in jedem Fall strafbar.
    \item Mangelnde oder fehlerhafte Kenntnisse von den geltenden Berliner Gesetzen rechtfertigt weder Minderung der Schuld noch Schuldfreiheit.
\end{enumerate}

\subsection{Mord}
\begin{enumerate}[(1)]
    \item Mord beging, wer unter Anwendung gemeingefährlicher Mittel und aus niederen Beweggründen heraus eine Person tötet.
    \item Gesetzesgemäß verordnete Hinrichtungen werden nicht als Mord anerkannt.
\end{enumerate}

\subsection{Totschlag}
\begin{enumerate}[(1)]
    \item Wer einen Menschen tötet, ohne Mörder zu sein, begeht Totschlag.
    \item Gesetzesgemäß verordnete Hinrichtungen werden nicht als Totschlag anerkannt.
\end{enumerate}

\subsection{Diebstahl}
Entwendet, stiehlt oder raubt man vom oder auf dem Territorium einer Vertragspartei, so muss man die Ware mitsamt ihres doppelten Warenwerts, zurückerstatten.

\subsection{Verunglimpfung fraktioneller Insignien und Symbole}
\begin{enumerate}[(1)]
    \item Als Verunglimpfung fraktioneller Insignien und Symbole gilt die absichtliche Verunglimpfung oder unerlaubte Entfernung staatlicher Symbole und Insignien einer Vertragspartei.
    \item Hierzu zählt zusätzlich das unerlaubte Tragen von offiziellen Orden und Uniformen einer Vertragspartei.
\end{enumerate}

\subsection{Betrug}
Als Betrug wird die Bereicherung von sich oder einer dritten Person unter Vorspiegelung falscher Tatsachen bezeichnet.

\subsection{Geldwäsche}
Die Herstellung von Münzen oder Wertpapieren ohne Genehmigung der staatlichen Bank der betroffenen Vertragspartei ist strafbar.

\subsection{Hehlerei}
\begin{enumerate}[(1)]
    \item Der Verkauf von Diebesware ist strafbar.
    \item Gewerbsmäßige Hehlerei ist ein Kapitalverbrechen.
\end{enumerate}

\subsection{Schulden}
Jegliche Schulden, die man bei einer Vertragspartei oder einem Bürger der Vertragspartei hat, müssen innerhalb einer Frist von 10 Tagen\footnote{Echtzeit} beglichen werden.

\subsection{Betreten des Staatsgebiets}
\begin{enumerate}[(1)]
    \item Das Betreten des Staatsgebiets einer Vertragspartei darf nur auf ausdrückliche Genehmigung oder unter Nachweis einer gültigen Einreiseerlaubnis erfolgen.
    \item Zuwiderhandlungen legitimieren einen Schießbefehl ohne Vorwarnung.
\end{enumerate}

\subsection{Spionage}
Taktische Aufklärung und Spionage auf dem Gebiet einer Vertragspartei sind strengstens untersagt.

\subsection{Hochverrat}
\begin{enumerate}[(1)]
    \item Als Hochverräter gilt, wer:
    \begin{enumerate}[1.]
        \item Staatsgeheimnisse ohne ausdrückliche Genehmigung verbreitet oder versucht, auf diese unerlaubt zuzugreifen.
        \item Eine absichtliche Schwächung des Staates herbeiführt.
        \item Befehle des Staatsoberhaupts oder Regierungschefs verweigert.
    \end{enumerate}
    \item Es gilt nicht als Hochverrat, sich aus einer Notlage, die durch den Staat ohne nachvollziehbaren Grund herbeigeführt wurde, gegen diesen zu stellen.
\end{enumerate}

\subsection{Vertragsbruch}
\begin{enumerate}[(1)]
    \item Wer die festgesetzten Bestimmungen einen unterschriebenen und inkraftgetretenen Vertrags verletzt, begeht Vertragsbruch.
    \item ${^1}$Ein Vertrag tritt erst dann inkraft, wenn dieser von allen Parteien unterschrieben wurde. ${^2}$Dies gilt auch für vertragliche Änderungen.
    \item ${^1}$Verträge verlieren ihre Gültigkeit, sofern eine Passage des Vertrags gegen die nationale Verfassung, den Kaiserpakt oder das Abkommen von Berlin verstößt. ${^2}$Sofern im Vertrag explizit geregelt, fallen nur die betroffenen Passagen weg.
    \item Es steht Staaten frei, die Beglaubigung durch eine dritte Partei zur Gültigkeit eines Vertrags zu fordern.
\end{enumerate}

\subsection{Verstöße gegen die Abkommen}
Strafbar macht sich, wer gegen die Vertragsbestimmungen des Kaiserpakts oder des Abkommens von Berlin verstößt.

\section{Jurisdiktion}
\subsection{Begriffliche Definitionen}
\begin{enumerate}[(1)]
    \item Als Klagende wird die Partei bezeichnet, die die Vorwürfe erhoben hat.
    \item Als Beklagte wird die Partei bezeichnet, die Gegenstand der Vorwürfe ist.
    \item Als Richterschaft wird die Partei bezeichnet, die zwischen den Parteien mit der Aufgabe, eine gerechte Einigung herbeizuführen, vermittelt.
    \item ${^1}$Als Richterschaftsvorsitz wird der Vorsitzende der Richterschaft bezeichnet, sofern die Richterschaft aus mehr als einer Person besteht. ${^2}$Der Richterschaftsvorsitz erteilt das Wort und ist für die Führung des Prozesses verantwortlich. Im Falle einer Stimmgleichheit ist er ermächtigt, die Entscheidung mittels einer zusätzlichen Stimme zu fällen.
\end{enumerate}

\subsection{Verpflichtungen der Abkommensstaaten in Hinsicht auf die eigene Verfassung}
\begin{enumerate}[(1)]
    \item ${^1}$Jeder Abkommensstaat muss der Beklagten die uneingeschränkte Möglichkeit bieten, sich gegen Urteile und Vorwürfe zu wehren, sofern diese nicht von der letzten Instanz gefällt wurden. ${^2}$Die letzte Instanz bei Verstößen, die die Berliner Gesetze betreffen, ist der Internationale Strafgerichtshof. ${^3}$Auch muss jedem Klagenden die uneingeschränkte Möglichkeit geboten werden, Vorwürfe vor Gericht zu bringen und gegen Urteile in Berufung zu gehen.
    \item Die Verfassung muss der Beklagten und der Klagenden die Möglichkeit bieten, gegen Urteile, die nicht gänzlich oder überhaupt nicht nachvollziehbar sind, in Berufung gehen zu können, sofern wie in Absatz 1 definiert, nicht von der höchsten Instanz gefällt.
    \item Ebenfalls muss Bestandteil der Verfassungen sein, dass erwähnte Staaten Präzedenzfallregelungen dahingehend nicht ausnutzen können als dass sie diese als Möglichkeit nutzen, um einen Schuldspruch zu erzwingen.
    \item Die Nutzung von Gesetzeslücken muss in den Staaten legal sein.
    \item Weiterhin darf die Beklagte und der Klagende keiner willkürlichen Rechtsprechung ausgesetzt sein.
\end{enumerate}

\subsection{Verurteilung}
\begin{enumerate}[(1)]
    \item Um eine Verurteilung zu rechtfertigen, muss durch die Beklagte die Tat vorsätzlich oder im Zuge grober, beziehungsweise herkömmlicher Fahrlässigkeit, begangen worden sein.
    \item Der Strafsatz muss sich nach der Schwere der Tat und der vergangenen Straffälligkeit des Täters richten. Hierbei darf der Strafsatz nicht die gesetzlichen Vorgaben überschreiten oder unterschreiten und auch nicht auf Grundlage der Anzahl der Verstöße aufsummiert werden.
    \item Bei Wiederholungstaten liegt es je nach Häufigkeit und Schwere der Tat im Ermessen des zuständigen Gerichts, ob weiterhin derselbe oder ein verhärteter Strafsatz geltend gemacht werden sollte.
    \item Die aufgeführten Strafsätze dienen lediglich zur Orientierung und sind daher nicht verpflichtend.
\end{enumerate}

\subsection{Internationaler Strafgerichtshof}
\begin{enumerate}[(1)]
    \item Die Rechtsprechung in Hinsicht auf die Berliner Gesetze unterliegen in höchster Instanz dem Internationalen Strafgerichtshof.
    \item Jeder Vertragsstaat des Abkommens von Berlin stellt einen Richter.
    \item ${^1}$Der Präsident am Internationalen Strafgerichtshof verfügt über den Richterschaftsvorsitz. ${^2}$Diese Position dürfen ausschließlich Personen mit umfassenden Rechtskenntnissen und Kenntnissen des Kaiserpakts und des Abkommens von Berlin einnehmen.
    \item Der Internationale Strafgerichtshof wird in einem Gebiet errichtet, das zum neutralen Gebiet zwischen allen Vertragsparteien erklärt wird.
\end{enumerate}

\subsection{Internationales Verfassungsgericht}
\begin{enumerate}[(1)]
    \item Der Aufgabenbereich des Internationalen Verfassungsgerichts umfasst die Entgegennahme und Bearbeitung von Beschwerden bezüglich der Abkommen und nationaler Verfassungen der Vertragsparteien, sowie von Entscheidungen, die im Konflikt mit den Rechten stehen, die den Parteien im Abkommen von Berlin zugesichert werden.
    \item Das Internationale Verfassungsgericht gilt ebenfalls als Instanz, die den Verfassungsgerichten der Vertragsparteien übergeordnet ist.
    \item ${^1}$Der Präsident am Internationalen Verfassungsgericht verfügt über den Richterschaftsvorsitz. ${^2}$Diese Position dürfen ausschließlich Personen mit umfassenden Rechtskenntnissen und Kenntnissen des Kaiserpakts und des Abkommens von Berlin einnehmen.
    \item Zur Meidung der Durchsetzung von Partikularinteressen werden zwei weitere Richter von den Vertragsstaaten monatlich neu gewählt. Es gilt hierbei das Iterationsverbot.
\end{enumerate}

\subsection{Vorgehen im Falle richterlicher Befangenheit}
\begin{enumerate}[(1)]
    \item Es dürfen nur die Richter am Prozess teilnehmen, die weder als Zeuge, noch als Verteidiger, Beklagte oder Klagende in diesem auftreten.
    \item Ist ein Vertragsstaat die Klagende Partei und ist keine andere Vertragspartei als Beklagte involviert, fungiert die Richterschaft dennoch als Kläger.
    \item Andernfalls müssen die beteiligten Parteien als Beklagte und Klagende auftreten und dürfen ihren Sitz in der Richterschaft nicht beanspruchen.
    \item Im Falle des Internationalen Verfassungsgerichts müssen für die Zeit der Befangenheit unbefangene Richter die befangenen Richter ersetzen.
    \item Ist der Richterschaftsvorsitz befangen, oder muss ein anderer befangener Richter ersetzt werden, so müssen die Vertragsparteien einen Stellvertreter wählen, der in dieser Zeit die Position wahrnimmt.
\end{enumerate}

\subsection{Beweisführung}\label{zeugen}
\begin{enumerate}[(1)]
    \item Man darf Personen in den Zeugenstand rufen.
    \item Diese darf man unter den gegebenen Regeln befragen.
    \item Diese Regeln lauten:
        \begin{enumerate}[1.]
            \item Die Zeugen stehen unter Eid, sobald sie ihr erstes Wort im Zeugenstand erheben.
            \item Die Zeugen müssen daher alles wahrheitsgemäß beantworten.
            \item Jegliche ungenauen Aussagen der Zeugen werden nicht ins Protokoll aufgenommen (siehe hierzu \ref{eordnung} Absatz 5).
            \item Zeugen dürfen dem Prozess nur während ihrer Aussage beiwohnen. Davor und danach dürfen sie diesem erst zur Urteilsverkündung wieder beiwohnen.
        \end{enumerate}
    \item Zeugen, wie auch Beweise müssen vor dem Prozess angemeldet werden.
    \item Wenn die Beweise bei Prozessbeginn noch nicht vorliegen, muss deren Beschaffung angemeldet werden. Der Richter muss dann einen weiteren Prozess anberaumen, in dem diese Beweismittel auch geklärt werden können.
    \item Beweise und Zeugen dürfen nicht manipuliert werden.
\end{enumerate}

\subsection{Anwälte}
Man darf einen Anwalt einstellen. Hierbei muss jedoch beachtet werden, dass kein Anrecht auf einen Pflichtverteidiger besteht.

\subsection{Anwaltszulassung}
\begin{enumerate}[(1)]
    \item Einem jeden Bürger der Vertragsparteien steht auch ohne Studium eine Anwaltszulassung zu.
    \item Bei Missbrauch oder mangelhafter, beziehungsweise fehlerhafter Wahrnehmung der Pflichten des Anwalts, kann einem die Zulassung entzogen werden.
    \item Die Entziehung kann durch ein bestandenes, durch das Deutsche Reich anerkanntes Staatsexamen widerrufen werden. Im Falle, dass man es bereits bestanden hat, muss man es wiederholen und erneut bestehen.
\end{enumerate}
    
\subsection{Einspruchsordnung}\label{eordnung}
\begin{enumerate}[(1)]
    \item Einsprüche sind erlaubt und bilden eine Ausnahme zu \ref{gordnung} Absatz 1.
    \item Sie können durch die Richterschaft abgewiesen werden.
    \item Bei einmaliger Ablehnung eines Einspruchs darf dieser nicht auf dieselbe Aussage erneut angewandt werden.
    \item Auf die Ankündigung eines Einspruchs muss stets die Ankündigung des Grundes folgen.
    \item Rechtlich zulässige Gründe sind:
        \begin{enumerate}[1.]
            \item Nicht aussagekräftig/unverständlich/mehrdeutig: Die Aussage oder Frage ist aufgrund seiner nicht aussagekräftigen Natur unzulässig.
            \item Bereits beantwortet: Die gleiche Frage wurde mehrfach gestellt, obwohl sie bereits beantwortet wurde.
            \item Unbewiesene Vermutung: Der Anwalt behauptet etwas, ohne sich auf vorliegende Beweise zu stützen.
            \item Fordert Spekulationen: Der Anwalt fordert den Zeugen auf, zu spekulieren.
            \item Zu viele Fragen: Der Anwalt fragt mehr als eine Frage gleichzeitig.
            \item Mangelnde Kenntnisse: Die Kenntnisse des Zeugens über das gefragte Thema sind unzureichend nachgewiesen.
            \item Ohne Priorität: Die Frage ist dem Prozess beziehungsweise der Befragung nicht dienlich.
            \item Gerücht: Die Antwort der Partei baut auf außergerichtlichen Aussagen auf.
            \item Hinterfragt die Staatsautorität: Eine Partei fechtet, hinterfragt oder beleidigt die Staatsautorität beziehungsweise die Autorität des Kaisers. Wird dieser Einspruch bewilligt, wird derjenige, der die Aussage gebracht hat, hinterher wegen Verstoßes gegen §5 SDelGB vor Gericht gestellt.
            \item Anfechtung: Der getätigte Einspruch ist nicht rechtmäßig, da dessen Gegenstand in diesem spezifischen Kontext keinen Verstoß gegen die Befragungsordnung darstellt.
        \end{enumerate}
    \item Wird ein Einspruch stattgegeben, so muss der Befragende bei der Befragung mit der nächsten Frage fortfahren. Der Zeuge darf die vorherige Frage nicht beantworten oder seine Aussage wird im Fall, dass er sie bereits getätigt hat oder dennoch antwortet, gestrichen. Erhebt ein Richter diesen Einspruch, so ist dem sofort stattgegeben, sofern der Gerichtsvorsitzende dem nicht widerspricht. 
    \item Der Richterschaftsvorsitz darf den Einspruch nur ablehnen, wenn der Gegenstand des Einspruchs in keinem vermutbaren Kontext einen Verstoß gegen die Befragungsordnung darstellt.   
\end{enumerate}

\subsection{Plädoyer}\label{pleed}
\begin{enumerate}[(1)]
    \item Jede, zur Aussage verpflichteten Partei, hat die Pflicht, ihre Position durch ein Schlussplädoyer zu verteidigen.
    \item Dieses Plädoyer muss den geltenden Konventionen entsprechen.
    \item Die Schlussfolgerung des Plädoyers ist die Einlassung des Mandanten, sofern es sich um das Schlussplädoyer der Beklagten handelt, und die Strafempfehlung.
    \item Eine Strafempfehlung darf nur durch einen Anwalt geäußert werden.
\end{enumerate}

\subsection{Prozessverlauf}\label{verlauf}
Das Abkommen von Berlin sieht den nachfolgenden Verlauf für Gerichtsverfahren vor:
\begin{enumerate}[1.]
    \item Alle Parteien mit Ausnahme der Richterschaft betreten den Raum.
    \item Die Richterschaft versammelt sich. Währenddessen muss jeder Anwesende stehen.
    \item Der Gerichtsvorsitzende eröffnet den Prozess und die weiteren Richter setzen sich.
    \item Die Beklagte verliest die Anklageschrift.
    \item Der Kläger muss den Strafbestand aus seiner Sicht darlegen.
    \item Der Beklagte hat das Wort und darf seine Darstellung des Sachverhalts darlegen.
    \item Von nun an entscheidet der Gerichtsvorsitzende, wer das Wort erhält.
    \item Sobald alle Beweise und Aussagen der beiden Parteien dargelegt wurden, müssen die Schlussplädoyers gemäß \ref{pleed} gehalten werden.
    \item Die Richterschaft tritt zurück und berät sich in einem separaten Gespräch. Hierbei wird über die Strafe beratschlagt und anschließend entschieden. Bei Stimmgleichheit verfügt der Gerichtsvorsitzende eine zweite Stimme.
    \item Die Richterschaft betritt den Saal, wobei erneut jeder stehen muss, und verkündet im Anschluss die Strafe.
    \item Bis der letzte Richter den Saal verlassen hat müssen alle Teilnehmer stehen und dürfen den Saal nicht verlassen.    
\end{enumerate}

\subsection{Ordnungspflicht}\label{gordnung}
\begin{enumerate}[(1)]
    \item Man darf nicht unaufgefordert sprechen.
    \item Man muss sich für den Prozess angemessen kleiden. Dementsprechend dürfen die Anwesenden keine Kopfbedeckungen mit sich führen und müssen einen Anzug in einer angemessenen Farbe tragen.
    \item Personen ist es nicht erlaubt, Symbole sichtbar mit sich zu führen, die die außergerichtliche Autorität oder Macht darstellen.
    \item Richter müssen schwarze Richterroben tragen.
    \item Im Falle des Internationalen Verfassungsgerichts müssen die Richter rote Anzüge tragen.
    \item Verstöße gegen die Gerichtsordnung unter Inbezugnahme von \ref{zeugen} und \ref{verlauf} werden, sofern bereits eine Verwarnung erteilt wurde mit einem Bußgeld geahndet. Liegen nach Ermessen der Richterschaft zu viele Verstöße vor, können sie die schuldige Partei ungeachtet ihrer Relevanz für diesen Prozess aus dem Saal verweisen und das Verfahren anschließend in dessen Abwesenheit fortfahren.
\end{enumerate}

\subsection{Gerichtliche Vorladung}
\begin{enumerate}[(1)]
    \item Sofern ein Verfahren bestätigt wurde kann unter Vereinbarung mit beiden Parteien ein Gerichtstermin festgelegt werden. Dies wird als außerordentliche Vorladung angesehen.
    \item Legt der Gerichtspräsident einen Termin fest, so muss dieses beide Parteien in einem Schreiben deutlich über das Verhandlungsdatum informieren. Hierbei handelt es sich um eine ordentliche Vorladung
    \item Der Termin und Ort einer Verhandlung muss spätestens zwölf Stunden vor Prozessbeginn bekanntgegeben werden.
    \item Ein Antrag auf Aufschub kann bis zu zwei Stunden vor Prozessbeginn eingereicht werden.
    \item Wird diesem Antrag durch den Gerichtsvorsitzenden des Verfahrens stattgegeben, so wird das Verfahren vertagt.
    \item Andernfalls, oder wenn kein Antrag besteht, müssen die Parteien erscheinen, ansonsten wird in ihrer Abwesenheit verhandelt.
    \item Erscheint keine Partei, so wird zugunsten der Beklagten entschieden.
    \item Jeder gemäß Absatz 6 abwesenden Partei droht eine Bußgeldstrafe.
    \item Abwesende Richter haben kein Mitentscheidungsrecht in dem Verfahren.
    \item Anträge auf Aufschub, die von Richtern gestellt wurden, können vom Gerichtspräsidenten abgelehnt werden, sofern es sich um einen Aufschub von über drei Tagen handelt oder der Prozess insgesamt bereits um mehr als zwei Wochen verschoben wurde.
    \item Die Richterschaft ist fähig zu tagen, sofern mindestens drei Richter anwesend sind.
    \item Gerichtliche Vorladungen müssen vom Gerichtspräsidenten bestätigt werden. Tut dieser dies innerhalb der nächsten drei Tage nicht, gilt sie dennoch.
\end{enumerate}

\subsection{Verfahrensfehler}
Im Falle eines Verfahrensfehlers muss der Gerichtsprozess wiederholt werden.

\subsection{Präzedenzfallordnung}\label{prec}
\begin{enumerate}[(1)]
    \item Entscheidet der Internationale Strafgerichtshof, dass es sich bei einem rechtlichen Ausnahmefall um eine Straftat handeln müsste, so muss dies umgehend in die Berliner Gesetze aufgenommen werden und sofern nach Ermessen der Richterschaft ein Bewusstsein des Verstoßes gegen moralische Normen durch die Beklagte und eine Tat gegen die körperliche Unversehrtheit des Menschen oder die strukturelle Integrität einer Vertragspartei vorliegen sollte, der Strafe entsprechend geurteilt werden.
    \item Damit das beschlossene Gesetz weiter Bestand hat, erfordert es, dass diesem von einer einfachen Mehrheit der innerhalb der nächsten zwei Wochen abgegebenen Stimmen, dafür spricht.
    \item Wird gegen das Gesetz entschieden, bleiben die Verurteilungen auf Grundlage dieses Gesetzesentwurfs dennoch gültig.
\end{enumerate}

\subsection{Gerichtsbarkeit der Staaten}
Prozesse, bei denen ein Staat die Beklagte ist, werden ausschließlich vor den internationalen Gerichtshöfen dieses Abkommens verhandelt.

\subsection{Außergerichtliche Gesetzesentscheidungen}
\begin{enumerate}[(1)]
    \item Die Verabschiedung von Berliner Gesetzen und anderen Erlässen außerhalb des Rahmens von \ref{prec} unterliegt der mehrheitlichen Zustimmung durch die Vertragsparteien.
    \item Abstimmungen können nur in der für die Entscheidung über die Angelegenheit anberaumten Sitzung vorgenommen werden.
    \item Die Sitzung und ihre Tagesordnung werden vom Präsidenten des Gerichtshofs mindestens zwei Tage vorher bekanntgegeben.
    \item Wird ein dringender Antrag gestellt, so kann der Präsident des Gerichts die in Absatz 2 genannte Frist auf zwölf Stunden verkürzen.
    \item Ein Antrag auf Vertagung kann bis zu zwei Stunden vor Beginn der Sitzung gestellt werden.
    \item Die Sitzung ist beschlussfähig, wenn mindestens ein Drittel der Mitglieder der Sitzung vertreten ist.
    \item Im Falle einer nicht gerichtlichen Beschlussfassung sind die Versammlungsmitglieder die Vertragsmitglieder sowie die Richterschaft.
\end{enumerate}

\end{document}