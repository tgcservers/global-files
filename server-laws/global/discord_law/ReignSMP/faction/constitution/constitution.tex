\documentclass{article}

% XeLaTeX packages
\usepackage{fontspec}
\usepackage{xeCJK}     
\usepackage{titlesec}
\usepackage{fancyhdr}

\pagestyle{fancy}
\fancyhf{}
\fancyhead[L]{\leftmark}

\renewcommand{\sectionmark}[1]{\markboth{#1}{}}

\renewcommand{\thesection}{\Roman{section}}
\counterwithout{subsection}{section}
\renewcommand{\thesubsection}{条 \arabic{subsection}}
\titleformat{\subsection}[block]{\normalfont\bfseries\centering}{\large 条 \arabic{subsection}\\}{0pt}{}
\newcommand{\longO}{ō}
\newcommand{\pl}{PLACEHOLDER\ }
\newcommand{\state}{\pl}

\setCJKmainfont{NotoSerifJP-VariableFont_wght.ttf}

\title{二神の条}
\author{村空の義治}
\date{N/A}

\begin{document}
\maketitle\newpage
\section{Hoheitsordnung}
\subsection{Permanenz der Hoheitsordnung}\label{permanency}
\begin{enumerate}
    \item Die Hoheitsordnung, sowie \ref{npwl} - \ref{accusation} sind permanent und unveränderlich.
    \item Die Versammlung der Daimy\longO\ ist der Hoheitsordnung untergeordnet.
    \item Wird die Hoheitsordnung außer Kraft gesetzt oder verändert, so wird das Reich \state unumkehrbar und unverhinderlich aufgelöst.
\end{enumerate}

\subsection{Gültigkeit der Gesetze}
\begin{enumerate}
    \item Die hier festgesetzten Gesetze sind reichsweit gültig und können durch die Versammlung der Daimy\longO\ in ihrer Aussagekraft und ihrem Sinn verändert oder außer Kraft gesetzt werden.
    \item Dieses Gesetz bildet keine Ausnahme zu \ref{permanency}.
\end{enumerate}

\subsection{Gültigkeit von Erlässen}
\begin{enumerate}
    \item Erlässe sind mit Gesetzen gleichwertig und können somit Gesetze und andere Erlässe ändern oder außer Kraft setzen.
    \item Im Gegensatz zu Gesetzen überlebt ein Erlass die Personen, die diesen erlassen hat nicht und müssen somit bis zum Tod derer gesetzlich festgeschrieben werden, um ohne Unterbrechung fortwährend gültig zu sein.
    \item Dieses Gesetz bildet keine Ausnahme zu \ref{permanency}.
\end{enumerate}

\subsection{Hoheit des Staates}
\begin{enumerate}
    \item Der Staat ist jeder Person und Entität auf dem Boden des Staates übergeordnet und verlangt von diesen bedingungslose Treue.
    \item Jede Person und Entität auf dem Boden des Staats muss den hier geltenden Gesetzen folgeleisten.
\end{enumerate}

\subsection{Hoheit der Versammlung der Daimy\longO}
\begin{enumerate}
    \item Kein Mensch, keine Institution und keine Versammlung ist befugt, sich der Versammlung der Daimy\longO\ überzuordnen oder sich über diese hinwegzusetzen.
    \item Die Entscheidungen der Versammlung der Daimy\longO\ können nur von dieser selbst außer Kraft gesetzt werden.
    \item Entscheidungen, die die Reichsebene betreffen, können nur von der Versammlung der Daimy\longO\ getroffen werden.
    \item Die Versammlung muss stets ausschließlich aus allen Daim\longO\ des Reichs zusammengesetzt sein.
    \item Jeder muss in der Versammlung stets das gleiche Stimmrecht haben.
    \item Die einzige Ausnahme hierzu bildet das alleinige Entscheidungsrecht der beiden Gründerhäuser Murakara und \pl in Bezug auf die Neuaufnahme von Häusern in das Reich \state.
    \item Die Versammlung darf niemals aufgelöst oder zeitweise seiner Rechte entledigt werden.
\end{enumerate}

\subsection{Daimy\longO}
\begin{enumerate}
    \item Der Daimy\longO\ verfügt über absolute Vollmachten auf seinem Lehen, die nur durch die Versammlung der Daimy\longO\ eingeschränkt werden können.
    \item Nur die Versammlung der Daimy\longO\ kann Personen in den Stand des Daimy\longO\ erheben.
    \item Wird jemand in den Stand des Daimy\longO\ erhoben, so wird sein Grundbesitz zum Daimyat erklärt.
    \item Ein Daimy\longO\ kann nur auf eine direkten Anordnung der Versammlung der Daimy\longO\ hin aus dem Stand des Daimy\longO\ gebannt werden.
    \item Ist ein Daimyat herrenlos, so entscheidet die Versammlung der Daimy\longO\ über dessen Fortbestand.
    \item Es kann nur den Titel des Daimy\longO\ fortführen, wer gemäß traditioneller Erbfolge oder anderweitiger Bestimmung durch den Vorgänger diesen nach Tod des vorigen Daimy\longO\ erbt.
    \item Das Erben des Titels kann nur erfolgen, sofern der Vorgänger in Ehre das Amt verließ oder die Versammlung der Daimy\longO\ einer unbeirrten Fortführung des Amts eindeutig zustimmt.
\end{enumerate}

\subsection{Versammlung der Daimy\longO}
\begin{enumerate}
    \item Die Versammlung der Daimy\longO\ tritt mindestens einmal in der Woche zur Erörterung und Entscheidung reichsweiter Angelegenheiten zusammen.
    \item Die Versammlung ist erst entscheidungsfähig, wenn die beiden Gründerhäuser vertreten sind.
    \item Daimy\longO\ können Vertreter für die Versammlung nominieren. Damit diese stimmberechtigt sind, müssen diese allerdings dokumentarisch nachweisen können, dass
          ihr Feudalherr ihnen die Stimmberechtigung erteilt hat.
\end{enumerate}

\subsection{Gesetzesvollmachten}
\begin{enumerate}
    \item Jeder Landesherr ist dazu befugt, Gesetze auf eigener Landesebene zu beschließen, 
          die allerdings nicht den Gesetzen der übergeordneten Landesebene widersprechen dürfen.
    \item Diese Gesetze können allerdings durch die übergeordnete Landesebene aufgehoben werden.
    \item Gleiches gilt für das Begnadigen von Personen auf eigener Landesebene.
\end{enumerate}

\subsection{Amtsmoral}
Jedes Amt im Staatsdienst ist zur pflichtbewussten und pflichtge,äßen Erfüllung dieses Dienstes verpflichtet.

\subsection{Anerkennung des Staatsdienstes}
\begin{enumerate}
    \item Verlässt eine Person ein Amt im Staatsdienst zu Lebzeiten, so muss diese für die Arbeit dem Amte und der Verdienste angemessen
    entlohnt werden.
    \item Hat sich eine Person außerordentlich verdient gemacht und konnte zu Lebzeiten nicht mehr angemessen entlohnt werden, so muss man dessen
    Erbe gemäß Erbfolge belohnen.
\end{enumerate}

\subsection{Adelstitel}
Jeglicher Träger eines Adelstitels befindet sich im Amt im Staatsdienst.

\newpage
\section{Grundordnung}
\subsection{Keine Strafe ohne Gesetz}\label{npwl}
\begin{enumerate}
    \item Es ist nicht zulässig, eine Person auf Grundlage eines Gesetzes zu
    verurteilen, das zur Tatzeit noch nicht bestand.
    \item Einem Gericht ist es möglich, dies dennoch zu tun, sofern ein begründeter
    und nachvollziehbarer Verdacht besteht, dass der Täter die Tat im Gewissen verübte, gegen
    die geltende moralische Norm des Reichs zu verstoßen.
\end{enumerate}

\subsection{Strafbarkeit des Versuchs und der Täterschaft}
\begin{enumerate}
    \item Jeglicher Versuch, eine Straftat zu begehen, ist mit der Straftat äquivalent zu
    behandeln.
    \item Ebenfalls gleicht vor dem Gesetz die Mitwisserschaft, die Beihilfe und die
    Mittäterschaft der Täterschaft und ist daher mit gleichen Maßnahmen aufzuwiegen.
\end{enumerate}

\subsection{Rechte des Beschuldigten}\label{accusation}
\begin{enumerate}
    \item Mangelnde oder fehlerhafte Kenntnisse der Rechtslage gewähren keine rechtliche
    Immunität.
    \item Bei eigenverschuldeter und unentschuldbarer Abwesenheit vor Gericht, dürfen Prozesse
    in Abwesenheit der fehlenden Partei abgehalten werden.
    \item Einem Beschuldigten steht es zu, die Beschuldigung vor Gericht anzufechten.
    \item Es gibt kein rechtskräftiges Urteil ohne Prozess.
    \item Advokaten müssen von der Beklagten oder der Klagenden selbst gestellt werden.
    \item Es besteht kein Grundrecht auf einen Advokaten.
    \item Jedem, der sich durch ein Urteil nachvollziehbar in seinen Rechten verletzt sieht,
    steht es zu, dieses anzufechten. Tut man dies, so wird die Rechtschaffenheit des Urteils
    von der übergeordneten Instanz geprüft.
    \item Befindet man sich bereits in der höchsten Instanz, so ist das Urteil rechtskräftig
    und final.
    \item Diese Rechte dürfen keinem verwehrt werden
\end{enumerate}

\newpage
\section{Verwaltungswesen}
\subsection{Öffentliches Amt}
\begin{enumerate}
    \item Als öffentliches Amt gilt jedes, nicht erbliche Amt im Staatsdienst.
    \item Es wird demjenigen die Amtszulassung entzogen, wer sich der Amtsmoral nicht beugt.
    \item Ämter können nur durch Staatsbürger besetzt werden.
\end{enumerate}

\subsection{Amtszeit des öffentlichen Amts}
\begin{enumerate}
    \item Die Amtszeit darf höchstens einen Monat nach Amtsantritt enden.
    \item Ein Amt darf nicht zwei aufeinanderfolgende Male durch dieselbe Person besetzt werden.
\end{enumerate}

\end{document}
