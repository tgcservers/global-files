\documentclass{article}

% XeLaTeX packages
\usepackage{fontspec}
\usepackage{xeCJK}     
\usepackage{titlesec}
\usepackage{fancyhdr}

\pagestyle{fancy}
\fancyhf{}
\fancyhead[L]{\leftmark}

\renewcommand{\sectionmark}[1]{\markboth{#1}{}}

\renewcommand{\thesection}{\Roman{section}}
\counterwithout{subsection}{section}
\renewcommand{\thesubsection}{条 \arabic{subsection}}
\titleformat{\subsection}[block]{\normalfont\bfseries\centering}{\large 条 \arabic{subsection}\\}{0pt}{}
\newcommand{\longO}{ō}
\newcommand{\pl}{PLACEHOLDER\ }
\newcommand{\state}{\pl}

\setCJKmainfont{NotoSerifJP-VariableFont_wght.ttf}

\title{二神の条}
\author{村空の義治}
\date{N/A}

\begin{document}
\maketitle\newpage
\section{主権秩序}
\subsection{主権秩序の永続性}\label{permanency}
\begin{enumerate}
    \item 主権秩序、および\ref{npwl} - \ref{accusation}は永久であり、変更不可能である。
    \item 大名会議は主権秩序に従属する。
    \item 主権秩序が無効化または変更された場合、国家は不可逆かつ防ぎようのない形で解散される。
\end{enumerate}

\subsection{法律の有効性}
\begin{enumerate}
    \item ここで定められた法律は国家全体に適用され、大名会議によってその効力や意味を変更または無効化されることができる。
    \item この法律は\ref{permanency}の例外を形成しない。
\end{enumerate}

\subsection{布告の有効性}
\begin{enumerate}
    \item 布告は法律と同等であり、法律や他の布告を変更または無効化することができる。
    \item 法律と異なり、布告は布告者の死後には効力を失い、そのため布告は布告者の死前に法的に記録される必要があり、途切れることなく有効でなければならない。
    \item この法律は\ref{permanency}の例外を形成しない。
\end{enumerate}

\subsection{国家の主権}
\begin{enumerate}
    \item 国家は国家内のすべての個人および組織に優越し、無条件の忠誠を要求する。
    \item 
\end{enumerate}

\subsection{大名会議の主権}
\begin{enumerate}
    \item いかなる人間、機関、会議も、大名会議に優越する権限を持たず、その決定を無視することはできない。
    \item 大名会議の決定は、会議自身によってのみ無効化されることができる。
    \item 国家レベルに関わる決定は、大名会議のみが行うことができる。
    \item 会議は常にすべての国の大名で構成されなければならない。
    \item すべての者は会議において平等な投票権を持たなければならない。
    \item これに対する唯一の例外は、新たな家系を国家に加える決定権が、創設家である村唐家と\pl に専有されることである。
    \item 大名会議は決して解散されることなく、またその権利が一時的に剥奪されることもない。
\end{enumerate}

\subsection{大名}
\begin{enumerate}
    \item 大名は、その領地において絶対的な権限を有し、その権限は大名会議によってのみ制限されることができる。
    \item 大名の地位に昇格させることができるのは、大名会議のみである。
    \item 大名の地位に昇格した場合、その土地は大名領と宣言される。
    \item 大名の地位を剥奪することができるのは、大名会議による直接の命令のみである。
    \item 大名領が無主の状態になった場合、その存続については大名会議が決定する。
    \item 大名の称号を継承できるのは、伝統的な継承法または前任者による他の決定に基づいて、その地位を前任者の死後に継承する者のみである。
    \item 称号の継承は、前任者が名誉のうちにその職を辞した場合、または大名会議が職務の継続を承認した場合にのみ行われる。
\end{enumerate}

\subsection{大名会議}
\begin{enumerate}
    \item 大名会議は、国家全体に関わる問題を議論し決定するために、少なくとも週に一度開催される。
    \item 会議は、創設家が代表されている場合にのみ、決定権を持つ。
    \item 大名は、会議の代表者を指名することができる。ただし、これらの代表者が投票権を持つためには、封建領主が投票権を付与したことを文書で証明する必要がある。
\end{enumerate}

\subsection{法律制定権}
\begin{enumerate}
    \item 各領主は、自身の領地内で法律を制定する権限を有するが、それらの法律は上位の領地の法律と矛盾してはならない。
    \item これらの法律は、上位の領地によって廃止されることがある。
    \item 同様に、自身の領地内で人々を恩赦することも可能である。
\end{enumerate}

\subsection{公務の道徳}
国家公務におけるすべての職位は、職務を忠実かつ適切に遂行する義務を負う。

\subsection{国家公務の認知}
\begin{enumerate}
    \item 生前に国家公務の職位を離れた者は、その職務と功績に応じて適切に報酬を受ける権利を有する。
    \item ある者が特に功績を上げ、生前に適切な報酬を受けることができなかった場合、その者の相続人は相続法に基づいて報酬を受ける権利を有する。
\end{enumerate}

\subsection{貴族の称号}
貴族の称号を持つ者はすべて、国家公務において職位を有する。

\newpage
\section{基本秩序}
\subsection{法律なくして罰なし}\label{npwl}
\begin{enumerate}
    \item 犯行時に存在しなかった法律に基づいて、人を裁くことは許されない。
    \item ただし、犯罪者が意識的に国家の道徳的規範に違反したとする合理的かつ理解可能な疑いがある場合、裁判所はこれを行うことができる。
\end{enumerate}

\subsection{試みと共犯の刑罰}
\begin{enumerate}
    \item 犯罪を試みた者は、その犯罪と同等に扱われる。
    \item 同様に、共犯や幇助、従犯も主犯と同等に扱われ、同じ措置が取られる。
\end{enumerate}

\subsection{被告の権利}\label{accusation}
\begin{enumerate}
    \item 法律の不知や誤認は、法的免責を与えない。
    \item 故意であり、かつ無責任な裁判への欠席の場合、裁判は欠席者なしで行われることがある。
    \item 被告には、裁判での告訴に異議を申し立てる権利がある。
    \item 裁判なしには有効な判決は存在しない。
    \item 弁護人は被告または原告自身によって選出されなければならない。
    \item 弁護人を持つ基本的な権利は存在しない。
    \item 判決によって自身の権利が侵害されたと納得できる者には、その判決に異議を申し立てる権利がある。異議を申し立てると、上級審で判決の正当性が審査される。
    \item 最高裁において既に審査されている場合、その判決は最終的かつ確定的である。
    \item これらの権利は、誰にも剥奪されることはない。
\end{enumerate}

\newpage
\section{行政組織}
\subsection{公職}
\begin{enumerate}
    \item 公職とは、国家公務において世襲でないすべての職位を指す。
    \item 公職の道徳に従わない者は、公職の許可を取り消される。
    \item 公職は国家公民によってのみ占有されることができる。
\end{enumerate}

\subsection{公職の任期}
\begin{enumerate}
    \item 任期は、就任後最長1か月で終了しなければならない。
    \item 同じ人物が連続して2回同じ職位を占めることはできない。
\end{enumerate}

\end{document}
