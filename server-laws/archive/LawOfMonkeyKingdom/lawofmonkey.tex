\documentclass{article}
\usepackage[utf8]{inputenc}
\usepackage{textcomp}
\usepackage{amsmath}
\usepackage{enumerate}
\usepackage{ragged2e}
\usepackage{blindtext}

\renewcommand{\thesection}{\Roman{section}}
\counterwithout{subsection}{section}
\renewcommand{\thesubsection}{§\arabic{subsection}}

\title{Codex Titus}
\author{Duke Ordovis of Nokron}
\date{07. Februar 2023}

\begin{document}
\maketitle
\vspace*{\fill}
\paragraph{Recht von Monkey Kingdom gemäß Erster Gesetzesreform [Edictum alteratum] des vierten Regierungsjahres Seiner Majestät Imperator Titus.}

\newpage
\topskip0pt
\vspace*{\fill}
\begin{Center}
\textbf{2. Auflage}
\end{Center}
\begin{flushright}
\textit{Berlin, den 07. Februar 2023}
\end{flushright}
\textbf{Auf Anordnung Seiner Majestät Imperator Titus hin, kamen HG Herzog Ordovis von Nokron umfassende Befähigungen bezüglich der Gesetzgebung zu. Das Fundament dieser Auflage bildet ein durchgesetztes imperiales Edikt vom 06. Februar 2023. Zur Meidung aufkommender Unannehmlichkeiten für Seine Majestät, wird der Leser gebeten, bei Verfassungsbeschwerden zunächst an den Autoren heranzutreten (auf dem Server als "Richter" markiert) und bei äußerster Notwendigkeit diese der Königskammer vorzulegen.}
\\\\
\textbf{- HG Duke Ordovis of Nokron}
\\\\\\\\\\\\\\\\\\\\\\\\\\\\\\\\
\vspace*{\fill}
%
\newpage
\tableofcontents
\newpage
\section{Präambel}
Im Bestreben, eine Nation der Rechtsstaatlichkeit zu schaffen, in welcher jeder dem Auge der Jurisdiktion als gleich erscheint, wurde im Namen 
Seiner Majestät Imperator Titus, erster seines Namens, durch die richtende Hand Seiner Gnaden Herzog Ordovis von Nokron, im vierten Regierungsjahr
die Verfassung des Königreichs erlassen, welche den Namen des Allgerechten trägt. Konstituiert wurde dieses Recht unter dem Schein der
allmächtigen Legasthenie.\\
Dem Imperator sei treu, wer Herr seiner Vernunft ist.\\\\
\textit{Ewigkeit dem Imperator und seinem Königreich}
\newpage
\section{Feudalordnung}
\subsection{Feudalstruktur}
(1) Dem Staate unterstehen jegliches Territorium auf dessen Gebiet und die Einwohner, sowie Besucher jenes Gebietes.\\
(2) Die Pflicht jedes Menschen auf dem Territorium des Staats ist es, den Befehlen des Imperators ausnahmslos folgezuleisten.\\

\subsection{Der Imperator}
(1) Der Imperator ist das Staatsoberhaupt und der Regierungschef des Monkey Kingdoms und daher in der Hierarchie an oberster Stelle. \\ 
(2) Jedes Mitglied der Nation ist ihm zu Treue verpflichtet.\\
(3) Er regiert das Königreich und verfügt daher über absolute Entscheidungsvollmachten.\\
(4) Der Imperator steht über dem Gesetz.\\
(5) Der Imperator verfügt sowohl über das Besitz- als auch Verwaltungsrecht seiner Domänen.\\
(6) Die direkte Anrede lautet "Eure Majestät".\\
(7) Die indirekte Anrede lautet "Seine Majestät" und kann ebenfalls mit HM abgekürzt werden.\\
(8) Die indirekte Adressierung verstorbener Imperator lautet "His Late Majesty", auch als "HLM" abgekürzt.\\
(9) Im rechtlichen Kontext ist der Begriff des Imperators synonym mit dem Begriff des Monkey Kingdoms und dessen weiterer Synonyme.\\
(10) Die Königin ist die Stellvertreterin den Imperators.

\subsection{Der Königsrat}
Als Königsrat wird die Versammlung bezeichnet, die aus dem Imperator und den Ratsherzögen (\ref{koenige} Abs. 6) besteht. Sie fungiert nicht als Parlament, da die Ratsabstimmungen lediglich beratender Funktion sind und daher nicht vom Imperator berücksichtigt werden müssen.

\subsection{Der Hochadel}
Dem Hochadel gehören der gesamte Land- und Verwaltungsadel der Herzogs- und Königreichsebene an.

\subsection{Die Herzöge}\label{koenige} 
(1) Ein Herzog besitzt auf seinem Territorium uneingeschränkte Entscheidungsvollmachten. Er ist dazu verpflichtet, dem Imperator Tribut zu zollen.  \\
(2) Der Titel besteht lebenslänglich und ist erblich.  \\
(3) Die Herzog sind in der Lage, ihre Vasallen selbst zu wählen.  \\
(4) \textit{weggefallen}\\
(5) Als Großherzöge werden jene bezeichnet, die über zusätzliche, administrative Vollmachten auf dem Server selbst, wie auch auf dem Discord-Server verfügen.\\
(6) Angehörige des Königsrats werden als Ratsherzöge bezeichnet. Zu ihnen gehören die folgenden Herzöge:  \\
\begin{enumerate}
	\item HRGD Großherzog Bonnie
	\item HG Herzog Ordovis von Nokron  
\end{enumerate}
(7) Die direkte Anrede des Herzogs lautet "Euer Gnaden".\\
(8) Die dementsprechende indirekte Anrede hat "Seine/Ihre Gnaden" zu lauten, zudem wird diese als "HG" abgekürzt.\\
(9) Großherzöge werden mit "Euer Königlicher Großherzog" direkt und mit "Sein/Ihr Königlicher Großherzog", dessen synonymes Akronym "HRGD" lautet, indirekt adressiert.\\

\subsection{Die Grafen}
(1) Grafen verwalten ihr Territorium, müssen ihren Herzog jedoch bei Entscheidungen um Erlaubnis bitten.  \\
(2) Sie müssen ihrem Lehnsherrn Tribut zollen.  \\
(3) Die direkte Anrede der Grafen lautet "Eure Durchlaucht". \\
(4) Die indirekte Anrede lautet "Seine/Ihre Durchlaucht", welche auch durch das Akronym "HSH" ersetzt werden kann.  

\subsection{Die Ritter}
(1) Der Titel des Ritters ist der niedrigste Titel des Adels. Er befindet sich im Besitz eines staatlich erwählten Helden beziehungsweise Kirchenspenders.  \\
(2) Ritter werden als "Sir" adressiert.  

\subsection{Sonderrechte}
Dem Imperator ist es gestattet, durch die Anwendung des Lex Votum Imperatoris jegliche Entscheidung jeglicher Instanz aufzuheben.

\subsection{Die Herrschaft des Imperators}
(1) Der Imperator regiert uneingeschränkt bis zu seinem Lebensende und wählt vor seinem Tod einen Nachfolger aus seiner Linie.  \\
(2) Die Autorität des Imperators darf nicht angezweifelt werden.  \\
(3) Der Imperator kann Ausnahmen zu allen Gesetzen machen.  \\

\subsection{Staatsbürger}
Als Staatsbürger werden die Einwohner vom Monkey Kingdom bezeichnet, denen eine Staatsbürgerschaft gewährt wurde.

\subsection{Bauern}
Alle Bauern des Königreichs sind vogelfrei und dennoch zum Zollen von Tribut verpflichtet.

\subsection{Banner}
(1) Jeder Feudalherr, als welche da gelten:
\begin{enumerate}
	\item Die Grafen
	\item Die Herzöge
	\item Die Großherzöge
\end{enumerate}
sind verpflichtet, ein eigenes Banner zu führen.\\
(2) Die Banner dürfen weder anstößige, noch auf sonstige Weise auf diesem Server, dem Discord-Server oder gemäß deutschem Recht verbotene Symbolik aufweisen.\\
(3) Das Banner muss über eine angemessene Komplexität verfügen.

\section{Struktur der Judikative}

\subsection{Gerichtliche Instanzen}
(1) Erhebt eine Partei Anklage, so beginnt der Rechtsstreit in der untersten Instanz. Sofern man gemäß \ref{verlauf} Nr. 10 in Berufung gegangen ist, wird das Verfahren von der nächsten Instanz behandelt.\\
(2) Die Instanzen in aufsteigender Folge sind:
\begin{enumerate}
	\item Kammergericht
	\item Herzogskammer
	\item Königskammer
\end{enumerate}
(3) Der Imperator kann nach eigenem Ermessen den Richterschaftsvorsitz eines Prozesses jederzeit übernehmen.

\subsection{Grafenkammer}
Die Grafenkammer ist für Rechtsstreitigkeiten auf Grafschaftsebene zuständig. Ihr sitzt der jeweilige Graf vor.

\subsection{Herzogskammer}
Der Herzogskammer sitzt der jeweilige Herzog vor. Dementsprechend ist sie auf der Herzogtumsebene tätig.

\subsection{Königskammer}
Das Königskammer ist der höchste Gerichtshof des Monkey Kingdoms und besteht aus dem Königsrat. Sein Vorsitzender ist der Imperator.

\subsection{Zeugen}\label{zeugen}
Man darf Personen in den Zeugenstand berufen.\\
(1) Diese darf man unter den gegebenen Regeln befragen  \\
(2) Diese Regeln lauten:  \\
\begin{enumerate}
	\item Die Zeugen stehen automatisch unter Eid, sobald sie ihr erstes Wort im Zeugenstand erheben.
	\item Die Zeugen müssen daher alles wahrheitsgemäß beantworten.
	\item Jegliche ungenauen Aussagen der Zeugen werden nicht ins Protokoll aufgenommen (siehe hierzu \ref{eordnung} Abs. 6).
\end{enumerate}

\subsection{Anwälte}
Man darf einen Anwalt einstellen. Hierbei muss jedoch beachtet werden, dass kein Anrecht auf einen Pflichtverteidiger besteht.

\subsection{Einspruchsordnung}\label{eordnung}
(1) Einsprüche sind erlaubt und bilden eine Ausnahme zu \ref{gordnung} Abs. 1.\\
(2) Sie können durch die Richterschaft abgewiesen werden.\\
(3) Bei einmaliger Ablehnung eines Einspruchs darf dieser nicht auf dieselbe Aussage erneut angewandt werden.\\
(4) Auf die Ankündigung eines Einspruchs muss stets die Ankündigung des Grundes folgen.\\
(5) Rechtlich zulässige Gründe sind:
\begin{enumerate}
\item Nicht aussagekräftig/unverständlich/mehrdeutig: Die Aussage oder Frage ist aufgrund seiner nicht aussagekräftigen Natur unzulässig.
\item Bereits beantwortet: Die gleiche Frage wurde mehrfach gestellt, obwohl sie bereits beantwortet wurde.
\item Unbewiesene Vermutung: Der Befragende behauptet etwas, ohne sich auf vorliegende Beweise zu stützen.
\item Fordert Spekulationen: Der Befragende fordert den Zeugen auf, zu spekulieren.
\item Supra interrogatio (über Befragung hinaus): Der Anwalt fragt mehr als eine Frage gleichzeitig.
\item Mangelnde Kenntnisse: Die Kenntnisse des Zeugens über das gefragte Thema sind unzureichend nachgewiesen.
\item Ohne Priorität: Die Frage ist dem Prozess beziehungsweise der Befragung nicht dienlich.
\item Gerücht: Die Antwort der Partei baut auf außergerichtlichen Aussagen auf.
\item Hinterfragt die Staatsautorität: Eine Partei fechtet, hinterfragt oder beleidigt die Staatsautorität beziehungsweise die Autorität des Imperators. Wird dieser Einspruch bewilligt, wird derjenige, der die Aussage gebracht hat, hinterher wegen Verstoßes gegen \ref{verrat} vor Gericht gestellt.
\end{enumerate}
(6) Wird ein Einspruch stattgegeben, so muss der Befragende bei der Befragung mit der nächsten Frage fortfahren. Der Zeuge darf die vorherige Frage nicht beantworten oder seine Aussage wird im Fall, dass er sie bereits getätigt hat oder dennoch antwortet, gestrichen. Erhebt ein Richter diesen Einspruch, so ist dem sofort stattgegeben, sofern der Gerichtsvorsitzende dem nicht widerspricht.

\subsection{Prozessverlauf}\label{verlauf}
Das Recht des Monkey Kingdoms sieht den nachfolgenden Verlauf für Gerichtsverfahren vor.\\
\begin{enumerate}
	\item Alle Parteien mit Ausnahme der Richterschaft betreten den Raum.
	\item Die Richterschaft versammelt sich. Währenddessen muss jeder Anwesende stehen.
	\item Der Gerichtsvorsitzende eröffnet den Prozess und die weiteren Richter setzen sich.
	\item Der Gerichtsvorsitzende verliest die Anklageschrift.
	\item Der Kläger muss den Strafbestand aus seiner Sicht darlegen.
	\item Der Beklagte hat das Wort und darf seine Darstellung des Sachverhalts darlegen.
	\item Von nun an entscheidet der Gerichtsvorsitzende, wer das Wort erhält.
	\item Sobald alle Beweise und Aussagen der beiden Parteien dargelegt wurden, dürfen die beklagte Partei und die klagende Partei, beziehungsweise deren Vertreter, je ein Strafmaß, beziehungsweise den Freispruch, empfehlen.
	\item Die Richterschaft tritt zurück und berät sich in einem separaten Gespräch. Hierbei wird über die Strafe beratschlagt und anschließend entschieden. Bei Stimmgleichheit verfügt der Gerichtsvorsitzende eine zweite Stimme.
	\item Die Richterschaft betritt den Saal, wobei erneut jeder stehen muss, und verkündet im Anschluss die Strafe. Daraufhin fragt der Gerichtsvorsitzende, ob eine Partei in Berufung gehen möchte, sofern denn eine höhere Instanz besteht. Andernfalls ist die Strafe final.
	\item Bis der letzte Richter den Saal verlassen hat müssen alle Teilnehmer stehen und dürfen den Saal nicht verlassen.
\end{enumerate}

\subsection{Gerichtsordnung}\label{gordnung}
(1) Man darf nicht unaufgefordert sprechen\\
(2) Verstöße gegen die Gerichtsordnung unter Inbezugnahme von \ref{zeugen} und \ref{verlauf} werden, sofern bereits eine Verwarnung erteilt wurde mit 10 HTK Bußgeld geahndet. Liegen nach Ermessen der Richterschaft zu viele Verstöße vor, können sie die schuldige Partei ungeachtet ihrer Relevanz für diesen Prozess aus dem Saal verweisen und das Verfahren anschließend in dessen Abwesenheit fortfahren.\\
(3) Von Absatz 2 ist lediglich der Kaiser ausgenommen.

\subsection{Gerichtliche Vorladung}
Sofern ein Verfahren bestätigt wurde kann unter Vereinbarung mit beiden Parteien ein Gerichtstermin festgelegt werden. Dies wird als außerordentliche Vorladung angesehen.\\
(1) Legt das Gericht einen Termin fest, so muss dieses beide Parteien in einem Schreiben deutlich über das Verhandlungsdatum informieren. Hierbei handelt es sich um eine ordentliche Vorladung\\
(2) Der Termin und Ort einer Verhandlung muss spätestens zwölf Stunden vor Prozessbeginn bekanntgegeben werden.\\
(3) Ein Antrag auf Aufschub kann bis zu zwei Stunden vor Prozessbeginn eingereicht werden.\\
(4) Wird diesem Antrag durch den Gerichtsvorsitzenden des Verfahrens stattgegeben, so wird das Verfahren vertagt.\\
(5) Andernfalls, oder wenn kein Antrag besteht, müssen die Parteien erscheinen, ansonsten wird in ihrer Abwesenheit verhandelt.\\
(6) Erscheint keine Partei, so wird der Termin ebenfalls vertagt.\\
(7) Jeder gemäß Absatz 5 abwesenden Partei droht eine Bußgeldstrafe in Höhe von 20 HTK.\\

\subsection{Anrede des Richters}
Steht man vor Gericht, so hat man den Richter mit der, ihm zustehenden Adressierung anzureden. Tut man dies nicht, wwird gemäß \ref{gordnung} verfahren.\\

\subsection{Rechtliche Immunität}
(1) Mangelnde oder fehlerhafte Kenntnisse des Gesetzes gewähren keine rechtliche Immunität, da das Informieren über die Gesetzeslage Pflicht ist.\\
(2) Der Imperator darf Personen rechtliche Immunität verleihen.

\subsection{Vergehen am Hochadel}\label{vergehen}
(1) Vergehen an dem Hochadel werden mit dem dreifachen Strafsatz vergolten.\\
(2) Vergehen an dem Imperator werden mit dem zehnfachen Strafsatz vergolten.\\
(3) Vergehen an dem Staat gelten als Vergehen an dem Imperator.

\subsection{Bußgeldstrafe}
Bußgeldstrafen sind Strafen, die die Schadensvergütung im Gegenständlichen, wie auch Geistigen und Symbolischen, in finanzieller Form anstreben.\\
(1) Sie werden bei Straftaten minderer Schwere als umfassendes Strafmaß erhoben.\\
(2) Bei Straftaten besonderer Schwere sind sie als zusätzliche Strafe angeführt.\\
(3) Die Bußgeldstrafen werden gemäß aktuellem Wechselkurs des Hamavarischen Torrekhen (HTK) in Coins errechnet.\\
(4) Bußgeldstrafen können auch durch Waren äquivalenten Werts ersetzt werden. Hierbei muss die Richterschaft jedoch die Waren als angemessen betrachten, andernfalls müssen andere Waren angeboten werden.

\subsection{Freiheitsstrafe}
(1) Eine Haftstrafe kann bei Beschluss des Gerichts entweder als Strafersatz oder Strafzusatz angewendet werden.\\
(2) Bei Ausbruchsversuchen und Ausbrüchen werden stets zehn Minuten zusätzliche Haft angeordnet.\\
(3) Beihilfe bei Ausbrüchen werden mit dem Verordnen der gleichen Haftstrafe für die helfende Partei bestraft.\\
(4) Abgesessen hat man die Strafe, sobald man die jeweilige Zeit nachweislich online war.\\
(5) Der Staat haftet für keine Gegenstände, die während der Haftstrafe verlorengehen, sofern für den Häftling genügend Zeit bestand, die Gegenstände anderweitig zu lagern.\\
(6) Der Strafsatz bemisst sich in 5-Minuten-Sätzen

\subsection{Hinrichtung}
(1) Hinrichtungen sind als Strafmaßnahme für Kapitalverbrechen vorgesehen.\\
(2) Hinrichtungen sind erst dann erlaubt, wenn das Gericht eindeutig eine Hinrichtung verhängt hat.\\
(3) Diese Strafe kann nur auf Empfehlung des Klägers hin aufgehoben werden\\
(4) Im Falle eines Verstoßes gegen das Staatsrecht verliert Absatz 3 seine Wirksamkeit.

\subsection{Verbindlichkeit von Strafsätzen}
(1) Die aufgeführten Strafsätze dienen lediglich zur Orientierung und sind daher nicht verpflichtend.\\
(2) Dies gilt nicht für Hinrichtungen.\\
(3) Bei Wiederholungstaten liegt es je nach Häufigkeit und Schwere der Tat im Ermessen des zuständigen Gerichts, ob weiterhin derselbe oder ein verhärteter Strafsatz geltend gemacht werden sollte.\\
(4) Bei äußerster Häufigkeit oder relativer Häufigkeit von Taten besonderer Schwere, haben Wiederholungstaten die Todesstrafe zur Folge.

\subsection{Untersuchungshaft}
Besteht die Gefahr, dass ein Tatverdächtiger bis zu seinem Prozess flieht oder befragt werden muss, muss eine Unterbringung in der Untersuchungshaft angeordnet werden.\\

\subsection{Unterbringung in Hochsicherheitseinrichtungen}
(1)	Freiheitsstrafen in Höhe von mehr als zwanzig Minuten müssen in Hochsicherheitseinrichtungen abgesessen werden.\\
(2)	Besteht eine akute Fluchtgefahr, so kann dies auch bei kürzerer Haft angeordnet werden.

\subsection{Unterbringung in einer Sonderverwahrung}
Personen, die sich eines Kapitalverbrechens schuldig gemacht haben und daher hingerichtet werden sollen, müssen sofern zusätzlich eine Freiheitsstrafe angeordnet wurde, in einer Todeszelle untergebracht werden. Mit Ende ihrer Haftstrafe werden sie hingerichtet\footnote{Vgl. CP-01/80: Hamavar gg. Marrkan-Bennetal}.

\subsection{Entzug von Titeln}
Es ist der Königskammer gestattet, bestimmten Personen den Titel zu entziehen, sofern sie dessen Macht missbrauchen oder mit ihr anderweitig nicht umgehen können.

\subsection{Präzedenzfälle}
Sofern ein rechtlicher Ausnahmefall vorliegt, ist der Fall unter sofortiger Wirkung der Königskammer zu übertragen.\\
(1) Entscheidet dieser, dass es sich bei dem vorliegenden Fall um eine Straftat handelt, so muss dies umgehend in die Gesetze aufgenommen werden und
sofern nach Ermessen der Königskammer ein Bewusstsein des Verstoßes gegen moralische Normen durch die Beklagte vorliegen sollte, der Strafe entsprechend
geurteilt werden.

\subsection{Generationenrecht}
(1) Verstirbt ein Kläger oder Opfer eines Verbrechens, so darf das Haus des Geschädigten Anklage erheben oder die Geschädigte vor Gericht vertreten.\\
(2) Verstirbt ein Täter, so muss sich das Haus des Täters für dessen Straftaten verantworten.\\
(3) Das Haus wird stets durch dessen Oberhaupt vertreten. Besteht keines, so wird dieses vom zuständigen Gericht gewählt.\\
(4) Gemäß Absatz 2 können demnach auch die nachfolgenden Oberhäupter zur Rechenschaft gezogen werden\footnote{Vgl. Akz. CP-01/05: Marrkan gg. Hòirran}.

\subsection{Strafverfolgung}
(1) Entzieht man sich der Strafverfolgung des Reichs, wird man auf dem Gebiet für vogelfrei erklärt, es sei denn, man stellt sich freiwillig vor das zuständige Gericht.\\
(2) Man darf sich ebenfalls nicht der Strafverfolgung verbündeter Reiche auf dem Gebiet des Kaiserreichs entziehen.\\
(3) Absatz 1 und 2 treten nur dann ein, wenn einer Person kein Asyl gewährt wurde.\\
(4) Einer Person darf Asyl gewährt werden, wenn sie in einem anderen Staat eine Straftat beging, die auf dem Gebiet des Kaiserreichs nicht als Verbrechen anerkannt wird.\\
(5) Das Recht auf Asyl darf einer Person jederzeit entzogen werden\\
(6) Behindert man die Justiz absichtlich, so muss man eine Bußgeldstrafe in Höhe von 30 HTK zahlen.\\
(7) Nicht vollständig ausgezahlte Bußgelder werden gemäß \ref{schulden} gehandhabt.

\section{Strafrecht}
\subsection{Diebstahl}
(1) Stiehlt man vom oder auf dem Territorium des Königreichs, so muss man die Ware mitsamt ihres doppelten Warenwerts, sofern vorhanden, zurückerstatten. Andernfalls muss der doppelte Warenwert gemäß üblichem Marktpreis gezahlt werden.\\
(2) Dies gilt für alle Gegenstände, die dem Staatsgebiet entstammen oder einer Person auf dem Staatsgebiet gehören und widerrechtlich entwendet wurden.\\
(3) Auch gilt dies für Gegenstände, die gelöscht wurden.

\subsection{Mord}
(1) Tötet man eine Person vorsätzlich, so muss man die Person mit 1000 HTK entschädigen und wird hingerichtet.
(2) Dieses Recht unterscheidet nicht zwischen Mord und Totschlag.

\subsection{Körperverletzung}
Wer eine Person auf dem Gebiet des Königreichs physisch verletzt, muss mit einer Strafe von 15 HTK rechnen.

\subsection{Schwere Körperverletzung}
Verletzt man eine Person vorsätzlich so schwer, dass sie mindestens die Hälfte ihrer Leben verloren hat, so muss man 50 HTK zahlen.

\subsection{Verunglimpfung fraktioneller Insignien und Symbole}
(1) Wer fraktionelle Symbole von Hamavar, dessen Vasallen oder Verbündeten verunglimpft oder absichtlich entfernt, muss 50 HTK zahlen.\\
(2) Hierzu zählt ebenfalls das unerlaubte Tragen von Orden und Uniformen, beziehungsweise das Tragen von Orden zu einer inoffiziellen Uniform.

\subsection{Effekte und Fähigkeiten}
Man darf keine Effekte ohne Genehmigung haben. Verstöße werden mit 30 HTK Bußgeld vergolten.

\subsection{Verbotene Gegenstände}
Man darf keine verbotenen Gegenstände mit sich führen, ansonsten droht eine Hinrichtung.

\subsection{Betrug}
Wer sich oder einen Dritten durch Vorspiegelung falscher Tatsachen bereichern oder einen Vorteil verschaffen möchte, muss Bußgeld zahlen. Der Betrag wird an die Schwere der Straftat angepasst.

\subsection{Menschenexperimente}
Menschenexperimente sind nur unter staatlicher Aufsicht erlaubt.\\
(1) Dies erfordert kein Einverständnis der Testperson.\\
(2) Der Staat kann Einspruch gegen die Wahl der Testperson erheben und somit die Entscheidung annullieren.

\subsection{Geldwäsche}
Wer sich ohne Genehmigung der Königlichen Bank Coins prägt, muss eine Haftstrafe absitzen. Weiterhin wird das Konto der Person geleert und ihr temporär alle Geldzufuhren abgestellt. Die Person verliert somit ihre Kreditfähigkeit und all ihre Immobilien. Alles weitere wird gemäß \ref{vergehen} gehandhabt.

\subsection{Siegelfälschung}
Wer ein Schwarzsiegel, staatliches Zertifikat oder einen historischen Gegenstand ungenehmigt dupliziert, muss 1000 HTK Strafe zahlen. Zudem muss der Gewinn, der dadurch erwirtschaftet wurde, zurückgezahlt werden.

\subsection{Hehlerei}
Wer illegale Waren verkauft, muss 50 HTK Strafe zahlen.\\
(1) Gewerbsmäßige Hehlerei wird zusätzlich mit dem Tode vergolten.

\section{Zivilrecht}
\subsection{Sachbeschädigung}
Wer fremdes Eigentum auf dem Gebiet des Königreichs beschädigt, muss für die Schäden vollständig aufkommen und zusätzlich 100 HTK zahlen.

\subsection{Rechte des Eigentümers}
Wer auf staatlichem Grund rechtmäßig Eigentum erworben hat, darf dieses nutzen und verändern, wie er möchte, solange diese Handlungen ausschließlich gesetzeskonform sind.\\
(1) Erwirbt man ein Haus, so gehört einem nur das Innere des Hauses und nicht die Fassade, weshalb diese nicht verändert werden darf.\\
(2) Für Territorien gilt, dass man sie erst mit Genehmigung des Lehnsherrn bebauen darf.

\subsection{Schulden}\label{schulden}
(1) Jegliche Schulden, die man beim Kaiserreich, dem Adel oder den Bürgern des Kaiserreichs hat, müssen innerhalb von 10 Tagen zurückgezahlt werden.\\
(2) Tut man dies nicht, verliert man bis zur Rückzahlung zusammen mit zusätzlichen 80 HTK oder Gegenständen mit äquivalentem Wert die Kreditfähigkeit im Kaiserreich.\\
(3) Die Strafe nach dreifachem Aufschub liegt im Ermessen des zuständigen Gerichts.

\section{Staatsrecht}

\subsection{Betreten des Staatsgebietes}
Das Betreten des Staatsgebietes darf nur mit einer ausdrücklichen Genehmigung erfolgen. Betritt man das Staatsgebiet ohne diese Aufenthaltsgenehmigung, so muss man 50 HTK Strafe zahlen.

\subsection{Spionage}
Strategische Aufklärung und Spionage auf dem Staatsgebiet sind nicht erlaubt und daher strafbar. Aufgrund der besonderen Schwere wird dies mit einer Hinrichtung und 1000 HTK Strafe vergolten.\\
(1) Dies gilt nicht für Operationen, die durch den Staat ausdrücklich genehmigt wurden. \\
(2) Man darf ebenso wenig ohne Genehmigung das hamavarische Territorium im Zuschauermodus durchqueren, denn gilt dies ebenfalls als Spionage.

\subsection{Finanzeller Status des Reichs}
Der Imperator kann auf nationaler Ebene nicht verschuldet sein.

\subsection{Hochverrat}\label{verrat}
(1)	Als Hochverräter gilt, wer
\begin{enumerate}
	\item Staatsgeheimnisse ohne Genehmigung verbreitet oder versucht auf diese unerlaubt zuzugreifen.
	\item Eine absichtliche Schwächung des Staates herbeiführt
	\item Die Befehle des Imperator verweigert
\end{enumerate}
(2)	Der Strafsatz gleicht dem Strafsatz des Mordes an dem Imperator.

\subsection{Religiöse Gegenstände}
Wer Gegenstände religiöser Natur beschädigt oder zerstört oder auf religiösem Boden Verbrechen begeht, muss eine Bußgeldstrafe in Höhe von 200 HTK zahlen und wird hingerichtet.\\
(1) Entweiht man religiöse Gebäude kommt dies dem dreifachen Strafsatz gleich.

\subsection{Majestätsbeleidigung}
Beleidigt man den Imperator, den Staat oder übergeordnete Staatsvertreter, so muss man 1000 HTK Strafe zahlen und wird anschließend hingerichtet.

\subsection{Blasphemie}
Wer die Existenz des Pantheons der allmächtigen Legastheniker leugnet, deren Anhängerschaft, Andenken oder Weltbild verunglimpft, oder dem Wort derer, ausgetragen durch den allgerechten Imperator widerspricht, hat 1000 HTK zu zahlen und wird anschließend hingerichtet.
\end{document}