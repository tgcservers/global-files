\documentclass{article}
\usepackage[utf8]{inputenc}
\usepackage{enumerate}
\usepackage{ragged2e}
\usepackage{etoc}

\renewcommand{\thesection}{}
\renewcommand{\thesubsection}{§\arabic{subsection}}

\title{Baurecht}
\author{Zweiter Konvent von Fonxangcu}
\date{31. Juli 1991}

\begin{document}
\maketitle
\vspace*{\fill}
\paragraph{Baurecht des altehrwürdigen Kaiserreichs Júxanbong}

\newpage
\topskip0pt
\vspace*{\fill}
\begin{Center}
\textbf{1. Fassung}
\vspace*{\fill}
\end{Center}
\newpage
\tableofcontents
\newpage
\section{Baugesetzbuch (BauGB)}
\localtableofcontents

\subsection{Antragstellung}
\begin{enumerate}[(1)]
    \item Um einen Bau vollziehen zu dürfen, muss man einen Bauantrag stellen, der gemäß \ref{baugen} genehmigt werden muss.
    \item Er muss nachfolgende Informationen beinhalten:
    \begin{enumerate}[1.]
        \item Den Auftraggeber
        \item Den Grundstücksbesitzer
        \item Das Grundstück
        \item Kurzgefasste Beschreibung des Gebäudes
        \item Das Bauunternehmen gemäß \ref{bauunt}
    \end{enumerate}
    \item Für mehrere Projekte kann man auch einen zusammengefassten Gesamtantrag stellen, der allerdings alle Details für jedes Gebäude beinhält.
\end{enumerate}

\subsection{Besitzungshoheit}
\begin{enumerate}[(1)]
    \item Man darf nur Grundstücke erwerben, die im júxanischen Reichsgrundbuch eingetragen wurden.
    \item Ist man nicht Besitzer des Grundstücks, so benötigt man eine Vollmacht vom Grundstücksbesitzer, die den Regulatorien von §1 HGB entspricht.
\end{enumerate}

\subsection{Baugenehmigung} \label{baugen}
\begin{enumerate}[(1)]
    \item Eine Baugenehmigung kann nur durch ein staatliches Rechtsgeschäft gemäß §1 HGB erteilt werden.
    \item Dieses Rechtsgeschäft muss mit Genehmigung durch den Lehnsherrn geschehen.
    \item Ihm direkt übergeordnete Lehnsherrn dürfen die Genehmigung für ungültig erklären.
    \item Dies ist kein absoluter Akt und kann daher vor dem für diese Region und Ebene zuständigen Gericht angefochten werden.
\end{enumerate}

\subsection{Eingetragenes Bauunternehmen} \label{bauunt}
\begin{enumerate}[(1)]
    \item Grundstücksbebauungen können nur unter Beaufsichtigung durch ein staatlich anerkanntes Bauunternehmen vorgenommen werden.
    \item Unternehmen können sich durch die júxanische Regierung als Bauunternehmen eintragen lassen.
\end{enumerate}

\subsection{Baumaterial}
\begin{enumerate}[(1)]
    \item Das Baumaterial muss den geltenden Gesetzen entsprechen und muss auf legalem Wege erworben und zur Baustelle transportiert worden sein.
    \item Es ist nicht genehmigt, gefährliches oder aus sonstigem Grund verbotenes Material im Haus zu verbauen.
\end{enumerate}

\section{Grundbesitzordnung (GrBO)}
\localtableofcontents
\subsection{Allgemeiner Teil}
\subsection{Gültigkeit}
\begin{enumerate}[(1)]
    \item Gültig ist ein Grundbesitz, solange er durch eine explizit für diese Region zuständige oder einer solchen übergeordneten, von dem
    Kaiserreich Júxanbong als solche ernannten Behörde gemäß \ref{rgbuch} in das Reichsgrundbuch eingetragen wurde.
    \item Mit ausreichender und nachvollziehbarer Begründung können die in Absatz 1 genannten Behörden diesen Eintrag aus dem Reichsgrundbuch streichen.
    \item Eine Eintragung im Reichsgrundbuch gewährt vollständige Eigentumsrechte gemäß GrBO.
\end{enumerate}
\subsection{Reichsgrundbuch}\label{rgbuch}
\end{document}