\documentclass{article}
\usepackage[utf8]{inputenc}
\usepackage{enumerate}
\usepackage{ragged2e}

\renewcommand{\thesection}{\Roman{section}}
\counterwithout{subsection}{section}
\renewcommand{\thesubsection}{§\arabic{subsection}}

\title{Codex Corae}
\author{Erster Konvent von Fonxangcu}
\date{12. Juli 1991}

\begin{document}
\maketitle
\vspace*{\fill}
\paragraph{Verfassung des altehrwürdigen Kaiserreichs Júxanbong}

\newpage
\topskip0pt
\vspace*{\fill}
\begin{Center}
\textbf{1. Fassung}
\end{Center}
\begin{flushright}
\textit{Berlin, den 12. Juli 2023}
\end{flushright}
Der Wunsch nach ewiger Vernunft begründete diesen Staat und so soll er auf ewig für die Rechtschaffenheit stehen, die ihm zugrundeliegt.
Diese Verfassung legt den Grundstein für Gerichte, die weder arm noch reich, bürgerlich noch fremd und weder Kaiser noch Bauer kennen.
Jahre eingehendster Auseinandersetzung mit dem Recht des Einzelnen führten zu jenem Resultat, welches das Manifest einer nie dagewesenen
Rechtsstaatlichkeit bildet.\\
Sowohl dem Beherrschten als auch dem Herrschenden soll als Mahnung gelten: Fügt euch dem Recht mit Verstand und Vernunft, denn ist dies der einzige Weg, eine
harmonischen Gesellschaft zu begründen.
\\\\
\textbf{Caroa ku Júxanbong Bü Ri Mangdi}
\\\\\\\\\\\\\\\\\\\\\\\\\\\\\\\\
\vspace*{\fill}

\newpage
\tableofcontents
\newpage
\section{Tabula prima: De legum}
\subsection{Lex votum motivum}
Die \textit{lex votum motivum} besagt, dass ein Amt in einer Versammlung bei Stimmgleichheit eine zweite Stimme erhält.

\subsection{Lex votum maiestatis}
Einem führenden Amt wird die Fähigkeit zugesichert, ohne Begründung ein Majestätsvotum zu veranlassen, bei welchem nur ein geringer Kreis an Mitgliedern teilnehmen darf.

\subsection{Lex votum imperatoris}
Dem Kaiser steht es zu, Beschlüsse niederrangiger Instanzen zu annullieren. Dieses Recht steht dem Hochkonzil nicht zu.

\subsection{Annuitätsprinzip}
Das Annuitätsprinzip besagt, dass eine Amtszeit stets ein Jahr dauern muss.

\subsection{Sitte des Prinzipats}
Der Kaiser soll sich niemals als erhabener oder majestätischer als der Hochkonzil darzustellen.

\subsection{Mos consensus}
Dem Kaiser ist es nicht gestattet, sich über Beschlüsse des Hochkonzils hinwegzusetzen oder den Hochkonzil abzuschaffen oder zu entmachten. Ebensowenig kann der Hochkonzil den amtierenden oder zukünftigen Kaiser absetzen oder entmachten.

\subsection{Passivus est activus}
Jegliche verbotene aktive Tat ist auch als passive Tat untersagt.

\subsection{Clausula praesidii legis}
Mangelnde oder fehlerhafte Kenntnisse der Rechtslage gewähren keine rechtliche Immunität.

\subsection{Clausula absentiae}
Bei selbstverschuldeter und unentschuldbarer Abwesenheit vor Gericht, dürfen Prozesse in Abwesenheit der fehlenden Partei abgehalten werden.

\subsection{Clausula reverentiae}
Man muss der Richterschaft Respekt zollen.

\section{Tabula secunda: De re publica}
\subsection{Der Kaiser}
\begin{enumerate}[(1)]
    \item Der Kaiser ist die Reichsgewalt und repräsentiert himmlische, wie auch irdische Macht. Ihm kommen die \textit{lex votum motivum}, die \textit{lex votum maiestatis} und die \textit{lex votum imperatoris} zu.
    \item Der Kaiser ist schuldfähig, doch ist kein Gericht der Welt in der Lage, ihn ohne seine Zustimmung zur Verantwortung zu ziehen.
    \item Der Kaiser trägt das Adelsprädikat "Erhabene Majestät".
    \item Der Kaiser darf nur dann eine Frau sein, wenn dies das Kaiserhaus vor einem vorzeitigen Ende bewahrt.
    \item Es ist keiner Person erlaubt, den Kaiser ohne dessen Einverständnis zu bevormunden.
    \item Friedensabkommen erfordern die Einverständnis des Kaisers.
\end{enumerate}

\subsection{Der Hochkonzil}
\begin{enumerate}[(1)]
    \item Die oberste Versammlung des Kaiserreichs ist der Hochkonzil. Diesem kommen die Befugnisse der höchsten ordentlichen Gerichtsbarkeit und der Verfassungsgebung zu. Die Mitglieder nennen sich Tribune. In seiner Form als Gericht wird es als \textit{tribunal dignitatis} bezeichnet.
    \item Dem Hochkonzil sitzt der Premierminister vor.
    \item Dem Hochkonzil und deren Mitgliedern steht es nicht zu, den Kaiser abzusetzen.
    \item In Entscheidungen des Hochkonzils hat der Kaiser ebenfalls eine Stimme, die sich bei Stimmgleichheit auf die \textit{lex votum motivum} ausweitet.
    \item Der Hochkonzil ist in der Lage, neue Gesetze zu beschließen, welche allerdings weder der \textit{tabula prima}, noch sonstigen Rechtsgrundsätzen der Verfassung widersprechen dürfen.
    \item Der Hochkonzil wird vom Kaiser gewählt.
\end{enumerate}

\subsection{Der Premierminister}
\begin{enumerate}[(1)]
    \item Der Premierminister ist die rechte Hand des Kaisers und gleichzeitig Regierungschef.
    \item Er vertritt den Kaiser in dessen Abwesenheit
\end{enumerate}

\subsection{Der Großkommandant}
\begin{enumerate}[(1)]
    \item Der Großkommandant ist der Befehlshaber der Kaiserlichen Streitkräfte.
    \item Kriegserklärungen müssen sowohl vom Kaiser als auch dem Großkommandanten zugestimmt werden. Dies dient der Absicherung, dass die Streitkräfte bereit zum Einsatz sind und die Lage als einer Kriegserklärung angemessen erachtet wird.
    \item Friedensabkommen obliegen der Entscheidung des Kaisers.
\end{enumerate}

\subsection{Die Großfürstliche Exzellenz}
\begin{enumerate}[(1)]
    \item Die Großfürstliche Exzellenz ist der Verwalter der Staatsgelder.
    \item Besteht ein drohender Bankrott, so darf die Großfürstliche Exzellenz zur Wahrung des Staatshaushalts dem Hochkonzil Ausgaben verbieten.
\end{enumerate}

\subsection{Potestas concilii}
Bei gerechtfertigtem Verdacht auf die Ausnutzung oder den sonstig fehlerhaften Einsatz der Amtsgewalt, ist es möglich, mittels eines außerordentlichen Ratsbeschlusses eine Person seines Amtes zu entheben.

\subsection{Die Könige}
\begin{enumerate}[(1)]
    \item Ein König ist das Oberhaupt eines Königreichs.
    \item Ihm steht das Adelsprädikat "Königliche Hoheit" zu.
\end{enumerate}

\subsection{Der Fürst}
\begin{enumerate}[(1)]
    \item Den Fürsten obliegt die Verwaltung der Kommandaturen.
    \item Sie unterstehen ihrem König.
\end{enumerate}

\section{Tabula tertia: De delicto}
\subsection{Natura delictorum}
\begin{enumerate}[(1)]
    \item Ein \textit{crimen contra publicum} (Verbrechen gegen die Öffentlichkeit) ist ein Verstoß gegen den Menschen in seiner Funktion als Bürger, sprich wenn er als dieser Schädigung erfährt, die geringerer Natur sind als ein \textit{crimen contra hominem}.
    \item Ein \textit{crimen contra hominem} (Verbrechen gegen den Menschen) ist ein grober Verstoß gegen die vorherrschende Moral.
    \item Ein \textit{crimen contra rem publica} (Verbrechen gegen den Staat) ist ein Verbrechen, das sich wegen seiner Natur nur direkt gegen den Staat selbst richten kann. Diese werden zur Wahrung einer geordneten und funktionierenden Gesellschaft als Delikt gesonderter Schwere betrachtet.
\end{enumerate}

\subsection{Lex advocatio}
\begin{enumerate}[(1)]
    \item Einem Beschuldigten steht es zu, die Beschuldigung vor Gericht anzufechten.
    \item Dieses Recht darf keinem verwehrt werden.
    \item Advokaten müssen von der Beklagten oder dem Klagenden selbst gestellt werden.
    \item Es besteht kein Grundrecht auf einen Advokaten.
\end{enumerate}

\subsection{Nullum iudicium sine audientia}
Es gibt kein rechtskräftiges Urteil ohne Gerichtsprozess.

\subsection{Strafsätze}
Strafsätze sollten an der Schwere und Häufigkeit des Verbrechens des Einzelnen bemessen werden.

\subsection{Präjudizien}
\begin{enumerate}[(1)]
    \item Ein Gericht muss stets nach richtungsweisenden Gerichtsentscheidungen urteilen.
    \item Erachtet ein Gericht eine Gerichtsentscheidung als unpassend, kann es diese aufheben und durch die eigene ersetzen, sofern das Gericht dem vorigen übergeordnet ist und für die betroffene Region zuständig ist. Nur das \textit{tribunal dignitatis} kann eigene Entscheidungen aufheben.
    \item Ebenfalls sind Gerichte in der Lage, eigene Entscheidungen zu fällen, sofern keine geeignete Präjudiz für den Fall besteht.
\end{enumerate}

\subsection{In culpa est, qui suus culpa sentit}
Es gilt der Grundsatz, dass ein Gericht nur dann einen Schuldspruch ohne vorliegende Präjudiz tätigen darf, wenn nach Ermessen der Richterschaft ein Bewusstsein des Verstoßes gegen im Kaiserreich geltende moralische Normen besteht.

\subsection{Urteilsanfechtung}
\begin{enumerate}[(1)]
    \item Jedem, der sich durch das Urteil geschädigt sehen kann, steht es zu, dieses anzufechten. Tut man dies, so wird die Rechtschaffenheit des Urteils von der nächsthöheren Instanz überprüft.
    \item Befindet man sich bereits in der höchsten Instanz, so ist das Urteil rechtskräftig und final.
\end{enumerate}

\subsection{Instanzen}
Die Instanzen des Kaiserreichs lauten wie folgt:
\begin{enumerate}[1.]
    \item Die Fürstenkammer ist ein Gericht der Bezirksgerichtsbarkeit und übernimmt Entscheidungen auf der Ebene der Komtureien. Ihr steht der amtierende Fürst vor.
    \item Die Königskammer urteilt über Angelegenheiten der Landesgerichtsbarkeit, bewegt sich also auf der Ebene der Königreiche. Ihr steht der amtierende König vor.
    \item Das \textit{tribunal dignitatis} urteilt über Reichsangelegenheiten und ist dementsprechend Repräsentant der ordentlichen Reichsgerichtsbarkeit. Es besteht aus dem Hochkonzil.
\end{enumerate}

\subsection{Reichsverfassungsgericht}
\begin{enumerate}[(1)]
    \item Das Reichsverfassungsgericht ist ein Gericht, das der außerordentlichen Reichsgerichtsbarkeit angehört. Dies bedeutet, dass es kein Teil der gerichtlichen Instanzen ist.
    \item Zum Aufgabenbereich des Reichsverfassungsgerichts gehören Verfassungsbeschwerden und rechtswidrige Urteilssprüche.
    \item Entscheidungen des Reichsverfassungsgerichts kann man nicht anfechten.
    \item Das Reichsverfassungsgericht steht über allen Reichsgerichten.
    \item Die Richterschaft besteht aus dem Kaiser, sowie einem von ihm gewählten Vertreter, dem Premierminister und einem von ihm gewählten Vertreter.
\end{enumerate}

\end{document}