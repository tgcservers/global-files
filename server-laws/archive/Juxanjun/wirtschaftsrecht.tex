\documentclass{article}
\usepackage[utf8]{inputenc}
\usepackage{enumerate}
\usepackage{ragged2e}
\usepackage{etoc}

\renewcommand{\thesection}{}
\renewcommand{\thesubsection}{\Roman{subsection}}
\counterwithout{subsubsection}{subsection}
\renewcommand{\thesubsubsection}{§\arabic{subsubsection}}

\title{Wirtschaftsrecht}
\author{S.E.M. Der Kaiser}
\date{19. Juli 1991}

\begin{document}
\maketitle
\vspace*{\fill}
\paragraph{Wirtschaftsrecht des altehrwürdigen Kaiserreichs Júxanbong. Die nachfolgenden Gesetze stammen ausnahmslos aus der Feder S.E.M. des Kaisers Bü Ri Mangdi und entsprechen den von-Preußen-Aeternium-Konventionen (§§1-14), den von-Preußen-Konventionen (§§15f.), und den Mangdi-Konventionen (§§16ff.). Es wäre daher freundlich, wenn niemand behaupten würde, dass ich die Gesetze vom Kaiserreich Deutschland kopiert habe, zumal ich sie für dieses in meiner Funktion als dessen Reichskanzler in Absprache mit Primordial Deity verfasst habe.}
\newpage
\tableofcontents
\newpage
\section{Handelsgesetzbuch (HGB)}
\localtableofcontents
\subsection{Grundregelungen}
\subsubsection{Rechtsgeschäft} \label{rechtsg}
\begin{enumerate}[(1)]
    \item Als gültiges Rechtsgeschäft gilt jeder der nachfolgenden Rechtsakte, sofern dieser gänzlich gesetzeskonform ist:
    \begin{enumerate}[1.]
        \item Testamentarische Verfügung
        \item Vertragsgeschäfte
    \end{enumerate}
    \item Ein Rechtsgeschäft verliert im Kontext von §23 CdxC seine Gültigkeit auch dann, wenn nach gerichtlichem Urteil kein Bewusstsein der Schuld vorliegt.
    \item Es bedarf einer Beglaubigung durch einen Notar, der vom Kaiserreich Júxanbong ernannt wurde, um vor Gericht gültig zu sein.
    \item Finanztransaktionen müssen immer mit dem Konto des Betroffenen ausgeführt werden.
\end{enumerate}


\subsubsection{Firmenführung} \label{firm}
\begin{enumerate}[(1)]
    \item Wer eine Firma gründet, muss diese in das Júxanische Reichshandelsregister (\ref{register}) eintragen lassen.
    \item Diese Firmen müssen präzise Buch führen (\ref{buch}).
    \item Eine Person kann erst dann rechtskräftig zum Eigentümer einer Gesellschaft ernannt werden, sofern ein Rechtsgeschäft (\ref{rechtsg}) vorliegt, in welchem der vorherige Eigentümer die Gesellschaft dem neuen Eigentümer nachweislich überträgt und der neue Eigentümer in das Reichshandelsregister (\ref{register}) eingetragen wurde.
    \item Sobald ein Eigentümer beabsichtigt, zurückzutreten und kein neuer Eigentümer gemäß Absatz 3 nachfolgt, ist der Eigentümer für die offenen Geschäfte des Unternehmens verantwortlich. Laufen diese aus, so darf dieser zurücktreten.
    \item Solange kein Eigentümer gemäß Absatz 3 nachfolgt, darf die Gesellschaft keine neuen Geschäfte aufnehmen.
    \item Absatz 3 ff. gilt nur dann, wenn es sich um kein Familienunternehmen (\ref{familien}) handelt.
\end{enumerate}


\subsubsection{Handelsregister} \label{register}
\begin{enumerate}[(1)]
    \item Einträge im Handelsregister werden von der Großfürstlichen Exzellenz vorgenommen.
    \item Jegliche Gesellschaft muss mit dem Namen des Eigentümers, der Adresse der Hauptzweigstelle, der Rechtsform gemäß \ref{gesellform} und ihrer Entsprechung im Register Internationaler Rechtsformen, sowie dem Namen und der Marke der Gesellschaft in das Handelsregister eingetragen werden.
    \item Hoflieferanten können zusätzlich die, ihnen zustehenden Garantiesätze eintragen lassen.
\end{enumerate}


\subsubsection{Auslandshandelsgebühren}
\begin{enumerate}[(1)]
    \item Gesellschaften, die im Ausland Tochterunternehmen eröffnen, müssen zusätzliche Gebühren an das Kaiserreich Júxanbong zahlen.
    \item Diese Gebühren müssen den Empfehlungen der Großfürstlichen Exzellenz entsprechen. 
    \item Sind die Gebühren zur Eröffnung der Zweigstelle im Ausland billiger, so muss die Gesellschaft die Kostendifferenz zur Empfehlung (Absatz 2) an das Júxanische Reich zahlen.
    \item Andernfalls muss es die Zusatzgebühren der Empfehlung entsprechend erstatten.
    \item Von dieser Regelung ausgenommen sind Handelsgesellschaften logistischer Tätigkeit. Im Falle von Absatz 3 müssen sie keine Zusatzgebühren bezahlen und im Falle von Absatz 4 übernimmt das Kaiserreich Júxanbong die Kostendifferenz zur Empfehlung.
    \item Letzteres verliert seine Wirksamkeit, sofern die Großfürstliche Exzellenz die Unterstützungen innerhalb des fraglichen Staates verwehrt.
    \item Eröffnet eine ausländische Gesellschaft im Kaiserreich Júxanbong eine Zweigstelle, so muss sie die Gebühren vollständig erstatten.
\end{enumerate}


\subsubsection{Familienunternehmen} \label{familien}
\begin{enumerate}[(1)]
    \item Familienunternehmen dürfen nur von júxanischen Staatsbürgern gegründet werden.
    \item Sie dürfen lediglich von Familienmitgliedern geführt werden.
    \item Absatz 2 verliert seine Wirksamkeit, sobald das fragliche Mitglied kein júxanischer Staatsbürger ist.
    \item Die Gesellschaft kann nur dann an andere Familien ausgehändigt werden, sofern der Eigentümer verfügt, dass die Gesellschaft zu einer nicht-familiären Gesellschaft umgewidmet und anschließend an den außerfamiliären Eigentümer übertragen wird.
    \item Im Falle von Absatz 4 kann das Unternehmen nicht zu Lebzeiten des neuen Eigentümers zu einem Familienunternehmen umgewidmet werden.
    \item Verstirbt das letzte Mitglied der Familie, so verliert das Familienunternehmen seine Geschäftsfähigkeit gemäß \ref{firm} Abs. 5, sofern keine rechtsgültige Nachfolge bewirkt wurde.
    \item Gesellschaften mit erweiterten Anteilsrechten können nicht als Familienunternehmen eingetragen werden.
\end{enumerate}


\subsubsection{Buchführung} \label{buch}
\begin{enumerate}[(1)]
    \item Eine Gesellschaft ist verpflichtet, ohne Auslassungen Buch zu führen.
    \item Zu jedem Geschäft muss folgendes vermerkt werden:
    \begin{enumerate}[1.]
        \item Verkäufer (sofern er von der Hauptgesellschaft abweicht)
        \item Kunde (sofern er von der Hauptgesellschaft abweicht)
        \item Gesamtpreis (dies schließt auch Tauschwaren ein)
        \item Gehandelte Gegenstände, beziehungsweise Kommentar zu gegenstandslosen Transaktionen
    \end{enumerate}
    \item Liegt eine gegenstandlose Transaktion, beispielsweise Schenkung oder Spenden in finanzieller oder gegenständlicher Form  vor, so müssen die Kommentare sinnig sein und für die Großfürstliche Exzellenz ersichtlich sein.
    \item Die Gesellschaft muss monatlich der júxanischen Großfürstlichen Exzellenz die Buchhaltung zukommen lassen.
    \item Verstöße gegen die Buchhaltungbestimmungen haben das Strafmaß gemäß \ref{apored} Abs. 3 zur Folge.
    \item In den ersten drei Monaten nach der Gründung muss insgesamt ein eindeutiger Gewinn von fremder Seite vorliegen.
    \item Danach muss dies im Abstand von einem Jahr regelmäßig eingehalten werden.
    \item Andernfalls muss das Unternehmen Strafgebühren zahlen, die von der Großfürstlichen Exzellenz beschlossen werden.
    \item Im Falle, dass nur Verluste registriert werden, muss das Unternehmen geschlossen werden.
\end{enumerate}


\subsubsection{Insolvenz} \label{apored}
\begin{enumerate}[(1)]
    \item Verfügt eine Gesellschaft nur noch über die Hälfte des Stammkapitals, muss es Konkurs anmelden.
    \item Eine Gesellschaft, welche bankrott geht und zuvor nicht an einen neuen Eigentümer überschrieben wurde, verliert die Genehmigung, Geschäfte auszuüben und wird aus dem Handelsregister ausgetragen.
    \item Geht die Gesellschaft Absatz 1 oder 2 nicht nach, so muss der Eigentümer die Haftung ungeachtet der Rechtsform übernehmen und die Gesellschaft wird umgehend aus dem Handelsregister ausgetragen und ist somit nicht länger fähig, ihren Eigentümer zu wechseln.
\end{enumerate}


\subsubsection{Internationaler Handel}
\begin{enumerate}[(1)]
    \item Um Transaktionen in das Ausland und Inland vorzunehmen, muss man eine Zweigstelle auf júxanischem Territorium im Reichshandelsregister registriert haben.
\item Zweigstellen müssen über eine júxanische Rechtsform verfügen.
\item Es ist júxanischen Staatsbürgen untersagt, die Hauptzweigstelle im Ausland zu gründen.
\item Der Buchführungspflicht gemäß \ref{buch} unterliegen jegliche júxanischen Hauptzweigstellen und deren Zweigstellen, sowie jegliche Hauptzweigstellen und deren Zweigstellen, sofern sie eine Zweigstellung im Kaiserreich Júxanbong haben.
\end{enumerate}

\subsubsection{Haftung}
\begin{enumerate}[(1)]
    \item Haftung für die Waren übernimmt derjenige, der sie zurzeit besitzt.
    \item Dies gilt sowohl auf júxanischem Grunde als auch für Gesellschaften mit Zweistelle oder Hauptzweigstelle auf júxanischem Grund.
    \item Die Haftung gegenüber dem Staat unterliegt stets dem Gesellschafter.
    \item Im Falle, dass der Gesellschafter nicht die juristische Person der Gesellschaft ist, trifft Absatz 3 nur dann zu, wenn die juristische Person zahlungsunfähig ist.
\end{enumerate}

\subsubsection{Banken}
\begin{enumerate}[(1)]
    \item Júxanische Banken müssen die Kontoinformationen der Kunden bei Anfrage durch die Großfürstliche Exzellenz offenlegen.
    \item Ohne Beschluss der Großfürstlichen Exzellenz dürfen sie keine Konten einfrieren.
    \item Júxanischen Gesellschaften ist es untersagt, sich bei Banken zu registrieren, die keine Vertragspartei im Bankenabkommen oder ähnlichen Abkommen sind.
    \item Verstöße gegen Absatz 3 werden für Gesellschaften gemäß \ref{apored} Abs. 3 geahndet.
\end{enumerate}

\subsubsection{Umsatzsteuern}
\begin{enumerate}[(1)]
    \item Jeder Umsatz eines gewinnorientierten Unternehmens muss gemäß aktuellem Steuersatz von Júxanbong besteuert werden.
    \item Fehlerhafte Besteuerung ist strafbar und wird gemäß \ref{apored} Abs. 3 geahndet.
\end{enumerate}

\subsection{Rechtsformen}
\subsubsection{Rechtsform} \label{gesellform}
\begin{enumerate}[(1)]
    \item Die Rechtsform einer Gesellschaft bestimmt die Haftungs-, sowie Handelsbedingungen.
    \item Im Kaiserreich Júxanbong anerkannte Rechtsformen sind:
        \begin{enumerate}[1.]
            \item Company with right to transport (RTT)
            \item Limited liability company (Ltd.)
            \item His Sublime Majesty's Company (HSMC)
            \item Private non-limited company (PNL)
            \item Corporation with extended share rights (Ext.)
            \item Corporation with extended share rights and right to transport (Ext. RTT)
            \item Imperial company (IC)
        \end{enumerate}
\item Gesellschaften haften mit dem Kapital der juristischen Person.
\item Ebenso gehört der Umsatz der Gesellschaft der juristischen Person.
\item Die juristische Person von Handelsgesellschaften ist die Handelsgesellschaft selbst.
\item Eigentümer von Privatunternehmen sind dessen juristische Person.
\item Die juristische Person von einer Reichsgesellschaft ist der Staat.
\item Im Falle von Absatz 5 müssen die Eigentümer im Handelsregister den Anteil am Gewinn der Gesellschaft registrieren.
\item Gesellschaften haftungsbeschränkter Rechtsformen müssen bei ihrer Gründung über ein Mindeststammkapital verfügen, welches von der Großfürstlichen Exzellenz beschlossen wird.
\end{enumerate}


\subsubsection{Private non-limited company}
\begin{enumerate}[(1)]
    \item Die Private non-limited company ist eine haftungsunbeschränkte Rechtsform.
    \item Gesellschaften dieser Rechtsformen verfügen über kein separates Stammkapital.
    \item Die juristische Person dieser Gesellschaft ist der Gesellschafter.
    \item Gesellschaften dieser Rechtsform können keine Gesellschaftsanteile verkaufen.
\end{enumerate}

\subsubsection{Limited liability company}\label{llc}
\begin{enumerate}[(1)]
    \item Die Limited liability company ist eine haftungsbeschränkte Rechtsform.
    \item Die juristische Person dieser Gesellschaft ist das Unternehmen selbst.
    \item Es bestehen keine Leistungseinschränkungen für diese Rechtsform.
    \item Gesellschaften dieser Rechtsform können nicht weniger als Zehntelanteile verkaufen.
\end{enumerate}

\subsubsection{Company with right to transport}
\begin{enumerate}[(1)]
    \item Das Gründungsrecht und die juristische Person der Company with right to transport entspricht der Limited liability company.
    \item Gesellschaften mit Transportrecht unterliegen nur teilweise den Auslandshandelsgebühren.
    \item Sie dürfen nur mit Dienstleistungen handeln, die mit dem Transport von Waren und Personen zusammenhängen. Sachleistungen dürfen sie nicht erbringen. 
\end{enumerate}

\subsubsection{His Sublime Majesty's Company}
\begin{enumerate}[(1)]
    \item Gesellschaften Seiner Erhabenen Majestät sind Gesellschaften im staatlichen Besitz.
    \item Sie unterliegen keinen Leistungseinschränkungen.
    \item Jegliche Einnahmen gehören dem Kaiserreich.
    \item Gesellschaften dieser Rechtsform dürfen keine Anteile verkaufen.
\end{enumerate}

\subsubsection{Imperial company}
\begin{enumerate}[(1)]
    \item Kaiserliche Unternehmen sind Unternehmen gesonderten staatlichen Besitzes.
    \item Sie können teilweise auch Eigentum von Privatpersonen sein.
    \item Der Staat ist hier lediglich Teilhaber.
    \item Imperial companies unterliegen keinen Leistungseinschränkungen.
    \item Es können keine Unternehmen jeglicher Extended share rights als Imperial company registriert werden.
\end{enumerate}

\subsubsection{Corporation with extended share rights}\label{ext}
\begin{enumerate}[(1)]
    \item Eine Corporation with extended share rights verfügt über das Recht, eigene Unternehmensaktien auszustellen.
    \item Dies bedingt, dass das Unternehmen beliebig große Anteile an ihrem Unternehmen verkaufen dürfen.
    \item Diese Rechtsform ist nicht leistungsbeschränkt.
    \item Gesellschaften dieser Rechtsform sind haftungsbeschränkte Gesellschaften gemäß \ref{llc}, verfügen allerdings über erweiterte Anteilsrechte, wie in Absatz 1 beschrieben.
    \item Corporation with extended share rights sind nicht verpflichtet, einen, den Anteilen entsprechenden Gewinnanteil auszuzahlen, sondern können sich diesem mit einer regelmäßigen Auszahlung pro Aktie annähern.
\end{enumerate}

\subsubsection{Corporation with extended share rights and right to transport}
Diese Rechtsform entspricht einer Company with right to transport mit erweiterten Anteilsrechten gemäß \ref{ext} Abs. 1f., 5.

\subsubsection{Organization}
\begin{enumerate}[(1)]
    \item Die Organization ist eine Non-Profit-Organisation, die keine Privateinnahmen generieren darf.
    \item Sie dürfen unversteuerte Spenden entgegennehmen.
    \item Sie dürfen keine Sachleistungen erbringen.
\end{enumerate}

\subsubsection{Hoflieferanten}
\begin{enumerate}[(1)]
    \item Jegliches Haus und jeglicher Titel, der vom Kaiser das Recht ausgeschrieben bekommen hat, Hoflieferanten auszuwählen, darf nur bei diesen einkaufen.
    \item Hoflieferanten müssen mit einem offiziellen Schreiben vom Ausstellenden oder einem, von ihm ausgewählten Vertreter, ernannt werden.
    \item Diese dürfen das Wappen des Ernennenden in Verbindung mit ihrer Marke tragen, dürfen dies allerdings weder als eigene Marke, noch anderweitig als alleinstehendes Symbol ohne Garantiesatz genutzt werden.
    \item Der Garantiesatz verweist auf den Dienst des Unternehmens als Hoflieferant. Er beinhält sowohl den Ernennenden als auch die Kategorie des Unternehmens und den ausgeschriebenen Markennamen.
    \item Der Garantiesatz kann beliebig mit oder ohne Wappen vom Unternehmen auf Produkten, Produktbeschreibungen, Dokumenten und im Handelsregister vermerkt werden.
    \item Lieferanten des Kaiserhauses tragen den Rechtsformzusatz ``B.I.A.`` (``By imperial appointment``)
\end{enumerate}

\end{document}