\documentclass{article}
\usepackage[utf8]{inputenc}
\usepackage{textcomp}
\usepackage{amsmath}
\usepackage{enumerate}
\usepackage{ragged2e}
\usepackage{blindtext}

\renewcommand{\thesection}{\Roman{section}}
\counterwithout{subsection}{section}
\renewcommand{\thesubsection}{§\arabic{subsection}}

\title{Verfassung der VR Swagistan}
\author{the.god.emperor}
\date{19. August 2023}

\begin{document}
\maketitle
\newpage
\tableofcontents
\newpage


\section{Grundsätzliches}
\subsection{Loyalitätsgebot}
\begin{enumerate}[(1)]
    \item Jeder Bürger muss zu jeder Zeit unnd um jeden Preis die Würde und das Erbe der VR Swagistan beschützen.
    \item Wer der Nation den Rücken zuwendet, sei es durch die ungenehmigte Unterstützung eines Feindes, den Überlauf zu diesem oder den Austritt und die anschließende Gründung einer neuen Nation ohne die Genehmigung der Volksrepublik, wird auf die Blacklist gesetzt.
    \item Jegliche Person auf der Blacklist muss ausnahmslos verfolgt und zur Strecke gebracht werden.
    \item Verrätern der Volksrepublik sollte selbst beim Zeigen von Reue niemals verziehen werden.
\end{enumerate}

\subsection{Oberster Sowjet}
\begin{enumerate}[(1)]
    \item Regierungsentscheidungen werden durch den Obersten Sowjet entschieden.
    \item Der Vorsitzende des Obersten Sowjets ist Genosse Generalsekretär $jaap\_stam$.
    \item Weitere stimmberechtigte Mitglieder des Obersten Sowjets sind der Leiter des KGB Genosse $robertogmor$ und der Marschall von Swagistan Genosse $the.god.emperor$.
    \item Die Ressortleiter sind ebenfalls Mitglied des Obersten Sowjets, sind allerdings nicht stimmberechtigt.
    \item Im Falle einer Stimmgleichheit verfügt der Generalsekretär über eine zweite Stimme.
\end{enumerate}

\subsection{Streitkräfte der Volksrepublik Swagistan}
\begin{enumerate}[(1)]
    \item Die Streitkräfte der Volksrepublik Swagistan sind für den Schutz und die Durchsetzung der Autorität der Volksrepublik verantwortlich.
    \item Es besteht eine allgemeine Wehrpflicht.
    \item Den Streitkräften sitzt das `Armed Forces Command' (Streitkräftekommando) vor.
\end{enumerate}

\subsection{Streitkräftekommando}
\begin{enumerate}[(1)]
    \item Dem Streitkräftekommando sitzt der Marschall von Swagistan vor.
    \item Das Streitkräftekommando ist für die Organisation der Armee und derer Einsätze verantwortlich.
    \item Es ist dem Ministerium der Verteidigung und dem Ministerium des Krieges übergeordnet.
\end{enumerate}

\subsection{Armeeränge}
\begin{enumerate}[(1)]
    \item Die Armee ist nach nachfolgender Hierarchie strukturiert:
    \begin{enumerate}[1.]
        \item Der \textbf{Marschall von Swagistan} ist der Oberbefehlshaber der Armee und leitet das Streitkräftekommando.
        \item Die \textbf{Generale der Teilstreitkräfte} sind die Befehlshaber der zwei Teilstreitkräfte (Angriffstruppen und Verteidigungstruppen).
        \item Der \textbf{Generaloberst} ist der Stellvertreter eines Generals einer Teilstreitkraft.
        \item Der \textbf{Oberst} ist der Einsatzleiter von Truppen.
        \item Der \textbf{Major} ist der Stellvertreter des Oberst.
        \item Der \textbf{Leutnant} ist ein angehender Offizier.
        \item Der \textbf{Feldwebel} ist ein erfahrenes Truppenmitglied.
        \item \textbf{Unteroffizier}
        \item \textbf{Gefreiter}
    \end{enumerate}
\end{enumerate}

\subsection{Auszeichnungen}
\begin{enumerate}[(1)]
    \item Der Oberste Sowjet kann auf Nominierung hin Personen Auszeichnungen verleihen.
    \item Die Auszeichnungen, ihre Stufen und ihre Anforderungen werden im Auszeichnungsregister genannt.
\end{enumerate}

\section{Discord-Server}

\subsection{Chatverhalten}\label{verhalten}
\begin{enumerate}[(1)]
	\item $^{1}$Spam, Beleidigungen, Drohungen und Provokationen gegen andere Spieler sind verboten und werden zu Sanktionen führen. $^{2}$Als Spammen wird das Verschicken von mehreren Nachrichten in einem geringen Zeitintervall bezeichnet. Ab fünf Nachrichten in kürzester Zeit kann es Konsequenzen nach sich ziehen. $^{3}$Das unerlaubte Nutzen von Pings ist aufgrund seiner provokanten Natur ebenfalls untersagt.
	\item $^{1}$Rassistische, politische, ethisch inakzeptable Inhalte (Äußerungen, Bilder, etc.) sind verboten und führen zu einem permanenten Ausschluss auf dem gesamten Discord-Server. $^{2}$Dies gilt auch für pornografische Inhalte. $^{3}$Auch gilt dies für die absichtliche Anordnung von Reaktionen zu derartigen Äußerungen.
	\item Für pornografische Zwecke explizit angelegte Textkanäle sind von Absatz 2 Satz 2 ausgeschlossen.
	\item \textit{weggefallen}
	\item Weiterhin dürfen Kanäle nur für den Zweck verwendet werden, für den sie vorgesehen sind. Bestehen Unklarheiten über den Verwendungszweck, so muss man sich vor dem Verfassen einer Nachricht an den Support (\ref{support}) wenden.
\end{enumerate}

\subsection{Teammitglieder}\label{members}
\begin{enumerate}[(1)]
	\item Anweisungen von befehlsbefugten Teammitgliedern sind verbindlich und stets zu befolgen.
	\item Teammitglieder werden durch eine Rangbezeichnung, beziehungsweise Rolle gekennzeichnet.
	\item Zu den befehlsbefugten Teammitgliedern gehören:
	\begin{enumerate}
		\item Der Generalsekretär
		\item Der Präsident am Obersten Gerichtshof
		\item Der Leiter des KGB
		\item Der Leiter der Organisation
		\item Die KGB-Agenten
		\item Die Minister
	\end{enumerate}
    \item Die in Absatz 3 genannte Hierarchie ist nach der Befehlsgewalt der Ränge sortiert.
	\item Die befehlsbefugten Mitglieder dürfen nur Befehle in ihrem Zuständigkeitsbereich erteilen.
\end{enumerate}

\subsection{Verhalten im Sprachchat}
\begin{enumerate}[(1)]
	\item Abgesehen von den Regelungen aus \ref{verhalten} gelten für Sprachkanäle zusätzlich folgende Bestimmungen.
	\item Die Nutzung von Stimmenverzerrern und Soundboards ist erlaubt, sofern die Teammitglieder keine Einwände erheben.
	\item Es ist nicht gestattet, Personen ohne deren Einverständnis aufzuzeichnen.
\end{enumerate}

\subsection{Gültigkeit}
\begin{enumerate}[(1)]
	\item Tritt man diesem Server bei, akzeptiert man die hier festgesetzten Bestimmungen.
	\item Die Server-Administration behält sich das Recht vor, diese Regeln jederzeit zu ändern.
	\item \textit{weggefallen}
	\item Die Regelungen treten erst in Kraft, sobald sie in dem Textkanal für Regeln veröffentlicht werden. Dementsprechend gilt keine Regelung rückwirkend.
	\item Mangelnde oder fehlerhafte Kenntnisse der Serverbestimmungen gewähren keine rechtliche Immunität, da das Informieren über die aktuelle Gesetzeslage des Servers Pflicht ist.
	\item Ebenfalls muss man sich bei Unklarheiten an den Zuständigen wenden.
	\item Verstößt eine der Bestimmungen gegen die Verfassung des Landes einer betroffenen Person, so wird für diese lediglich die rechtswidrige Passage aufgehoben\footnote{Salvatorische Klausel}.	
\end{enumerate}

\subsection{Zweitaccounts}
Man muss Zweitaccount als solche markieren. Hierfür muss man sein Serverprofil derartig bearbeiten, dass es jedem möglich ist, anhand dieses Profils nachvollziehen zu können, um wessen Zweitaccount es sich handelt.

\subsection{Strafmaß}
\begin{enumerate}[(1)]
	\item Es wird im Allgemeinen zwischen drei Strafen differenziert:
	\begin{enumerate}[1.]
		\item Eine Verwarnung ist eine Vorstufe zu tatsächlichen Strafmaßnahmen. Jedes Mitglied bekommt für minder schwere Verstöße eine Verwarnung.
		\item Ein Timeout bezeichnet einen temporären Ausschluss vom Server.
		\item Ein permanenter Bann ist ein unwiderruflicher, zeitlich unbegrenzter Ausschluss vom Server.
	\end{enumerate}
	\item Das Strafmaß wird selten nach der Schwere des Verstoßes, sondern zumeist nach folgender Vorgabe bemessen:
	\begin{enumerate}[1.]
		\item Erste Verwarnung
		\item Zweite Verwarnung
		\item 24-Stunden-Timeout
		\item Einwöchiges Timeout
		\item Ein-Monat-Timeout
		\item 1-Jahr-Timeout
		\item Permanenter Bann
	\end{enumerate}
	\item Jede Strafe muss ausnahmslos widerrufen werden, sofern die bestrafte Person die Unrechtmäßigkeit der Strafe nachweisen kann.
	\item Unrechtmäßigkeit liegt vor, sofern es sich bei der fraglichen Tat um keinen Verstoß seitens des Bestraften handelt, bei der Bestrafung gegen Absatz 1 und 2 verstoßen wurde oder die Tat fälschlicherweise als Straftat besonderer Schwere eingestuft wurde.
	\item Handelt es sich bei der Tat um einen schweren Verstoß, so kann je nach Schwere des Verstoßes ein sofortiges Timeout bishin zu einem sofortigen permanenten Bann erfolgen. Die Einschätzung der Schwere unterliegt dem Zuständigen, muss jedoch nachvollziehbar sein. 
	\item Sofern Zweifel bestehen, kann das Urteil von dem Präsident am Obersten Gerichtshof oder durch eine qualifizierte Mehrheit durch den Vorstand aufgehoben und rückgängig gemacht oder in eine andere Strafe umgewandelt werden.
	\item Jegliche rechtswidrigen Nachrichten müssen in Form eines Screenshots bis zum Anbruch der übernächsten Woche zwischengespeichert werden, damit im Zweifelsfall die Rechtswidrigkeit angefochten werden kann\footnote{Dies begründet sich in vergangenen Schwierigkeiten, die Rechtswidrigkeit von Aussagen im Nachhinein zu bewerten.}, danach kann man bei dem Vorstand eine Löschung beantragen, die jedoch mit einer qualifizierten Mehrheit bestätigt werden muss.
	\item Nach jedem Timeout steigt die Schwere der Straftat so, dass der erste Verstoß nach einem Timeout gemäß Strafhierarchie aus Absatz 2 Nummer 1 \- 7 aufgrund seiner Schwere im Verhältnis zur vorherigen Grundstrafe\footnote{Die erste Strafe nach einem Timeout, beziehungsweise die insgesamt erste Strafe.} erhöht wird.
	\item Von Absatz 8 sind permanente Banns teils ausgeschlossen. Diese sollen lediglich im Falle äußerster Schwere oder beim Zutreffen von Absatz 8 verhängt werden, sofern keine akute Verhaltensbesserung vorliegt und der Vorstand in einem einfachen Mehrheitsbeschluss dafür stimmt.
	\item Es ist nicht gestattet, entgegen der ausdrücklichen Erlaubnis des Präsidenten am Obersten Gerichtshof, Personen auf dem Server zu entbannen. Dies gilt als strafbar und wird ungeachtet der Position in der Strafhierarchie nach Interpretation gemäß Absatz 8 mit einem Timeout bestraft. Die widerrechtlich entbannte Person ist zudem umgehend gebannt zu werden.
	\item $^{1}$Verstöße besonderer Schwere durch Mitglieder eines Ministeriums resultieren in einem permantenten Ausschluss aus diesem und weiteren Ministerien. $^{2}$Diese Regelung betrifft weder $the.god.emperor$ noch $jaap\_stam$ oder $robertogmor$ oder deren Zweitaccounts.
	\item Mittäterschaft, Beihilfe, Anstiftung und Versuch werden äquivalent betraft.
\end{enumerate}

\subsection{Inhaber}
\begin{enumerate}[(1)]
	\item Der Begriff des Inhabers entspricht dem Begriff des Teilhabers gemäß §3 Abs. 1 TeilhB.
	\item Jegliche Beschlüsse der Inhaberschaft erfordern eine qualifizierte Mehrheit, es sei denn, die Teilhaberschaftsbestimmungen widersprechen dem.
	\item Auf Anordnung der Inhaber hin kann ein Adminkonzil einberufen werden, bei welchem die Admins und Moderatoren, die keine Teilhaber sind, je eine Stimme bekommen und die Entscheidung für die Teilhaberversammlung übernehmen.
\end{enumerate}

\subsection{Vorstandsvorsitz}
\begin{enumerate}[(1)]
	\item Den Vorstandsvorsitz gemäß §8 Abs. 2 TeilhB nimmt der Generalsekretär ein.
	\item Das Amt wird ständig von $jaap\_stam$ besetzt.
	\item Sollte dieser aus dem Vorstand austreten, so muss entweder durch diesen oder durch den Vorstand mittels einer einfachen Minderheit ein neuer Vorstandsvorsitzender gewählt werden.
\end{enumerate}

\subsection{Ressorts}
\begin{enumerate}[(1)]
	\item Die nachfolgenden Ressorts gemäß §9 TeilhB bestehen auf dem Server:
		\begin{enumerate}[1.]
			\item Abteilung der Organisation
			\item Ministerium der Justiz
			\item Ministerium der Verteidigung
			\item Ministerium des Krieges
			\item KGB
			\item Ministerium des Äußeren
			\item Ministerium des Bergbaus und Erzsammelns
			\item Ministerium der Landwirtschaft und Nahrungsbeschaffung
			\item Ministerium der Erkundung und Lootbeschaffung
			\item Ministerium der Infrastruktur und des Baus
			\item Ministerium des Handels und der Dorfbewohner
			\item Ministerium der Finanzen
			\item Ministerium der Technologie
			\item Diplomatische Garde
		\end{enumerate}
	\item An den Vorsitz dieser Ressorts wird die dementsprechende Rolle vergeben.
	\item In den Ressorts dürfen nur Personen angestellt werden, die auf dem jeweiligen Gebiet über ausreichende Kenntnisse und Fähigkeiten verfügen.
\end{enumerate}

\subsection{Abteilung der Organisation}
\begin{enumerate}[(1)]
    \item Die Abteilung der Organisation ist für die angebrachte Organisation des Servers zuständig.
    \item Dies bezieht die Zuweisung von Rollen und dessen Verwaltung ein.
    \item Den Vorsitz der Abteilung hält der Leiter der Abteilung der Organisation (HDO) inne.
\end{enumerate}

\subsection{Ministerium der Justiz}
\begin{enumerate}[(1)]
	\item In den Aufgabenbereich des Ministeriums der Justiz fallen:
	\begin{enumerate}[1.]
		\item Rechtliche Fragen zur Serververfassung
		\item Anfragen rechtlichen Beistands
		\item Anfechtungen servergerichtlicher und sonstiger Urteile
		\item Verfassungsbeschwerden
		\item Gesetzesvorschläge
		\item Behandlung von Verstößen gegen Serverrichtlinien
		\item Behandlung von Verstößen gegen die Teilhaberschaftsbestimmungen
		\item Prüfung der Urteile der Moderation
	\end{enumerate}
	\item Gesetzesvorschläge, die von der Änderung oder Abschaffung bereits bestehender Gesetze sprechen, gelten als Verfassungsbeschwerden.
	\item Sowohl Gesetzesverstöße und Berufung, als auch Verfassungsbeschwerden gelten als ausreichende Begründung für einen vollwertigen Prozess.
	\item Der Ressortleiter ist der Minister der Justiz (MoJ).
	\item In seiner Funktion als richtendes Organ wird das Ministerium der Justiz als `Supreme Court' (Oberster Gerichtshof) bezeichnet.
	\item Mitglieder des Obersten Gerichtshofs tragen die Amtsbezeichnung `Supreme Judge' (Oberster Richter). Außerhalb ihrer richtenden Tätigkeit werden sie als `Server lawyer' (Serveranwalt) bezeichnet.
	\item Dem Hohen Gericht steht der `President of the Supreme Court' (Präsident am Obersten Gerichtshof) vor.
	\item Der Präsident am Obersten Gerichtshof ist der MdJ.\@
	\item Sofern keine neuen, ausreichenden Beweise vorliegen, trifft Absatz 3 nicht zu.
	\item Urteile durch die Moderation und der Oberste Gerichtshof müssen von dem Präsidenten am Obersten Gerichtshof bestätigt werden und können daher abgewiesen werden.
	\item Die Abweisung von Urteilen muss gerechtfertigt sein und begründet werden.
\end{enumerate}

\subsection{Ministerium der Verteidigung}
\begin{enumerate}[(1)]
    \item Dem Ministerium der Verteidigung obliegt die Aufgabe, die VR Swagistan zu beschützen.
    \item Mitglieder dieses Ministeriums gehören den Verteidigungstruppen an.
    \item Der Minister der Verteidigung (MoD) ist der Vorsitzende des Ministeriums.
\end{enumerate}

\subsection{Ministerium des Krieges}
\begin{enumerate}[(1)]
    \item Das Ministerium des Krieges ist für die Vollziehung von Angriffsplänen und die Jagd nach Personen auf der Blacklist verantwortlich.
    \item Dies bezieht auch das Griefen oder jedwede andere Akte der Sabotage ein.
    \item Den Vorsitz des Ministeriums hält der Minister des Krieges (MoW) inne.
\end{enumerate}

\subsection{KGB}\label{support}
\begin{enumerate}[(1)]
	\item Jegliche Fragen bezüglich des Discord- und Minecraft-Servers, die nicht in den rechtlichen Bereich fallen, fallen in den Aufgabenbereich des KGB. Bestehen Unklarheiten bezüglich des Zuständigkeitsbereichs, sollte man sich ebenfalls an die Moderation wenden.
	\item Der KGB dient zur Kontrolle der Einhaltung der Serverrichtlinien.
	\item Dies bedingt, dass sie in der Lage sind, ohne eine Genehmigung Strafen zu vollziehen, die allerdings an das Ministerium der Justiz mitsamt des Kontexts weitergeleitet werden und endgültig bestätigt werden müssen.
	\item Den Vorstand des KGB hat der Leiter des KGB, welcher auch administrative Aufgaben erfüllt.
	\item Von Absatz 3 ausgeschlossen sind jegliche Verstöße gegen die Teilhaberschaftsbestimmungen, da diese von dem Ministerium der Justiz und VIRTSTAX behandelt werden.
	\item Mitglieder des KGB werden als `KGB agent' (KGB-Agent) aufgeführt.
\end{enumerate}

\subsection{Ministerium des Äußeren}
\begin{enumerate}[(1)]
    \item Das Ministerium des Äußeren ist sowohl für den fraktionsübergreifenden Handel, als auch Diplomatie verantwortlich.
    \item Dies schließt auch die Entsendung von Diplomaten ein.
    \item Der Minister des Äußeren (MoFA) ist der Vorsitzende des Ministeriums.
\end{enumerate}

\subsection{Ministerium des Bergbaus und Erzsammelns}
\begin{enumerate}[(1)]
    \item Das Ministerium des Bergbaus und Erzsammelns ist für den Abbau von Erzen und sonstigen, für den Untergrund typischen Materialien, verantwortlich.
    \item Der Vorsitzende dieses Ministeriums ist der Minister des Bergbaus und Erzsammelns (MoMOG).
\end{enumerate}

\subsection{Ministerium der Landwirtschaft und Nahrungsbeschaffung}
\begin{enumerate}[(1)]
    \item Das Ministerium der Landwirtschaft und Nahrungsbeschaffung ist für den Anbau und das Organisieren von Nahrungsmitteln verantwortlich.
    \item Den Vorsitz dieses Ministeriums hält der Minister der Landwirtschaft und Nahrungsbeschaffung (MoAFG) inne.
\end{enumerate}

\subsection{Ministerium der Erkundung und Lootbeschaffung}
\begin{enumerate}[(1)]
    \item Das Ministerium der Erkundung und Lootbeschaffung ist für die Erkundung von Gebieten verantwortlich.
    \item Dies bezieht auch die Suche nach Strukturen und dessen Plünderung ein.
    \item Der Vorsitzende dieses Ministeriums ist der Minister der Erkundung und Lootbeschaffung (MoALG).
\end{enumerate}

\subsection{Ministerium der Infrastruktur und des Baus}
\begin{enumerate}[(1)]
    \item Das Ministerium der Infrastruktur und des Baus ist für die Errichtung von Gebäuden verantwortlich.
    \item Der Minister der Infrastruktur und des Baus (MoIB) ist der Vorsitzende des Ministeriums.
\end{enumerate}

\subsection{Ministerium des Handels und der Dorfbewohner}
\begin{enumerate}[(1)]
    \item Das Ministerium des Handels und der Dorfbewohner ist für die Bestände der Dorfbewohner und den Handel mit diesen verantwortlich.
    \item Der Vorsitzende dieses Ministeriums ist der Minister des Handels und der Dorfbewohner (MoTV).
\end{enumerate}

\subsection{Ministerium der Finanzen}
\begin{enumerate}[(1)]
	\item Das Ministerium der Finanzen ist dafür verantwortlich, die Serverfinanzen im Blick zu behalten.
	\item Dies bezieht auch die Verwaltung der Unternehmensanteile ein.
	\item Der Minister der Finanzen (MoF) ist der Transaktionsberechtigte gemäß §7 TeilhB.
\end{enumerate}

\subsection{Ministerium der Technologie}
\begin{enumerate}[(1)]
	\item Das Ministerium der Technologie dient der Wartung aller Server und Dienstleistungen, die der VR Swagistan unterstehen.
	\item Dies bezieht auch die Realisierung neuer Funktionalitäten auf diesen Servern ein.
	\item Den Vorsitz hat der Minister der Technologie (MoT).
	\item Mitglieder dieser Abteilung werden als `Technicians' (Techniker) bezeichnet.
\end{enumerate}

\subsection{Diplomatische Garde}
\begin{enumerate}[(1)]
	\item Die Diplomatische Garde dient dem Eskort diplomatischer Gesandter.
	\item Sie untersteht direkt dem Ministerium des Äußeren.
\end{enumerate}

\section{Minecraft-Server}

\subsection{Anwendung}
Die Gesetze dieses Abschnitts finden nur dann Anwendung, wenn die VR Swagistan Minecraft-Server besitzt.

\subsection{Bürgschaft}
\begin{enumerate}[(1)]
	\item Es dürfen nur Personen auf die Whitelist gesetzt werden, für die jemand nachweislich bürgt.
	\item Begeht eine Person einen Verstoß besonderer Schwere, so wird sie mitsamt des Bürgenden vom Server gebannt.
	\item Wird ein Bürgender vom Server gebannt, geschieht dies denen gleich, für die dieser bürgt.
	\item Bürgschaften kann man nicht nachträglich zurückziehen.
	\item Die Administration ist von Abs. 2f. ausgenommen.
\end{enumerate}

\subsection{Rechtliche Separation}
\begin{enumerate}[(1)]
	\item Das Serverrecht ist eindeutig von der internen Rechtssituation auf den Ablegern des The God Complex Servers zu unterscheiden.
	\item Als internes Recht werden nicht von der Inhaberschaft in ihrer Funktion als Teilhaberversammlung anerkannte Verfassungen und Regeln, wie beispielsweise fraktionseigene Gesetzestexte bezeichnet.
	\item Die Einsicht und Nutzung von, internen Regelungen übergeordneten, Serverdaten und sonstigen, nur für die Administration zugänglichen Informationen, wie Spielerdaten oder Logs, darf nicht zur Beweisführung für Prozesse und ähnliches dienen, die nicht von dem Obersten Servergericht in dessen Funktion vollzogen werden\footnote{So dürfen beispielsweise In-Game-Morde nicht über Logs nachgewiesen werden}.
\end{enumerate}

\subsection{Grundsätzliche Regeln}
\begin{enumerate}[(1)]
	\item Das Minecraft-Serverrecht untersteht dem Discordserverrecht.
	\item Es ist verboten, auf Methoden zurückzugreifen, die gegenüber anderen Spielern, ungeachtet dessen, ob sie die Methode einsetzen oder nicht, einen Vorteil verschaffen, die allgemein nicht als gerecht anerkannt werden.
	\item Jegliches, von derartigen Methoden nicht betroffenes Verhalten, ist nicht strafbar.
	\item Auf dem Minecraft-Server muss man sich den jeweiligen Regeln des Discord-Servers entsprechend verhalten.
	\item Das generelle Serverrecht unterscheidet nicht zwischen Fraktionen, weshalb diese lediglich eine interne Organisation darstellt, die keine Deckung durch jegliche serverweite Gesetze erfährt und somit Verbrechen gegen diese im Einzelnen kein Gegenstand serverweiter Urteile sein können.
	\item Absatz 3 tritt nur ein, wenn sich die Verstöße nicht gegen die Regeln des Discordservers oder Abs. 1f. richten.
\end{enumerate}

\subsection{Fraktionen}
\begin{enumerate}[(1)]
	\item Als Fraktion gilt jegliche Gruppierung mit mehr als einem Spieler.
	\item Der Begriff des Spielers ist nicht mit dem Begriff des Minecraft-Kontos synonym und rechtfertigt daher keine Gründung, wenn es sich bei dem anderen Konto um ein Zweitkonto der Person handelt.
	\item Fraktionen haben das Anrecht auf eine eigene Kategorie, in der sie jegliche Kanäle auf Anfrage hin einrichten können.
	\item Die Moderation kann derartige Anliegen ablehnen, sofern diese keinen gerechtfertigten Grund für eine Einrichtung feststellen können.
	\item Aufgrund der Tätigkeit und Aufgaben der Moderation ist diese jederzeit berechtigt, in die Kanäle einzusehen, um Verstöße gegen das geltende Recht erkennen zu können.
	\item Jeder Fraktion wird eine eigene Rolle zugesichert, die zum Zweck hat, dass diese Kanäle nicht durch Mitglieder anderer Fraktionen eingesehen werden können.
\end{enumerate}

\subsection{Spezifisches Serverrecht}
\begin{enumerate}[(1)]
	\item Es ist untersagt, Orte zu zerstören, die ein Ort des Wissens sind.
	\item Absatz 1 darf nur dann gebrochen werden, wenn eine Fraktion diese Regel ausnutzt, um sich einen Vorteil zu verschaffen.
	\item Jede Fraktion muss bei ihrer Einrichtung ein Territorium beanspruchen.
	\item Dieses Territorium muss aus einem überirdischen Biom bestehen und darf zunächst nicht über dessen Grenzen hinausragen.
	\item Es dürfen nur Materialen verwendet werden, die aus diesem Territorium stammen oder von anderen erworben wurden.
	\item Man darf mit jeder Erweiterung, die serverweit von der Administration angekündigt wurde, sein Gebiet gemäß Abs. 3ff. auf ein benachbartes Biom ausweiten.
	\item Verschiedene Varianten eines Bioms werden nicht als eigenes Biom erachtet.
\end{enumerate}

\end{document}