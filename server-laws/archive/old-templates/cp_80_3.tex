\documentclass{article}
\usepackage[utf8]{inputenc}
\usepackage{textcomp}
\usepackage{amsmath}
\usepackage{enumerate}

\title{MORRDORRYON gg. MACEDONÍAYON}
\author{Akz. CP-3/80}
\date{26. Januar 1180}

\begin{document}
\maketitle
\vspace*{\fill}
\paragraph{Hamavarisches Recht gemäß Recht nach HSGM Kaiser Ellendir IV. Daimitrayon Ellendir}
\newpage
\section{Inhalt}
\section{Anlagen}
\subsection{Vorladung №1}
\paragraph{An HRH König Damian III. Lúinnayon Macedoníayon,\\\\}
Hiermit werden Sie gemäß XXXIIa darüber informiert, dass das Atilengericht Hrrátim beabsichtigt, bezüglich der durch Julian III. Morrdorryon eingereichten Klage, 
am 30. Dezember 1178 eingereichten Klage, am folgenden Freitag, dem 27. Januar, um 19 Uhr MESZ eine Verhandlung im Prozess Morrdorryon gg.
Macedoníayon (Akz. CP-3/80), abzuhalten.\\
Gemäß XXXIIb ist es Ihnen gestattet, bis zwei Stunden vor Verhandlungsbeginn einen Antrag auf Aufschub des Gerichtstermins einzureichen.
Kommen Sie dem nicht nach, so haben Sie zu erscheinen. Andernfalls wird die Verhandlung ohne Sie begonnen und ein Bußgeld in Höhe von 20
HTK verhängt.
\paragraph{Kurze Wiedergabe des Sachverhalts\\}
Die Klagende erhob am 30. Dezember 1178 beim Hohen Tribunal Anklage gegen das Hause Macedoníayon \textit{hierzulande Könige von Lúinna} wegen Mordes
(LIII). Aufgrund einer gesetzlichen Änderung, geltend gemacht am 20. Januar 1180 durch HSGM Kaiser Ellendir IV. Daimitrayon Ellendir, wird jedoch gemäß 
XXIIa ein Atilengericht den Fall zunächst bearbeiten.\\\\
Die klagende Partei, Julian III. Morrdorryon \textit{hierzulande titellos} wird ebenfalls eine Vorladung erhalten.

\paragraph{Atilengericht Reichsbezirk Hrrátim, Hrratá Nirid 8, Weißer Bezirk (Nibtimva Cahadirr)}
\newpage
\subsection{Vorladung №2}
\paragraph{An Julian III. Morrdorryon \textit{hierzulande titellos},\\\\}
Hiermit werden Sie gemäß XXXIIa darüber informiert, dass das Atilengericht Hrrátim beabsichtigt, bezüglich Ihrer, 
am 30. Dezember 1178 eingereichten Klage, am folgenden Samstag, dem 27. Januar, um 19 Uhr MESZ eine Verhandlung im Prozess Morrdorryon gg.
Macedoníayon (Akz. CP-2/80), abzuhalten.\\
Gemäß XXXIIb ist es Ihnen gestattet, bis zwei Stunden vor Verhandlungsbeginn einen Antrag auf Aufschub des Gerichtstermins einzureichen.
Kommen Sie dem nicht nach, so haben Sie zu erscheinen. Andernfalls wird die Verhandlung ohne Sie begonnen und ein Bußgeld in Höhe von 20
HTK verhängt.
\paragraph{Kurze Wiedergabe des Sachverhalts\\}
Die Klagende erhob am 30. Dezember 1178 beim Hohen Tribunal Anklage gegen das Hause Macedoníayon \textit{hierzulande Könige von Lúinna} wegen Mordes
(LIII). Aufgrund einer gesetzlichen Änderung, geltend gemacht am 20. Januar 1180 durch HSGM Kaiser Ellendir IV. Daimitrayon Ellendir, wird jedoch gemäß 
XXIIa ein Atilengericht den Fall zunächst bearbeiten.\\\\
Der Stellvertreter der beklagten Partei, das Oberhaupt des Hauses Macedoníayon \textit{hierzulande König von Lúinna}, wird ebenfalls eine Gerichtsvorladung erhalten.\\

\paragraph{Atilengericht Reichsbezirk Hrrátim, Hrratá Nirid 8, Weißer Bezirk (Nibtimva Cahadirr)}
\subsection{Ermahnung}
\paragraph{An die beteiligten Parteien des Prozesses Macedoníayon gg. Morrdorryon (Akz. CP-2/80)\\}
Da Sie beide nicht zum Prozess erschienen sind und keinen Antrag auf Aufschub gestellt haben, wird gemäß XXXIIf,g der Prozess vertagt und gegen beide Parteien ein Bußgeld in Höhe von 20 HTK verhängt.
\paragraph{Atilengericht Reichsbezirk Hrrátim, Hrratá Nirid 8, Weißer Bezirk (Nibtimva Cahadirr)}
\newpage
\subsection{Verhandlungsdaten}
\textbf{Morrdorryon gg. Macedoníayon - I. Instanz}\\
27. Januar 1180, 19:00 Uhr MESZ \textit{per decreto}\\
Atilengericht Hrrátim\\
Gerichtsvorsitz HSGM Kaiser Ellendir IV. Daimitrayon Ellendir

\newpage
\section{Anklageschrift}
Auf Beschwerde Julians III. Morrdorryon, hierzulande titellos, eingereicht am 30. Dezember 1178, hin\\
-UND-\\
Durch den Staate, vertreten durch HSGM Kaiser Ellendir IV. Daimitrayon Ellendir, am 02. Januar 1179 bestätigt,\\
wurde der beklagten Partei, Damian II. Macedoníayon, hierzulande einstiger König von Lúinna, zur Last gelegt,
\begin{enumerate}
    \item Am 30. Dezember 1178 Julian II. Morrdorryon vorsätzlich angegriffen und ermordet zu haben.
\end{enumerate}

Den geschilderten Tatumständen liegen gemäß Codex Ellendir Verstöße gegen

\begin{enumerate}
    \item LIII (Mord)
\end{enumerate}

vor.\\\\

Da ein Sonderfall gemäß XLVIa,b vorliegt, wird die Klagende durch Julian III. Morrdorryon und die Beklagte durch Damian III. Macedoníayon vertreten.

\newpage
\section{Urteil}
\paragraph{Das Atilengericht des Reichsbezirks Hrrátim und dessen Richterschaft verkünden\\}
Dass die Beklagte in allen Anklagepunkten schuldiggesprochen wurde. Zudem wurde der Forderung nach dem
Höchstmaß stattgegeben. Daher ist der Beklagte und dementsprechend verurteilte verpflichtet,
eine Bußgeldzahlung von 15 HTK zu entrichten.\\
Beiden Parteien steht es gemäß XXII zu, eine Berufung gegen das Urteil einzulegen. Die Klagende hat die Berufung bereits
abgelehnt. Die Beklagte hat Berufung eingelegt.\\
Einspruch gegen Ermahnung durch die Beklagte nicht zugelassen. Möglichkeit rechtzeitiger Antragstellung auf Aufschub nachgewiesen.
\end{document}