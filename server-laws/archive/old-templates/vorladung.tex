\documentclass{article}
\usepackage[utf8]{inputenc}
\usepackage{textcomp}
\usepackage{amsmath}
\usepackage{enumerate}

\title{DAS KAISERREICH HAMAVAR gg. MARRKAN-BENNETAL}
\author{Akz. RP-1/97}
\date{23. September 1197}

\begin{document}
\maketitle
\vspace*{\fill}
\paragraph{Hamavarisches Recht gemäß Recht nach HSGM Kaiser Ellendir IV. Daimitrayon Ellendir}
\newpage
\section{Inhalt}
\section{Anlagen}
\subsection{Vorladung №1}
\paragraph{An HEH Lordkanzler und Hochkönig Sir Redhir IV. von Hamavar GIH,\\\\}
Hiermit werden Sie gemäß XXXIIIa darüber informiert, dass das Atilengericht Hrrátim beabsichtigt, bezüglich Ihrer, 
im April 1188 eingereichten Klage, am folgenden Montag, dem 25. September, um 19 Uhr eine Verhandlung im Prozess Das Kaiserreich Hamavar gg.
Marrkan-Bennetal (Akz. RP-1/93), abzuhalten.\\
Gemäß XXXIIIb ist es Ihnen gestattet, bis zwei Stunden vor Verhandlungsbeginn einen Antrag auf Aufschub des Gerichtstermins einzureichen.
Kommen Sie dem nicht nach, so haben Sie zu erscheinen. Andernfalls wird die Verhandlung ohne Sie begonnen und ein Bußgeld in Höhe von 20
HTK verhängt.
\paragraph{Kurze Wiedergabe des Sachverhalts\\}
Die Klagende erhob im April 1188 beim Atilengericht Hrrátim Anklage gegen Bennet III. Marrkan-Bennetal \textit{hierzulande titellos} wegen Entweihung religiösen Bodens
(LXXVI).\\\\
Die beklagte Partei, Bennet IV. Marrkan-Bennetal \textit{hierzulande titellos}, wird ebenfalls eine Gerichtsvorladung erhalten.\\

\paragraph{Atilengericht Reichsbezirk Hrrátim, Hrratá Erik III 1, Hèturrir-Distrikt}
\newpage
\subsection{Vorladung №2}
\paragraph{An Bennet IV. Marrkan-Bennetal \textit{hierzulande titellos},\\\\}
Hiermit werden Sie gemäß XXXIIIa darüber informiert, dass das Atilengericht Hrrátim beabsichtigt, bezüglich der durch die Generalstaatsanwaltschaft des Kaiserreichs Hamavar, 
im April 1188 eingereichten Klage, am folgenden Montag, dem 25. September, um 19 Uhr eine Verhandlung im Prozess Das Kaiserreich Hamavar gg. Marrkan-Bennetal (Akz. RP-1/97), abzuhalten.\\
Gemäß XXXIIIb ist es Ihnen gestattet, bis zwei Stunden vor Verhandlungsbeginn einen Antrag auf Aufschub des Gerichtstermins einzureichen.
Kommen Sie dem nicht nach, so haben Sie zu erscheinen. Andernfalls wird die Verhandlung ohne Sie begonnen und ein Bußgeld in Höhe von 20
HTK verhängt.
\paragraph{Kurze Wiedergabe des Sachverhalts\\}
Die Klagende erhob im April 1188 beim Atilengericht Hrrátim Anklage gegen Bennet III. Marrkan-Bennetal \textit{hierzulande titellos} wegen Entweihung religiösen Bodens
(LXXVI).\\\\
Die klagende Partei, der Generalstaatsanwalt HEH Lordkanzler und Hochkönig Sir Redhir IV. von Hamavar GIH, wird ebenfalls eine Gerichtsvorladung erhalten.\\

\paragraph{Atilengericht Reichsbezirk Hrrátim, Hrratá Erik III 1, Hèturrir-Distrikt}
\subsection{Ermahnung}
\paragraph{An die beteiligten Parteien des Prozesses Macedoníayon gg. Morrdorryon (Akz. CP-2/80)\\}
Da Sie beide nicht zum Prozess erschienen sind und keinen Antrag auf Aufschub gestellt haben, wird gemäß XXXIIf,g der Prozess vertagt und gegen beide Parteien ein Bußgeld in Höhe von 20 HTK verhängt.
\paragraph{Atilengericht Reichsbezirk Hrrátim, Hrratá Nirid 8, Weißer Bezirk (Nibtimva Cahadirr)}
\newpage
\subsection{Verhandlungsdaten}
\textbf{Macedoníayon gg. Morrdorryon - I. Instanz}\\
21. Januar 1180, 19:00 Uhr MESZ \textit{per decreto}\\
Atilengericht Hrrátim\\
Gerichtsvorsitz HSGM Kaiser Ellendir IV. Daimitrayon Ellendir
\\\\
\textbf{Macedoníayon gg. Morrdorryon - II. Instanz}\\
26. Januar 1180, 20:00 Uhr MESZ \textit{per decreto}\\
Fürstenkammer Daimitra\\
Gerichtsvorsitz HSGM Kaiser Ellendir IV. Daimitrayon Ellendir

\newpage
\section{Anklageschrift}
Auf Beschwerde HRH König Damians III. Macedoníayon, hierzulande König von Lúinna, eingereicht am 30. Dezember 1178, hin\\
-UND-\\
Durch den Staate, vertreten durch HSGM Kaiser Ellendir IV. Daimitrayon Ellendir, am 02. Januar 1179 bestätigt,\\
Wird der beklagten Partei, Julian III. Morrdorryon, hierzulande titellos, zur Last gelegt,
\begin{enumerate}
    \item Am 30. Dezember 1178 HRH König Damian III. Macedoníayon vorsätzlich angegriffen und verletzt zu haben.
\end{enumerate}

Den geschilderten Tatumständen liegen gemäß Codex Ellendir Verstöße gegen

\begin{enumerate}
    \item LII (Körperverletzung)
\end{enumerate}

vor.
\newpage
\section{Urteil}
\paragraph{Das Atilengericht des Reichsbezirks Hrrátim und dessen Richterschaft verkünden\\}
Dass die Beklagte in allen Anklagepunkten schuldiggesprochen wurde. Zudem wurde der Forderung nach dem
Höchstmaß stattgegeben. Daher ist der Beklagte und dementsprechend verurteilte verpflichtet,
eine Bußgeldzahlung von 15 HTK zu entrichten.\\
Beiden Parteien steht es gemäß XXII zu, eine Berufung gegen das Urteil einzulegen. Die Klagende hat die Berufung bereits
abgelehnt. Die Beklagte hat Berufung eingelegt.\\
Einspruch gegen Ermahnung durch die Beklagte nicht zugelassen. Möglichkeit rechtzeitiger Antragstellung auf Aufschub nachgewiesen.
\end{document}