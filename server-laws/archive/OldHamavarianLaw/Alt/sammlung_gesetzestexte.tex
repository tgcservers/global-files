\documentclass{article}
\usepackage[utf8]{inputenc}
\usepackage{textcomp}
\usepackage{amsmath}
\usepackage{enumerate}

\title{Sammlung von Gesetzestexten}
\author{Kaiserlicher Verlag Hrrátim}
\date{08. Februar 1185}

\begin{document}
\maketitle
\newpage
\section{Codex Primus}
\subsection{Originalfassung von 1103}
1. Betritt man das Territorium des Königreichs, geht die Justiz automatisch davon aus, dass man die, hier geltenden Gesetze gelesen und akzeptiert hat.\\
a. Daher gewährt mangelnde Kenntnis von den Gesetzen keine rechtliche Immunität.
\\\\
2. Es gilt als Verbrechen, Diplomaten hinzurichten oder zu töten, es sei denn, sie haben gegen ein Gesetz verstoßen, das die Hinrichtung fordert.
\\\\
3. Hinrichtungen gelten NICHT als feindlicher Akt gegen einen anderen Staat,
sofern sie laut Gesetz berechtigt sind.\\
a. Daher ist es ein Verbrechen, dies als Kriegsgrund gegen das Königreich zu verwenden.\\
b. Fahrlässigkeit wird äquivalent bestraft.
\\\\
4. Man hat die Könige angemessen mit einer Verbeugung (1x schleichen) und "Seid gegrüßt, erhabene Könige" zu begrüßen.\\
a. Tut man dies nicht, so wird man dazu aufgefordert.\\
b. Geht man der Aufforderung nicht nach, darf die Person
hingerichtet werden.
\\\\
5. Kommt es zu einer Hinrichtung, hat das Königreich die Pflicht, die Gegenstände des Hingerichteten aufzubewahren.\\
a. Für Verlust von XP kommt das Königreich nicht auf.\\
b. Das Tribunal darf
eine angemessene Entschädigung verlangen.
\\\\
6. Gegen jede Verurteilung kann man nach ihrer Vollstreckung Klage beim Hohen Tribunal bei Gora einlegen.
\\\\
7. Auf unerlaubtes Betreten des Königsbezirkes von
Daimitra steht die Todesstrafe.\\
a. Es reicht eine nachweisbare Genehmigung durch einen der Könige.
\\\\
8. Bei mutwilliger Beschädigung muss man mit einer Neubeschaffung der Materialien aufkommen.\\
a. Mutwillige Beschädigung im Königsbezirk oder an Feldern wird zusätzlich mit dem Tode bestraft.
\\\\
9. Beleidigung und Diskreditierung der Könige werden mit dem Tod bestraft. Weiterhin muss man mit einem Goldblock Strafe aufkommen.\\
a. Nicht angemessenes Verhalten vor den Königen gilt NICHT als Beleidigung.
\\\\
10. (Anhang zu 4)\\
4. Gilt nicht für den Großkönig.
\\\\
11. Für Diebstahl muss man mit der doppelten Anzahl jedes gestohlenen Gutes aufkommen.\\
a. Auf Diebstahl im Königsbezirk steht zusätzlich die Todesstrafe.
\\\\
12. Blutvergießen im Königsbezirk wird mit dem Tode, sowie mit der Zahlung von fünf Goldblöcken oder zwei Diamantenblöcken bestraft.\\
a. Ist ein Opfer ein
Diplomat, muss man mit der doppelten zahlung aufkommen.\\
b. Ist ein Opfer ein König, muss man mit der fünffachen Zahlung aufkommen.\\
c. Jede Zahlung wird pro Opfer der jeweiligen Kategorie gezählt und summiert.\\
d. Rechtlich zugelassene Hinrichtungen sind immun gegen diese Regelung.
\\\\
13. Leute, die sich auf diesem Gebiet aufhalten, dürfen auch gemäß Recht eines anderen Staates hingerichtet werden, sofern sie auf dem Gebiet dieses Staates
eine Straftat begingen und verurteilt wurden.
\\\\
14. Prozesse des Tribunals können auch in Abwesenheit des Angeklagten geschehen, sofern dieser nicht innerhalb der nächsten zwei Tage erscheint.
Jegliche Verurteilung innerhalb dieser Frist ist ungültig, sofern
der Angeklagte nicht erscheint.\\
a. Auf Anfrage durch das Tribunal hin, kann der Großkönig eine Ausnahme ausrufen.
\\\\
\textit{Aus König Elnirr Ellendir: Informationen zur gesetzlichen Lage für Besucher im Königreich Hamavar, Königsbezirk 1103}
\section{Codex Elnirr}
\subsection{Originalfassung von 1105}
I. Diebstahl:\\Stiehlt man von dem Gebiet des Königreichs, so muss man die Ware mit dem vierfachen Wert und einem Diamanten entschädigen.\\Dies gilt für alle Ressourcen, die dem Staatsgebiet entstammen und man widerrechtlich entwendet hat.\\Zu den Ressourcen gehören auch Teile des Staatsgebiets.
\\\\
II. Mord:\\Tötet man eine Person vorsätzlich, so wird man hingerichtet und muss die Person mit 4 Diamanten entschädigen.
\\\\
III. Widerrechtliches Betreten des Staatsgebietes:\\Betritt man das Staatsgebiet widerrechtlich, so wird man hingerichtet.
\\\\
IV. Politische Neutralität von Strafen:\\Jegliche rechtmäßig verhängte Strafen sind politisch neutral und dürfen dementsprechend nicht als Kriegs- oder Sanktionsgrund verwendet werden.
\\\\
V. Sachbeschädigung:\\Wer fremdes Eigentum auf dem Gebiet von Hamavar beschädigt, muss für die Schäden vollständig aufkommen und zusätzlich 4 Diamanten zahlen.
\\\\
VI. Hausfriedensbruch:\\Wer sich ohne Genehmigung des Besitzers Zutritt zu dessen Eigentum verschafft, darf durch diesen hingerichtet werden und muss zusätzlich 5 Diamanten zahlen.
\\\\
VII. Berufung\\Hält man eine Entscheidung des Königreichs für rechtswidrig oder unangemessen, so darf man, nachdem man bestraft wurde, Berufung bei dem Hohen Tribunal einlegen.\\Spricht das Hohe Tribunal einen frei, so muss das Königreich die Person entschädigen.
\\\\
VIII. Strafmaß bei Zahlungsunfähigkeit:\\Ist eine Person nicht imstande die geforderte Bußgeldstrafe zu bezahlen, so darf das Königreich eine andere, von ihnen festgelegte Strafe fordern. Hierbei hat das Königreich vollkommene Entscheidungsfreiheit.
\\\\
IX. Rechtliche Immunität:\\Diplomaten verfügen über beschränkte rechtliche Immunität, doch werden Vergehen an ihnen mit dem doppelten Strafsatz bestraft.\\Mangelnde oder fehlerhafte Kenntnisse des Gesetzes gewähren keine rechtliche Immunität, da das Informieren über die Gesetzeslage Pflicht ist.

\subsection{Fassung von 1179}
VERFASSUNG DES KAISERREICHS VON HAMAVAR
\\\\
I. Diebstahl:\\Stiehlt man von dem Gebiet des Kaiserreichs, so muss man die Ware mit dem doppelten Wert und 4 HTK pro Stack entschädigen (<1 Stack = 2 HTK).\\Dies gilt für alle Ressourcen, die dem Staatsgebiet entstammen und man widerrechtlich entwendet hat.\\Jeglicher Handel mit Dorfbewohnern kostet einen Festpreis von 1 Smaragd pro Handel.
\\\\
II. Mord:\\Tötet man eine Person vorsätzlich, so wird man hingerichtet und muss die Person mit 5 Diamanten entschädigen.
\\\\
III. Widerrechtliches Betreten des Staatsgebietes:\\Betritt man das Staatsgebiet widerrechtlich, so muss man 10 HTK zahlen.\\Wiederholungstäter werden zusätzlich hingerichtet.
\\\\
IV. Politische Neutralität von Strafen:\\Jegliche rechtmäßig verhängte Strafen sind politisch neutral und dürfen dementsprechend nicht als Kriegs- oder Sanktionsgrund verwendet werden.
\\\\
V. Sachbeschädigung:\\Wer fremdes Eigentum auf dem Gebiet von Hamavar beschädigt, muss für die Schäden vollständig aufkommen und zusätzlich 10 HTK zahlen.
\\\\
VI. Rechte des Eigentümers:\\Wer auf hamavarischem Grund rechtmäßig Eigentum erworben hat, darf dieses nutzen und verändern, wie er möchte, solange diese Handlungen ausschließlich gesetzeskonform sind.\\Erwirbt man ein Haus, so gehört einem nur das Innere des Hauses und nicht die Fassade, weshalb diese nicht verändert werden darf.
\\\\
VII. Hausfriedensbruch:\\Wer sich ohne Genehmigung des Besitzers Zutritt zu dessen Eigentum verschafft, darf durch diesen hingerichtet werden und muss zusätzlich 5 HTK zahlen.
\\\\
VIII. Duellierrecht:\\Jeder Mord darf einmalig von einem Nachkommen des Opfers in Form eines Duells auf den Tod gerächt werden. Verliert dieser jedoch den Kampf, hat er das Recht auf Rache in diesem einen Fall verwirkt. Dies gilt auch für den Fall, dass er gewinnt.
\\\\
IX. Anrede des Richters:\\Steht man vor Gericht, so hat man den Richter als 'ehrenwerter Atile' anzureden. Tut man dies nicht, muss man 5 HTK zahlen.\\Steht man vor dem Hohen Tribunal, so hat man den Kaiser als 'Eure erhabene und glorreiche Majestät' und die Könige als 'Eure Majestät' anzureden. Verweigert man dies, muss man 10 HTK zahlen und wird hingerichtet.\\In beiden Fällen muss man erst nach einer Aufforderung dem nachkommen. Pro Verhandlung steht einem eine Aufforderung frei.
\\\\
X. Berufung:\\Hält man ein Urteil für rechtswidrig oder unangemessen, darf man Berufung einlegen. Stand man zuvor vor einem Atilengericht, wird diese Berufung von dem Tribunal behandelt.\\Hält man eine Entscheidung des Tribunals für rechtswidrig oder unangemessen, so darf man, nachdem man bestraft wurde, Berufung bei dem Hohen Tribunal einlegen.\\Spricht das Hohe Tribunal einen frei, so muss das Königreich die Person entschädigen.
\\\\
XI. Strafen in der Königsstadt:\\Die Strafen in der Königsstadt Hrrátim haben die fünffachen Bußgeldsätze der normalen Strafen.\\Im Falle von Mord muss man mit 10 Diamantenblöcken oder 30 HTKblöcken aufkommen.\\In der Königsstadt wird zusätzlich für jede Straftat die Todesstrafe verhängt.
\\\\
XII. Verteidigung vor Gericht:\\Man darf sich vor Gericht verteidigen.\\Man darf Personen in den Zeugenstand rufen.\\Diese darf man unter den gegebenen Regeln befragen\\Diese Regeln lauten:\\Die Zeugen stehen automatisch unter Eid, sobald sie ihr erstes Wort im Zeugenstand erheben.\\Die Zeugen müssen daher alles wahrheitsgemäß beantworten\\Jegliche ungenauen Aussagen der Zeugen werden nicht ins Protokoll aufgenommen (siehe hierzu XIIdc)\\Man darf einen Anwalt einstellen.\\Man darf nicht unaufgefordert sprechen.\\Einsprüche sind erlaubt, können jedoch abgewiesen werden.\\Bei einmaliger Ablehnung eines Einspruchs, darf dieser nicht erneut getätigt werden.\\Bei einem Einspruch muss man stets die Begründung des Einspruchs ankündigen.\\Genehmigte Begründungen sind:\\Nicht aussagekräftig/unverständlich/mehrdeutig: Die Aussage oder Frage ist aufgrund seiner nicht aussagekräftigen Natur unzulässig.\\Bereits beantwortet: Die gleiche wurde  Frage mehrfach gestellt, obwohl sie beteits beantwortet wurde.\\Unbewiesene Vermutung: Der Anwalt behauptet etwas, ohne sich auf vorliegende Beweise zu stützen.\\Fordert Spekulationen: Der Anwalt fordert den Zeugen auf, zu spekulieren.\\Supra interrogatio (über Befragung hinaus): Der Anwalt fragt mehr als eine Frage gleichzeitig.\\Mangelnde Kenntnisse: Die Kenntnisse des Zeugens über das gefragte Thema sind unzureichend nachgewiesen.\\Ohne Priorität: Die Frage ist dem Prozess beziehungsweise der Befragung nicht dienlich.\\Gerücht: Die Antwort der Partei baut auf außergerichtlichen Aussagen auf.\\Lex Accusantis: Ein Mitglied des Hohen Tribunals hat einen Einwand gegen eine Entscheidung des Kaisers\\Dieser Einspruch kann nur durch Mitglieder des Hohen Tribunals getätigt werden, die als richtende Partei im Prozess dienen.\\Hinterfragt die Staatsautorität: Eine Partei fechtet, hinterfragt oder beleidigt die Staatsautorität beziehungsweise die Autorität des Kaisers. Wird dieser Einsprich bewilligt, wird derjenige, der die Aussage gebracht hat, hinterher wegen Verstoßes gegen XXXVII vor Gericht gestellt.\\Wird ein Einspruch stattgegeben, so muss der Anwalt bei der Befragung mit der nächsten Frage fortfahren. Der Zeuge darf die vorherige Frage nicht beantworten oder seine Aussage wird im Fall, dass er sie bereits getätigt hat, gestrichen.\\Beantwortet ein Zeuge eine Frage, nachdem sie durch die Stattgabe eines Einspruchs als unzulässig bewertet wurde, so wird er sofort aus dem Zeugenstand verwiesen und die Aussage gestrichen.\\Verstößt man gegen diese Regel muss man zehn HTK pro Verstoß zahlen.\\Man muss am Ende eines Prozesses ein Schlussplädoyer in Form einer Rede halten.\\Das hamavarische Recht sieht folgenden Ablauf für einen Prozess vor:\\Der Angeklagte und der Kläger betreten den Raum\\Das Hohe Tribunal versammelt sich, in der Zeit muss jeder stehen.\\Der Kaiser setzt sich und eröffnet den Prozess. Erst müssen sich die weiteren Mitglieder des Hohen Tribunals setzen, dann die weiteren Anwesenden.\\Der Kläger muss den Strafbestand darlegen.\\Der Angeklagte hat das Wort und darf seine Darstellung der Umstände darlegen.\\Von nun an entscheidet der Kaiser, wer das Wort erhält.\\Der Kaiser darf jegliche Anwesende, die nach seinem Ermessen zu oft gegen die Gerichtsordnung (XII, ausgenommen XIId) verstoßen haben, aus dem Saal verweisen. Die Strafen gemäß XIIg sind Verwarnungen. Hierzu gibt es keine Ausnahmen.\\Sobald alle Beweise und Aussagen der beiden Parteien dargelegt wurden, tritt das Hohe Tribunal zurück und berät sich in einem separaten Gespräch, dem nur die Mitglieder beiwohnen dürfen. Hierbei wird über die Strafe beratschlagt. Bei Stimmgleichheit erhält der Kaiser eine zweite Stimme.\\Das Hohe Tribunal verkündet die Strafe und verlässt den Saal. Hierbei müssen sich die weiteren Anwesenden erheben und dürfen anschließend, nachdem der letzte Richter den Saal verlassen hat, diesen ebenfalls verlassen.
\\\\
XIII. Rechtliche Immunität (Gemäß Verfassung von Tag 70):\\Diplomaten verfügen über beschränkte rechtliche Immunität, doch werden Vergehen an ihnen mit dem fünffachen Strafsatz bestraft, sowie einer Hinrichtung bestraft.\\Mangelnde oder fehlerhafte Kenntnisse des Gesetzes gewähren keine rechtliche Immunität, da das Informieren über die Gesetzeslage Pflicht ist.\\Die Kaiser dürfen Personen rechtliche Immunität verleihen.
\\\\
XIV. Vergehen an dem Hochadel:\\Vergehen an dem Hochadel werden mit dem dreifachen Strafsatz vergolten.\\Vergehen an den Kaisern werden mit dem zehnfachen Strafsatz vergolten.\\Vergehen an dem Staat gelten als Vergehen an den Kaisern.\\Bei Vergehen am Hochadel droht ebenfalls die Todesstrafe.
\\\\
XV. Verbannung:\\Verbannung dient im Falle von Zahlungsunfähigkeit als Ersatz für hohe Bußgeldstrafen. Die verzehnfachte Bußgeldstrafe entspricht der Anzahl der Tage einer Verbannung.\\Als Verbannter darf man das Gebiet des Königreichs nicht betreten.
\\\\
XVI. Entehrung:\\Beging man eine besonders ehrlose Tat, so ist das Tribunal imstande, den Täter zu entehren. In diesem Fall ist der Täter nicht fähig, jegliche Tätigkeiten auf dem Staatsgebiet zu unternehmen.\\Das Königreich erwartet von seinen Verbündeten, diesen Status für den Täter auch auf ihrem Gebiet auszurufen.
\\\\
XVII. Die Königsherrschaft:\\Jeder der beiden Könige regiert uneingeschränkt bis zu seinem Lebensende und wählt vor seinem Tod einen Nachfolger aus seiner Linie.\\Die Autorität der Könige darf nicht angezweifelt werden.\\Die Kaiser können Ausnahmen zu allen Gesetzen machen.
\\\\
XVIII. Körperverletzung:\\Wer eine Person auf dem Gebiet des Kaiserreichs physisch verletzt, muss mit einer Strafe von 15 HTK rechnen.
\\\\
XIX. Strafverfolgung:\\Entzieht man sich der Strafverfolgung des Reichs, wird man vor das Hochkonzil gestellt.\\Man darf sich ebenfalls nicht der Strafverfolgung verbündeter Reiche auf dem Gebiet des Kaiserreichs entziehen.\\XIXa tritt nur dann ein, wenn einer Person kein Asyl gewährt wurde.\\Einer Person darf Asyl gewährt werden, wenn sie in einem anderen Staat eine Straftat beging, die auf dem Gebiet des Kaiserreichs nicht als Verbrechen anerkannt wird.\\Das Recht auf Asyl darf einer Person jederzeit entzogen werden.
\\\\
XX. Entschädigungssteuern:\\Auf Entschädigungen werden zusätzlich zu den, im Recht fixierten Bußgeldsätzen, eine Steuer erhoben. \\Der Steuersatz wird alle 50 Tage von den Actimvardimen festgelegt.\\Die Steuern umfassen einen Mindestbetrag von 1 HTK und werden stets aufgerundet.
\\\\
XXI. Finanzieller Status des Reichs:\\Die Kaiser können auf nationaler Ebene nicht verschuldet sein.
\\\\
XXII. Schulden:\\Jegliche Schulden, die man beim Kaiserreich hat, müssen innerhalb von 100 Tagen zurückgezahlt werden.\\Tut man dies nicht, verliert man bis zur Rückzahlung zusammen mit 18 zusätzlichen HTK oder Gegenständen mit äquivalenten Wert die Kreditfähigkeit im Kaiserreich.
\\\\
XXIII. Pferde:\\Pferde sind innerhalb der Stadt nicht als Fortbewegungsmittel gestattet. Jeglicher Verstoß wird mit einer Bußgeldstrafe von 2 HTK geahndet.\\Reitet man mit einem Pferd in den Palasthof des Weißen Palasts, so muss man 6 HTK zahlen.
\\\\
XXIV. Handelsvertrag\\Wer Hamavarische Güter erwirbt, stimmt dem zu, dass er bei Weiterverkauf derer, derer Nebenprodukte oder derer Nachkommen (beispielsweise Setzlinge als Nachkommen eines Setzlings), 10% des Gewinns Hamavar bezahlt. Jeglicher Verstoß wird gemäß I und XIV geahndet.\\Dies gilt auch für jegliche Händler, die die Genehmigung haben, selbst Güter auf hamavarischem Grund anzubauen, zu produzieren, oder abzubauen.\\Zur Klasse der Güter gehört auch Land
\\\\
XXV. Staatssymbolik\\Das Verunglimpfen und das Abhängen von Staatssymbolik wird mit dem Tode und 36 HTK Bußgeld bestraft.
\\\\
XXVI. Staatsbürgerschaft\\Es ist Staatsbürger, wer durch den Staat als solcher bestätigt wurde.\\Anrecht auf eine Staatsbürgerschaft haben:\\Alle Besitzer hamavarischen Grundes\\Alle Angehörigen des hamavarischen Adels\\Alle Arbeitstätigen, die hauptsächlich auf hamavarischem Gebiet arbeiten
\\\\
XXVII. Hochverrat\\Als Hochverräter gilt, wer\\Staatsgeheimnisse ohne Genehmigung verbreitet oder versucht auf diese unerlaubt zuzugreifen.\\Eine absichtliche Schwächung des Staates herbeiführt\\       Der Strafsatz gleicht dem Strafsatz des Mordes an einem Kaiser.
\\\\
XXVIII. Polizeiliche Kontrollen\\Polizeiliche Kontrollen werden durch Staatspolizisten ausgeführt.\\Sie dürfen Leute, die illegale Dinge mit sich führen, hinrichten sofern diese der dritten Aufforderung, sie abzugeben, nicht nachgehen.\\Auch dürfen sie Leute, die kein gültiges Visum mit sich tragen und dennoch sich nach der dritten Aufforderung noch auf dem Gebiet aufhalten, ebenfalls hinrichten.
\\\\
XXIX. Visum\\Man muss ein gültiges Visum mit sich führen, ansonsten droht eine Bußgeldstrafe in Höhe von 20 HTK.
\\\\
XXX. Verbotene Gegenstände\\Man darf keine verbotenen Gegenstände mit sich führen, ansonsten droht eine Hinrichtung.
\\\\
XXXI. Religiöse Gegenstände\\Wer Gegenstände religiöser Natur beschädigt oder zerstört oder auf religiösem Boden Verbrechen begeht, muss eine Bußgeldstrafe in Höhe von 200 HTK zahlen und wird hingerichtet.
\\\\
XXXII. Haftstrafe\\Eine Haftstrafe kann bei Beschluss des Gerichts entweder als Strafersatz oder Strafzusatz angewendet werden.\\Der Gefangene muss hierbei all seine Sachen aus dem Inventar legen und einen doppelten Screenshot vom leeren Inventar oder vom vollen Inventar schicken.\\Anschließend werden alle Chorusfrüchte und Enderperlen aus dem Inventar entfernt und der Gefangene wird in den Abenteurermodus versetzt und verliert zusätzlich temporär alle Operatorrechte. Zudem muss der Gefangene in seiner Zelle einen Spawnpunkt setzen.\\Bei Ausbruchsversuchen und Ausbrüchen werden stets zehn Minuten zusätzliche Haft angeordnet.\\Der Gefangene darf bei Widerstand in die Zelle zurückteleportiert werden.\\Beihilfe bei Ausbrüchen werden mit dem Verordnen der gleichen Haftstrafe für die helfende Partei bestraft.\\Abgesessen hat man die Strafe, sobald man die jeweilige Zeit nachweislich online war.\\Der Staat haftet für keine Gegenstände, die während der Haftstrafe verloren gehen, sofern für den Häftling genügend Zeit bestand, die Gegenstände anderweitig zu lagern.
\\\\
XXXIII. Sklaverei\\Sklaverei und Menschenhandel werden mit 40 HTK Bußgeld und einer Todesstrafe bestraft.\\ 
\\
XXXIV. Menschenexperimente\\Menschenexperimente sind nur unter staatlicher Aufsicht erlaubt.\\Dies erfordert keine Einverständnis der Testperson.\\Der Staat kann Einspruch gegen die Wahl der Testperson erheben.
\\\\
XXXV. Geldwäsche\\Wer sich ohne Genehmigung der Lotus Bank HTK prägt, muss eine Haftstrafe absitzen. Weiterhin wird das Konto der Person geleert und ihr temporär alle Geldzufuhren abgestellt. Die Person verliert somit ihre Kreditfähigkeit und all ihre Immobilien. Alles weitere wird gemäß XIVb gehandhabt.
\\\\
XXXVI. Effekte\\Man darf in der Stadt keine Effekte ohne Genehmigung haben. Verstöße werden mit 10 HTK Bußgeld vergolten.\\Man darf ebenso wenig ohne Genehmigung das hamavarische Territorium im Zuschauermodus durchqueren
\\\\
XXXVII. Majestätsbeleidigung\\Beleidigt man den Kaiser, den Staat oder übergeordnete Staatsvertreter, so muss man 30 HTK Strafe zahlen.
\\\\
XXXVIII. Siegelfälschung\\Wer ein Schwarzsiegel, staatliches Zertifikat oder einen historischen Gegenstand ungenehmigt dupliziert, muss 1000 HTK Strafe zahlen. Zudem muss der Gewinn, der dadurch erwirtschaftet wurde, zurückgezahlt werden.
\\\\
XXXIX. Hehlerei\\Wer gestohlene Gegenstände verkauft, muss 50 HTK Strafe zahlen.\\Gewerbsmäßige Hehlerei wird mit dem Tode vergolten.
\end{document}