\documentclass{article}
\usepackage[utf8]{inputenc}
\usepackage{textcomp}
\usepackage{amsmath}
\usepackage{enumerate}

\renewcommand{\thesection}{\Roman{section}. Abschnitt:}
\counterwithout{subsection}{section}
\renewcommand{\thesubsection}{Artikel \arabic{subsection}}

\title{Kriegsrecht}
\author{Serverkonzil}
\date{20. Februar 1185}

\begin{document}
\maketitle
\vspace*{\fill}
\paragraph{Kriegsrecht gemäß reformierter Gesetzgebung Ellendirs IV.}

\newpage
\section{Allgemeine Bestimmungen}
\subsection{Gültigkeit}
Das Kriegsrecht ist international gültig und wird akzeptiert, sobald man einmal den Server betritt.

\subsection{Grundregeln}
Es besteht eine Gewichtung von 75\% theoretischer Kriegsführung und 25\% praktischer Kriegsführung.

\subsection{Jurisdiktion}
Verstößt man gegen das Kriegsrecht, so verliert man je nach Schwere eine Schlacht, eine Region, eine Provinz oder den Krieg.
\begin{enumerate}[(1)]
	\item Neutrale Kriegsparteien fungieren im Krieg als Kriegsrichter und entscheiden sowohl über Strafmaße als auch die Gültigkeit von Zügen.
	\item Als neutral werden jene bezeichnet, die sich nicht durch truppentechnische Unterstützung einmischen, also Truppen für den Krieg zur Verfügung stellen.
\end{enumerate}

\subsection{Stellung zu Kriegsverbrechen}
Folgende Kriegsverbrechen werden nicht als Verstoß gegen das Kriegsrecht erachtet:
\begin{enumerate}
	\item Völkermord
	\item Gezieltes Herbeiführen ziviler Opfer
	\item Angriffskrieg
	\item Misshandlung von Kriegsgefangenen
\end{enumerate}

\section{Theoretische Kriegsführung}
\subsection{Erster Krieg}\label{erster}
Vor Kriegsbeginn müssen alle Kriegsparteien eine Liste der Siedlungen zusammenstellen, aus denen sie die Truppen heranziehen.\\

\begin{enumerate}[(1)]
	\item Besteht der Verdacht, dass jemand aufgrund des bevorstehenden Krieges die Anzahl der Einwohner mit einer Methode erhöht hat, die gegen das mündliche Serverrecht verstößt, kann diese Stadt durch die Kriegsrichter von der Rekrutierung ausgeschlossen werden.
	\item Die Anzahl der Soldaten beläuft sich auf das 2.500-fache der Gesamtzahl einbezogener Einwohner.
	\item Die Anzahl der Schiffe beläuft sich auf das zehnfache der Schiffe, die im Reich der jeweiligen Partei bestehen.
	\item Schiffe müssen nicht eingetragen werden, da diese generell in Vollzahl eingezogen werden.
	\item Die Truppen müssen in Teilverbände, beispielsweise Armeen, Flotten, Bataillone, eingeteilt werden. Dies wird als Basis für alle nachfolgenden Kriege verwendet.
	\item Die Truppenstärke des Heeres und die Anzahl der Schiffe der Flotte müssen in realistischem Verhältnis zur Wirtschaftskraft und Reichsgröße stehen.
\end{enumerate}

\subsection{Erste Runde}
Handelt es sich nicht um den ersten Krieg, so beginnt jede Kriegspartei die Runde mit der Rekrutierung neuer Truppen. Truppenverluste werden hierbei
um 10\% des Ursprungsbestands pro vergangenes Jahr seit dem letzten Krieg beglichen. Dies kann jedoch höchstens zu einem vollständigen Begleichen der Verluste führen, allerdings nicht
zur Rekrutierung neuer Truppen. Die Rekrutierung neuer Truppen erfolgt, indem die Einwohner angegebener Dörfer ausgezählt werden. Kam es zu einem Wachstum, so wird gemäß \ref{erster} Absatz 2 gerechnet, ansonsten
gleichermaßen die Truppenstärke vermindert. Es gilt ebenfalls, \ref{erster} Absatz 1 zu beachten. Hebt jemand am Tag des Kriegsbeginn die Truppen nicht aus, so werden sie so verteilt, wie die Dorfbewohner positioniert sind.

\subsection{Phasen}
Jede Runde stellt einen Tag dar und besteht aus drei Phasen. Diese sind:
\begin{enumerate}
	\item Strategische Vorbereitung
	\item Angriffsphase
	\item Bewegungsphase
\end{enumerate}
Sie müssen nicht in der aufgeführten Reihenfolge ausgeführt werden. Auch dürfen die Phasen gemischt werden, da der Begriff
der Phase lediglich der organisatorischen, nicht jedoch der tatsächlichen Unterscheidung dient. Um längere Zeiten zu überbrücken können sich die Spieler auch auf das Überspringen einiger Runden einigen.

\subsection{Strategische Vorbereitung}
Zur strategischen Vorbereitung gehört das Schließen von Verträgen und Evaluieren strategischer Möglichkeiten. So kann man
sich beispielsweise stabile Nahrungsversorgung beschaffen, sofern die Möglichkeit besteht. Auch kann man an jeder beliebigen Stelle, die unter der eigenen Kontrolle steht, nicht verwendete Truppen ausheben. Löst man eine Armee auf, so kann sie woanders wieder ausgehoben werden. Dies ist nur in vollständig kontrollierten Gebieten und nicht in demselben Krieg möglich.

\subsection{Angriffsphase}
In der Angriffsphase entsendet man Truppenteile in eine gegnerische Region. Hierbei kann man variable Truppengrößen nutzen, muss
allerdings betiteln, welche Truppenverbände man aussendet. 
\begin{enumerate}[(1)]
	\item Im Gegensatz zu Risiko darf man auch alle Truppen entsenden. Auch kann man gleichzeitig aus mehreren angrenzenden Regionen Unterstützungstruppen heranziehen.
	\item Bei der Wertung des Erfolgs wird gemäß \ref{macht} vorgegangen.
	\item Man darf beliebig viele Angriffe ausführen.
	\item Es können nur Truppen innerhalb einer Reichweite von 200 Hamavarischen Meilen attackiert werden.
	\item Wurde ein Angriff verübt, so befinden sich die Truppen, sofern diese siegreich waren, an der Stellung, die sie angegriffen haben.
	\item Ein besiegter Gegner kann sich in einen Radius von 500 Hamavarischen Meilen zurückziehen.
	\item Landungsschiffe können erst angegriffen werden, wenn die Eskortflotte zerstört wurde oder sich zurückzieht.
\end{enumerate}

\subsection{Macht}\label{macht}
Als Macht wird die Stärke eines Truppenteils bezeichnet.
\begin{enumerate}[(1)]
	\item Die Macht wird anhand der Anzahl an Soldaten, sowie der Stärke des Befehlshabers errechnet.
	\item Ein Zufallsgenerator errechnet die Ergebnisse einer Schlacht. Zu diesen gehört die Frage, ob gesiegt wurde und wie viele Verluste es auf beiden Seiten gab. In seltenen Fällen kann auch der Befehlshaber sterben.
	\item Im Falle der Marine haben nicht alle Schiffe denselben Wert. Siehe hierzu Appendix II.
	\item Zur Marine kann nur gezählt werden, was von den Kriegsrichtern als kampftauglich eingestuft wurde.
\end{enumerate}

\subsection{Bewegungsphase}
In der Bewegungsphase darf man Truppen in verfügbare Regionen bewegen.
\begin{enumerate}[(1)]
	\item Als nicht verfügbar gelten Gebiete, die keiner Kriegspartei angehören.
	\item Eine Ausnahme zu Absatz 1 ist, dass man mit neutralen Parteien Bündnisse schließen kann, laut welchen man sich durch deren Territorium bewegen darf.
	\item Eine Truppenbewegung dauert ohne Hindernisse einen Tag pro 100 Hamavarische Meilen \footnote{Dies entspricht 100 Blöcken}.
	\item Hindernisse verlangsamen den Fortschritt, indem die Anzahl der Tage, die man für deren Durchquerung benötigte, wären sie keine Hindernisse, mit deren Multiplikator verrechnet wird.
	\item Folgende Hindernisse gelten für das Heer mit x = Basiswert:
	\begin{enumerate}[1.]
		\item Meer: 0.5x
		\item Dschungel: 3x
		\item Berge: 2x
		\item Sumpf: 3x
		\item Schnee: 2x
	\end{enumerate}
	\item Für die Marine gilt 1/3x auf dem Meer mit x = Basiswert.
	\item Bewegt sich eine Landstreitkraft mittels Landungsschiffe über das Meer, ist ihre Truppenmacht gleich einem Zehntel der sonstigen Truppenmacht.
\end{enumerate}

\subsection{Truppenverschleiß}
Während der Bewegungsphase muss der Truppenverschleiß beachtet werden. Der Truppenverschleiß setzt erst nach drei Runden ein. Im Falle von Absatz 1 Nummer 3, 4 und 5 setzt es jedoch sofort ein.
\begin{enumerate}[(1)]
	\item Im Falle des Heeres kann es zu folgenden Arten des Verschleiß kommen:
	\begin{enumerate}[1.]
		\item Belagerung (Handhabung gemäß \ref{belagerung})
		\item Gemäß \ref{nahrung} abgeschnittene Nahrungsversorgung: 5\% pro Runde
		\item Durchquerung arktischer Gebiete: 5\% pro Runde
		\item Durchquerung tropischer Gebiete: 3\% pro Runde
		\item Durchquerung von Wüsten: 7\% pro Runde
	\end{enumerate}
	\item Im Falle der Marine kann es zu folgenden Arten des Verschleiß kommen:
	\begin{enumerate}[1.]
		\item Passieren feindlicher Außenposten: Festungswert gemäß \ref{festung} multipliziert mit 1\% pro Runde
	\end{enumerate}
	\item Bei der Marine kommt es zunächst zu einem Abzug der Truppenmacht. Reicht der Abzug aus, um ein Schiff zu versenken, wird dieses der Truppenstärke abgezogen. Es handelt sich dabei jedoch um keinen zusätzlichen Abzug.
\end{enumerate}

\subsection{Belagerung}\label{belagerung}
Es sind während der Angriffsphase auch Belagerungen möglich.
\begin{enumerate}[(1)]
	\item Für das Heer gilt hierbei:
	\begin{enumerate}[1.]
		\item Es muss sich innerhalb einer Reichweite von 200 Hamavarischen Meilen befinden.
		\item Ist dies der Fall, so müssen sie vor die Stadt, beziehungsweise Festung vorrücken.
	\end{enumerate}
	\item Für die Marine gilt zusätzlich, dass ihre Angriffsstärke zwei Drittel der eigentlichen Macht beträgt.
	\item Die Dauer einer Belagerung wird anhand der Festungsstärke (\ref{festung}) errechnet.
	\item Belagerte Truppen können sich vor die Stadt bewegen.
	\item Bei einer Belagerung gilt, dass der Verschleiß der insgesamt stärkeren Partei 5\% pro Runde beträgt und der der schwächeren 10\%.
	\item Ist die Dauer aus Absatz 3 erschöpft, so muss man einen W4 würfeln. Wurde eine 1 gewürfelt, so ist die gesamte Bevölkerung verhungert. Andernfalls haben 10\% überlebt. In beiden Fällen hat man die Belagerung gewonnen und kann die Stadt einnehmen.
\end{enumerate}

\subsection{Nahrungsversorgung}\label{nahrung}
Eine stabile Nahrungsversorgung besteht, sofern ein Korridor besteht, der von Nahrungslieferanten betreten werden kann. Auf dem Meer kann dies durch eine Seeblockade unterbunden werden.

\subsection{Festungsstärke}\label{festung}
Die Festungsstärke lässt sich aus der Zusammenrechnung folgender Faktoren ermitteln:
\begin{enumerate}[1.]
	\item Lage: Bestehen auffällige Topographien in der Umgebung, die einen schnellen Vorstoß verhindern könnten, so legen die Kriegsrichter nach eigenem Ermessen einen Wert zwischen 1 und 6 fest. Dies gilt nicht für festgelegte Fälle:
	\begin{enumerate}[a.]
		\item Hügel: Der Ort befindet sich in einer hügeligen Region. Hinderniswert von 2
		\item Gebirge: Der Ort befindet sich in einer bergigen Region. Hinderniswert von 5
		\item Insel: Der Ort ist von einem Binnengewässer umschlossen. Hinderniswert von 4
		\item Fluss: Ein Fluss muss überquert werden. Hinderniswert von 1
		\item Berglage: Der Ort befindet sich auf einem Berg und ist zu Fuß unzugänglich. Hinderniswert von 6
	\end{enumerate}
	\item Befestigungen: Pro vollständigen Befestigungswall wird folgendermaßen gerechnet:
	\begin{enumerate}[a.]
		\item Dicke der Mauer: 1 pro Block
		\item Dicke des Kerns (sofern aus Obsidian): 5 pro Block
		\item Höhe der Mauer: 1/4 pro Block
	\end{enumerate}
	\item Größe der befestigten Anlage: Die Größe der Anlage wird nach folgendem Schema bemessen:
	\begin{enumerate}[a.]
		\item Siedlung: 0 (beispielsweise Seenstadt)
		\item Außenposten: 1 (beispielsweise Burg Klüftige Hügel)
		\item Kleinstadt: 1 (beispielsweise Black Schlong City)
		\item Burg: 2 (beispielsweise Festung von Lúinna / Festung von Gúrra)
		\item Stadt: 2 (beispielsweise Schwan City)
		\item Festung: 3 (beispielsweise Festung von Nordmakedonien - wäre sie denn irgendwann mal fertig)
		\item Großstadt: 4 (beispielsweise Húktim / Illúthrrin)
		\item Adelsburg: 5 (beispielsweise Durrve Alkanhèt)
		\item Metropole: 6 (Hrrátim)
	\end{enumerate}
\end{enumerate}

\section{Appendix I: Beispielhafte Errechnung der Festungsstärke Hrrátims}
\begin{itemize}
	\item 2 vollständige Befestigungswälle mit einer Breite und Höhe von vier Blöcken, kein Kern = 18
	\item 7 vollständige Befestigungswälle mit einer Breite von fünf Blöcken, 3 Blöcke Obsidiankern und einer Höhe von 16 Blöcken = 126
	\item Metropole: 6
	\item Eine innere Befestigung (Festung von Kingsford) mit der Größe einer Burg und einer 5 Blöcke breiten, 8 Blöcke hohen und mit 3 Blöcken Obsidian gefüllten Mauer: 18
\end{itemize}

= 168 (Tage)

\section{Appendix II: Macht verschiedener Kriegsschiffsklassen}
\begin{itemize}
	\item Ellendir-Klasse (Zweimaster, Galeere): 4
	\item Ferdinand-Klasse (Zweimaster, Brigge): 5
	\item Gúrra-Klasse (Einmaster, Langschiff): 4
	\item Havenstett-Klasse (Einmaster, Langschiff): 2
	\item Húktim-Klasse (Einmaster, Langschiff): 0
	\item Ellendir-Klasse Erster Ordnung (Dreimaster, Galeone): 12
	\item First-Order-Klasse (Zweimaster, Kogge): 4
	\item Ellendir-Klasse Zweiter Ordnung (Dreimaster, Galeone): 18
\end{itemize}

\end{document}